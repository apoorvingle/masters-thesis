\chapter{Programming in \qub{}}\label{chp:qub-programming}

In this chapter, we illustrate using examples how \qub{} is different
from other functional languages and how a powerful type system based on logic of \BI{}
would be used to track resources. The examples show how the resources use
can be tracked at compile time and resource leaks can be avoided.

\section{File Handles}\label{sec:file-handle-example}

In modern programming languages resources, such as files, are treated as normal variables.
It is the programmer's responsibility to check that that files are not closed twice
and no files that remain open when they are no longer in use. This seemingly trivial
responsibility becomes tedious and error prone as the program gets more complex.
Modern functional languages, such as Haskell, enforces the file input and output to be wrapped in a \HaskellF{IO} Monad.
This is more declarative than imperative languages, but the type system is not powerful enough
to detect whether a file handle is closed twice or is not closed at all.
Consider the type signatures for functions for file handling shown in \cref{fig:file-handling-function}. We explicitly
return the file resource i.e. \HaskellF{f} to keep track of the resource.
A simple program in Haskell that opens a file and reads a line from it and then closes the file handle using this
flavor of file handling functions is shown in \cref{fig:file-read-close}.

\begin{figure}[h]
  \begin{framed}
\begin{minted}[escapeinside=||,mathescape=true,linenos]{haskell}
openFile :: FilePath -> IO FileHandle
closeFile :: FileHandle -> IO ()
readLine :: FileHandle -> IO (String, FileHandle)
writeFile :: String -> FileHandle -> IO ((), FileHandle)
\end{minted}
  \end{framed}
  \caption{File Handling Functions}
  \label{fig:file-handling-function}
\end{figure}

\begin{figure}[h]
  \begin{framed}
\begin{minted}[escapeinside=||,mathescape=true,linenos]{haskell}
do f  <- openFile "sample.txt"
(s, f)  <- readLine f
() <- closeFile f
\end{minted}
  \end{framed}
  \caption{Reading from a file in Haskell}
  \label{fig:file-read-close}
\end{figure}

Consider an incorrect version of a program where the file handle is closed twice after reading a line from it as shown in \cref{fig:file-read-close-2times}.
It may not cause problems in a single threaded environment, but in a multi-threaded environment
the second close may accidently close the file handle that may have been reused in the background by another thread.
When another thread tries to write on this closed file handle, it would throw an exception.
Haskell's type system would accept this program but it might generate a runtime exception.
\begin{figure}[h]
  \begin{framed}
    \begin{minted}[escapeinside=||,mathescape=true,linenos]{haskell}
do f  <- openFile "sample.txt"
   (s, f)  <- readLine f
   () <- closeFile f
   () <- closeFile f
    \end{minted}
  \end{framed}
  \caption{Reading from a File and Closing it Twice}
  \label{fig:file-read-close-2times}
\end{figure}

Apple's goto fail bug that appeared in iOS 7.0 and caused a security vulnerability in 2012 is a similar example of closing the file twice.
The code snippet that caused the SSL/TLS handshake to be completely skipped looked like in \cref{fig:goto-fail}.
The second \mintinline{C}{goto fail;} on line 6 would always force the protocol to skip the
steps to be taken after the if-block. This made the system vulnerable to a man-in-middle attack.
\begin{figure}[h]
  \begin{framed}
\begin{minted}[escapeinside=||,mathescape=true, linenos]{C}
...
if ((err = SSLHashSHA1.update(&hashCtx, &serverRandom)) != 0)
    goto fail;
if ((err = SSLHashSHA1.update(&hashCtx, &signedParams)) != 0)
    goto fail;
    goto fail; // Cause of vulnerability
if ((err = SSLHashSHA1.final(&hashCtx, &hashOut)) != 0)
    goto fail;

err = sslRawVerify(...);
...
\end{minted}
  \end{framed}
  \caption{Goto Fail bug}
  \label{fig:goto-fail}
\end{figure}

Another example of incorrect way of using file handle is by not closing the file handle after using it shown in \cref{fig:file-read-noclose}.
In a short lived process the operating system closes the file handles that are not closed when the program exits.
In a long running process the operating system would run out of space to allocate new file handles. The whole process would crash with an error that
it cannot open any more file handles. Abnormal exit from the process may cause of loss of information as would interfere in the write process.
The operating system would close the file handle without waiting for the buffer to be completely
written on the file system.

\begin{figure}[h]
  \begin{framed}
    \begin{minted}[escapeinside=||,mathescape=true,linenos]{haskell}
do f <- openFile "sample.txt"
   (s, f)  <- readLine f
   return s
    \end{minted}
  \end{framed}
  \caption{Reading from a File and Not Closing File Handle}
  \label{fig:file-read-noclose}
\end{figure}

We take a deeper dive into this problem by inspecting the desugared version of the ``do'' notation.
Both the programs might be translated into bind ($>>=$) operations . Recall that type signature of bind function is given as
\HaskellF{(>>=) :: (Monad m) => m t -> (t -> m u) -> m u}. Desugared version of \cref{fig:file-read-close-2times}
is shown in \cref{fig:file-read-close-2times-desugared} and the desugared version of \cref{fig:file-read-noclose} is shown in \cref{fig:file-read-noclose-desugared}.

\begin{figure}[h]
\begin{framed}
\begin{minted}[escapeinside=||,mathescape=true,linenos]{haskell}
(>>=) (openFile "sample.txt" ReadMode) (\f ->
       >>= (readFile f) (\(s, f) -> return s)
\end{minted}
  \end{framed}
  \caption{Reading from a File in Haskell and Not Closing It (Desugared)}
  \label{fig:file-read-noclose-desugared}
\end{figure}

\begin{figure}[h]
  \begin{framed}
    \begin{minted}[escapeinside=||,mathescape=true, linenos]{haskell}
(>>=|$_1$|) (openFile "sample.txt") (\f ->
            (>>=|$_2$|) (readFile f) (\(s, f) ->
                        (>>=|$_3$|) (closeFile f) (\_ -> closeFile f)))
    \end{minted}
  \end{framed}
  \caption{Reading from a File in Haskell and Closing Twice (Desugared)}
  \label{fig:file-read-close-2times-desugared}
\end{figure}

In both the cases described above, the well typed looking program should be red flagged by the compiler, as they would
cause programs to crash at runtime. To solve this problem, we introduce the % concept of unrestricted type similar to Quill, and
concept of sharing and separation of resources from the logic of \BI{} in \qub{}. The type $\rightarrow$ now
has to be specified as either shared ($\shimp$) or separated ($\sepimp$). In \qub{} program code, we will use
\HaskellF{-*>} to mean $\sepimp$ and \HaskellF{-&>} to mean $\shimp$.
The file handling functions will have different type signatures in \qub{} as shown in \cref{fig:qub-file-handling-functions} to accommodate
the new function implication flavors. Similarly, the bind operation will also be typed differently as shown below as the resources of the function are
separate from its arguments.

\begin{figure}[h]
  \begin{framed}
    \begin{minted}[linenos]{haskell}
openFile :: FilePath -*> IO FileHandle
closeFile :: FileHandle -*> IO ()
readLine :: FileHandle -*> IO (String, FileHandle)
writeFile :: String -*> FileHandle -*> IO ((), FileHandle)

(>>=) :: (Monad m) => m t -*> (t -*> m u) -*> m u
    \end{minted}
  \end{framed}
  \caption{File Handling and Bind Function Type Signatures in \qub{}}
  \label{fig:qub-file-handling-functions}
\end{figure}

We now consider the types for the two faulty programs previously described with respect to \qub{}.
In \cref{fig:qub-close-file-twice} we see that the first \HaskellF{(>>=|$_1$|)}
and second bind \HaskellF{(>>=|$_2$|)} functions would have appropriate types
where each argument is separate from the the function. The file handing functions
return a new binding for the file resource \HaskellF{f}. However, we notice that the third bind operation
\HaskellF{(>>=|$_3$|)} would have a problem. It would be typed as
\HaskellF{IO () -*> (() -*> IO ()) -&> IO ()}\\
as both the arguments share the file variable \HaskellF{f}. This would be a type error
as the bind operation should have a type of \HaskellF{IO () -*> (() -*> IO ()) -*> IO ()}.
This mismatch in the types would be caught statically during the type checking phase of compilation.

\begin{figure}[h]
\begin{framed}
\begin{minted}[escapeinside=||, mathescape=true, linenos]{haskell}
(>>=|$_1$|) (openFile "sample.txt") (\f ->
            (>>=|$_2$|) (readFile f) (\(s, f) ->
                        (>>=|$_3$|) (close f) (\_ -> closeFile f)))


(>>=|$_1$|):: IO FileHandle -*> (FileHandle -*> IO ()) -*> IO ()
(>>=|$_1$|) (openFile "sample.txt" ReadMode) (\f -> ...)

(>>=|$_2$|) :: IO FileHandle
            -*> (FileHandle -*> IO (String, Filehandle))
            -*> IO (String, FileHandle)}
(>>=|$_2$|) (readLine f) (\(s,f) -> ... )

(>>=|$_3$|) :: IO () -*> (() -*> IO ()) -&> IO ()
(>>=|$_3$|) (closeFile f) (\_ -> closeFile f)}
\end{minted}
\end{framed}
\caption{Closing file twice in \qub{}}
\label{fig:qub-close-file-twice}
\end{figure}

We also have the concept of unrestricted types similar to Quill. A type that contains no resources, or which is implicitly
copied or dropped in the program is tagged as an unrestriced type. In the second faulty program,
shown in \cref{fig:file-read-noclose-desugared} the file handle \HaskellF{f} is not closed.
It is declared but not used in its scope. This would force the file handle to be be tagged as an unrestricted type by the \qub{} type checker.
This is a violation of the assumption that resources cannot be of the unrestricted type. Thus
the program would not type check due to mismatch of the file handle type to be inferred as unrestricted.

\section{Exception handling}
We expand on the file handling scenario and consider the code that can throw runtime exceptions.
The motivation to do so lies in the fact that memory leaks are caused because of runtime
exceptions where the part of code that is responsible to clean up resources. In Haskell, error handing is done using \HaskellF{MonadError}\citep{liang_monad_1995}.
The type class definition is shown in \cref{fig:haskell-monaderror}. \HaskellF{throwError} is used inside a monadic context to initiate
exception processing and \HaskellF{catchError} is used to handle or rethrow a previously thrown error and return
to a normal execution.

\begin{figure}[h]
  \begin{framed}
\begin{minted}[linenos=true, mathescape=true, linenos]{haskell}
class Monad m => MonadError e m | m -> e where
    throwError :: e -> m a
    catchError :: m a -> (e -> m a) -> m a
\end{minted}
  \end{framed}
  \caption{Haskell's MonadError Typeclass}
  \label{fig:haskell-monaderror}
\end{figure}

\noindent
A common way of handling error in Haskell is shown in \cref{fig:haskell-error-handling}.
We use the same file handling API as defined previously in \cref{fig:file-handling-function}
In normal code execution path, the first line of the file will be returned after closing the file.
In case of a runtime error, say \HaskellF{readLine f} throws an error, the \HaskellF{catchError} function will
invoke the handler function and return an appropriate error message. We notice that in case of an error the file handle \HaskellF{f}
is never closed and will cause a resource leak. The Haskell type system has no way to detect this memory leak.

\begin{figure}[h]
  \begin{framed}
\begin{minted}[linenos=true, escapeinside=||, mathescape=true]{haskell}
do f <- openFile "sample.txt"
   ((s, f)  <- readLine f
    let c = caps s
    () <- closeFile f
    return $ Right c) `catchError` (\_ ->
                             return $ Left "Error in reading file")
\end{minted}
  \end{framed}
  \caption{Handling Errors in Haskell}
  \label{fig:haskell-error-handling}
\end{figure}

We enforce resource clean-up in a systematic way in \qub{} with the help of the type system based on \BI{}.
For a concrete example, see \cref{fig:qub-file-exceptions}.
We will assume that \HaskellF{readLine} can throw an exception during runtime, where
it might fail to read a line due to the filehandle being stale and the file no longer exists.
For the sake of simplicity we will assume \HaskellF{closeFile}
does not throw exceptions. \HaskellF{openFile} throwing an exception would not cause a resource leak hence will not be a problem.
We have two kinds of IO operations, \HaskellF{IO} that does not fail or throw exceptions and is safe,
and \HaskellF{IOF} that can fail and throw exceptions. The function \HaskellF{throw} can be used by another function to start
the processing of the exception. The function \HaskellF{catch} can convert a code that can fail, to a code
that does not fail. It gives a way to clean up resources after the code throws an exception.
The sharing arrow between the two arguments of \HaskellF{catch} enables the clean up block of code to be executed on the same resources
that were used its first argument i.e. the same file handle will be shared between the arguments.

\begin{figure}[h]
  \begin{framed}
    \begin{minted}[linenos]{haskell}
openFile :: FilePath -*> IO FileHandle
closeFile :: FileHandle -*> IO ()
readFile :: FileHandle -*> IOF (String, FileHandle)
writeFile :: String -*> FileHandle -*> IOF ((), FileHandle)

throw :: Exception -*> IO a
catch :: IOF a -*> (Exception -*> IO a) -&> IO a

readFromFile :: FilePath -*> IO (Either String String)
readFromFile fpath =
do fh  <- openFile fpath
   ((s, fh)  <- readLine fh
   let l = caps s
   () <- closeFile fh
   return $ Right l) `catch`
           (\e -> do closeFile fh
                     return $ Left "Could not read file")
    \end{minted}
  \end{framed}
  \caption{Exception Handling in \qub{}}
  \label{fig:qub-file-exceptions}
\end{figure}


%%% Local Variables:
%%% mode: latex
%%% TeX-master: "../thesis-ku"
%%% End:
