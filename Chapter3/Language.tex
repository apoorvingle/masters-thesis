\chapter{Language Core Syntax and Types}

\begin{figure}[h]
  \begin{framed}
    \begin{flalign*}
      \text{Type Variables}\ \ \      t, u, v         &\in \text{TVar}  \nonumber\\
      \text{Kinds}\ \ \               \kappa          &::= \star \mid \kappa \rightarrow \kappa \nonumber\\
      \text{Type constructors}\ \ \   T^{\kappa}       &::= \mathcal{T}^{\kappa}\ \text{where}
                                                      \{\otimes, \with, \oplus, \parr, \overset{!}{\sepimp}, \sepimp, \xrightarrow{!}, \rightarrow \} \subseteq \mathcal{T}^{* \rightarrow * \rightarrow *}\nonumber\\
      \text{Types}\ \ \               \tau^{\kappa}    &::= t \mid T^{\kappa} \mid \tau^{\kappa' \rightarrow \kappa} \tau^{\kappa'}\nonumber\\
      \text{Predicates}\ \ \          \pi             &::= \texttt{Un}\ \tau \mid \texttt{SeFun}\ \tau \mid \texttt{ShFun}\ \tau \mid \tau \leq \tau' \nonumber\\
      \text{Qualified Types}\ \ \     \rho            &::= \tau^{*} \mid \pi => \rho \nonumber\\
      \text{Type schemes}\ \ \        \sigma          &::= \rho \mid \forall t. \sigma \nonumber
    \end{flalign*}
  \end{framed}
  \caption{Types and Kinds in \qub}
  \label{fig:qub-types}
\end{figure}
% Describe types
The type language consists of type variables and two kinds of binary type constructors the
sharing arrow ($\rightarrow$) and the separating arrow ($\sepimp$). The sharing arrow
would mean that the function shares resources with its argument and the separating
arrow would mean that the function does not share resources with its arguments.
We would write both the arrows in an infix notation. The kind system is simple where we use
$\star$ to denote all the types. We use $\tau$, $v$ and $\phi$ to denote types of any kind.
We also include additive and multiplicative sums or product types in our core language. The
definitions of all the four type constructors is described in \cref{sec:pairs} and \cref{sec:sums}.
The system is powerful enough to let programmers define their own types using the type constructors.
% We support user defined data types following
% Jones' [\cite{jones_system_1993}] by adding kind support to our language.

% Describe Predicates
The predicate system enhances the expressibility of the type system. Following the same route taken
by Quill [\cite{morris_best_2016}] we use the predicate \texttt{Un} $\tau$ to denote
that the type $\tau$ is unrestricted. We write \texttt{ShFun} $\tau$ to describe that type $\tau$ may share resources with its
argument types and \texttt{SeFun} $\tau$ to describe that $\tau$ is
does not share any resources from its argument types. The function types can also be of the unrestricted type.
Thus if a type $\tau$ is unrestricted i.e. it qualifies with predicate \texttt{Un} and it is also one of the function types
i.e. \texttt{SeFun} or \texttt{ShFun}, we write them as $\overset{!}{\sepimp}$ and $\xrightarrow{!}$ respectively.
This can be considered as an improving substitution following Jones notion of improvement of qualified types [\cite{jones_simplifying_1995}].
We also define an ordering on types by using the predicate $\geq$. The predicate $\tau \geq \tau'$ holds if the type $\tau'$
is less restricting than $\tau$. The predicate entailment relations $P => Q$ are given in \cref{fig:entailment-rules}.
\TODO{Explain what does it mean for a type to be less that other type}
\begin{figure}[h]
  \begin{framed}
    \begin{minipage}{0.20\linewidth}
      \begin{prooftree}
        \AxiomC{$\pi \in P$}
        \UnaryInfC{$P => \pi$}
      \end{prooftree}
    \end{minipage}
    \begin{minipage}{0.20\linewidth}
      \begin{prooftree}
        \AxiomC{$\bigwedge_{\pi \in Q} P => \pi$}
        \UnaryInfC{$P => Q$}
      \end{prooftree}
    \end{minipage}
    \begin{minipage}{0.20\linewidth}
      \begin{prooftree}
        \AxiomC{}
        \UnaryInfC{$P => \Un{(\tau \sepimp \tau')}$}
      \end{prooftree}
    \end{minipage}
    \begin{minipage}{0.20\linewidth}
      \begin{prooftree}
        \AxiomC{}
        \UnaryInfC{$P => \Un{(\tau \rightarrow \tau')}$}
      \end{prooftree}
    \end{minipage}
    \begin{minipage}{0.20\linewidth}
      \begin{prooftree}
        \AxiomC{$\tau = \sepimp \vee \tau = \overset{!}{\sepimp}$}
        \UnaryInfC{$P => \SeFun{\tau}$}
      \end{prooftree}
    \end{minipage}
    \begin{minipage}{0.20\linewidth}
      \begin{prooftree}
        \AxiomC{$\tau = \rightarrow \vee \tau = \overset{!}{\rightarrow}$}
        \UnaryInfC{$P => \ShFun{\tau}$}
      \end{prooftree}
    \end{minipage}
    \begin{minipage}{0.20\linewidth}
      \begin{prooftree}
        \AxiomC{$P => \Un{\tau}$}
        \UnaryInfC{$P => \tau \geq (v \overset{!}{\sepimp} v')$}
      \end{prooftree}
    \end{minipage}
    \begin{minipage}{0.20\linewidth}
      \begin{prooftree}
        \AxiomC{$P => \Un{\tau}$}
        \UnaryInfC{$P => \tau \geq (v \overset{!}{\rightarrow} v')$}
      \end{prooftree}
    \end{minipage}
    \begin{minipage}{0.5\linewidth}
      \begin{prooftree}
        \AxiomC{$$}
        \UnaryInfC{$P => \tau \geq (v \sepimp v')$}
      \end{prooftree}
    \end{minipage}
    \begin{minipage}{0.5\linewidth}
      \begin{prooftree}
        \AxiomC{$$}
        \UnaryInfC{$P => \tau \geq (v \rightarrow v')$}
      \end{prooftree}
    \end{minipage}
    \begin{minipage}{0.5\linewidth}
      \begin{prooftree}
        \AxiomC{$P => \tau \geq \phi t$}
        \AxiomC{$t\ \text{fresh}$}
        \BinaryInfC{$P => \tau \geq \phi$}
      \end{prooftree}
    \end{minipage}
    \begin{minipage}{0.5\linewidth}
      \begin{prooftree}
        \AxiomC{$P => \tau t \geq \phi$}
        \AxiomC{$t\ \text{fresh}$}
        \BinaryInfC{$P => \tau \geq \phi$}
      \end{prooftree}
    \end{minipage}
  \end{framed}
  \caption{Entailment Rules}
  \label{fig:entailment-rules}
\end{figure}


\begin{figure}[h]
  \begin{framed}
    \begin{flalign*}
      \text{Environments}\ \ \      \Gamma,\Delta     &::= \epsilon \mid x^{\bar{y}}:\sigma \mid \Gamma, \Delta \mid \Gamma; \Delta \nonumber\\
      \text{Environment Context}\ \ \ H, H'           &::= \epsilon \mid H,H' \mid H;H' \mid \square \nonumber\\
  \end{flalign*}
\end{framed}
  \caption{Typing Context}
  \label{fig:typing-context}
\end{figure}
% Describe Typing judgments
In normal type systems, the contexts are represented as sets or lists. In \BI\ they are represented as binary trees.
The leaf nodes contain the pair of variable and types. Internal nodes of the context tree are
connectives which can either be a semicolon ($;$) or a comma ($,$).
If a bunch $\Delta$ is a subtree of $\Gamma$, then we denote a subtree relation by $\Gamma(\Delta)$.
Two context are equivalent ($\Gamma \equiv \Delta$)if they can be transformed into one another by renaming the identifiers.
The bunches have a restriction that no identifier appears more than once. We restrict certain structural rules on the context
depending on the connectives being used. If contexts are combined using a comma ($,$), contraction and weakening is not admissible,
but if the contexts are combined using a semicolon ($;$) then it can undergo contraction and weakening. Exchange rule is admissible
in both the connectives. This distinction enables us to have a special treatment for resources in our language.
By associating a resource with a comma constructor, our type system will not disposed it off by using the contraction rule.
While, non-resourceful objects (or normal propositions) can be combined using the semi-colon constructor.
An example bunch is shown in \cref{fig:bunches-bi}. a and b have a shared context while c is separate from the bunch a and b.
If $\Gamma$ represents the complete bunch of \cref{fig:bunches-bi}, $\Delta \equiv (a:A; b:B)$ and $\Delta' \equiv (c:C)$
then $\Gamma \equiv \Delta,\Delta'$ and $\Gamma(\Delta)$.

\begin{figure}[h]
  \centering
  \tikzset{every tree node/.style={minimum width=2em},
         blank/.style={draw=none},
         edge from parent/.style=
         {draw,edge from parent path={(\tikzparentnode) -- (\tikzchildnode)}},
         level distance=1.5cm}
\begin{tikzpicture}
\Tree
[.,
    [.;
        [.a:A ]
        [.b:B ]
    ]
    [.c:C ]
    ]
\end{tikzpicture}
\caption{Bunches in \textbf{\em BI}}
\label{fig:bunches-bi}
\end{figure}

We try to emulate the behaviour of bunches using multisets and denote multiset union using $\Gamma \sqcup \Gamma'$.
The type assignment will be a collection of variables with an added annotation of sharing. If a resource $x$ is in
sharing with one or many resources $\bar{y}$ we would represent it as $x^{\bar{y}}:\tau$.
Further $\Gamma, x^{\bar{y}}:\tau$ would mean $\Gamma \sqcup \{x^{\bar{y}}:\tau\}$. We define a few auxilary functions on the
type assigments. Vars($\Gamma$) is the set of all the term variables in $\Gamma$. Shared($\Gamma$) computes
the set of all the term variables that are in sharing with each other. Used($\Gamma$) computes the
union of all the term variables in the type assignment and the term variables shared by each of those.
We define two partial operators on type assigments as shown in \cref{fig:context-operation}.
Two type assigments are said to be in disjoint union ($\circledast$) if they have no common used term.
If the type assignments have an exact overlapping of terms being used, it is said to be in a sharing union ($\varoplus$).
\begin{figure}[h]
  \begin{framed}
    \noindent
    \begin{flalign*}
      \text{Vars}(\Gamma, x^{\bar{y}}) &= \text{Vars}(\Gamma) \cup \{ x \}\\
      \text{Shared}(\Gamma, x^{\bar{y}}) &= \text{Shared}(\Gamma) \cup \{ \bar{y} \}\\
      \text{Used}(\Gamma) &= \text{Vars}(\Gamma) \cup \text{Shared}(\Gamma)\\
      (\Gamma, x^{\bar{y}})^{[a \mapsto \bar{b}]} &= \begin{cases}
        x \notin \bar{y}\ \ \ \ (\Gamma^{[a \mapsto \bar{b}]}, x^{\bar{y}}:\tau)\\
        x \in \bar{y}\ \ \ \  (\Gamma^{[a \mapsto \bar{b}]}, x^{(\bar{y}\backslash a)\cup\bar{b}}:\tau)
      \end{cases}\\
      \Gamma^{[\bar{a} \mapsto \bar{b}]} &= (\dots((\Gamma^{[a_1 \mapsto \bar{b}]})^{[a_2 \mapsto \bar{b}]})^{\dots})^{[a_2 \mapsto \bar{b}]}
    \end{flalign*}
  \end{framed}
  \caption{Auxilary Functions on Multisets}
  \label{fig:multiset-aux-function}
\end{figure}
\begin{figure}[h]
  \begin{framed}
    \begin{flalign*}
      \Gamma \circledast \Gamma' &= \Gamma \sqcup \Gamma' => \text{if}\ \text{Vars}(\Gamma) \# \text{Used}(\Gamma')\ \text{and}\ \text{Vars}(\Gamma')\# \text{Used}(\Gamma) \\
      \Gamma \varoplus \Gamma'   &= \Gamma \sqcup \Gamma' => \text{if}\ \text{Used}(\Gamma) \equiv \text{Used}(\Gamma')
    \end{flalign*}
  \end{framed}
  \caption{Context Operations}
  \label{fig:context-operation}
\end{figure}


\begin{figure}[h]
  \begin{framed}
    \begin{flalign*}
      \text{Term Variables}\ \ \  x, y, z  &\in \text{Var} \nonumber\\
      % \text{Patterns}\ \ \        p        &::= x \mid C \vec{x}\nonumber\\
      \text{Expressions}\ \ \     M, N     &::= x \mid \lambda^{*}x. M \mid \lambda^{\alpha}x. M \mid M N\nonumber\\
      &\mid \CaseSh{M}{\{\texttt{inl}_{sh}\ x \mapsto N ; \texttt{inr}_{sh}\ y \mapsto N'\}}\mid \texttt{inl}_{sh}\ x \mid \texttt{inr}_{sh}\ y \nonumber\\
      &\mid \CaseSe{M}{\{\texttt{inl}_{se}\ x \mapsto N ; \texttt{inr}_{se}\ y \mapsto N'\}}\mid \texttt{inl}_{se}\ x \mid \texttt{inr}_{se}\ y \nonumber\\
      &\mid \Let{x}{M}{N} \mid \Pair{M,N} \mid \Pair{M;N} \mid \texttt{fst}_{sh}\ M \mid \texttt{snd}_{sh}\ M \mid \texttt{fst}_{se}\ M
      \mid \texttt{snd}_{se}\ M \nonumber
    \end{flalign*}
  \end{framed}
  \caption{Language Syntax}
  \label{fig:quill-terms}
\end{figure}
% Describe the language here

% Describe terms and patterns
Our term language is similar to that of simply typed lambda calculus involving variables and application
but we have 2 different kind of lambdas. The alpha lambda ($\lambda^{\alpha} x. M$) denotes sharing
of the argument term with the expression $M$ and the separating lambda term ($\lambda^{*} x. M$) that implies
the argument term has a separating context with the expression $M$. We also have polymorphic $\texttt{let}$
expressions to be able to define functions with a limited scope.

% The type constructors are added in order to allow programmers to define their own data types. They can be used to define sum and product types.
% \texttt{case} expression can be used to pattern match on the expression to express it in terms
% of individual sum types. Patterns are either term variables or constructor terms.


\section{Conventions and Notations}
The type assignment $\Gamma$ is a finite multiset of pairs of terms and types such that if two term variables occur
more than once, all of them must have same type. i.e. if $x:\sigma \in \Gamma$ and $x:\sigma' \in \Gamma$ then $\sigma \equiv \sigma'$.
The expression $P \mid \Gamma \vdash M : \sigma$ denotes the assersion that the term $M$ is of type $\sigma$
when the predicates in $P$ are satisfied and the free type variables in $M$ are specifed in type assigment $\Gamma$.
$\Gamma_{x}$ denotes the type assigment excluding the type variable $x$. $TV(\Gamma)$ denotes free type
variables in the environment $\Gamma$. $\Gamma \odot \Delta$ means that the contexts can either
be sharing union ($\Gamma \varoplus \Delta$) or separating union ($\Gamma \circledast \Delta$).

\section{Typing Rules}
% Structural Rules
% Connective Rules
% forall, => Qualified type rules
We split our type system into multiple sections for legibility. The first section includes structrual rules
as shown in \cref{fig:structural-rules} and the second section includes connectives with introduction and elemination rules
as shown in \cref{fig:typing-rules}.

The tautology rule ([ID]) is a simple type assigment lookup for checking the type of the term.
The exchange rule ([EXCH]) justifies the fact that the order in which the type assignments are witnessed
in the multiset is insignificant and we can commute the individual term typing schemes within the multiset context
with no effect to the typing derivations. The contraction sharing rule [CTR-SH] and weakening sharing rule [WKN-SH]
says that we can duplicate or drop certain pairs of type assigments as per our convience as we know they are in sharing with other
terms that remain in the context. The contraction separation rule [CTR-UN] and weakening separation rule [WKN-SH] can be
applied to terms only if we can prove that they are of unrestricted type which is captured by introducting
the ($\Delta$ \texttt{un}) predicate on the type that is dropped or duplicated.

\begin{figure}[h]
  \begin{framed}
    % var
    \begin{minipage}{.5\textwidth}
      \begin{prooftree}
        \AxiomC{} \RightLabel{[ID]}
        \UnaryInfC{$P \mid x^{\bar{y}} : \sigma \vdash x : \sigma $}
      \end{prooftree}
    \end{minipage}
    % EXCH
    \begin{minipage}{.5\textwidth}
      \begin{prooftree}
        \AxiomC{$\Gamma \odot \Delta \vdash M : \sigma$} \RightLabel{[EXCH]}
        \UnaryInfC{$P \mid \Delta \odot \Gamma \vdash M : \sigma $}
      \end{prooftree}
    \end{minipage}
        % CTR UN
    \begin{minipage}{.5\textwidth}
      \begin{prooftree}
        \AxiomC{$P \mid \Gamma \circledast \Delta \vdash M : \sigma$}
        \AxiomC{$P'\vdash \Delta\ \texttt{un}$} \RightLabel{[CTR-UN]}
        \BinaryInfC{$P \mid \Gamma \vdash M : \sigma$}
      \end{prooftree}
    \end{minipage}
    % WKN UN
    \begin{minipage}{.5\textwidth}
      \begin{prooftree}
        \AxiomC{$P \mid \Gamma \vdash M : \sigma$}
        \AxiomC{$P' \vdash \Delta\ \texttt{un}$} \RightLabel{[WKN-UN]}
        \BinaryInfC{$P \mid \Gamma \circledast \Delta \vdash M : \sigma$}
      \end{prooftree}
    \end{minipage}
    % CTR Sh
    \begin{minipage}{.5\textwidth}
      \begin{prooftree}
        \AxiomC{$P \mid \Gamma \varoplus \Delta \vdash M : \sigma$} \RightLabel{[CTR-SH]}
        \AxiomC{$P \mid \Gamma \vdash M : \sigma$} \RightLabel{[CTR-SH]}
        \BinaryInfC{$P \mid \Gamma \vdash M : \sigma$}
      \end{prooftree}
    \end{minipage}
    % WKN Sh
    \begin{minipage}{.5\textwidth}
      \begin{prooftree}
        \AxiomC{$P  \mid \Gamma \vdash M : \sigma$} \RightLabel{[WKN-SH]}
        \UnaryInfC{$P \mid \Gamma \varoplus \Delta \vdash M : \sigma$}
      \end{prooftree}
    \end{minipage}
   \end{framed}
    \caption{Structural Typing Rules}
    \label{fig:structural-rules}
  \end{figure}

  The connective rules explained here.


  \begin{figure}[h]
    \begin{framed}
    % let
    \begin{minipage}{1\textwidth}
      \begin{prooftree}
        \AxiomC{$P \mid \Gamma \vdash M : \sigma$}
        \AxiomC{$P' \mid \Gamma_{x} \odot x: \sigma \vdash N: \tau$} \RightLabel{[LET]}
        \BinaryInfC{$P \cup P' \mid \Gamma \vdash \Let{x}{M}{N}: \tau$}
      \end{prooftree}
    \end{minipage}
    % forall I
    \begin{minipage}{0.5\textwidth}
      \begin{prooftree}
        \AxiomC{$P \mid \Gamma \vdash M: \sigma$}
        \AxiomC{$t \notin TV(\Gamma) \cup TV(P)$}\RightLabel{$[\forall I]$}
        \BinaryInfC{$P \mid \Gamma \vdash M: \forall t. \sigma$}
      \end{prooftree}
    \end{minipage}
    % forall E
    \begin{minipage}{0.5\textwidth}
      \begin{prooftree}
        \AxiomC{$P \mid \Gamma \vdash M: \forall t.\sigma$}\RightLabel{$[\forall E]$}
        \UnaryInfC{$P \mid \Gamma \vdash M: [\tau \backslash t] \sigma $}
      \end{prooftree}
    \end{minipage}
    % => I
    \begin{minipage}{0.5\textwidth}
      \begin{prooftree}
        \AxiomC{$P, \pi \mid \Gamma \vdash M : \rho$} \RightLabel{$[=> I]$}
        \UnaryInfC{$P \mid \Gamma \vdash M : \pi \Rightarrow \rho$}
      \end{prooftree}
    \end{minipage}
    % => E
    \begin{minipage}{0.5\textwidth}
      \begin{prooftree}
        \AxiomC{$P \mid \Gamma \vdash M : \pi \Rightarrow \rho$}
        \AxiomC{$P \Rightarrow \pi$} \RightLabel{$[=> E]$}
        \BinaryInfC{$P \mid \Gamma \vdash M: \rho$}
      \end{prooftree}
    \end{minipage}
    % -&> I
    \begin{minipage}{1\textwidth}
      \begin{prooftree}
        \AxiomC{$P \mid \Gamma^{[\emptyset\mapsto \{x\}]},x^{\text{Vars}(\Gamma)}: \tau \vdash M : \tau'$}
        \AxiomC{$P \Rightarrow \texttt{ShFun}\ \phi$}
        \AxiomC{$P \vdash \Gamma \geq \phi$} \RightLabel{$[\rightarrow I]$}
        \TrinaryInfC{$P \mid \Gamma \vdash \lambda^{\alpha}x. M : \phi \tau \tau'$}
      \end{prooftree}
    \end{minipage}
    % -&> E
    \begin{minipage}{1\textwidth}
      \begin{prooftree}
        \AxiomC{$P \mid \Gamma \vdash M : \phi \tau \tau'$}
        \AxiomC{$P \mid \Delta \vdash N : \tau'$}
        \AxiomC{$P => \texttt{ShFun}\ \phi$}
        \AxiomC{$P \vdash \Gamma \geq \phi$} \RightLabel{$[\rightarrow E]$}
        \QuaternaryInfC{$P \mid \Gamma \varoplus \Delta \vdash M N : \tau'$}
      \end{prooftree}
    \end{minipage}
    % -*> I
    \begin{minipage}{1\textwidth}
      \begin{prooftree}
        \AxiomC{$P \mid \Gamma,x^{\emptyset}: \tau \vdash M : \tau'$}
        \AxiomC{$P => \texttt{SeFun}\ \phi$}
        \AxiomC{$ P \vdash \Gamma \geq \phi$} \RightLabel{$[\sepimp I]$}
        \TrinaryInfC{$P \mid \Gamma \vdash \lambda^{*}x. M : \phi \tau \tau'$}
      \end{prooftree}
    \end{minipage}
    % -*> E
    \begin{minipage}{1\textwidth}
      \begin{prooftree}
        \AxiomC{$P \mid \Gamma \vdash M : \phi \tau \tau'$}
        \AxiomC{$P \mid \Delta \vdash N : \tau$}
        \AxiomC{$P => \texttt{SeFun}\ \phi$}
        \AxiomC{$P \vdash \Gamma \geq \phi$} \RightLabel{$[\sepimp E]$}
        \QuaternaryInfC{$P \mid \Gamma \circledast \Delta \vdash M N : \tau'$}
      \end{prooftree}
    \end{minipage}
  \end{framed}
    \caption{Connective Typing Rules}
    \label{fig:typing-rules}
\end{figure}

\begin{figure}[h]
  \begin{framed}
  \begin{minipage}{0.5\linewidth}
    \begin{prooftree}
      \AxiomC{$P => \texttt{Un}\ \tau$}\RightLabel{[\texttt{Un}-$\tau$]}
      \UnaryInfC{$P \vdash \tau\ \texttt{un}$}
    \end{prooftree}
  \end{minipage}
  \begin{minipage}{0.5\linewidth}
    \begin{prooftree}
      \AxiomC{$P,\pi \vdash \rho\ \texttt{un}$}\RightLabel{[\texttt{Un}-$\rho$]}
      \UnaryInfC{$P \vdash \pi => \rho\ \texttt{un}$}
    \end{prooftree}
  \end{minipage}
  \begin{minipage}{0.5\linewidth}
    \begin{prooftree}
      \AxiomC{$P, \texttt{Un}\ t \vdash \sigma\ \texttt{Un}$}\RightLabel{[\texttt{Un}-$\sigma$]}
      \UnaryInfC{$P \vdash \forall t.\sigma\ \texttt{un}$}
    \end{prooftree}
  \end{minipage}
  \begin{minipage}{0.5\linewidth}
    \begin{prooftree}
      \AxiomC{$\bigwedge_{x:\sigma \in \Gamma}P \vdash \rho\ \texttt{un}$}\RightLabel{[\texttt{Un}-$\Gamma$]}
      \UnaryInfC{$P \vdash \Gamma\ \texttt{un}$}
    \end{prooftree}
  \end{minipage}

  \begin{minipage}{0.5\linewidth}
    \begin{prooftree}
      \AxiomC{$P => \tau \geq \phi$}\RightLabel{[$\geq$-$\tau$]}
      \UnaryInfC{$P \vdash \tau \geq \phi$}
    \end{prooftree}
  \end{minipage}
  \begin{minipage}{0.5\linewidth}
    \begin{prooftree}
      \AxiomC{$P,\pi \vdash \rho \geq \phi$}\RightLabel{[$\geq$-$\rho$]}
      \UnaryInfC{$P \vdash (\pi => \rho) \geq \phi$}
    \end{prooftree}
  \end{minipage}
  \begin{minipage}{0.5\linewidth}
    \begin{prooftree}
      \AxiomC{$P, \texttt{Un}\ t \vdash \sigma \geq \phi$}\RightLabel{[$\geq$-$\sigma$]}
      \UnaryInfC{$P \vdash (\forall t.\sigma) \geq \phi$}
    \end{prooftree}
  \end{minipage}
  \begin{minipage}{0.5\linewidth}
    \begin{prooftree}
      \AxiomC{$\bigwedge_{x:\sigma \in \Gamma}P \vdash \rho \geq \phi$}\RightLabel{[$\geq$-$\Gamma$]}
      \UnaryInfC{$P \vdash \Gamma \geq \phi$}
    \end{prooftree}
  \end{minipage}
\end{framed}
  \caption{Typing Rules for Base cases}
  \label{fig:bi-base-typing-rules}
\end{figure}

\section{Syntax Directed Typing rules}
The typing rules explained in the previous section are not syntax directed and will not be fit
to develop an algorithm to develop a type inference algorithm. In this section we
will define syntax directed typing rules that will simplify our system.

\TODO{We will also prove equivance between the typing rules and syntax directed typing rules of our system}

\section{Modified Algorithm $\M$}
The type inference algorithm is given in \cref{fig:algorithm-m}
$\Sigma$ keeps track of all the used variables
$S$ is the substitution that is generated by the algorithm to infer the type of the
term using the context $\Gamma$. $X$ are the predicates for the qualified types.

$\mathcal{U}$ is a modified Robinson's unification algorithm[\cite{robinson_machine-oriented_1965}].
It generates kind preserving substitutions to unify types. The algorithm fails, if no such substitution exists.
We write $C \overset{U}{\sim}_{\kappa} C'$ for assertion that $U$ is the unifier
of the constructor types $C$ and $C'$.

\begin{flalign*}
  \tau \sim \tau' => \exists S. S \tau = \tau'
\end{flalign*}
\begin{figure}[h]
  \begin{framed}
    \begin{minipage}{0.5\linewidth}
      \begin{flalign*}
        Leq(\phi, \Gamma)  = \bigcup_{(x:\tau) \in \Gamma} \{P \mid P \vdash \phi \leq \tau \}
      \end{flalign*}
    \end{minipage}
    \begin{minipage}{0.5\linewidth}
      \begin{flalign*}
        \texttt{Un}(\Gamma)  = \bigcup\{P \mid (y:\sigma) \in \Gamma, P \vdash \sigma\ \texttt{un} \}
      \end{flalign*}
    \end{minipage}
    \begin{minipage}{0.5\linewidth}
      \begin{flalign*}
        \texttt{Weaken}(x, \sigma, \Sigma)  = \begin{cases}
          P\ \ \ \ &\text{if}\ x \notin \Sigma, P \vdash \sigma\ \texttt{un}\\
          \emptyset\ \ \ &otherwise
        \end{cases}
      \end{flalign*}
    \end{minipage}
    \begin{minipage}{0.5\linewidth}
      \begin{flalign*}
        \texttt{GenI}(\Gamma, P &=> \tau)  = \forall (ftv(S P, \tau)).S P => \tau \nonumber\\
        \text{where}\ &S\ \text{improves}\ \texttt{ftv}(P) \backslash \texttt{ftv}(\Gamma, \tau)\ \text{in}\ P
      \end{flalign*}
    \end{minipage}
    \begin{minipage}{1\linewidth}
      \begin{flalign*}
        \mathcal{C}(P, \Gamma, \Sigma)  = \{ x_i, x_j \mid \Gamma(x_i:\tau_1;x_j:\tau_2) \}
      \end{flalign*}
    \end{minipage}
  \end{framed}
  % Un
  % Weaken
  % GenI
  % \mathcal{C}(\Gamma, \Sigma)
  \caption{Auxilary definitions}
  \label{fig:aux-defs}
\end{figure}

\begin{figure}[h]
  \begin{framed}
    \begin{minipage}[ht]{1\linewidth}
      \centering
      \fbox{
        $\M(S, X;\Gamma \vdash M : \tau) = P, S', \Sigma$
      }
    \end{minipage}

    % x var
    \begin{minipage}{1\linewidth}
      \begin{flalign*}
        \M(S, X;\Gamma \vdash x : \tau) &= ([\vec{u} / \vec{t}]P), S' \circ S, \{x\} \nonumber \\
        \text{where}\ (x : \forall \vec{t}. P => \nu) &\in S \Gamma \nonumber\\
        S' &= \Unf([\vec{u} / \vec{t}]\nu, S \tau)
      \end{flalign*}
    \end{minipage}

    % \*x. M: t
    \begin{minipage}{1\linewidth}
      \begin{flalign*}
        \M(S, X;\Gamma \vdash \lambda ^{*} x. M : \tau) &= (P \cup Q), S', \Sigma \backslash x \nonumber \\
        \text{where}\ P; S'; \Sigma &= \M(\Unf(\tau, u_1 u_2 u_3) \circ S, X; \Gamma, x:u_2 \vdash M: u_3) \nonumber\\
        Q &= \{\SeFun{u_1}\} \cup \text{Leq}(u_1, \Gamma\mid_{\Sigma}) \cup \text{Weaken}(x, u_2, \Sigma)
      \end{flalign*}
    \end{minipage}

    % \&x. M: t
    \begin{minipage}{1\linewidth}
      \begin{flalign*}
        \M(S, X;\Gamma \vdash \lambda ^{\alpha} x. M : \tau) &= (P \cup Q), S', \Sigma \backslash x \nonumber \\
        \text{where}\ P; S'; \Sigma &= \M(\Unf(\tau, u_1 u_2 u_3) \circ S, X; \Gamma, x:u_2 \vdash M: u_3) \nonumber\\
        Q &= \{\ShFun{u_1}\} \cup \text{Leq}(u_1, \Gamma|_{\Sigma}) \cup \text{Weaken}(x, u_2, \Sigma)
      \end{flalign*}
    \end{minipage}

    % M N: t
    \begin{minipage}{1\linewidth}
      \begin{flalign*}
        \M(S, X;\Gamma \vdash M N : \tau) &= Q, R', \Sigma \cup \Sigma' \nonumber \\
        \text{where}\ P, R, \Sigma &= \M(S, X; \Gamma M:  u_1 u_2 \tau) \nonumber \\
        P', R', \Sigma' &= \M(R, X; \Gamma N: u_2) \nonumber\\
        \text{if}\ \mathcal{C}(\Gamma, \Sigma) &= \mathcal{C}(\Gamma, \Sigma') \nonumber\\
        \text{then}\ Q &= P \cup P' \cup \{\ShFun{u_1}\} \nonumber\\
        \text{else}\ Q &= P \cup P' \cup \{\SeFun{u_1}\} \cup \text{Un}(\Gamma|_{\Sigma \cap \Sigma'})
      \end{flalign*}
    \end{minipage}

    % let x = M in N: t
    \begin{minipage}{1\linewidth}
      \begin{flalign*}
        \M(S, X;\Gamma \vdash \Let{x}{M}{N} : \tau) &= (P \cup Q), R', \Sigma \cup \{\Sigma' \backslash x \} \nonumber\\
        \text{where}\ P, R, \Sigma &= \M(S, X;\Gamma \vdash M:u_1) \nonumber \\
        \sigma &= \text{GenI}(R\Gamma; R(P => u_1)) \nonumber\\
        P', R', \Sigma' &= \M(R, X;\Gamma, x:\sigma \vdash N : \tau) \nonumber\\
        Q &= \text{Un}(\Gamma|_{\Sigma \cap \Sigma'}) \cup \text{Weaken}(x, \sigma, \Sigma')
      \end{flalign*}
    \end{minipage}

    \begin{minipage}{1\linewidth}
      \begin{flalign*}
        \M(S, X;\Gamma \vdash \Case{M}{\{p_i \mapsto N_i\}_i}) &= (P_M \cup \{P_i\}), Q, \Sigma_M \cup \{\Sigma_i\} \nonumber\\
        \text{where}\ P_M, R_M, \Sigma_M &= \M(S, X;\Gamma \vdash M:u_1 \oplus u_2 \oplus \cdots \oplus u_n) \nonumber \\
        P_1, R_1, \Sigma_1 &= \M(R_M, X;\Gamma \vdash p_1:u_1 \vdash N_1: v) \nonumber \\
        P_i, R_i, \Sigma_i &= \M(R_{i-1}, X;\Gamma \vdash p_i:u_i \vdash N_i: v) \nonumber \\
        \Sigma' &= \bigcup_{i,j \leq n} (\Sigma_i \cap \Sigma_{j}) \nonumber\\
        Q &= \text{Un}(\Gamma|_{\Sigma'}) \cup \{\text{Weaken}(p_i, u_i, \Sigma_i)\}
      \end{flalign*}
    \end{minipage}
  \end{framed}
  \caption{Type Inference Algorithm $\mathcal{M}$}
  \label{fig:algorithm-m}
\end{figure}

\begin{figure}[h]
  \begin{framed}
    \begin{minipage}{1\linewidth}
      \begin{flalign*}
        \M(S, X;\Gamma \vdash C\ x) &= (P_M \cup \{P_i\}), Q, \Sigma_M \cup \{\Sigma_i\} \nonumber\\
        \text{where}\ C &= \forall \vec{t_1}. (\forall \vec{t_2}. \exists \vec{t_3}. Q => v') \sepimp v \nonumber\\
        P_M, R_M, \Sigma_M &= \M(S, X;\Gamma \vdash M:u_1 \oplus u_2 \oplus \cdots \oplus u_n) \nonumber \\
        P_1, R_1, \Sigma_1 &= \M(R_M, X;\Gamma \vdash p_1:u_1 \vdash N_1: v) \nonumber \\
        P_i, R_i, \Sigma_i &= \M(R_{i-1}, X;\Gamma \vdash p_i:u_i \vdash N_i: v) \nonumber \\
        \Sigma' &= \bigcup_{i,j \leq n} (\Sigma_i \cap \Sigma_{j}) \nonumber\\
        Q &= \text{Un}(\Gamma|_{\Sigma'}) \cup \{\text{Weaken}(p_i, u_i, \Sigma_i)\}
      \end{flalign*}
    \end{minipage}
  \end{framed}
  \caption{Type Inference Algorithm $\mathcal{M}$ (continued)}
  \label{fig:algorithm-m-cont}
\end{figure}

Here is how the naive recursive algorithm works

For Lambda
\begin{enumerate}
\item Get the bound variable
\item Assign a new type for bound variable
\item assign a new type variable for the body
\item Evaluate type of the body and assign it to the body type variable
\item return the type as (type of var) -> (type of body)
\end{enumerate}

Body can be made up of another lambda. In this case the same 5 steps will be triggered
or it is an application of 2 or more variables
In case of application 2 new type variable:
\begin{enumerate}
\item introduce type variables for left expression and right expression
  left expression a type $A \rightarrow B$ right expression gets the type $A$
\item recursively type check both of them
\item return the type of the complete application as $B$ if the type checking in previous step is successfully
\end{enumerate}


In Quill there are 2 kinds of lambdas:
\begin{enumerate}
\item Sharing Lambda $\lambda^{\alpha}$
\item Separating Lambda $\lambda^{*}$
\end{enumerate}
This helps us specify if the 2 variables are separated or have sharing
of resources between them.
The church encoding of a sharing pair will be represented as\\
$shPair = \lambda^{*} x \rightarrow \lambda^{\alpha} y \rightarrow \lambda^{*} sh \rightarrow sh\ x\ y$\\
This means that x and y may share resources
A separating pair is represented as\\
$sePair = \lambda^{*} x \rightarrow \lambda^{*}y \rightarrow \lambda^{*}se \rightarrow se\ x\ y$\\
This means that x and y do not share resources.
% By default we may assume that resources are always separating unless explicitly specified
% that they are sharing.

\section{Typing environment}

The typing environment in standard Milner-Damas algorithm
is a pair of identifier and its type.
We need to modify the typing environment so that it describes sharing.
% There can be many different ways of doing it.
In the current implementation we have
extended the typing environment to hold 2 more entities along with the
type of the identifier, a list of list of identifiers --- that describes the sharing of variables,
a scope tag---that identifies if the variable is global in the complete module or local to the definition.
Global variables can be used anywhere in the file or other code file if it is imported
All function names will be defaulted to global scope.
Local variables can be used only after they have been bound in the typing environment.
The new typing environment can be realized as:
\begin{minted}{haskell}
  type Env = Map Id (Type, [[Id]], Scope)
\end{minted}

% how is the list of list of ids help in identifying sharing

% how do you define a closure

% What do you mean by having a break in the closure

% The used field in the type-checker state

\section{Modification to Typechecking Algorithm}

To incorporate the sharing, we have to modify the typechecking algorithm.
The main reason to do that is to avoid adding $Un$ predicates to
shared variables that are not used. Take an example of $fst$ function
that returns the first argument of a shared pair
\begin{verbatim}
fst = \x -> \&y -> x
\end{verbatim}
In this case y should not be assigned an $Un$ predicate because
it is shared with x and x is indeed used.

The type checking algorithm has to be tweaked so that we keep track
of what all variables are shared and which ones are separate.

When we encounter an Alpha Lambda
\begin{enumerate}
\item Get the bound variable
\item Assign a new type for bound variable
\item Assign a new type variable for the body
\item add the variable in a sharing context
\item Evaluate type of the body and assign it to the body type variable
\item return the type as (type of var) -> (type of body)
\end{enumerate}


% The main problem is that we do not know when a sharing variable should be kept in scope
% or removed. In some cases we may need to keep it around for introducing the (>:=) predicates
% Some cases are 2 types:
% 1) The variable is used
% 2) The variable is not used

The type checking calls occur from left to right recursively.
While going down the recursion we keep on adding variables to the environment
We stop recursing at application. where we compute whether we have complete sharing to assign ShFun or SeFun.

While folding out of of the recursion we do 2 things:
1) keep track of the used variables (including implicitly used variables due to sharing)
2) generate new goals
a) depending on whether the the introduced variable was used
or its sharing member was used. Assign Un predicates and weaken if it is not used at all
b) introduce lesser-restricted predicates
3) Generate other new assumptions

We cannot determine when to get rid of the complete bunch.
\begin{verbatim}
\z -> \*x -> \&y -> y
\end{verbatim}
hence, when we are folding out of the final recursive but 1 call ie. for Lambda bound variable x

\section{Sharing}
What do we exactly mean by sharing?
There are 2 interpretations of sharing that i can think of
1) We have a resource $\mathcal{R}$, and 2 pointers $\alpha$, $\beta$. we say $alpha$, $\beta$ share if both of them point to the same resource $\mathcal{R}$
$\mathcal{R}$ is never exposed to the user space and can be manipulated only by using $\alpha$, $\beta$.
2) We have resource 


%%% Local Variables:
%%% mode: latex
%%% TeX-master: "../thesis-ku.tex"
%%% End:
 