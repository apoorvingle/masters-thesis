\chapter{Implementing Typechecking in Quill}


Here is how the naive recursive algorithm works

For Lambda
\begin{enumerate}
\item Get the bound variable
\item Assign a new type for bound variable
\item assign a new type variable for the body
\item Evaluate type of the body and assign it to the body type varialbe
\item return the type as (type of var) -> (type of body)
\end{enumerate}

Body can be made up of another lambda. In this case the same 5 steps will be triggered
or it is an application of 2 or more variables
In case of application 2 new type variable:
\begin{enumerate}
\item introduce type variables for left expression and right expression
    left expression a type $A \rightarrow B$ right expression gets the type $A$
\item recursively type check both of them
\item return the type of the complete application as $B$ if the type checking in previous step is successfull
\end{enumerate}


In Quill there are 2 kinds of lambdas:
\begin{enumerate}
\item Sharing Lambda $\&$
\item Separating Lambda $\*$
\end{enumerate}

This helps specify if the 2 variables are independent or have some sort of sharing between them.
eg. A sharing pair will be represented as
\begin{verbatim}
shPair = \x ->\&y -> \sh -> sh x y
\end{verbatim}
This means that x and y may share resources

The other way pair is a sparating pair represented as
\begin{verbatim}
sePair = \x -> \*y -> \se -> se x y
\end{verbatim}
This means that x and y do not share variables

\section{Typing environment}

The typing environment is a pair of identifier and its type

We need to modify the typing envrionment or typing judgements
also describes the sharing. There can be many different
ways of doing it.
In the current implmenetation we have
extended the typing environment to hold 2 more things
a list of list of identifiers, that describes the sharing of variables
a scope tag that identifies if the variable is global in the complete module or local to the definition.

Global variables can be used anywhere in the compelete file.
All function names will be defaulted to global scope.
Local variables can be used only after they have been bound in the typing environment.

% how is the list of list of ids help in identifying sharing

% how do you define a closure

% What do you mean by having a break in the closure

% The used field in the typechecker state

\section{Modification to Typechecking Algorithm}

To incorporate the sharing, we have to modify the typechecking algorithm.
The main reason to do that is to avoid adding $Un$ predicates to
shared variables that are not used. Take an example of $fst$ function
that returns the first argument of a shared pair
\begin{verbatim}
fst = \x -> \&y -> x
\end{verbatim}
In this case y should not be assigend an $Un$ predicate because
it is shared with x and x is indeed used.

The type checking algorithm has to be tweeked so that we keep track
of what all variables are shared and which ones are separate.

When we encounter an Alpha Lambda
\begin{enumerate}
\item Get the bound variable
\item Assign a new type for bound variable
\item Assign a new type variable for the body
\item add the variable in a sharing context
\item Evaluate type of the body and assign it to the body type variable
\item return the type as (type of var) -> (type of body)
\end{enumerate}


% The main problem is that we do not know when a sharing variable should be kept in scope
% or removed. In some cases we may need to keep it around for introducing the (>:=) predicates
% Some cases are 2 types:
% 1) The variable is used
% 2) The variable is not used

The type checking calls occur from left to right recursively.
While going down the recursion we keep on adding variables to the envionment
We stop recursing at application. where we compute whether we have complete sharing to assign ShFun or SeFun.

While folding out of of the recursion we do 2 things:
1) keep track of the used variables (including implicitly used variables due to sharing)
2) generate new goals
   a) depending on whether the the introduced variable was used
      or its sharing member was used assign Un predicates and weaken if it is not used at all
   b) introduce lesser-restricted predicates
3) Generate new assumptions

We cannot determine when to get rid of the complete bunch.
\begin{verbatim}
\z -> \*x -> \&y -> y
\end{verbatim}
hence, when we are folding out of the final recursive but 1 call ie. for Lambda bound variable x


\section{Typing rules}

\begin{minipage}{.5\textwidth}
  \begin{prooftree}
    \AxiomC{} \RightLabel{$[Id]$}
    \UnaryInfC{$\langle !A \rangle \vdash !A$}
    \AxiomC{$[\Gamma], [A] \vdash B$} \RightLabel{$[!\text{--}E]$}
    \BinaryInfC{$[\Gamma], \langle !A \rangle \vdash B$}
    \RightLabel{$[\rightspoon \text{--} I]$}
    \UnaryInfC{$[\Gamma] \vdash !A \rightspoon B$}
  \end{prooftree}
\end{minipage}
\begin{minipage}{.5\textwidth}
  \begin{prooftree}
    \AxiomC{$[\Gamma] \vdash !A \rightspoon B $}
    \AxiomC{$[\Delta] \vdash A$} \RightLabel{$[!\text{--}I]$}
    \UnaryInfC{$[\Delta] \vdash !A$}
    \RightLabel{$\rightspoon\text{--}E$}
    \BinaryInfC{$[\Gamma], [\Delta] \vdash B$}
  \end{prooftree}
\end{minipage}




%%% Local Variables:
%%% mode: latex
%%% TeX-master: "Typechecking"
%%% End:
