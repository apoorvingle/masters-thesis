\chapter{Type Extensions}

We add 2 new type classes to our language
\begin{itemize}
\item Unrestricted values will have a type Un
  that represents the fact that the resource can be
  duplicated/discarded or they cannot.
  This expressivity helps us understand how the functions
  use the resources. If the function uses a file handle as a resource
  in a restricted fashion. It would not call duplicate or drop on it.
  On the other hand if the file handle is being opened or closed
  by a function, it will have to be in an unrestricted fashion.
\item All the functions are of class $->$ can potentially have any of the 4 choices
\begin{verbatim}
              ShFun   SeFun

Un             -!&>    -!*>

Substructural   -&>     -*>
\end{verbatim}
  $-\&>$ implies that the function application is of the sharing type.
  i.e. the resources used in the function have a may have a sharing closure.

  $-*>$ implies that the function applciation is of a separating type.
  the resources used do not have anything in common

  $-\!\&>$ implies that the function itself can be discarded
  or used multiple number of times. i.e. its closure
  can be duplicated or discarded.

  $-!*>$ implies that the function itself can be discarded.
  This is useful in case of higher order functions. in the monad instance
  of Maybe, in bind operation

\begin{lstlisting}
instance Monad (-!*>) Maybe
   return a = Just a
   (>>=) a f = case a of
                 Nothing -> Nothing
                 Just v -> f v
\end{lstlisting}

  In the first branch of case expression, f is never used and discarded. Hence
  the type of the function should be $-!*>$
  This may not always be the case. As with a non-empty list
  the operation f is performed atleast once.

\begin{lstlisting}
instance Monad (-!*>) NEList
   return a   = Last a
   (>>=) m f  = case a of
                  Last a    -> f a
                  Cons a as -> concat (f a) (as >>= f)
\end{lstlisting}

\end{itemize}


%%% Local Variables:
%%% mode: latex
%%% TeX-master: t
%%% End:
