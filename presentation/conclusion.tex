\section{Conclusion and Future Work}\label{sec:conclusion}

\begin{frame}[fragile, c]
  \frametitle{Motivation Revisited}
  \begin{center}
    {\LARGE
      What about filehandles, exceptions and memory leaks and runtime crashes?
    }
\end{center}

\end{frame}

\begin{frame}[fragile, c]
  \frametitle{Filehandles Revisited}
  \begin{center}
  File Handling API in \qub{}
  \begin{haskell}
     openFile  :: FilePath   $\sepimp$ IO FileHandle

     closeFile :: FileHandle $\sepimp$ IO ()

     readLine  :: FileHandle $\sepimp$ IO (String, FileHandle)

     writeFile :: String     $\sepimp$ FileHandle
                             $\sepimp$ IO FileHandle




     ($\gg\!=$) :: IO a $\sepimp$ (a $\sepimp$ IO b) $\sepimp$ IO b
   \end{haskell}
\end{center}
\end{frame}

\begin{frame}[fragile, c]
  \frametitle{Filehandles Revisited}
  \begin{center}
  {\LARGE \color{white}Fails Typecheck!}

  \begin{haskell}
               do f  <- openFile "sample.txt"
                  (s, f)  <- readLine f
                  () <- closeFile f
                  () <- closeFile f
            \end{haskell}
{\LARGE      $\downlsquigarrow \qquad \downlsquigarrow \qquad \downlsquigarrow$}
            \begin{haskell}
               ($\gg\!=$) (openFile "sample.txt") (\ f ->

               ($\gg\!=$) (readLine f) (\ (s, f) ->

               ($\gg\!=$) (closeFile f) (\ _ -> closeFile f)
            \end{haskell}
          \end{center}

\end{frame}

\begin{frame}[fragile, c]
  \begin{center}
  {\LARGE \color{white}Fails Typecheck!}

  \frametitle{Filehandles Revisited}
\begin{haskell}
               do f  <- openFile "sample.txt"
                  (s, f)  <- readLine f
                  () <- closeFile f
                  () <- closeFile f
               \end{haskell}
{\LARGE      $\downlsquigarrow \qquad \downlsquigarrow \qquad \downlsquigarrow$}
               \begin{haskell}
               ($\gg\!=$) (openFile "sample.txt") (\ f ->

               ($\gg\!=$) (readLine f) (\ (s, f) ->

               @(${\color{red}\gg\!=}$) (closeFile f) (\ _ -> closeFile f)@
               \end{haskell}
          \end{center}

\end{frame}

\begin{frame}[fragile, c]
  \begin{center}
  \frametitle{Filehandles Revisited}
  {\LARGE \color{red}Fails Typecheck!}

\begin{haskell}
                do f  <- openFile "sample.txt"
                   (s, f)  <- readLine f
                   () <- closeFile f
                   () <- closeFile f
               \end{haskell}
{\LARGE      $\downlsquigarrow \qquad \downlsquigarrow \qquad \downlsquigarrow$}
               \begin{haskell}
               {- (${\color{dkgreen}\gg\!=}$) :: IO a ${\color{dkgreen}\sepimp}$ (a ${\color{dkgreen}\sepimp}$ IO b) ${\color{dkgreen}\sepimp}$ IO b -}
               ($\gg\!=$) (openFile "sample.txt") (\ f ->
               {- (${\color{dkgreen}\gg\!=}$) :: IO a ${\color{dkgreen}\sepimp}$ (a ${\color{dkgreen}\sepimp}$ IO b) ${\color{dkgreen}\sepimp}$ IO b -}
               ($\gg\!=$) (readLine f) (\ (s, f) ->
               {- (${\color{dkgreen}\gg\!=}$) :: IO a ${\color{dkgreen}\sepimp}$ (a ${\color{dkgreen}\sepimp}$ IO b) ${\color{dkgreen}\shimp}$ IO b -}
               @(${\color{red}\gg\!=}$) (closeFile f) (\ _ -> closeFile f)@
\end{haskell}
\end{center}
\end{frame}

\begin{frame}[fragile, c]
  \frametitle{Exceptions Revisited}
  \begin{center}
    \begin{haskell}
          openFile :: FilePath $\sepimp$ IO FileHandle
          closeFile :: FileHandle $\sepimp$ IO ()
          readFile :: FileHandle $\sepimp$ IOF (String, FileHandle)
          writeFile :: String $\sepimp$ FileHandle $\sepimp$ IOF FileHandle

          throw :: Exception $\sepimp$ IOF a
          catch :: IOF a $\sepimp$ (Exception $\sepimp$ IO a) ->> IO a
          \end{haskell}
\begin{itemize}
\item May not fail \texttt{IO a}
\item May fail \texttt{IOF a}
\end{itemize}
\end{center}
\end{frame}

\begin{frame}[fragile, c]
  \frametitle{Exceptions Revisited}
  \begin{center}
    \begin{haskell}
          readFromFile :: FilePath $\sepimp$ IO (Either String String)
          readFromFile fpath =
          (do @fh@  <- openFile fpath
             (s, @fh@)  <- readLine @fh@
             let l = caps s
             () <- closeFile @fh@
             return \dollar Right l) `catch`
                     (\e -> do closeFile @fh@
                               return \dollar Left "Could not read file")
                        \end{haskell}
                      \end{center}
     \begin{itemize}
     \item Filehandle \texttt{fh} is shared between the \texttt{catch} arguments
       \begin{haskell}
         catch :: IOF a $\sepimp$ (Exception $\sepimp$ IO a) ->> IO a
       \end{haskell}
     \item Avoids memory leak
\end{itemize}
\end{frame}


\begin{frame}
  \frametitle{Contributions revisited}

  \begin{itemize}

  \item {\color{red}Language design}
    \begin{itemize}
    \item {\color{red}Resources are ``first class citizens''}
    \item {\color{red}Resources(variables) can be in sharing or separate}
    \end{itemize}

  \item {\color{red}\qub{} is logic of \BI on steroids}
    \begin{itemize}
    \item {\color{red}Typing Environments as graphs}
    \end{itemize}
  \item {\color{red}Working examples}
  \item Formalizing and proving important properties of \qub{}
    \begin{itemize}
    \item Type system
    \item Syntax directed type system (sound and complete)
    \item Type inference algorithm $\M$ (sound)
    \end{itemize}

  \end{itemize}

\end{frame}

\begin{frame}[fragile, c]
  \frametitle{Future Work}
p  \begin{itemize}
  \item Type inference algorithm $\M{}$ is incomplete

    Terms can have two types
    \begin{itemize}
    \item $\{\texttt{Un}\ A\} \mid \emptyset \vdash \lambda^{\sepimp}f. \lambda^{\sepimp}x. f x x: (A \sepimp A \sepimp B) \sepimp A \sepimp B$
    \item $\emptyset \mid \emptyset \vdash \lambda^{\sepimp}f. \lambda^{\sepimp}x. f x x: (A \sepimp A \shimp B) \sepimp A \shimp B$
    \end{itemize}
  \item Current semantics: call-by-value assumed

    Formalize resource correctness
  \end{itemize}
\end{frame}



\begin{frame}[fragile, c]
  \begin{center}
    \begin{tabular}[h]{c}
      \Huge  Thank You!
    \end{tabular}
  \end{center}
\end{frame}

\begin{frame}[fragile, c]
  \begin{center}
    \begin{tabular}[h]{c}
      \Huge Q \& A
    \end{tabular}
  \end{center}
\end{frame}