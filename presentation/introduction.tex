\section{Introduction and Motivation}\label{sec:introduction}
% (5 mins)

\begin{frame}[c]
  \frametitle{Introduction and Motivation}
  {\large
    \begin{center}
      Hard problems in programming

      {\color{red}Naming variables}
    \end{center}
    }
\end{frame}

\begin{frame}
  \frametitle{Introduction and Motivation}
  \begin{large}
  \begin{center}
    Hard problems in programming

    {\color{red}Resource management}
    \\{\normalsize  in evolving production code}

    {\normalsize Resources: Files, database connections, entity with a shared state}
\end{center}
\end{large}
\end{frame}

\begin{frame}[fragile, c]
  \frametitle{Resource Management: File Handling}
  \begin{center}
    \begin{itemize}
    \item Modified File Handling API in Haskell
    \end{itemize}
    \begin{tabular}[h]{c}
      \begin{haskell}
     openFile  :: FilePath   -> IO FileHandle

     closeFile :: FileHandle -> IO ()

     readLine  :: FileHandle -> IO (String, FileHandle)

     writeFile :: String     -> FileHandle
                             -> IO ((), FileHandle)


     upper     :: String     -> String
      \end{haskell}
    \end{tabular}
\end{center}
\end{frame}

\begin{frame}[fragile,c]
  \frametitle{Resource Management: File Handling}
  \begin{center}
    \begin{itemize}
    \item File Handling in Haskell
    \end{itemize}
    \begin{tabular}[h]{c}
      \begin{haskell}
        do f  <- openFile "sample.txt"
           (s, f)  <- readLine f
           let c = upper s
           ((), f) <- writeLine f c
                  .
                  .
                  .
           () <- closeFile f
      \end{haskell}
    \end{tabular}
  \end{center}
\end{frame}

\begin{frame}[fragile, c]
  \frametitle{Resource Management: File Handling}
  \begin{center}
    \begin{itemize}
    \item File Handling in Haskell Gone Wrong (Part I)
    \end{itemize}
    \begin{tabular}[h]{c}
    \begin{haskell}
      do f  <- openFile "sample.txt"
         (s, f)  <- readLine f
         let c = upper s
         ((), f) <- writeLine f c
              .
              .
              .
         () <- closeFile f
              .
              .
              .
         () <- closeFile f
         return c
    \end{haskell}
    \end{tabular}
  \end{center}
\end{frame}

\begin{frame}[fragile, c]
  \frametitle{Resource Management: File Handling}
  \begin{center}
    \begin{itemize}
    \item File Handling in Haskell Gone Wrong (Part I)
    \end{itemize}
    \begin{tabular}[h]{c}
    \begin{haskell}
      do f  <- openFile "sample.txt"
         (s, f)  <- readLine f
         let c = upper s
         ((), f) <- writeLine f c
              .
              .
              .
        @() <- closeFile f@
              .
              .
              .
        @() <- closeFile f@
         return c
    \end{haskell}
    \end{tabular}
    \begin{itemize}
    \item File is closed twice: Run time crash
    \end{itemize}
  \end{center}
\end{frame}

\begin{frame}[fragile, c]
  \frametitle{Resource Management: File Handling}
  \begin{center}

  \begin{itemize}
  \item File Handling in Haskell Gone Wrong (Part II)
  \end{itemize}
  \begin{tabular}[h]{c}
    \begin{haskell}
    do f  <- openFile "sample.txt"
       (s, f)  <- readLine f
       let c = upper s
       ((), f) <- writeLine f c
           .
           .
           .
       return c
     \end{haskell}
  \end{tabular}

\end{center}
\end{frame}

\begin{frame}[fragile, c]
  \frametitle{Resource Management: File Handling}
  \begin{center}

  \begin{itemize}
  \item File Handling in Haskell Gone Wrong (Part II)
  \end{itemize}
  \begin{tabular}[h]{c}
    \begin{haskell}
    do f  <- openFile "sample.txt"
       (s, f)  <- readLine f
       let c = upper s
       ((), f) <- writeLine f c
           .
           .
           .
       return c /*File not closed!!*/
     \end{haskell}
  \end{tabular}
  \begin{itemize}
  \item File not closed: Memory leak
  \end{itemize}
  \end{center}
\end{frame}

\begin{frame}[fragile, c]
  \frametitle{Resource Management: Exception Handling}
  \begin{center}
  \begin{itemize}
  \item \texttt{MonadError}\citep{liang_monad_1995} in Haskell
  \end{itemize}
  \begin{tabular}[h]{c}
    \begin{haskell}
class Monad m => MonadError e m | m -> e where
    throwError :: e -> m a
    catchError :: m a -> (e -> m a) -> m a
     \end{haskell}
  \end{tabular}
  \begin{itemize}
  \item \texttt{throwError} starts exception processing
  \item \texttt{catchError} exception handler
  \end{itemize}
  \end{center}
\end{frame}

\begin{frame}[fragile, c]
  \frametitle{Resource Management: Exception Handling}
  \begin{center}

  \begin{itemize}
  \item Using \texttt{MonadError} in Haskell
  \end{itemize}
  \begin{tabular}[h]{c}
    \begin{haskell}
      do f <- openFile "sample.txt"
         ((s, f)  <- readLine f
         let c = upper s
         () <- closeFile f
         return \dollar Right c)
             `catchError` (\_ ->
                    return \dollar Left "Error in reading file")
     \end{haskell}
  \end{tabular}
  \begin{itemize}
  \item Exception may cause memory leak
  \end{itemize}
  \end{center}
\end{frame}

\begin{frame}[c]
  \frametitle{Introduction and Motivation}
  \begin{center}
    \uncover<+-> {\LARGE
      \begin{aquote}{R. Milner}
        Well typed programs do not go wrong.
      \end{aquote}
    }
    \vspace{2cm}
    % \begin{description}
    % \item<2-> \qquad \qquad {\color{red}Can we do better?}
    % \end{description}
    \uncover<+-> {\LARGE
      \begin{aquote}{Coldplay}
        \sout{Lights}{\color{red} Types} will guide you home
      \end{aquote}
    }
  \end{center}
\end{frame}

\begin{frame}
  \frametitle{Contributions}
  \begin{itemize}
  \item
    Design and implement \qub{} type system
    \begin{framed}
      \begin{itemize}
      \item Resources as first class citizens
      \item Program objects are restricted or unrestricted
      \item Functions that share resources with their arguments or are separate.
      \end{itemize}
    \end{framed}

  \item Formalizing and proving important properties of \qub{}
  \item \qub{} is logic of \BI with steroids
    \begin{itemize}
    \item Environments as graphs
    \end{itemize}
  \item Working examples
  \end{itemize}
\end{frame}

%%% Local Variables:
%%% mode: latex
%%% TeX-master: "defense-slides"
%%% End:
