\section{Introduction and Motivation}\label{sec:introduction}
% (5 mins)

\begin{frame}[c]
  \frametitle{Introduction and Motivation}
  {\large
    \begin{center}
      Hard problems in programming

      {\LARGE \color{red}Naming variables}
    \end{center}
    }
\end{frame}

\begin{frame}
  \frametitle{Introduction and Motivation}
  \begin{center}
    {\large Hard problems in programming}

    {\LARGE \color{red}Resource management}
    \\{\normalsize  in evolving production code}

    {\normalsize Resources: Files, database connections, shared mutable state}
\end{center}
\end{frame}

\begin{frame}[fragile, c]
  \frametitle{Resource Management: File Handling}
  \begin{center}
    \begin{itemize}
    \item Modified File Handling API in Haskell
    \end{itemize}
    \begin{tabular}[h]{c}
      \begin{haskell}

     openFile   :: FilePath   -> IO FileHandle

     closeFile  :: FileHandle -> IO ()

     readLine   :: FileHandle -> IO (String, FileHandle)

     writeFile  :: String     -> FileHandle
                             -> IO FileHandle


     upper      :: String     -> String

     ($\gg\!=$) :: IO FileHandle -> (FileHandle -> IO b) -> IO b
      \end{haskell}
    \end{tabular}
\end{center}
\end{frame}

\begin{frame}[fragile,c]
  \frametitle{Resource Management: File Handling}
  \begin{center}
    \begin{itemize}
    \item File Handling in Haskell
    \end{itemize}
    \begin{tabular}[h]{c}
      \begin{haskell}
        do f  <- openFile "sample.txt"
           (s, f)  <- readLine f
           let c = upper s
           f <- writeLine f c
                  .
                  .
                  .
           () <- closeFile f
      \end{haskell}
    \end{tabular}
  \end{center}
\end{frame}

\begin{frame}[fragile, c]
  \frametitle{Resource Management: File Handling}
  \begin{center}
    \begin{itemize}
    \item File Handling in Haskell Gone Wrong (Part I)
    \end{itemize}
    \begin{tabular}[h]{c}
    \begin{haskell}
      do f  <- openFile "sample.txt"
         (s, f)  <- readLine f
         let c = upper s
         f <- writeLine f c
              .
              .
              .
         () <- closeFile f
              .
              .
              .
         () <- closeFile f
         return c
    \end{haskell}
    \end{tabular}
  \end{center}
\end{frame}

\begin{frame}[fragile, c]
  \frametitle{Resource Management: File Handling}
  \begin{center}
    \begin{itemize}
    \item File Handling in Haskell Gone Wrong (Part I)
    \end{itemize}
    \begin{tabular}[h]{c}
    \begin{haskell}
      do f  <- openFile "sample.txt"
         (s, f)  <- readLine f
         let c = upper s
         f <- writeLine f c
              .
              .
              .
        @() <- closeFile f@
              .
              .
              .
        @() <- closeFile f@
         return c
    \end{haskell}
    \end{tabular}
    \begin{itemize}
    \item File is closed twice: Run time crash
    \end{itemize}
  \end{center}
\end{frame}

\begin{frame}[fragile, c]
  \frametitle{Resource Management: File Handling}
  \begin{center}

  \begin{itemize}
  \item File Handling in Haskell Gone Wrong (Part II)
  \end{itemize}
  \begin{tabular}[h]{c}
    \begin{haskell}
    do f  <- openFile "sample.txt"
       (s, f)  <- readLine f
       let c = upper s
       f <- writeLine f c
           .
           .
           .
       return c
     \end{haskell}
  \end{tabular}

\end{center}
\end{frame}

\begin{frame}[fragile, c]
  \frametitle{Resource Management: File Handling}
  \begin{center}

  \begin{itemize}
  \item File Handling in Haskell Gone Wrong (Part II)
  \end{itemize}
  \begin{tabular}[h]{c}
    \begin{haskell}
    do f  <- openFile "sample.txt"
       (s, f)  <- readLine f
       let c = upper s
       f <- writeLine f c
           .
           .
           .
       return c {- File not closed!! -}
     \end{haskell}
  \end{tabular}
  \begin{itemize}
  \item File not closed: Memory leak
  \end{itemize}
  \end{center}
\end{frame}

\begin{frame}[fragile, c]
  \frametitle{Resource Management: Exception Handling}
  \begin{center}
    \begin{itemize}
    \item \texttt{MonadError}\footnote[frame]{\footfullcite{liang_monad_1995}}  type class in Haskell
      \begin{tabular}[c]{c}
        \begin{haskell}
      class Monad m => MonadError e m | m -> e where
          throwError :: e -> m a
          catchError :: m a -> (e -> m a) -> m a
        \end{haskell}
      \end{tabular}

    \item\texttt{MonadError} instance with \texttt{IO} and \texttt{Exception}
      \begin{tabular}[c]{c}
        \begin{haskell}
          throwError :: Exception -> IO a
          catchError :: IO a -> (Exception -> IO a) -> IO a
        \end{haskell}
      \end{tabular}

    \item \texttt{throwError} start exception processing
    \item \texttt{catchError} exception handler
    \end{itemize}
  \end{center}
\end{frame}

\begin{frame}[fragile, c]
  \frametitle{Resource Management: Exception Handling}
  \begin{center}

  \begin{itemize}
  \item Using \texttt{MonadError} in Haskell
  \end{itemize}
  \begin{tabular}[h]{c}
    \begin{haskell}
  (do f <- openFile "sample.txt"
     @(s, f)  <- readLine f@         {- Exception raised here -}
      let c = upper s
      () <- closeFile f
      return \dollar Right c)
          `catchError` (\_ ->
                 return \dollar Left "Error in reading file")
     \end{haskell}
  \end{tabular}
  \begin{itemize}
  \item Exception may cause memory leak
  \end{itemize}
  \end{center}
\end{frame}

\begin{frame}[c]
  \frametitle{Introduction and Motivation}
  \begin{center}
    \uncover<+-> {\LARGE
      \begin{aquote}{R. Milner}
        Well typed programs do not go wrong.
      \end{aquote}
    }
    \vspace{2cm}
    % \begin{description}
    % \item<2-> \qquad \qquad {\color{red}Can we do better?}
    % \end{description}
    \uncover<+-> {\LARGE
      \begin{aquote}{Coldplay}
        \sout{Lights}{\color{red} Types} will guide you home$\dots$
      \end{aquote}
    }
  \end{center}
\end{frame}

\begin{frame}
  \frametitle{Contributions revisited}

  \begin{itemize}

  \item {\color{red}Language design}
    \begin{itemize}
    \item {\color{red}Resources are ``first class citizens''}
    \item {\color{red}Resources(variables) can be in sharing or separated}
    \end{itemize}

  \item {\color{red}\qub{} is logic of \BI on steroids}
    \begin{itemize}
    \item {\color{red}Typing Environments as graphs}
    \end{itemize}
  \item Formalizing and proving important properties of \qub{}
    \begin{itemize}
    \item Type system
    \item Syntax directed type system (sound and complete)
    \item Type inference algorithm $\M$ (sound)
    \end{itemize}
  \item {\color{red}Working examples}

  \end{itemize}

\end{frame}


%%% Local Variables:
%%% mode: latex
%%% TeX-master: "defense-slides"
%%% End:
