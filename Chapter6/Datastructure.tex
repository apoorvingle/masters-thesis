\chapter{\qub{} Extention and Datatypes}\label{chp:datatypes}
In this chapter we discuss how \qub{} can be extended to have a kind system which makes
the type language powerful enough to accept user defined datatypes in \cref{sec:kind-system}. We also discuss
how we can encode sums and multiplicative and additive products in \cref{sec:pairs-types}.

\section{Kind System}\label{sec:kind-system}
In the original system, it would be tedious to extend the system with new types as we would have to introduce new syntax and associated typing rules for
each of the new syntax in our core language. The kind system generalizes the concept of adding new types by abstracting them as type
constructors. This generalization alleviates the burden of modifying the core language by
treating all types to be instances of type constructors. The idea was introduced by Barendregt \citeyearpar{barendregt_1991} and
is used by Jones \citeyearpar{jones_system_1993} for qualified types. We will follow Jones' approach
to add the language of type constructors and kinds in our system.
The modified type system is shown in \cref{fig:qub-kind-extension}.
All types have kind $\star$ and the kind of the type constructors depends on its arity.
\begin{figure}[h]
  \begin{framed}
    \begin{flalign*}
      \text{Type Variables}\ \ t, u, \upsilon         &\in \text{Type Variables}  \\
      \text{Kinds}\ \ \               \kappa          &::= \kappa \mid \kappa' \rightarrow \kappa\\
      \text{Types}\ \ \             \tau^{\kappa}      &::= t^\kappa \mid T^{\kappa} \mid \tau^{\kappa' \rightarrow \kappa}\tau^{\kappa'}\\
      \text{Type Constructors}\ \ \   T^{\kappa}       &\in \mathcal{T}^{\kappa}\ 
                         \text{where}\ \{\otimes, \with, \oplus, \overset{!}{\sepimp}, \sepimp, \xrightarrow{!}, \rightarrow \} \subseteq \mathcal{T}^{\star \rightarrow \star \rightarrow \star}\\
      \text{Predicates}\ \ \          \pi             &::= \texttt{Un}\ \tau \mid \texttt{SeFun}\ \tau \mid \texttt{ShFun}\ \tau \mid \tau \geq \tau' \\
      \text{Qualified Types}\ \ \     \rho            &::= \tau \mid \pi => \rho \\
      \text{Type schemes}\ \ \        \sigma          &::= \rho \mid \forall t. \sigma
    \end{flalign*}
  \end{framed}
  \caption{Extended \qub{} Types and Kinds}
  \label{fig:qub-kind-extension}
\end{figure}

Addition of a kind system changes the treatement to the type inference algorithm in some detail.
All types and type constructors have to be annotated with their kind. $T^\kappa$ denotes type constructors
and $\tau^{\kappa' \rightarrow \kappa}\tau^{\kappa'}$ denotes application of types.
The type constructor application rule that computes kinds is given in \cref{fig:constructor-application} where
$\tau$ is of kind $\kappa' \rightarrow \kappa$ and $\tau'$ is of kind $\kappa'$. The application
of both the constructors would result in a kind $\kappa$.
$\sepimp$, $\overset{!}{\sepimp}$, $\rightarrow$ and $\overset{!}{\rightarrow}$
are now treated as type constructors with an arity of two and would have a kind $\star \rightarrow \star \rightarrow \star$, The
type constructor for List will have a kind $\star \rightarrow \star$ while
types like Int and Float will have a kind $\star$.

\begin{figure}
  \begin{framed}
    \begin{prooftree}
      \AxiomC{$\tau :: \kappa' \rightarrow \kappa$}
      \AxiomC{$\tau' :: \kappa'$}
      \BinaryInfC{$\tau \tau' :: \kappa$}
    \end{prooftree}
  \end{framed}
  \caption{Constructor Application Rule}
  \label{fig:constructor-application}
\end{figure}

The unification of types is now done via a modified version of Robinson's algorithm \citeyearpar{robinson_machine-oriented_1965}
is used in order to deduce the most general unifier for type constructors.
Formally we define $S$ to be the {\it most general unifier} for type constructors $T$ and $T'$ if:
\begin{enumerate}
  \item $S$ is a unifier for $T$ and $T'$.
  \item For every unifier $S'$ of $T$ and $T'$ we can write $T'$ in a form of
    $R S$ where $R$ some kind preserving substitution.
\end{enumerate}
We write $T \overset{S}{\sim}_{\kappa} T'$ for assertion that $S$ is the unifier
of the constructor types $T, T' \in T^{\kappa}$. The rules in \cref{fig:unify-types}
describe the unification algorithm for type constructors. [KVar] and [KVar'] contain
and additional constraint of the type variable $t$ to not be free in the type constructor's $T$
type variables to ensure the unification does not lead to infinite types. The [KApply] rule
states that type constructors of the form $T T'$ can be unified with $H H'$ only if $T$ and $H$
can be unified which asserts that they have to have the same kind $\kappa' \rightarrow \kappa$.

\begin{figure}[h]
  \begin{framed}\centering
    \begin{minipage}[l]{0.5\linewidth}
      \begin{flalign*}
        t \overset{id}{\sim}_{\kappa} t \tag{[ID-KVar]}
      \end{flalign*}
    \end{minipage}
    \begin{minipage}[r]{0.5\linewidth}
      \begin{flalign*}
        T \overset{id}{\sim}_{\kappa} T \tag{[ID-KConst]}
      \end{flalign*}
    \end{minipage}
    \begin{minipage}[l]{0.5\linewidth}
      \begin{flalign*}
        t \overset{[T/t]}{\sim}_{\kappa} T, t \notin \texttt{fvs}(T) \tag{[KVar]}
      \end{flalign*}
    \end{minipage}
    \begin{minipage}[r]{0.5\linewidth}
      \begin{flalign*}
        T \overset{[T/t]}{\sim}_{\kappa} t, t \notin \texttt{fvs}(T) \tag{[KVar']}
      \end{flalign*}
    \end{minipage}
    \begin{minipage}[c]{1.0\linewidth}
      \begin{prooftree}
        \AxiomC{$T \overset{S}{\sim}_{\kappa'\rightarrow \kappa} T'$}
        \AxiomC{$S H \overset{S'}{\sim}_{\kappa'} S H'$}\RightLabel{([KApply])}
        \BinaryInfC{$T T' \overset{SS'}{\sim}_{\kappa} H H'$}
      \end{prooftree}
    \end{minipage}
  \end{framed}
  \caption{Kind Preserving Unification of Type Constructors}
  \label{fig:unify-types}
\end{figure}

\begin{figure}[h]
  \begin{framed}
    \begin{flalign*}
      \text{Term Variables}\ \ \  x, y, z  &\in \text{Var}\\
      \text{Expressions}\ \ \     M, N     &::= x \mid \lambda^{*}x. M \mid \lambda^{\alpha}x. M \mid M N \mid \Let{x}{M}{N}\\
      &\mid \Pair{M,N} \mid \Let{\Pair{x,y}}{M}{N} \mid \Pair{M;N} \mid \Fst{M} \mid \Snd{M}\\
      &\mid \Case{M}{\{\texttt{inl}\ x \mapsto N ; \texttt{inr}\ y \mapsto N'\}}\mid \Inl{x} \mid \Inr{y}\\
    \end{flalign*}
  \end{framed}
  \caption{Extended \qub{} Language Syntax}
  \label{fig:ext-qub-terms}
\end{figure}

\TODO{add more details of how this helps and proofs remain the same}

\section{Pairs and Sums in  \qub{}}\label{sec:pairs-types}
Introduction of two kinds of arrows in our type system leads to different flavors of pairs. This distinction cannot be made
in intuitionistic logic as the structural rules allow re-use of propositions. But due to restrictions in weakening and
contraction we obtain two kinds of pairs, additive and multiplicative. In this section we illustrate how the
extented \qub{} can be used to introduce new types. We introduce syntax and type constructors for multiplicative pairs in \cref{subsec:mul-pairs}
and the same for additive pairs in \cref{subsec:add-pairs}. We then introduce the
syntax and type constructors for sum types in \cref{subsec:sums} and illustrate that they indeed work as expected.

\subsection{Multiplicative Pair Type}\label{subsec:mul-pairs}
Lambda encoding of muliplicative pairs is given in \cref{fig:mul-pair}. The typing rules
are given in \cref{fig:mul-pair-rules}. We give the proofs of the typing rules in \cref{subsec:mul-pairs-deriv}.
The meaning of a multiplicative pair can be thought of as
having separate resouce entities together in the program environment context and they would have to be
explicitly disposed off. Failure to do so, would raise a type error regarding the resources
not being unrestricted.

\begin{figure}[h]
  \begin{framed}
    \begin{flalign*}
      \otimes &\in \mathcal{T}^{\star \rightarrow \star \rightarrow \star}\\
      \tau \otimes \tau' &= \tau \sepimp \tau' \sepimp (\tau \sepimp \tau' \sepimp \upsilon) \sepimp \upsilon\\
      (,) &= \lambda^{*} x. \lambda^{*} y. \lambda^{*} f. f x y
    \end{flalign*}
  \end{framed}
\caption{Multiplicative Pair}
\label{fig:mul-pair}
\end{figure}
\begin{figure}[h]
  \begin{framed}
    \begin{minipage}{1\linewidth}
      \begin{prooftree}
        \AxiomC{$P \mid \Gamma  \vdash M : \tau$}
        \AxiomC{$P \mid \Delta \vdash N: \tau'$} \RightLabel{$[\otimes I]$}
        \BinaryInfC{$P \mid \Gamma \circledast \Delta \vdash \Pair{M,N}: \tau \otimes \tau'$}
      \end{prooftree}
    \end{minipage}
    \begin{minipage}{1\linewidth}
      \begin{prooftree}
        \AxiomC{$P \mid \Gamma \vdash M: \tau \otimes \tau'$}
        \AxiomC{$P \mid \{ x^{\{\vec{z} \mid \vec{z} \subseteq \texttt{Vars}(\Gamma')\}}:\tau \} \circledast
          \{ y^{\{\bar{z'} \mid \bar{z'} \subseteq \texttt{Vars}(\Gamma')\}} : \tau'\} \sqcup \Gamma'_{x,y} \vdash N:\upsilon$}\RightLabel{$[\otimes E]$}
        \BinaryInfC{$P \mid \Gamma \sqcup \Gamma' \vdash (\Let{\Pair{x, y}}{M}{N}): \upsilon$}
      \end{prooftree}
    \end{minipage}
  \end{framed}
  \caption{Derivable Typing Rules for Multiplicative Pair}
  \label{fig:mul-pair-rules}
\end{figure}

\subsection{Additive Pair Type}\label{subsec:add-pairs}
The lambda encoding for additive pairs is given in \cref{fig:add-pairs}. The typing rules are given in \cref{fig:add-pair-rules}. We give
proof of derivations for the rules in \cref{subsec:add-pairs-deriv}. Additive pairs can be thought of as program
entities that have sharing resources. We use \Fst{} and \Snd{} as projection functions on the additive pair to obtain the
constituent objects. It is worthwhile to note that we do not provide these projection functions for multiplicative pairs
so as to restrict the programmer from implicitly discarding one of the components of the multipicative pair.

\begin{figure}[h]
  \centering
  \begin{framed}
    \begin{flalign*}
      \with &\in \mathcal{T}^{\star \rightarrow \star \rightarrow \star}\\
      \tau \with \tau' &= \tau \sepimp \tau' \rightarrow (\tau \sepimp \tau' \rightarrow \upsilon) \rightarrow \upsilon\\
      (;) &= \lambda^{*} x. \lambda^{\alpha} by. \lambda^{\alpha} f. f x y
    \end{flalign*}
  \end{framed}
  \caption{Additive Pair}
  \label{fig:add-pair}
\end{figure}
\begin{figure}[h]
  \begin{framed}
    \begin{minipage}{1\linewidth}
      \begin{prooftree}
        \AxiomC{$P \mid \Gamma  \vdash M : \tau$}
        \AxiomC{$P \mid \Delta \vdash N: \tau'$} \RightLabel{$[\with I]$}
        \BinaryInfC{$P \mid \Gamma \varoplus \Delta \vdash \Pair{M;N}: \tau \with \tau'$}
      \end{prooftree}
    \end{minipage}
    \begin{minipage}{.5\linewidth}
      \begin{prooftree}
        \AxiomC{$P \mid \Gamma \vdash M: \tau \with \tau'$} \RightLabel{$[\with E_1]$}
        \UnaryInfC{$P \mid \Gamma \vdash \texttt{fst}\ M: \tau$}
      \end{prooftree}
    \end{minipage}
    \begin{minipage}{.5\linewidth}
      \begin{prooftree}
        \AxiomC{$P \mid \Gamma \vdash M: \tau \with \tau'$} \RightLabel{$[\with E_2]$}
        \UnaryInfC{$P \mid \Gamma \vdash \texttt{snd}\ M: \tau'$}
      \end{prooftree}
    \end{minipage}
  \end{framed}
  \caption{Derivable Typing Rules for Additive Pair}
  \label{fig:add-pair-rules}
\end{figure}


\subsection{Sum Type}\label{subsec:sums}
The lambda encoding of sum types is given in \cref{fig:sum-types} and the derivable typing rules are given in \cref{fig:bi-sum-types}.
We give proof of derivations in \cref{sec:sums-deriv}. Sum types can be thought of as a choice between two types. At a given point of
time only one of the two types would exist. We provide two functions to be used has handlers for both the types. The $\Case{M}{\{f;g\}}$ construct
is the deconstructor that would decide which function to use $f$ or $g$ depending on the value evaluated for $M$. The return type
of both the functions $f$ and $g$ would have to be the same for them to have correct typing. Encoding sum types within the language would be
helpful in defining data structures such as lists, trees and option types.

\begin{figure}[h]
  \begin{framed}\centering
    \begin{flalign*}
      \oplus &\in \mathcal{T}^{\star \rightarrow \star \rightarrow \star}\\
      \tau \oplus \tau' &= (\tau \rightarrow \upsilon) \rightarrow (\tau' \rightarrow \upsilon) \rightarrow \upsilon\\
      \Inl{} &: \tau \sepimp (\tau \oplus \tau')\\
      \Inl{} &= \lambda^{*}x. \lambda^{\alpha} f. \lambda^{\alpha} g. f x\\
      \Inr{} &: \tau' \sepimp (\tau \oplus \tau')\\
      \Inr{} &= \lambda^{*}y. \lambda^{\alpha} f. \lambda^{\alpha} g. g y
    \end{flalign*}
  \end{framed}
  \caption{Sum Type}
  \label{fig:sum-types}
\end{figure}

\begin{figure}[h]
\begin{framed}
% sum type I_l
\begin{minipage}{0.5\textwidth}
  \begin{prooftree}
    \AxiomC{$P \mid \Gamma \vdash M: \tau$}\RightLabel{[$\oplus$I$_l$]}
    \UnaryInfC{$P \mid \Delta \vdash \texttt{inl}\ M: \tau \oplus \tau'$}
  \end{prooftree}
\end{minipage}
% sum type I_r
\begin{minipage}{0.5\textwidth}
  \begin{prooftree}
    \AxiomC{$P \mid \Gamma \vdash M: \tau'$}\RightLabel{[$\oplus$I$_r$]}
    \UnaryInfC{$P \mid \Delta \vdash \texttt{inr}\ M: \tau \oplus \tau'$}
  \end{prooftree}
\end{minipage}
% sum type E
\begin{minipage}{1\textwidth}
  \begin{prooftree}
    \AxiomC{$P \mid \Gamma \vdash M: \tau \oplus \tau'$}
    \AxiomC{$P \mid \Gamma \circledast x:\tau \vdash N :\upsilon$}
    \AxiomC{$P \mid \Gamma \circledast y:\tau' \vdash N':\upsilon$}\RightLabel{[$\oplus$E]}
    \TrinaryInfC{$P \mid \Gamma \vdash \Case{M}{\{\texttt{inl}\ x \mapsto N; \texttt{inr}\ y \mapsto N'\}} : \upsilon $}
  \end{prooftree}
\end{minipage}
\end{framed}
\caption{Derivable Typing Rules for Sum Type}
\label{fig:bi-sum-types}
\end{figure}


\section{Generic Type Constructors}\label{sec:type-constructors}

To leverage the full power of introducing the sharing concept in a programming language,
we would want to have users be able to define datatypes with sharing and separation of its
constituent resources.  We add two new typing rules as shown in \cref{fig:ud-datatypes}
for handling generic datatypes using type constructors.
In the current implimentation, we only consider a specific case of type constructors where all the type variables
involved are either all shared or all separate. % and give an insight of how this
% can be generalized to have a mix of sharing and separation between the type parameters
% for a type constructor.

\TODO{How do we write about type constructors?}
\begin{figure}[h]
  \begin{framed}\centering
    % sharing constructor
    \begin{minipage}{0.45\textwidth}
      \begin{prooftree}
        \AxiomC{$P \mid \Gamma \varoplus x:\tau^{\kappa} \vdash C:\tau^{\kappa} \rightarrow T$}\RightLabel{$[\text{C-sh}]$}
        \UnaryInfC{$P \mid \Gamma \vdash C x : T $}
      \end{prooftree}
    \end{minipage}
    % Separating constructor
    \begin{minipage}{0.45\textwidth}
      \begin{prooftree}
        \AxiomC{$P \mid \Gamma \circledast x:\tau^{\kappa} \vdash C:\tau^{\kappa} \sepimp T$}\RightLabel{$[\text{C-se}]$}
        \UnaryInfC{$P \mid \Gamma \vdash C x : T $}
      \end{prooftree}
    \end{minipage}
  \end{framed}
  \caption{User Defined Datatypes}
  \label{fig:ud-datatypes}
\end{figure}

We illustrate using examples how the type system works.
The sharing pair has shared resources and using \Fst{} and \Snd{} works
to get both the individual constituents as shown in \cref{fig:sharing-pair}
The (!!) denots that a and b are in sharing. The fstp and sndp functions
typecheck as expected.

\begin{figure}[h]
  \begin{framed}
\begin{minted}{haskell}
data Pair' a b = ShP !! a b

fstp::Pair' a b -> a
fstp (ShP x y) = x

sndp :: Pair' a b -> b
sndp (ShP x y) = y
\end{minted}
  \end{framed}

  \caption{Shared Pairs}
  \label{fig:sharing-pair}
\end{figure}

Separating pair is defined in \cref{fig:sep-pair}. The functions fst and snd do not type check
as on of the resources is being dropped and we would need the resource to be unrestricted.

\begin{figure}[h]
  \begin{framed}
\begin{minted}{haskell}
data Pair a b = SeP a b

fst :: Pair' a b -> a
fst (SeP x y) = x

snd :: Pair' a b -> b
snd (SeP x y) = y
\end{minted}
  \end{framed}

  \caption{Separating Pair}
  \label{fig:sep-pair}
\end{figure}


% Having two arrows indroduces two types of pairs. A sharing pair (\texttt{Pair$_{sh}$}) and a separating pair (\texttt{Pair$_{sep}$}) depending on
% which arrow we use. The church encoding of both the pairs and their respective types are given in \ref{fig:bi-pairs-typing}.
% Derivation proofs of both the pairs are given in \cref{sec:pairs}.


% We can now introduce 2 types in our type language $\otimes$ and $\with$
% which represent separating pair and sharing pair respectively as shown in \cref{fig:pair-typing-rules}.
% We also have a sum type whose rules are shown in \cref{fig:bi-sum-types}.
% Proof derivation and definitions for sum type is given in \cref{sec:sums}.


% The datatypes are user defined types:
% There are 2 main kinds of non-recursive datatypes

% 1) data Choice a b = L a | R b

% where the types a and b do not appear in both the constructors.
% Choice here is a sum type a + b. How can we handle sum types in the current scenario?
% 1 way is to have a and b both not share resources. in this case we will have
% a and b to be both un restricted.

% a and b cannot direclty share resources. As the meaning of a sum type is that
% either a or b can exist and not both. So we will always have separating arrow
% for the sum types

% Write the elemination rules for Either

% for functions the closure has to be equal.

% 2) data Pair a b = P a b

% The constructor uses both the resources a and b
% Here we can have 2 subtypes:

% a and b share resources or can be completely different.
% We would want the programmer to specify whether the arguments are shared or separate.

% For current purposes this syntactical specification is given via a modified syntax

% data SeparatingPair a b = MkSeP a, b
% data SharingPair a b = MkShP !! a b

% We can of-course mix and match the sharing or separation.

% data MixTuple a b c d e = MkMixTuple a,b;c,d

% The question now that needs to be answered is how should the sharing be interpreted?
% There are various options.

% 1. interpreted right to left. i.e. MkMixTuple (a, (b; (c,d)))
%    so the b is shared with separating pair (c,d) and a is separate than b

% 2. interpreted left to right. i.e. MkMixTuple (((a, b);c),d)
%    so the c is shared with separating pair (a,b) and d is separate from c

% 3. Sharing ; has precedence over separating , i.e. MkMixTuple a,(b;c),d

% 4. Separating , has lower precedence over ; i.e. MkMixTuple (a,b);(c,d)

% This should be handled during the AST generating while parsing.
% The type checking algorithm should be robust enough
% to perform type checking either of these 4 types of sharing as they are
% all perfectly valid scenarios.

% Recursive datatypes have some kind of structural similarities within themselves

% data Tree a = Leaf a | Node a (Tree a) (Tree a)

% data Tree a b = Leaf b | Node a (Tree a b) (Tree a b)

% Open questions:
% How do we specify sharing of data between nodes?
% eg. How can we say that all the leaves have shared data? and the data in nodes is sparating


% What happens when lesser unrestrictedness with datatypes? Choice a b >:= f


% \section{Examples}

% \subsection{Pairs}

% Our language has a notion of 2 kinds of pairs. The pairs representing resources that do not share
% are denoted using separating pair $A \otimes B$. This means that when ever 



%%% Local Variables:
%%% mode: latex
%%% TeX-master: "../thesis-ku.tex"
%%% End:
