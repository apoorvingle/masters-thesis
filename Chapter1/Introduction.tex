\chapter{Introduction}
\section{Motivation}
Resources are treated as normal values in Hindley-Milner based type systems. Thus, statically typed modern programming languages such as
Haskell, ML, etc. do little to track resource usage at compile time to detect errors, such as, not closing open file handles
or closing the file handle more than once. Substructural logic systems such as linear logic, restrict the use of
structural rules---contraction and weakening---to view values as resources. There have been several attempts to create a practical
type system for programming language but they have not gained popularity due to practical limitations.
We develop a type system based on the logic of bunched implications and implement a type inference algorithm $\M$ for the language.
We extend the system with kinds that allows users to define their own datatypes.


\section{Contributions}
\begin{itemize}
\item \qub{}, A type system based on logic of \BI{} that treates resources as first class citizens in the programming language.
  We introduce two kinds of program objects---restricted and unrestriced---and two kinds of functions---the ones that share resrouces with
  their arguments and the other that do not.
\item Design a sound and complete syntax directed type system and implement a type inference algorithm based on modified Algorithm $\M$.
\item Examples related to file handling and exceptions that show how \qub{} is more expressive than standard
  Hindley-Milner type system for by tracing resource use at compile time to detect anomalies.
\end{itemize}