\chapter{Derivations for Products and Sums}\label{chp:derivation}
We start this section by adding few auxilary defitions for terms and types from first principles.
By convention, we denote an empty typing context by $I$ and empty predicate context with $\emptyset$

\section{Derivable Typing Rules For Product Types (Addiditive and Multiplicative Pairs)}\label{sec:pairs-deriv}
Pairs now have two meanings. Either they have sharing resources or they have separating resources. We define each
of them below.

\subsection{Additive Pairs}\label{subsec:add-pairs-deriv}
If the resources are in sharing we say they are additive pairs.

\begin{minipage}[h]{1.0\linewidth}
  \begin{prooftree}
    \AxiomC{$$}\RightLabel{[ID]}
    \UnaryInfC{$\emptyset \mid y^{xf}: \tau' \vdash y: \tau' $}

    \AxiomC{$$}
    \UnaryInfC{$\emptyset \mid x^{fy}: \tau \vdash x: \tau$}\RightLabel{[ID]}

    \AxiomC{$$}\RightLabel{[ID]}
    \UnaryInfC{$\emptyset \mid f^{xy}: \tau \shimp \tau' \shimp \upsilon \vdash f: \tau \shimp \tau' \shimp \upsilon$}\RightLabel{[$\sepimp E$]}
    \BinaryInfC{$\emptyset \mid x^{xy}:\tau \varoplus f^{xy}: \tau \shimp \tau' \shimp \upsilon \vdash f x: (\tau' \shimp \upsilon)$}\RightLabel{[$\shimp E$]}

    \BinaryInfC{$\emptyset \mid y^{xf}:\tau' \varoplus x^{yf}:\tau \varoplus f^{xy}:\tau \shimp \tau' \shimp \upsilon \vdash f x y: \upsilon$}\RightLabel{[$EXCH$]}
    \UnaryInfC{$\emptyset \mid x^{yf}:\tau \varoplus y^{xf}:\tau' \varoplus f^{xy}: \tau \shimp \tau' \shimp \upsilon \vdash f x y: \upsilon$}\RightLabel{[$\shimp I$]}
    \UnaryInfC{$\emptyset \mid x^{y}:\tau \varoplus y^{x}:\tau' \vdash \lambda^{\shimp}f. f x y: (\tau \shimp \tau' \shimp \upsilon) \shimp \upsilon$}\RightLabel{[$\shimp I$]}
    \UnaryInfC{$\emptyset \mid x^{\emptyset}:\tau \vdash \lambda^{\shimp}y. \lambda^{\shimp}f. f x y: \tau' \shimp (\tau \shimp \tau' \shimp \upsilon) \shimp \upsilon$}\RightLabel{[$\sepimp$ I]}
    \UnaryInfC{$\emptyset \mid I \vdash \lambda^{\sepimp}x. \lambda^{\shimp}y. \lambda^{\shimp}f. f x y: \tau \sepimp \tau' \shimp (\tau \shimp \tau' \shimp \upsilon) \shimp \upsilon$}
  \end{prooftree}
\end{minipage}\\

\noindent
We assign a new type symbol ($\with$) to describe type of additive pair
\begin{framed}\centering
  $\tau \with \tau' = (\tau \shimp \tau' \shimp \upsilon) \shimp \upsilon$
\end{framed}

\noindent
We now define sharing (additive) pair constructor as:
\begin{framed}\centering
  \begin{flalign*}
    ; &: \tau \sepimp \tau' \shimp (\tau \with \tau')\\
    ; &= \lambda^{\sepimp}x. \lambda^{\shimp}y. \lambda^{\shimp}f. f x y
  \end{flalign*}
\end{framed}

\noindent
We now derive left and right projections or deconstructors for sharing pairs below:

\begin{minipage}[h]{1.0\linewidth}
  \begin{prooftree}
    \AxiomC{$$}\RightLabel{[ID]}
    \UnaryInfC{$\emptyset \mid x^{y}:\tau \vdash x: \tau$}\RightLabel{[WKN-SH]}
    \UnaryInfC{$\emptyset \mid x^{y}:\tau \varoplus y^{x}:\tau' \vdash x: \tau$}\RightLabel{[$\shimp I$]}
    \UnaryInfC{$\emptyset \mid x^{\emptyset}:\tau \vdash \lambda^{\shimp}y. x : \tau' \shimp \tau $}\RightLabel{[$\shimp$I]}
    \UnaryInfC{$\emptyset \mid I \vdash \lambda^{\shimp}x. \lambda^{\shimp}y. x: \tau \shimp \tau' \shimp \tau$}
  \end{prooftree}
\end{minipage}
\begin{framed}
  \begin{flalign*}
  \texttt{fst} &: \tau \shimp \tau' \shimp \tau\\
  \texttt{fst} &= \lambda^{\shimp}x. \lambda^{\shimp}y. x
\end{flalign*}
\end{framed}

\begin{minipage}[h]{1.0\linewidth}
  \begin{prooftree}
    \AxiomC{$$}\RightLabel{[ID]}
    \UnaryInfC{$\emptyset \mid y^{x}:\tau' \vdash y: \tau' $}\RightLabel{[WKN-SH]}
    \UnaryInfC{$\emptyset \mid x^{y}:\tau \varoplus y^{x}:\tau' \vdash y: \tau' $}\RightLabel{[$\shimp I$]}
    \UnaryInfC{$\emptyset \mid x^{\emptyset}:\tau \vdash \lambda^{\shimp}y: \tau' \shimp \tau'$}\RightLabel{[$\shimp I$]}
    \UnaryInfC{$\emptyset \mid I \vdash \lambda^{\shimp}x. \lambda^{\shimp}y. y: \tau \shimp \tau' \shimp \tau'$}
  \end{prooftree}
\end{minipage}

\begin{framed}\centering
  \begin{flalign*}
    \texttt{snd} &: \tau \shimp \tau' \shimp \tau'\\
    \texttt{snd} &= \lambda^{\shimp}x. \lambda^{\shimp}y. y
  \end{flalign*}
\end{framed}

\subsection{Multiplicative Pairs}\label{subsec:mul-pairs-deriv}
If the resources are separate we say they are multiplicative or separating pairs.

\begin{minipage}[h]{1.0\linewidth}
  \begin{prooftree}
    \AxiomC{$$}\RightLabel{[ID]}
    \UnaryInfC{$\emptyset \mid y^{\emptyset}:\tau' \vdash y: \tau' $}

    \AxiomC{$$}\RightLabel{[ID]}
    \UnaryInfC{$\emptyset \mid x^{\emptyset}: \tau \vdash x: \tau$}

    \AxiomC{$$}\RightLabel{[ID]}
    \UnaryInfC{$\emptyset \mid f^{\emptyset}: \tau \sepimp \tau' \sepimp \upsilon \vdash f: \tau \sepimp \tau' \sepimp \upsilon $}\RightLabel{[$\sepimp$E]}
    \BinaryInfC{$\emptyset \mid x^{\emptyset}: \tau \circledast f^{\emptyset}: \tau \sepimp \tau' \sepimp \upsilon \vdash f x: (\tau' \sepimp \upsilon)$}
        \RightLabel{[$\sepimp$E]}

    \BinaryInfC{$\emptyset \mid y^{\emptyset}:\tau' \circledast x^{\emptyset}:\tau \circledast f^{\emptyset}:\tau \sepimp \tau' \sepimp \upsilon \vdash f x y: \upsilon$}
        \RightLabel{[$EXCH$]}
    \UnaryInfC{$\emptyset \mid x^{\emptyset}:\tau \circledast y^{\emptyset}:\tau' \circledast f^{\emptyset}:\tau \sepimp \tau' \sepimp \upsilon \vdash f x y: \upsilon$}\RightLabel{[$\sepimp$ I]}
    \UnaryInfC{$\emptyset \mid x^{\emptyset}:\tau \circledast y^{\emptyset}:\tau' \vdash \lambda^{\sepimp}f. f x y: (\tau \sepimp \tau' \sepimp \upsilon) \sepimp \upsilon$}\RightLabel{[$\sepimp$ I]}
    \UnaryInfC{$\emptyset \mid x^{\emptyset}:\tau \vdash \lambda^{\sepimp}y. \lambda^{\sepimp}f. f x y: \tau' \sepimp (\tau \sepimp \tau' \sepimp \upsilon) \sepimp \upsilon$}\RightLabel{[$\equiv$]}
    \UnaryInfC{$\emptyset \mid I \vdash \lambda^{\sepimp}x. \lambda^{\sepimp}y. \lambda^{\sepimp}f. f x y: \tau \sepimp \tau' \sepimp (\tau \sepimp \tau' \sepimp \upsilon) \sepimp \upsilon$}
  \end{prooftree}
\end{minipage}\\

\noindent
We assign a type symbol ($\otimes$) for the multiplicative pair
\begin{framed}\centering
  $\tau \otimes \tau' = (\tau \sepimp \tau' \sepimp \upsilon) \sepimp \upsilon$
\end{framed}

\noindent
We can now define separating (multiplicative) pair as:
\begin{framed}
  \begin{flalign*}
    , &: \tau \sepimp \tau' \sepimp (\tau \otimes \tau')\\
    , &= \lambda^{\sepimp}x. \lambda^{\sepimp}y. \lambda^{\sepimp}f. f x y
  \end{flalign*}
\end{framed}

We will abuse the notation of lambda calculus for $;$ and $,$ as use them as infix operators for syntactic convinence
\begin{flalign*}
  \langle x , y \rangle \equiv (,) x y\\
  \langle x ; y \rangle \equiv (;) x y
\end{flalign*}

We are now in a position to write the proof derivations for \cref{fig:add-pair-rules} and \cref{fig:mul-pair-rules} using
the auxilary definitions from above.
\begin{framed}
    \begin{minipage}{1\linewidth}
      \begin{prooftree}
        \AxiomC{$P \mid \Gamma  \vdash M : \tau$}
        \AxiomC{$P \mid \Delta \vdash N: \tau'$} \RightLabel{[$\with$ I]}
        \BinaryInfC{$P \mid \Gamma \varoplus \Delta \vdash \Pair{M;N}: \tau \with \tau'$}
      \end{prooftree}
    \end{minipage}
    \begin{minipage}{.5\linewidth}
      \begin{prooftree}
        \AxiomC{$P \mid \Gamma \vdash M: \tau \with \tau'$} \RightLabel{[$\with$ E$_1$]}
        \UnaryInfC{$P \mid \Gamma \vdash \texttt{fst}\ M: \tau$}
      \end{prooftree}
    \end{minipage}
    \begin{minipage}{.5\linewidth}
      \begin{prooftree}
        \AxiomC{$P \mid \Gamma \vdash M: \tau \with \tau'$} \RightLabel{[$\with$ E$_1$]}
        \UnaryInfC{$P \mid \Gamma \vdash \texttt{snd}\ M: \tau'$}
      \end{prooftree}
    \end{minipage}

    \begin{minipage}{1\linewidth}
      \begin{prooftree}
        \AxiomC{$P \mid \Gamma  \vdash M : \tau$}
        \AxiomC{$P \mid \Delta \vdash N: \tau'$} \RightLabel{$[\otimes I]$}
        \BinaryInfC{$P \mid \Gamma \circledast \Delta \vdash \Pair{M,N}: \tau \otimes \tau'$}
      \end{prooftree}
    \end{minipage}
    \begin{minipage}{1\linewidth}
      \begin{prooftree}
        \AxiomC{$P \mid \Gamma \vdash M: \tau \otimes \tau'$}
        \AxiomC{$P \mid \{ x^{\{\vec{z} \mid \vec{z} \subseteq \texttt{Vars}(\Gamma')\}}:\tau \} \circledast
          \{ y^{\{\bar{z'} \mid \bar{z'} \subseteq \texttt{Vars}(\Gamma')\}} : \tau'\} \sqcup \Gamma'_{x,y} \vdash N:\upsilon$}\RightLabel{$[\otimes E]$}
        \BinaryInfC{$P \mid \Gamma \sqcup \Gamma' \vdash (\Let{\Pair{x, y}}{M}{N}): \upsilon$}
      \end{prooftree}
    \end{minipage}
\end{framed}

\section{Derivable Typing rules for Sum Types}\label{sec:sums-deriv}
Sum types can hold only one of the enclosing types at a given point of time.

\begin{landscape}
  % choice
  \noindent
  \begin{small}
  \begin{prooftree}
    \AxiomC{$$}\RightLabel{[ID]}
    \UnaryInfC{$\emptyset \mid  g^{cf}:(B \shimp E) \vdash g: (B \shimp E)$}

    \AxiomC{$$}\RightLabel{[ID]}
    \UnaryInfC{$\emptyset \mid f^{cg}: (A \shimp E) \vdash f: (A \shimp E)$}
    \AxiomC{$$}\RightLabel{[ID]}
    \UnaryInfC{$\emptyset \mid c^{fg}: ((A \shimp E) \shimp (B \shimp E) \shimp E)
      \vdash c:((A \shimp E) \shimp (B \shimp E) \shimp E)$}\RightLabel{[$\shimp$E]}
    \BinaryInfC{$\emptyset \mid c^{fg}:((A \shimp E) \shimp (B \shimp E) \shimp E)
      \varoplus f^{cg}:(A \shimp E) \vdash c f: (B \shimp E) \shimp E$}\RightLabel{[$\shimp$E]}

    \BinaryInfC{$\emptyset \mid  c^{fg}:((A \shimp E) \shimp (B \shimp E) \shimp E)  \varoplus
      f^{cg}:(A \shimp E)  \varoplus g^{cf}:(B \shimp E) \vdash c f g : E$}\RightLabel{[$\shimp$I]}
    \UnaryInfC{$\emptyset \mid  c^{f}:((A \shimp E) \shimp (B \shimp E) \shimp E) \varoplus
      f^{c}:(A \shimp E) \vdash \lambda^{\shimp} g. c f g :(B \shimp E) \shimp  E$}\RightLabel{[$\shimp$I]}
    \UnaryInfC{$\emptyset \mid  c^{\emptyset}:((A \shimp E) \shimp (B \shimp E) \shimp E)
      \vdash  \lambda^{\shimp} f. \lambda^{\shimp} g. c f g :(A \shimp E) \shimp (B \shimp E) \shimp E$}\RightLabel{[$\sepimp$I]}
    \UnaryInfC{$\emptyset \mid I \vdash \lambda^{\sepimp} c. \lambda^{\shimp} f. \lambda^{\shimp} g. c f g
      : ((A \shimp E) \shimp (B \shimp E) \shimp E) \sepimp (A \shimp E) \shimp (B \shimp E) \shimp E$}
  \end{prooftree}
\end{small}

\noindent
We now define sum type to be
\begin{framed}\centering
  $A \oplus B = (A \shimp E) \shimp (B \shimp E) \shimp E$\\
\end{framed}

We now define left and right {\it injections} or constructors for the sum type.

% left
\begin{prooftree}
  \AxiomC{$$}\RightLabel{[ID]}
  \UnaryInfC{$\emptyset \mid x^{fg}: A  \vdash x : A$}

  \AxiomC{$$}\RightLabel{[ID]}
  \UnaryInfC{$\emptyset \mid f^{gx}: (A \shimp E) \vdash f: (A \shimp E)$}\RightLabel{[$\shimp$E]}

  \BinaryInfC{$\emptyset \mid x^{fg}: A \varoplus f^{xg}: (A \shimp E)  \vdash f x : E$}\RightLabel{[WKN-UN]}
  \UnaryInfC{$\emptyset \mid x^{fg}: A \varoplus f^{xg}: (A \shimp E) \varoplus g^{fx}: (B \shimp E) \vdash f x : E$}\RightLabel{[$\shimp$I]}
  \UnaryInfC{$\emptyset \mid x^{f}: A \varoplus f^{x}: (A \shimp E) \vdash \lambda^{\shimp} g. f x : (B \shimp E) \shimp E$}\RightLabel{[$\shimp$I]}
  \UnaryInfC{$\emptyset \mid  x^{\emptyset}: A \vdash  \lambda^{\shimp} f. \lambda^{\shimp} g. f x
    :(A \shimp E) \shimp (B \shimp E) \shimp E$}\RightLabel{[$\sepimp$I]}\RightLabel{[$\sepimp$I]}
  \UnaryInfC{$\emptyset \mid I \vdash \lambda^{\sepimp} x. \lambda^{\shimp} f. \lambda^{\shimp} g. f x
    : A \sepimp  (A \shimp E) \shimp (B \shimp E) \shimp E$}
\end{prooftree}

\noindent Left injection defined below as:
\begin{framed}\centering
  $\texttt{inl} : A \sepimp A \oplus B$\\
  $\texttt{inl} = \lambda^{\sepimp} x. \lambda^{\shimp} f. \lambda^{\shimp} g. f x$
\end{framed}

% right
\noindent
\begin{prooftree}
  \AxiomC{$$}\RightLabel{[ID]}
  \UnaryInfC{$\emptyset \mid y^{fg}: B  \vdash y : B$}

  \AxiomC{$$}\RightLabel{[ID]}
  \UnaryInfC{$\emptyset \mid g^{yf}: (B \sepimp E) \vdash g: (B \sepimp E)$}\RightLabel{[$\shimp$I]}

  \BinaryInfC{$\emptyset \mid y^{fg}: B \varoplus g^{yf}: (B \shimp E) \vdash g y : E$}\RightLabel{[WKN-UN]}
  \UnaryInfC{$\emptyset \mid y^{fg}: B \varoplus f^{yg}: (A \shimp E) \varoplus g^{yf}: (B \shimp E) \vdash g y : E$}\RightLabel{[$\shimp$I]}
  \UnaryInfC{$\emptyset \mid y^{f}: B \varoplus f^{y}: (A \shimp E) \vdash \lambda^{\shimp} g. g y : (B \shimp E) \shimp E$}\RightLabel{[$\shimp$I]}
  \UnaryInfC{$\emptyset \mid  y^{\emptyset}: B \vdash \lambda^{\shimp} f. \lambda^{\shimp} g. g y
    :(A \shimp E) \shimp (B \shimp E) \shimp E$}\RightLabel{[$\sepimp$I]}\RightLabel{[$\sepimp$I]}
  \UnaryInfC{$\emptyset \mid I \vdash \lambda^{\sepimp} y. \lambda^{\shimp} f. \lambda^{\shimp} g. g y: B \sepimp  (A \shimp E) \shimp (B \shimp E) \shimp E$}
\end{prooftree}

\noindent Right injection defined below as:
\begin{framed}\centering
  $\texttt{inr} : B \sepimp A \oplus B$\\
  $\texttt{inr} = \lambda^{\sepimp} y. \lambda^{\shimp} f. \lambda^{\shimp} g. g y$
\end{framed}
\end{landscape}

We can now derive sum types in our language using the auxilary definitions given above
and provide a new syntax for deconstructing the sum type by matching on its structure by a case statement.
\begin{framed}\centering
    $\texttt{case}\ {c}\ \texttt{of}\ {\{ f;g \}} = \lambda^{\sepimp}c. \lambda^{\shimp}f. \lambda^{\shimp}g. c f g$
\end{framed}

\begin{framed}
  \begin{minipage}[h]{0.5\linewidth}
    \begin{prooftree}
      \AxiomC{$\emptyset \mid \texttt{inl}: A \sepimp A \oplus B \vdash x : A$}\RightLabel{[$\oplus$I$_1$]}
      \UnaryInfC{$\emptyset \mid I \vdash \texttt{inl}\ x: A \oplus B$}
    \end{prooftree}
  \end{minipage}
  \begin{minipage}[h]{0.5\linewidth}
    \begin{prooftree}
      \AxiomC{$\emptyset \mid \texttt{inr}: B \sepimp A \oplus B \vdash y : B$}\RightLabel{[$\oplus$I$_2$]}
      \UnaryInfC{$\emptyset \mid I \vdash \texttt{inr}\ y: A \oplus B$}
    \end{prooftree}
  \end{minipage}
  \begin{minipage}[h]{1.0\linewidth}
    \begin{prooftree}
      \AxiomC{$\emptyset \mid \Gamma \vdash M : A \oplus B$}\RightLabel{[$\oplus$E]}
      \AxiomC{$\emptyset \mid \Gamma \varoplus x:A \vdash N_1 : E$}
      \AxiomC{$\emptyset \mid \Gamma \varoplus y:B \vdash N_2 : E$}
      \TrinaryInfC{$\emptyset \mid \Gamma \vdash \Case{M}{\{ \texttt{inl}\ x \mapsto N_1; \texttt{inr}\ y \mapsto N_2\}}:E$}
    \end{prooftree}
  \end{minipage}
\end{framed}


%%% Local Variables:
%%% mode: latex
%%% TeX-master: "../thesis-ku"
%%% End:
