\chapter{Appendix}

\section{Proofs}
We start this section by adding few auxilary defitions for terms and types from first principles.
By convention, we denote an empty typing context by $I$ and empty predicate context with $\emptyset$

\subsection{Derivable Typing Rules For Product Types (Addiditive and Multiplicative Pairs)}\label{sec:pairs}
Pairs can now have two meanings. Either they have sharing resources or they have separating resources. We define each
of them below.

\subsubsection{Additive Pairs}
If the resources are in sharing we say they are additive pairs.

\begin{minipage}[h]{1.0\linewidth}
  \begin{prooftree}
    \AxiomC{$$}\RightLabel{[ID]}
    \UnaryInfC{$\emptyset \mid y^{xf}:B \vdash y: B $}

    \AxiomC{$$}
    \UnaryInfC{$\emptyset \mid x^{fy}: A \vdash x: A$}\RightLabel{[ID]}

    \AxiomC{$$}\RightLabel{[ID]}
    \UnaryInfC{$\emptyset \mid f^{xy}: A \rightarrow B \rightarrow C \vdash f: A \rightarrow B \rightarrow C $}\RightLabel{[$\sepimp E$]}
    \BinaryInfC{$\emptyset \mid x^{xy}:A \varoplus f^{xy}: A \rightarrow B \rightarrow C \vdash f x: (B \rightarrow C)$}\RightLabel{[$\rightarrow E$]}

    \BinaryInfC{$\emptyset \mid y^{xf}:B \varoplus x^{yf}:A \varoplus f^{xy}:A \rightarrow B \rightarrow C \vdash f x y: C$}\RightLabel{[$EXCH$]}
    \UnaryInfC{$\emptyset \mid x^{yf}:A \varoplus y^{xf}:B \varoplus f^{xy}: A \rightarrow B \rightarrow C \vdash f x y: C$}\RightLabel{[$\rightarrow I$]}
    \UnaryInfC{$\emptyset \mid x^{y}:A \varoplus y^{x}:B \vdash \lambda^{\alpha}f. f x y: (A \rightarrow B \rightarrow C) \rightarrow C$}\RightLabel{[$\rightarrow I$]}
    \UnaryInfC{$\emptyset \mid x^{\emptyset}:A \vdash \lambda^{\alpha}y. \lambda^{\alpha}f. f x y: B \rightarrow (A \rightarrow B \rightarrow C) \rightarrow C$}\RightLabel{[$\sepimp$ I]}
    \UnaryInfC{$\emptyset \mid I \vdash \lambda^{*}x. \lambda^{\alpha}y. \lambda^{\alpha}f. f x y: A \sepimp B \rightarrow (A \rightarrow B \rightarrow C) \rightarrow C$}
  \end{prooftree}
\end{minipage}
\noindent
We can now define sharing (additive) pair as:
\begin{framed}\centering
    $; = \lambda^{*}x. \lambda^{\alpha}y. \lambda^{\alpha}f. f x y$
\end{framed}
and assign a new type to this pair and call it $\with$
\begin{framed}\centering
  $A \with B = (A \rightarrow B \rightarrow C) \rightarrow C$
\end{framed}

\noindent
We now define left and right projections or deconstructors for sharing pairs below:
\begin{minipage}[h]{1.0\linewidth}
  \begin{prooftree}
    \AxiomC{$$}\RightLabel{[ID]}
    \UnaryInfC{$\emptyset \mid x^{y}:A \vdash x: A $}\RightLabel{[WKN-SH]}
    \UnaryInfC{$\emptyset \mid x^{y}:A \varoplus y^{x}:B \vdash x: A $}\RightLabel{[$\rightarrow I$]}
    \UnaryInfC{$\emptyset \mid x^{\emptyset}:A \vdash \lambda^{\alpha}y. x : B \rightarrow A $}\RightLabel{[$\rightarrow$I]}
    \UnaryInfC{$\emptyset \mid I \vdash \lambda^{\alpha}x. \lambda^{\alpha}y. x: A \rightarrow B \rightarrow A$}
  \end{prooftree}
\end{minipage}
\begin{framed}
  \centering
  $\texttt{fst}_{sh} = \lambda^{\alpha}x. \lambda^{\alpha}y. x$
\end{framed}

\begin{minipage}[h]{1.0\linewidth}
  \begin{prooftree}
    \AxiomC{$$}\RightLabel{[ID]}
    \UnaryInfC{$\emptyset \mid y^{x}:B \vdash y: B $}\RightLabel{[WKN-SH]}
    \UnaryInfC{$\emptyset \mid x^{y}:A \varoplus y^{x}:B \vdash y: B $}\RightLabel{[$\rightarrow I$]}
    \UnaryInfC{$\emptyset \mid x^{\emptyset}:A \vdash \lambda^{\alpha}y: B \rightarrow B $}\RightLabel{[$\rightarrow I$]}
    \UnaryInfC{$\emptyset \mid I \vdash \lambda^{\alpha}x. \lambda^{\alpha}y. y: A \rightarrow B \rightarrow B$}
  \end{prooftree}
\end{minipage}

\begin{framed}\centering
    $\texttt{snd}_{sh} = \lambda^{\alpha}x. \lambda^{\alpha}y. y$
\end{framed}

\subsubsection{Multiplicative Pairs}
If the resources are separate we say they are multiplicative pairs.

\begin{minipage}[h]{1.0\linewidth}
  \begin{prooftree}
    \AxiomC{$$}\RightLabel{[ID]}
    \UnaryInfC{$\emptyset \mid y^{\emptyset}:B \vdash y: B $}

    \AxiomC{$$}\RightLabel{[ID]}
    \UnaryInfC{$\emptyset \mid x^{\emptyset}: A \vdash x: A$}

    \AxiomC{$$}\RightLabel{[ID]}
    \UnaryInfC{$\emptyset \mid f^{\emptyset}: A \sepimp B \sepimp C \vdash f: A \sepimp B \sepimp C $}\RightLabel{[$\sepimp E$]}
    \BinaryInfC{$\emptyset \mid x^{\emptyset}:A \circledast f^{\emptyset}: A \sepimp B \sepimp C \vdash f x: (B \sepimp C)$}\RightLabel{[$\sepimp E$]}

    \BinaryInfC{$\emptyset \mid y^{\emptyset}:B \circledast x^{\emptyset}:A \circledast f^{\emptyset}:A \sepimp B \sepimp C \vdash f x y: C$}\RightLabel{[$EXCH$]}
    \UnaryInfC{$\emptyset \mid x^{\emptyset}:A \circledast y^{\emptyset}:B \circledast f^{\emptyset}:A \sepimp B \sepimp C \vdash f x y: C$}\RightLabel{[$\sepimp I$]}
    \UnaryInfC{$\emptyset \mid x^{\emptyset}:A \circledast y^{\emptyset}:B \vdash \lambda^{*}f. f x y: (A \sepimp B \sepimp C) \sepimp C$}\RightLabel{[$\sepimp I$]}
    \UnaryInfC{$\emptyset \mid x^{\emptyset}:A \vdash \lambda^{*}y. \lambda^{*}f. f x y: B \sepimp (A \sepimp B \sepimp C) \sepimp C$}\RightLabel{[$\equiv$]}
    \UnaryInfC{$\emptyset \mid I \vdash \lambda^{*}x. \lambda^{*}y. \lambda^{*}f. f x y: A \sepimp B \sepimp (A \sepimp B \sepimp C) \sepimp C$}
  \end{prooftree}
\end{minipage}

We can now define separating (multiplicative) pair as:
\begin{framed}\centering
    $,= \lambda^{*}x. \lambda^{*}y. \lambda^{*}f. f x y$
\end{framed}
We assign a new type to this pair construct and call it $\otimes$
\begin{framed}\centering
  $A \otimes B = (A \sepimp B \sepimp C) \sepimp C$
\end{framed}

\begin{minipage}[h]{1.0\linewidth}
  \begin{prooftree}
    \AxiomC{$$}\RightLabel{[ID]}
    \UnaryInfC{$\Un{B} \mid x^{\emptyset}:A \vdash x:A $}

    \AxiomC{$$}
    \UnaryInfC{$\Un{B} => \Un{B}$}\RightLabel{[UN-$\Gamma$, UN-$\sigma$, UN-$\tau$]}
    \UnaryInfC{$\Un{B} \vdash y^{\emptyset}:B\ \texttt{un} $}\RightLabel{[WKN-UN]}

    \BinaryInfC{$\Un{B} \mid x^{\emptyset}:A \circledast y^{\emptyset}:B \vdash x: A $}\RightLabel{[$\sepimp I$]}
    \UnaryInfC{$\Un{B} \mid x^{\emptyset}:A \vdash \lambda^{*}y. x: B \sepimp A $}\RightLabel{[$\sepimp I$]}
    \UnaryInfC{$\Un{B} \mid I \vdash \lambda^{*}x. \lambda^{*}y. x: A \sepimp B \sepimp A$}
  \end{prooftree}
\end{minipage}

\begin{framed}\centering
    $\texttt{fst}_{sep} = \lambda^{*}x. \lambda^{*}y. x$
\end{framed}

\begin{minipage}[h]{1.0\linewidth}
  \begin{prooftree}
    \AxiomC{$$}\RightLabel{[ID]}
    \UnaryInfC{$\Un{A} \mid y^{\emptyset}:B \vdash y: B $}

    \AxiomC{$$}
    \UnaryInfC{$\Un{A} => \Un{A}$}\RightLabel{[UN-$\Gamma$, UN-$\sigma$, UN-$\tau$]}
    \UnaryInfC{$\Un{A} \vdash x^{\emptyset}:A\ \texttt{un} $}\RightLabel{[WKN-UN]}

    \BinaryInfC{$\Un{A} \mid x^{\emptyset}:A \circledast y^{\emptyset}:B \vdash y: B $}\RightLabel{[$\sepimp I$]}
    \UnaryInfC{$\Un{A} \mid x^{\emptyset}:A \vdash \lambda^{*}y.y: B \sepimp B $}\RightLabel{[$\sepimp I$]}
    \UnaryInfC{$\Un{A} \mid I \vdash \lambda^{*}x. \lambda^{*}y. y: A \sepimp B \sepimp B$}
  \end{prooftree}
\end{minipage}

\begin{framed}\centering
    $\texttt{snd}_{sep} = \lambda^{*}x. \lambda^{*}y. y$
\end{framed}

We will abuse the notation of lambda calculus for $;$ and $,$ as use them as infix operators for syntactic convinence
\begin{flalign*}
  \langle x , y \rangle \equiv (,) x y\\
  \langle x ; y \rangle \equiv (;) x y
\end{flalign*}

We are now in a position to write the proof derivations for \cref{fig:pair-typing-rules} using
the auxilary definitions from above.
\begin{framed}
  \noindent
  \begin{minipage}{0.5\linewidth}
    \begin{prooftree}
      \AxiomC{$\emptyset \mid \texttt{fst}_{sh}\ : A \rightarrow B \rightarrow A \vdash M: A \with B$}\RightLabel{$\with E_1$}
      \UnaryInfC{$\emptyset \mid I \vdash \texttt{fst}_{sh}\  M : A$}
    \end{prooftree}
  \end{minipage}
  \begin{minipage}{0.5\linewidth}
    \begin{prooftree}
      \AxiomC{$\emptyset \mid \texttt{snd}_{sh}\ : A \rightarrow B \rightarrow B \vdash M: A \with B $}\RightLabel{$\with E_2$}
      \UnaryInfC{$\emptyset \mid I \vdash \texttt{snd}_{sh}\  M : B$}
    \end{prooftree}
  \end{minipage}
  \noindent
  \begin{minipage}{1\linewidth}
    \begin{prooftree}
      \AxiomC{$P \mid \emptyset \vdash (;) : \with $}
      \AxiomC{$P \mid \Gamma  \vdash M : A$}
      \AxiomC{$P \mid \Delta \vdash N: B $} \RightLabel{$[\with I]$}
      \TrinaryInfC{$P \mid \Gamma \varoplus \Delta \vdash \langle M ; N \rangle: A \with B$}
    \end{prooftree}
  \end{minipage}
  \noindent
  \begin{minipage}{0.5\linewidth}
    \begin{prooftree}
      \AxiomC{$\Un{B} \mid \texttt{fst}_{sep}\ : A \sepimp B \sepimp A \vdash  M: A \otimes B $}\RightLabel{$\otimes E_1$}
      \UnaryInfC{$\Un{B} \mid I \vdash \texttt{fst}_{sep}\  M : A$}
    \end{prooftree}
  \end{minipage}
  \begin{minipage}{0.5\linewidth}
    \begin{prooftree}
      \AxiomC{$\Un{A} \mid \texttt{snd}_{sep}\ : A \sepimp B \sepimp B \vdash M: A \otimes B $}\RightLabel{$\otimes E_2$}
      \UnaryInfC{$\Un{A} \mid I \vdash \texttt{snd}_{sep}\  M : B$}
    \end{prooftree}
  \end{minipage}
  \noindent
  \begin{minipage}{1\linewidth}
    \begin{prooftree}
      \AxiomC{$P \mid \emptyset \vdash (,) : \otimes $}
      \AxiomC{$P \mid \Gamma  \vdash M : A$}
      \AxiomC{$P \mid \Delta \vdash N: B $} \RightLabel{$[\otimes I]$}
      \TrinaryInfC{$P \mid \Gamma \circledast \Delta \vdash \langle M , N \rangle: A \otimes B$}
    \end{prooftree}
  \end{minipage}
\end{framed}

\subsection{Derivable Typing rules for (Multiplicative and Additive) Sum Types}\label{sec:sums}
Sum types can hold only one of the enclosing types.
\subsubsection{Muliplicative Sums}
If the two resources are separate, we call them multipicative sums.
\begin{landscape}
  % choice
  \noindent
  \begin{small}
  \begin{prooftree}
    \AxiomC{$$}\RightLabel{[ID]}
    \UnaryInfC{$\emptyset \mid  g^{\emptyset}:(B \sepimp E) \vdash g: (B \sepimp E)$}

    \AxiomC{$$}\RightLabel{[ID]}
    \UnaryInfC{$\emptyset \mid f^{\emptyset}: (A \sepimp E) \vdash f: (A \sepimp E)$}
    \AxiomC{$$}\RightLabel{[ID]}
    \UnaryInfC{$\emptyset \mid c^{\emptyset}: ((A \sepimp E) \sepimp (B \sepimp E) \sepimp E)
      \vdash c:((A \sepimp E) \sepimp (B \sepimp E) \sepimp E)$}\RightLabel{[$\sepimp$E]}
    \BinaryInfC{$\emptyset \mid c^{\emptyset}:((A \sepimp E) \sepimp (B \sepimp E) \sepimp E)
      \circledast f^{\emptyset}:(A \sepimp E) \vdash c f: (B \sepimp E) \sepimp E$}\RightLabel{[$\sepimp$E]}

    \BinaryInfC{$\emptyset \mid  c^{\emptyset}:((A \sepimp E) \sepimp (B \sepimp E) \sepimp E)  \circledast
      f^{\emptyset}:(A \sepimp E)  \circledast g^{\emptyset}:(B \sepimp E) \vdash c f g : E$}\RightLabel{[$\sepimp$I]}
    \UnaryInfC{$\emptyset \mid  c^{\emptyset}:((A \sepimp E) \sepimp (B \sepimp E) \sepimp E) \circledast
      f^{\emptyset}:(A \sepimp E) \vdash \lambda^{*} g. c f g :(B \sepimp E) \sepimp E$}\RightLabel{[$\sepimp$I]}
    \UnaryInfC{$\emptyset \mid  c^{\emptyset}:((A \sepimp E) \sepimp (B \sepimp E) \sepimp E) \vdash  \lambda^{*} f. \lambda^{*} g. c f g :(A \sepimp E) \sepimp (B \sepimp E) \sepimp E$}\RightLabel{[$\sepimp$I]}\RightLabel{[$\sepimp$I]}
    \UnaryInfC{$\emptyset \mid I \vdash \lambda^{*} c. \lambda^{*} f. \lambda^{*} g. c f g : ((A \sepimp E) \sepimp (B \sepimp E) \sepimp E) \sepimp (A \sepimp E) \sepimp (B \sepimp E) \sepimp E$}
  \end{prooftree}
\end{small}

  \noindent
  We now define sum type to be
  \begin{framed}\centering
    $A \oplus B = (A \sepimp E) \sepimp (B \sepimp E) \sepimp E$\\
    $\CaseSe{c}{\{ f;g \}} = \lambda^{*}c. \lambda^{*}f. \lambda^{*}g. c f g$
  \end{framed}

  We now define left and right {\it injections} or constructors for the sum type.
  % left
  \noindent
  \begin{prooftree}
    \AxiomC{$$}\RightLabel{[ID]}
    \UnaryInfC{$\Un{A \sepimp E} \mid y: B  \vdash y : B$}

    \AxiomC{$$}\RightLabel{}
    \UnaryInfC{$\Un{A \sepimp E} \vdash \Un{A \sepimp E}$}\RightLabel{[WKN-UN]}
    \BinaryInfC{$\Un{A \sepimp E} \mid f: (A \sepimp E), y: B  \vdash y : B$}

    \AxiomC{$$}\RightLabel{[ID]}
    \UnaryInfC{$\Un{A \sepimp E} \mid g: (B \sepimp E) \vdash g: (B \sepimp E)$}\RightLabel{[$\sepimp$I]}

    \BinaryInfC{$\Un{A \sepimp E} \mid f: (A \sepimp E), y: B, g: (B \sepimp E) \vdash g y : E$}\RightLabel{[EXCH]}
    \UnaryInfC{$\Un{A \sepimp E} \mid y: B, f: (A \sepimp E), g: (B \sepimp E) \vdash g y : E$}\RightLabel{[$\sepimp$I]}
    \UnaryInfC{$\Un{A \sepimp E} \mid y: B, f: (A \sepimp E) \vdash \lambda^{*} g. g y : (B \sepimp E) \sepimp E$}\RightLabel{[$\sepimp$I]}
    \UnaryInfC{$\Un{A \sepimp E} \mid y: B \vdash \lambda^{*} y. \lambda^{*} f. \lambda^{*} g. g y :(A \sepimp E) \sepimp (B \sepimp E) \sepimp E$}\RightLabel{[$\equiv$]}
    \UnaryInfC{$\Un{A \sepimp E} \mid  I, y: B \vdash \lambda^{*} y. \lambda^{*} f. \lambda^{*} g. g y :(A \sepimp E) \sepimp (B \sepimp E) \sepimp E$}\RightLabel{[$\sepimp$I]}\RightLabel{[$\sepimp$I]}
    \UnaryInfC{$\Un{A \sepimp E} \mid I \vdash \lambda^{*} y. \lambda^{*} f. \lambda^{*} g. g y: B \sepimp  (A \sepimp E) \sepimp (B \sepimp E) \sepimp E$}
  \end{prooftree}
  \noindent Left injection defined below as:
  \begin{framed}\centering
    $inl : A \sepimp A \oplus B$\\
    $\texttt{inl} = \lambda x. \lambda f. \lambda g. f x$
  \end{framed}

  % right
  \begin{prooftree}
    \AxiomC{$$}\RightLabel{[ID]}
    \UnaryInfC{$\Un{B \sepimp E} \mid x: A  \vdash x : B$}

    \AxiomC{$$}\RightLabel{}
    \UnaryInfC{$\Un{B \sepimp E} \vdash \Un{B \sepimp E}$}\RightLabel{[WKN-UN]}
    \BinaryInfC{$\Un{B \sepimp E} \mid g: (B \sepimp E), x: A  \vdash x : A$}

    \AxiomC{$$}\RightLabel{[ID]}
    \UnaryInfC{$\Un{B \sepimp E} \mid f: (A \sepimp E) \vdash f: (A \sepimp E)$}\RightLabel{[$\sepimp$E]}

    \BinaryInfC{$\Un{B \sepimp E} \mid x: A, f: (A \sepimp E), g: (B \sepimp E) \vdash f x : E$}\RightLabel{[$\sepimp$I]}
    \UnaryInfC{$\Un{B \sepimp E} \mid x: A, f: (A \sepimp E) \vdash \lambda^{*} g. f x : (B \sepimp E) \sepimp E$}\RightLabel{[$\sepimp$I]}
    \UnaryInfC{$\Un{B \sepimp E} \mid x: A \vdash  \lambda^{*} f. \lambda^{*} g. f x :(A \sepimp E) \sepimp (B \sepimp E) \sepimp E$}\RightLabel{[$\equiv$]}
    \UnaryInfC{$\Un{B \sepimp E} \mid  I, x: A \vdash  \lambda^{*} f. \lambda^{*} g. f x :(A \sepimp E) \sepimp (B \sepimp E) \sepimp E$}\RightLabel{[$\sepimp$I]}\RightLabel{[$\sepimp$I]}
    \UnaryInfC{$\Un{B \sepimp E} \mid I \vdash \lambda^{*} x. \lambda^{*} f. \lambda^{*} g. f x: A \sepimp  (A \sepimp E) \sepimp (B \sepimp E) \sepimp E$}
  \end{prooftree}

  \noindent \noindent Right injection defined below as:
  \begin{framed}\centering
    $inr : B \sepimp A \oplus B$\\
    $\texttt{inr} = \lambda y. \lambda f. \lambda g. g y$
  \end{framed}
\end{landscape}

We can now add sum types in our language using the auxilary definitions given above:
\begin{framed}
  \begin{minipage}[h]{0.5\linewidth}
    \begin{prooftree}
      \AxiomC{$\emptyset \mid \texttt{inr}: A \sepimp A \oplus B \vdash x : A$}\RightLabel{[$\oplus$I$_1$]}
      \UnaryInfC{$\emptyset \mid I \vdash \texttt{inl}\ x: A \oplus B$}
    \end{prooftree}
  \end{minipage}
  \begin{minipage}[h]{0.5\linewidth}
    \begin{prooftree}
      \AxiomC{$\emptyset \mid \texttt{inl}: B \sepimp A \oplus B \vdash y : B$}\RightLabel{[$\oplus$I$_2$]}
      \UnaryInfC{$\emptyset \mid I \vdash \texttt{inl}\ y: A \oplus B$}
    \end{prooftree}
  \end{minipage}
  \begin{minipage}[h]{1.0\linewidth}
    \begin{prooftree}
      \AxiomC{$\emptyset \mid \Gamma \vdash M : A \oplus B$}\RightLabel{[$\oplus$E]}
      \AxiomC{$\emptyset \mid \Gamma \circledast x:A \vdash N_1 : E$}
      \AxiomC{$\emptyset \mid \Gamma \circledast y:B \vdash N_2 : E$}
      \TrinaryInfC{$\emptyset \mid \Gamma \vdash\Case{M}{\{ \texttt{inl}\ x \mapsto N_1; \texttt{inl}\ y \mapsto N_2\}}:E$}
    \end{prooftree}
  \end{minipage}
\end{framed}

\subsubsection{Addivite sums}
If the two resources are in sharing, we call them additive sums.

\begin{landscape}
  % choice
  \noindent
  \begin{small}
  \begin{prooftree}
    \AxiomC{$$}\RightLabel{[ID]}
    \UnaryInfC{$\emptyset \mid  g^{cf}:(B \rightarrow E) \vdash g: (B \rightarrow E)$}

    \AxiomC{$$}\RightLabel{[ID]}
    \UnaryInfC{$\emptyset \mid f^{cg}: (A \rightarrow E) \vdash f: (A \rightarrow E)$}
    \AxiomC{$$}\RightLabel{[ID]}
    \UnaryInfC{$\emptyset \mid c^{fg}: ((A \rightarrow E) \rightarrow (B \rightarrow E) \rightarrow E)
      \vdash c:((A \rightarrow E) \rightarrow (B \rightarrow E) \rightarrow E)$}\RightLabel{[$\rightarrow$E]}
    \BinaryInfC{$\emptyset \mid c^{fg}:((A \rightarrow E) \rightarrow (B \rightarrow E) \rightarrow E)
      \varoplus f^{cg}:(A \rightarrow E) \vdash c f: (B \rightarrow E) \rightarrow E$}\RightLabel{[$\rightarrow$E]}

    \BinaryInfC{$\emptyset \mid  c^{fg}:((A \rightarrow E) \rightarrow (B \rightarrow E) \rightarrow E)  \varoplus
      f^{cg}:(A \rightarrow E)  \varoplus g^{cf}:(B \rightarrow E) \vdash c f g : E$}\RightLabel{[$\rightarrow$I]}
    \UnaryInfC{$\emptyset \mid  c^{f}:((A \rightarrow E) \rightarrow (B \rightarrow E) \rightarrow E) \varoplus
      f^{c}:(A \rightarrow E) \vdash \lambda^{\alpha} g. c f g :(B \rightarrow E) \rightarrow  E$}\RightLabel{[$\rightarrow$I]}
    \UnaryInfC{$\emptyset \mid  c^{\emptyset}:((A \rightarrow E) \rightarrow (B \rightarrow E) \rightarrow E)
      \vdash  \lambda^{\alpha} f. \lambda^{\alpha} g. c f g :(A \rightarrow E) \rightarrow (B \rightarrow E) \rightarrow E$}\RightLabel{[$\sepimp$I]}
    \UnaryInfC{$\emptyset \mid I \vdash \lambda^{*} c. \lambda^{\alpha} f. \lambda^{\alpha} g. c f g
      : ((A \rightarrow E) \rightarrow (B \rightarrow E) \rightarrow E) \sepimp (A \rightarrow E) \rightarrow (B \rightarrow E) \rightarrow E$}
  \end{prooftree}
\end{small}

\noindent
We now define additive sum type to be
\begin{framed}\centering
  $A \parr B = (A \rightarrow E) \rightarrow (B \rightarrow E) \rightarrow E$\\
  $\texttt{case}_{sh}\ {c}\ \texttt{of}\ {\{ f;g \}} = \lambda^{*}c. \lambda^{\alpha}f. \lambda^{\alpha}g. c f g$
\end{framed}

We now define left and right {\it injections} or constructors for the sum type.

% left
\begin{prooftree}
  \AxiomC{$$}\RightLabel{[ID]}
  \UnaryInfC{$\emptyset \mid x^{fg}: A  \vdash x : A$}

  \AxiomC{$$}\RightLabel{[ID]}
  \UnaryInfC{$\emptyset \mid f^{gx}: (A \rightarrow E) \vdash f: (A \rightarrow E)$}\RightLabel{[$\rightarrow$E]}

  \BinaryInfC{$\emptyset \mid x^{fg}: A \varoplus f^{xg}: (A \rightarrow E)  \vdash f x : E$}\RightLabel{[WKN-UN]}
  \UnaryInfC{$\emptyset \mid x^{fg}: A \varoplus f^{xg}: (A \rightarrow E) \varoplus g^{fx}: (B \rightarrow E) \vdash f x : E$}\RightLabel{[$\rightarrow$I]}
  \UnaryInfC{$\emptyset \mid x^{f}: A \varoplus f^{x}: (A \rightarrow E) \vdash \lambda^{\alpha} g. f x : (B \rightarrow E) \rightarrow E$}\RightLabel{[$\rightarrow$I]}
  \UnaryInfC{$\emptyset \mid  x^{\emptyset}: A \vdash  \lambda^{\alpha} f. \lambda^{\alpha} g. f x
    :(A \rightarrow E) \rightarrow (B \rightarrow E) \rightarrow E$}\RightLabel{[$\sepimp$I]}\RightLabel{[$\sepimp$I]}
  \UnaryInfC{$\emptyset \mid I \vdash \lambda^{*} x. \lambda^{\alpha} f. \lambda^{\alpha} g. f x
    : A \sepimp  (A \rightarrow E) \rightarrow (B \rightarrow E) \rightarrow E$}
\end{prooftree}

\noindent Left injection defined below as:
\begin{framed}\centering
  $\texttt{inl}_{sh} : A \sepimp A \parr B$\\
  $\texttt{inl}_{sh} = \lambda^{*} x. \lambda^{\alpha} f. \lambda^{\alpha} g. f x$
\end{framed}

% right
\noindent
\begin{prooftree}
  \AxiomC{$$}\RightLabel{[ID]}
  \UnaryInfC{$\emptyset \mid y^{fg}: B  \vdash y : B$}

  \AxiomC{$$}\RightLabel{[ID]}
  \UnaryInfC{$\emptyset \mid g^{yf}: (B \sepimp E) \vdash g: (B \sepimp E)$}\RightLabel{[$\rightarrow$I]}

  \BinaryInfC{$\emptyset \mid y^{fg}: B \varoplus g^{yf}: (B \rightarrow E) \vdash g y : E$}\RightLabel{[WKN-UN]}
  \UnaryInfC{$\emptyset \mid y^{fg}: B \varoplus f^{yg}: (A \rightarrow E) \varoplus g^{yf}: (B \rightarrow E) \vdash g y : E$}\RightLabel{[$\rightarrow$I]}
  \UnaryInfC{$\emptyset \mid y^{f}: B \varoplus f^{y}: (A \rightarrow E) \vdash \lambda^{\alpha} g. g y : (B \rightarrow E) \rightarrow E$}\RightLabel{[$\rightarrow$I]}
  \UnaryInfC{$\emptyset \mid  y^{\emptyset}: B \vdash \lambda^{\alpha} f. \lambda^{\alpha} g. g y
    :(A \rightarrow E) \rightarrow (B \rightarrow E) \rightarrow E$}\RightLabel{[$\sepimp$I]}\RightLabel{[$\sepimp$I]}
  \UnaryInfC{$\emptyset \mid I \vdash \lambda^{*} y. \lambda^{\alpha} f. \lambda^{\alpha} g. g y: B \sepimp  (A \rightarrow E) \rightarrow (B \rightarrow E) \rightarrow E$}
\end{prooftree}

\noindent Right injection defined below as:
\begin{framed}\centering
  $\texttt{inr}_{sh} : B \sepimp A \parr B$\\
  $\texttt{inr}_{sh} = \lambda^{*} y. \lambda^{\alpha} f. \lambda^{\alpha} g. g y$
\end{framed}
\end{landscape}

We can now derive additive sum types in our language using the auxilary definitions given above:
\begin{framed}
  \begin{minipage}[h]{0.5\linewidth}
    \begin{prooftree}
      \AxiomC{$\emptyset \mid \texttt{inl}_{sh}: A \sepimp A \parr B \vdash x : A$}\RightLabel{[$\parr$I$_1$]}
      \UnaryInfC{$\emptyset \mid I \vdash \texttt{inl}_{sh}\ x: A \parr B$}
    \end{prooftree}
  \end{minipage}
  \begin{minipage}[h]{0.5\linewidth}
    \begin{prooftree}
      \AxiomC{$\emptyset \mid \texttt{inr}_{sh}: B \sepimp A \parr B \vdash y : B$}\RightLabel{[$\parr$I$_2$]}
      \UnaryInfC{$\emptyset \mid I \vdash \texttt{inr}_{sh}\ y: A \parr B$}
    \end{prooftree}
  \end{minipage}
  \begin{minipage}[h]{1.0\linewidth}
    \begin{prooftree}
      \AxiomC{$\emptyset \mid \Gamma \vdash M : A \parr B$}\RightLabel{[$\parr$E]}
      \AxiomC{$\emptyset \mid \Gamma \varoplus x:A \vdash N_1 : E$}
      \AxiomC{$\emptyset \mid \Gamma \varoplus y:B \vdash N_2 : E$}
      \TrinaryInfC{$\emptyset \mid \Gamma \vdash \texttt{case}\ M\ \texttt{of} {\{ \texttt{inl}_{sh}\ x \mapsto N_1; \texttt{inr}_{sh}\ y \mapsto N_2\}}:E$}
    \end{prooftree}
  \end{minipage}
\end{framed}



\subsection{Proofs for Soundness and Completeness}

\begin{theorem}
  Soundness of $\mathcal{M}$. If $\mathcal{M}(S, X; \Gamma \vdash M : \tau) = P, S', \Sigma$ then $S' P | S' (\Gamma\mid_{\Sigma}) \vdash M : S' \tau$
\end{theorem}

\begin{proof}
  Proof by induction on structure of $M$
  \begin{itemize}
  \item Case 1. $x$
  \item Case 2. $\lambda^{*} x. M$
  \item Case 3. $\lambda ^{\alpha}x. M$
  \item Case 4. $M\ N$
  \item Case 5. $\texttt{let}\ x = M\ \texttt{in}\ N$
  \end{itemize}

\end{proof}

\begin{theorem}
  Completeness of $\mathcal{M}$.
\end{theorem}
\begin{proof}
  Proof by induction on derivation on $M$
\end{proof}

\begin{theorem}
  $\mathcal{M}$ computes principal types.
\end{theorem}
\begin{proof}

\end{proof}