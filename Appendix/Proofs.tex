\chapter{Proofs}\label{chp:proofs}
\begin{theorem}[Soundness of $\vdashs$]\label{thm:soundness-syntax-directed}
   If $P \mid \Gamma \vdashs M:\tau$ then $P \mid \Gamma \vdash M:\tau$
\end{theorem}
\begin{proof}\label{prf:soundness-syntax-directed}
  Proof by induction on the derivation of $P \mid \Gamma \vdashs M:\tau$.
  \begin{itemize}
  \item{Case [VAR$^s$].}
    We have a derivation of $P \mid x:\sigma \vdash x:\sigma$ by [VAR] rule.
    We proceed by repeated application of [$\forall$E] as $(Q => \tau) \sqsubseteq \sigma$, and
    we construct a derivation of $P \mid x:\sigma \vdash x: Q => \tau$. Then as $P => Q$ we can
    repeatedly apply [$=>$E] to construct a derivation of $P \mid x:\sigma \vdash x:\tau$.
    Finally, depending on whether the bindings are in sharing with or separate from
    $x$ we repeatedly apply [WKN-SH] or [WKN-UN] respectively for all the bindings in $\Gamma$ to construct
    a derivation of $P \mid \Gamma \sqcup \{x:\sigma\} \vdash x:\tau$.
  \item{Case [$\shimp$I$^s$].}
    By induction hypothesis we have a derivation of $P \mid \Gamma \varoplus x:\tau \vdash M:\tau'$.
    We apply [$\shimp$I] and reuse the derivations for $\ShFun{\phi}$ and $\Gamma \geq \phi$  to
    construct a derivation of $P \mid \Gamma \vdash \lambda^{\shimp}x. M: \phi \tau \tau'$.
  \item{Case [$\sepimp$I$^s$].}
    By induction hypothesis we have a derivation of $P \mid \Gamma \circledast x:\tau \vdash M:\tau'$.
    Similar to previous case, we apply [$\sepimp$I] and reuse the derivations for $\SeFun{\phi}$ and $\Gamma \geq \phi$  to
    construct a derivation of $P \mid \Gamma \vdash \lambda^{\sepimp}x. M: \phi \tau \tau'$.
  \item{Case [App$^s$].}
    By induction hypothesis we have derivations of $P \mid \Gamma \circledast \Delta \vdash M: \phi \upsilon \tau$ and
    $P \mid \Gamma' \circledast \Delta \vdash N: \upsilon$ we check for $\texttt{Used}(\Gamma) = \texttt{Used}(\Gamma')$ and if it is true
    and apply [$\shimp$E] or check for $\texttt{Used}(\Gamma) \mathbin{\#}\texttt{Shared}(\Gamma') \wedge \texttt{Shared}(\Gamma') \mathbin{\#}\texttt{Used}(\Gamma)$
    and if it is true we apply [$\sepimp$E] to reuse derivations of $\ShFun{\phi}$ or $\SeFun{\phi}$ respectively to construct the derivation
    of $P \mid (\Gamma \varoplus \Gamma') \circledast \Delta \vdash M N:\tau$ or $P \mid (\Gamma \circledast \Gamma') \circledast \Delta \vdash M N:\tau$.
  \item{Case [Let$^s$].}
    By induction hypothesis have a derivation of $P \mid \Gamma \circledast \Delta \vdash M: \tau$ and $P \mid \Gamma' \sqcup x:\tau \circledast \Delta \vdash N:\tau$
    Applying [$\forall$I] and [$=>$I] on the first hypothesis we derive $\emptyset \mid \Gamma \circledast \Delta \vdash M: \sigma$. Now by applying
    the [LET] rule and reusing $P \vdash \Delta\ \texttt{un}$ we construct the derivation of
    $P \mid \Gamma \sqcup \Gamma' \circledast \Delta \vdash \Let{x}{M}{N}:\tau$.\qedhere
  \end{itemize}
\end{proof}


\begin{theorem}[Completeness of $\vdashs$]\label{thm:completeness-syntax-directed}
  If $P \mid \Gamma \vdash M:\sigma$ then
  $\exists Q, \tau$ such that $Q \mid \Gamma \vdashs M:\tau$
  and $(P \mid \sigma) \sqsubseteq \texttt{Gen}(\Gamma, Q => \tau)$
\end{theorem}

We first go over few intermediate results and definitions that will help us prove the final result.

\begin{lemma}\label{lemma:WKN-UN-helper}
  If $P \mid \Gamma \vdash M:\tau$, $\forall_{y \in \texttt{dom}(\Delta)} y$ is not free in $M$ and $P \vdash \Delta\ \texttt{un}$
  then $P \mid \Gamma \circledast \Delta \vdash M:\tau$.
\end{lemma}
\begin{proof}
  By induction on derivation of $P \mid \Gamma \vdash M:\tau$,
  \begin{itemize}
  \item{Case [VAR].} $M$ is an expression $x$ and $y \notin \texttt{fvs}(x)$ also, $P \vdash \sigma\ \texttt{un}$ for $(y:\sigma)$, thus
    using [VAR] and [WKN-UN] we can obtain the required derivation.
  \item{Case others.} [LET], [$\shimp$I], [$\shimp$E], [$\sepimp$I], [$\sepimp$E] are all straightforward.\qedhere
  \end{itemize}
\end{proof}

\begin{lemma}\label{lemma:WKN-SH-helper}
  If $P \mid \Gamma \vdash M:\tau$, $\forall_{y \in \texttt{dom}(\Delta)} y$ is not free in $M$ then $P \mid \Gamma \varoplus \Delta \vdash M:\tau$.
\end{lemma}
\begin{proof}
  By induction on derivation of $P \mid \Gamma \vdash M:\tau$,
  \begin{itemize}
  \item{Case [VAR].} $M$ is an expression $x$ and $y \notin \texttt{fvs}(x)$ also, $\Gamma \varoplus \{y:\sigma\}$ thus, by using
    [WKN-SH] and [VAR] the required derivation is obtained.
  \item{Case others.}  [LET], [$\shimp$I], [$\shimp$E], [$\sepimp$I], [$\sepimp$E] are all straightforward.\qedhere
  \end{itemize}
\end{proof}

\begin{lemma}\label{lemma:impl-helper}
  If $P \mid \Gamma \vdashs M:\tau$ and $\sigma = \texttt{Gen}(\Gamma, P \Rightarrow \tau)$,
  then for any $P' \Rightarrow \tau' \sqsubseteq \sigma$, $P' \mid \Gamma \vdashs M:\tau' $
\end{lemma}
\begin{proof}
  Let $\sigma = \forall \vec{t}. Q => \upsilon$. By definition, there are some $\vec{u}$, such that
  $\tau' = [\vec{u}/\vec{t}]\upsilon$ and $P' \Rightarrow [\vec{u}/\vec{t}]Q$. Hence, we have
  $P' \mid [\vec{u}/\vec{t}]\Gamma \vdash M :\tau'$, and as $\vec{t}$ is free in $\Gamma$, we get the
  desired result, $P' \mid \Gamma \vdash M:\tau'$\qedhere
\end{proof}

\begin{defn}[$\Gamma \sqsubseteq \Gamma'$]
  If $\texttt{dom}(\Gamma) = \texttt{dom}(\Gamma')$ and $\forall_{x \in \texttt{dom}(\Gamma)} \Gamma(x) \sqsubseteq \Gamma'(x)$
  then we say $\Gamma \sqsubseteq \Gamma'$.
\end{defn}

\begin{lemma}\label{lemma:let-helper}
  If $P \mid \Gamma \vdashs M: \tau$, and $\Gamma \sqsubseteq \Gamma'$ then $P \mid \Gamma' \vdashs M:\tau$.
\end{lemma}
\begin{proof}
  By induction on derivation of $P \mid \Gamma \vdashs M: \tau$.
  \begin{itemize}
  \item{Case [Var$^s$].} If $(P \Rightarrow \tau) \sqsubseteq \sigma$ and $\sigma \sqsubseteq \sigma'$ then $(P \Rightarrow \tau) \sqsubseteq \sigma'$.
    Now by applying [$\forall$E] we get the required derivation.
  \item{Case others.} All other cases, [Let$^s$],[$\shimp$I$^s$],[$\sepimp$I$^s$], and [App$^s$] are trivial.
    Essentially use [$\forall$E] rule to get the correct instance of the type to get the required derivation.\qedhere
  \end{itemize}
\end{proof}

\begin{proof}[Proof of \cref{thm:completeness-syntax-directed}]
  By induction on derivation of $P \mid \Gamma \vdash M:\sigma$.
  \begin{itemize}
    % structural rules
  \item{Case [ID].}
    We have $(x:\sigma) \in \Gamma$, where $\sigma = (\forall \vec{t}. Q => \tau)$. Pick
    fresh type variables $\vec{u}$ so that $([\vec{u}/\vec{t}]Q => [\vec{u}/\vec{t}]\tau) \sqsubseteq \sigma$,
    thus $P, [\vec{u}/\vec{t}]Q \mid \Gamma \vdash x : [\vec{u}/\vec{t}]\tau$ using [VAR$^s$].
    \begin{flalign*}
      \sigma' &= \texttt{Gen}(\Gamma, (P, [\vec{u}/\vec{t}]Q => [\vec{u}/\vec{t}]\tau))\\
      &= \forall u. (P, [\vec{u}/\vec{t}]Q) => [\vec{u}/\vec{t}]\tau
    \end{flalign*}
    as $\vec{u}$ are fresh i.e. $\vec{u} \notin \texttt{dom}(\Gamma)$ and $(P \mid \sigma) \sqsubseteq \sigma'$.
  \item{Case [CTR-UN].}
    The newly duplicated type assignments are in $\Delta$ which is separate from $\Gamma$ and $P \vdash \Delta\ \texttt{un}$.
    The induction hypothesis gives the required derivation of $P \mid \Gamma \circledast \Delta \vdash M:\tau$ such that
    $(P \mid \sigma) \sqsubseteq (\emptyset \mid \texttt{Gen}(\Gamma \circledast \Delta, Q => \tau))$
  \item{Case [WKN-UN].}
    This follows directly from \cref{lemma:WKN-UN-helper} and the induction hypothesis.
  \item{Case [CTR-SH].}
    The newly duplicated type assignments are in $\Delta$ which is in sharing with $\Gamma$.
    The induction hypothesis gives the required derivation of $P \mid \Gamma \varoplus \Delta \vdash M:\sigma$ such that
    $(P \mid \sigma) \sqsubseteq (\emptyset \mid \texttt{Gen}(\Gamma \varoplus \Delta, Q => \tau))$
  \item{Case [WKN-SH].}
    This follows directly from \cref{lemma:WKN-SH-helper} and the induction hypothesis.

    % connective rules
  \item{Case [$\sepimp$I].}
    By induction hypothesis and \cref{lemma:impl-helper}, we have a derivation of $Q \mid \Gamma \vdashs M:\tau'$ also, $P \mid \Gamma \geq \phi$
    thus by [$\sepimp$I$^s$] we have the required derivation.
  \item{Case [$\sepimp$E].}
    By induction hypothesis and \cref{lemma:impl-helper}, we have a derivation of $Q \mid \Gamma \vdashs M:\phi \tau \tau'$ and $Q \mid \Gamma' \vdashs N: \tau$.
    We can partition the environment into separating contexts $\Gamma \circledast \Gamma'$ and apply [App$^s$] to obtain the required derivation.
  \item{Case [$\shimp$I].}
    By induction hypothesis and \cref{lemma:impl-helper}, we have a derivation of $Q \mid \Gamma \vdashs M:\upsilon$ also, $Q \mid \Gamma \geq \phi$
    thus by using [$\shimp$I$^s$] we obtain the required derivation.
  \item{Case [$\shimp$E].}
    By induction hypothesis and \cref{lemma:impl-helper}, we have a derivation of $Q \mid \Gamma \vdashs M:\phi \tau \tau'$ and $Q \mid \Gamma' \vdashs N: \tau$.
    We can partition the environment into sharing contexts $\Gamma \varoplus \Gamma'$ and apply [App$^s$] to get the required derivation.
  \item{Case [$=>$E].}
    By induction hypothesis we have $Q \mid \Gamma \vdashs M:\tau$ such that, letting $\sigma = \texttt{Gen}(\Gamma, Q \Rightarrow \tau)$,
    $(P, \pi \mid \rho) \sqsubseteq \sigma$. As $(P \mid \pi \Rightarrow \rho) \sqsubseteq (P, \pi \mid \rho)$, we also have
    $(P \mid \pi \Rightarrow \rho) \sqsubseteq \sigma$
  \item{Case [$=>$I].}
    By induction hypothesis we have $Q \mid \Gamma \vdashs M: \tau$ such that, letting $\sigma = \texttt{Gen}(\Gamma, Q \Rightarrow \tau)$,
    $(P \mid \pi \Rightarrow \rho) \sqsubseteq \sigma$. Since $P \Rightarrow \pi$, we have $(P \mid \rho) \sqsubseteq (P \mid \pi \Rightarrow \rho)$, and hence,
    $(P \mid \rho) \sqsubseteq \sigma$.
  \item{Case [$\forall$I].}
    By the induction hypothesis, we have $Q \mid \Gamma \vdashs M :\tau$ such that, letting $\sigma = \texttt{Gen}(\Gamma, Q \Rightarrow \tau)$,
    $(P \mid \pi \Rightarrow \rho) \sqsubseteq \sigma$
  \item{Case [$\forall$E].}
    By the induction hypothesis we have $Q \mid \Gamma \vdashs M: \tau$ such that, letting $\sigma = \texttt{Gen}(\Gamma, Q \Rightarrow \tau)$,
    $(P \mid \pi \Rightarrow \rho) \sqsubseteq \sigma$, As $(P \mid [\tau/t]\sigma) \sqsubseteq (P \mid \sigma)$,
    $(P \mid [\tau / t]\sigma) \sqsubseteq \sigma$ (because $\sigma = (\emptyset \mid \sigma)$.
  \item{Case [Let].}
    By induction hypothesis, we have $$Q \mid \Gamma \circledast \Delta \vdashs M: \upsilon$$ and
    $$Q' \mid \Gamma' \circledast \Delta \sqcup \{x:\sigma\} \vdashs N:\tau$$
    such that, letting $\sigma' = \texttt{Gen}(\Gamma, Q \Rightarrow \upsilon)$, $(P \mid \forall t. \sigma) \sqsubseteq \sigma'$. Thus we conclude that
    $\Gamma \sqcup \{ x:\forall t. \sigma \} \sqsubseteq   \Gamma \sqcup \{ x:\sigma'\}$. Now by applying \cref{lemma:let-helper}, the induction hypothesis, and \cref{lemma:impl-helper},
    we have a derivation of $Q' \mid \Gamma \sqcup \{ x:\sigma'\} \vdash N:\tau$. Finally applying [LET$^s$] we get the required derivation.
      \qedhere
  \end{itemize}
\end{proof}


\begin{theorem}[Soundness of $\mathcal{M}$.]\label{thm:soundness-m}
   If $\mathcal{M}(S, \Psi, \Gamma \vdash M : \tau) = P, S', \Sigma, \Psi'$ then $S' P \mid S' \Gamma \vdash M : S' \tau$
 \end{theorem}

We go over a few intermediate results and definitions to prove the soundness result.

\begin{lemma}\label{lemma:strengthen}
  If $P \mid \Delta \vdashs M:\tau$ and $Q \Rightarrow P$, then $Q \mid \Delta \vdashs M: \tau$
\end{lemma}

\begin{lemma}\label{lemma:syntax-WKN-UN-helper}
  If $P \mid \Gamma \vdashs M:\tau$, $\forall_{y \in \texttt{dom}(\Delta)} y$ is not free in $M$ and $P \vdashs \Delta\ \texttt{un}$
  then $P \mid \Gamma \circledast \Delta \vdash M:\tau$.
\end{lemma}
\begin{proof}
  We see that the syntax directed system is complete with respect to the original type system hence, due to \cref{lemma:WKN-UN-helper} and \cref{thm:completeness-syntax-directed}, we can prove the required result.
\end{proof}

\begin{lemma}\label{lemma:syntax-WKN-SH-helper}
  If $P \mid \Gamma \vdashs M:\tau$, $\forall_{y \in \texttt{dom}(\Delta)} y$ is not free in $M$ then $P \mid \Gamma \varoplus \Delta \vdashs M:\tau$.
\end{lemma}
\begin{proof}
  Proof is similar to \cref{lemma:syntax-WKN-UN-helper}. We see that the syntax directed system is complete with respect to the original type system hence,
  due to \cref{lemma:WKN-SH-helper} and \cref{thm:completeness-syntax-directed}, we can prove the required result.
\end{proof}

\begin{lemma}\label{lemma:syntax-closed-under-sub}
  If $P \mid \Gamma \vdashs M:\tau$ then $S P \mid S \Gamma \vdashs S \tau$.
\end{lemma}
\begin{proof}
  Proof is by a simple induction on derivation of $P \mid \Gamma \vdashs M:\tau$
\end{proof}

\begin{defn}[Simple Constraint]
  A simple constraint in $t$ is of the form $\Un{t}$, $\ShFun{t}$, $\SeFun{t}$ or $\tau \geq t$ where t is a type variable.
\end{defn}

\begin{lemma}[Constraint Simplification]
  Suppose there is a non-trivial entailment $P \Rightarrow Q$.  $$\exists Q' \text{\ such that\ } P \Rightarrow Q' \Rightarrow Q$$ and $Q'$ is simple and
  $Q'$ is called simplification of $Q$.
\end{lemma}
\begin{proof}
  Proof is by induction on derivation on $P \Rightarrow Q$.
\end{proof}

\begin{defn}[Non-trivial Entailment]
  The entailment $P \Rightarrow Q$ is non-trivial if only uses of the assumption rule are for simple constraints.
\end{defn}

\begin{defn}[Improving Substitution]
  Suppose $P$ is simple, and $X$ is a set of type variables.
  A substitution S is called improving substitution for $X$ in $P$ if:
  \begin{itemize}
  \item If $\Un{t} \in P \wedge \ShFun{t} \in P$ then $S t = \tightoverset{\scalebox{0.5}{!}}{\shimp}$
  \item If $\Un{t} \in P \wedge \SeFun{t} \in P$ then $S t = \tightoverset{\scalebox{0.5}{!}}{\sepimp}$
  \item If $\Un{t} \notin P \wedge \ShFun{t} \in P$ then $S t = \shimp$
  \item If $\Un{t} \notin P \wedge \SeFun{t} \in P$ then $S t = \sepimp$
  \end{itemize}
\end{defn}

\begin{lemma}\label{lemma:improving-sub}
  If $P \mid \Gamma \vdashs M:\tau$ and $X = \texttt{fvs}(P) \backslash (\texttt{fvs}(\Gamma) \cup \texttt{fvs}(\tau))$,
  $P' \Rightarrow P$ and $S$ is an improving substitution for $X \in P'$, then $S P \mid \Gamma \vdashs M :\tau$
\end{lemma}
\begin{proof}
  Proof by induction on derivation of $P \mid \Gamma \vdashs M:\tau$.
  The improving substitution will be bound to a type variable $\phi$ if the rules used are [$\shimp$I$^s$], [$\sepimp$I$^s$] or [APP$^s$], the
  substitution $S \phi$ is a suitable type for further use.
\end{proof}

\begin{proof}[Proof of \cref{thm:soundness-m}]
  Proof by induction on structure of $M$
  \begin{itemize}
  \item{Case $x$.}
    We have $$(x^{\vec{y}}:\forall \vec{t}. P \Rightarrow \tau) \in \Gamma$$ where $\vec{y} \subseteq \texttt{dom}(\Gamma)$ and $\Sigma = \{ x \}$. We see that
    $$([\vec{u}/\vec{t}]P \Rightarrow [\vec{u}/\vec{t}]\tau) \sqsubseteq (\forall \vec{t}. P \Rightarrow \tau)$$
    So, we can apply [VAR$^s$] to construct
    $[\vec{u}/\vec{t}]P \mid \{x:\forall \vec{t}. P \Rightarrow \tau \} \vdashs x: [\vec{u}/\vec{t}]\tau$

  \item{Case $\lambda^{\shimp} x. M$.}
    We have $$P; S'; \Sigma; \Psi' = \M(\Unf(\tau, u_1 u_2 u_3) \circ S, \Psi, \Gamma, x:u_2 \vdash M: u_3)$$
    and by induction hypothesis $$S' P \mid S' ((\Gamma \sqcup \{ x^{\texttt{dom}(\Gamma)} : u_2 \})|_{\Sigma}) \vdashs M : S' u_3$$
    Let $Q = \{\ShFun{u_1}\} \cup \texttt{Leq}(u_1, \Gamma|_\Sigma) \cup \texttt{Weaken}(x, u_2, \Sigma, \Psi) \cup P$
    and let $\Sigma' = \Sigma \backslash \{x\}$. By \cref{lemma:strengthen} and \cref{lemma:syntax-WKN-UN-helper} we can construct a
    derivation of $$S' Q \mid S'  ((\Gamma \sqcup \{x:u_2\})|_{\Sigma'}) \vdashs M: S' u_3$$
    Now, using [$\shimp$I$^s$] we get the required result.

  \item{Case $\lambda^{\sepimp} x. M$.}
    We have $$P; S'; \Sigma; \Psi' = \M(\Unf(\tau, u_1 u_2 u_3) \circ S, \Psi, \Gamma, x:u_2 \vdash M: u_3)$$
    and by induction hypothesis $$S' P \mid S' ((\Gamma \sqcup \{ x^\emptyset : u_2 \})|_{\Sigma}) \vdashs M : S' u_3$$
    Let $Q = \{\SeFun{u_1}\} \cup \texttt{Leq}(u_1, \Gamma|_\Sigma) \cup \texttt{Weaken}(x, u_2, \Sigma, \Psi') \cup P$
    and let $\Sigma' = \Sigma \backslash \{x\}$. By \cref{lemma:strengthen} and \cref{lemma:syntax-WKN-UN-helper} we can construct a
    derivation of $$S' Q \mid S'  ((\Gamma \sqcup \{x:u_2\})|_{\Sigma'}) \vdashs M: S' u_3$$
    Now, using [$\sepimp$I$^s$] we get the required result.

  \item{Case $M N$.}
    We have that
    \begin{flalign*}
      P; R; \Sigma; \Psi' &= \M(S, \Psi, \Gamma \vdash M:  u_1 u_2 \tau)\\
      P'; R'; \Sigma'; \Psi'' &= \M(S \circ R, \Psi', S \Gamma \vdash N: u_2)
    \end{flalign*}
    We have two sub-cases here depending on whether the application is shared or separating.
    If we can prove $\mathcal{C}(S \Gamma, \Sigma, \Psi) \mathbin{\#} \mathcal{C}(S \circ R \Gamma, \Sigma', \Psi')$ then we let $Q = P \cup P' \cup \SeFun{u_1}$
    or if we can prove $\mathcal{C}(S \Gamma, \Sigma, \Psi) = \mathcal{C}((S \circ R) \Gamma, \Sigma', \Psi')$ then we let $Q = P \cup P' \cup \ShFun{u_1}$
    (else the inference algorithm fails). Let $\Delta = \texttt{dom}(\Gamma)\backslash \mathcal{C}(\Gamma, \Sigma \cup \Sigma', \Psi \circ \Psi') $
    Now by \cref{lemma:strengthen} and \cref{lemma:syntax-closed-under-sub} and the induction hypothesis we get
    $$R' Q' \mid R' \Gamma \circledast R' \Delta \vdashs M: R' (u_1 u_2 \tau)$$
    $$R' Q' \mid R' \Gamma \circledast R' \Delta \vdashs N: u_2$$
    where $Q' = P \cup P' \cup Q \cup \{\Un{\Delta}\}$. We can now build the desired result using [APP$^s$].

  \item{Case $\Let{x}{M}{N}$.}
    We have
    $$P; R; \Sigma; \Psi' = \M(S, \Psi, \Gamma \vdash M:u_1)$$
    $$P'; R'; \Sigma'; \Psi'' = \M(R, \Psi', \Gamma, x:\sigma \vdash N : \tau)$$
    where $\sigma = \texttt{GenI}(R\Gamma; R(P \Rightarrow u_1))$. Let $T$ improve $\texttt{fvs}(P) \ (\texttt{fvs}(\Gamma) \cup \texttt{fvs}(R \tau))$ in $P$,
    then there is a partition of $\Gamma$ into $\Gamma_M$, $\Gamma_N$ and $\Delta$ such that, by \cref{lemma:improving-sub}, \cref{lemma:syntax-closed-under-sub} and
    induction hypothesis we have
    $$(T \circ R')P \mid R' (\Gamma_M \circledast \Delta) \vdashs M: R u_1)$$
    $$R' P \mid R' (\Gamma_N \circledast \Delta \sqcup \{ x: \sigma \}) \vdashs N : R' \tau)$$
    now, using $R' P \vdash \Delta\ \texttt{un}$ and [LET$^s$] we get the desired result.\qedhere
  \end{itemize}
\end{proof}

%%% Local Variables:
%%% mode: latex
%%% TeX-master: "../thesis-ku"
%%% End:
