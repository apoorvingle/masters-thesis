\chapter{Datatypes with resources}

The datatypes are user defined types:
There are 2 main kinds of non-recursive datatypes

1) data Choice a b = L a | R b

where the types a and b do not appear in both the constructors.
Choice here is a sum type a + b. How can we handle sum types in the current scenario?
1 way is to have a and b both not share resources. in this case we will have
a and b to be both un restricted.

a and b cannot direclty share resources. As the meaning of a sum type is that
either a or b can exist and not both. So we will always have separating arrow
for the sum types

Write the elemination rules for Either

for functions the closure has to be equal.

2) data Pair a b = P a b

The constructor uses both the resources a and b
Here we can have 2 subtypes:

a and b share resources or can be completely different.
We would want the programmer to specify whether the arguments are shared or separate.

For current purposes this syntactical specification is given via a modified syntax

data SeparatingPair a b = MkSeP a, b
data SharingPair a b = MkShP a; b

We can of-course mix and match the sharing or separation.

data MixTuple a b c d e = MkMixTuple a,b;c,d

The question now that needs to be answered is how should the sharing be interpreted?
There are various options.

1. interpreted right to left. i.e. MkMixTuple (a, (b; (c,d)))
   so the b is shared with separating pair (c,d) and a is separate than b

2. interpreted left to right. i.e. MkMixTuple (((a, b);c),d)
   so the c is shared with separating pair (a,b) and d is separate from c

3. Sharing ; has precedence over separating , i.e. MkMixTuple a,(b;c),d

4. Separating , has lower precedence over ; i.e. MkMixTuple (a,b);(c,d)

This should be handled during the AST generating while parsing.
The type checking algorithm should be robust enough
to perform type checking either of these 4 types of sharing as they are
all perfectly valid scenarios.

Recursive datatypes have some kind of structural similarities within themselves

data Tree a = Leaf a | Node a (Tree a) (Tree a)

data Tree a b = Leaf b | Node a (Tree a b) (Tree a b)

Open questions:
How do we specify sharing of data between nodes?
eg. How can we say that all the leaves have shared data? and the data in nodes is sparating


What happens when lesser unrestrictedness with datatypes? Choice a b >:= f


\section{Examples}

example datatypes.

Functor, Monad, Applicative

Pair NonEmpty List, List


%%% Local Variables:
%%% mode: latex
%%% TeX-master: "../thesis-ku.tex"
%%% End:
