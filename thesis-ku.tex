%% LyX 2.2.0 created this file.  For more info, see http://www.lyx.org/.
%% Do not edit unless you really know what you are doing.
\documentclass[12pt,english]{kuthesis}
\usepackage{mathptmx}
\usepackage{MnSymbol}
\renewcommand{\sfdefault}{lmss}
\renewcommand{\ttdefault}{lmtt}
\usepackage[T1]{fontenc}
\usepackage[utf8]{inputenc}
\usepackage{geometry}
\geometry{verbose,tmargin=1in,bmargin=1in,lmargin=1in,rmargin=1in}
\setcounter{secnumdepth}{3}
\setcounter{tocdepth}{3}

\usepackage{xcolor}
\usepackage{babel}
\usepackage{url}
\usepackage{graphicx}
\usepackage{setspace}
\usepackage{esint}
\usepackage[authoryear]{natbib}

\usepackage{bussproofs}
\usepackage{semantic}
\usepackage{framed}
\usepackage{minted}
\usepackage{cmll}
\usepackage{amsthm}
\usepackage{pdflscape}
\usepackage{stmaryrd}
\usepackage{xspace}

\usepackage{tikz}
\usepackage{tikz-qtree}
%\usetikzlibrary{arrows.meta}

\renewcommand\qedsymbol{$\blacksquare$}

\newtheorem{theorem}{Theorem}[chapter]

\theoremstyle{definition}
\newtheorem{defn}{Definition}

\doublespacing
\usepackage[unicode=true,
 bookmarks=true,bookmarksnumbered=false,bookmarksopen=false,
 breaklinks=true,pdfborder={0 0 0},pdfborderstyle={},backref=false,colorlinks=true]
 {hyperref}
\hypersetup{pdftitle={University of Kansas Thesis Template},
 pdfauthor={Apoorv Ingle},
 pdfsubject={A Thesis},
 urlcolor={blue},citecolor={blue},allcolors={blue}}
\usepackage[capitalise]{cleveref}
\crefformat{section}{\S#2#1#3} % see manual of cleveref, section 8.2.1
\crefformat{subsection}{\S#2#1#3}
\crefformat{subsubsection}{\S#2#1#3}

\makeatletter

%%%%%%%%%%%%%%%%%%%%%%%%%%%%%% LyX specific LaTeX commands.
\providecommand{\LyX}{\texorpdfstring%
  {L\kern-.1667em\lower.25em\hbox{Y}\kern-.125emX\@}
  {LyX}}
%% Because html converters don't know tabularnewline
\providecommand{\tabularnewline}{\\}
%%%%%%%%%%%%%%%%%%%%%%%%%%%%%% User specified LaTeX commands.
\newcommand{\BI}{\textbf{\em BI}\xspace}
\newcommand\sepimp{\mathrel{-\mkern-6mu*}}

\newcommand{\M}{\mathcal{M}}
\newcommand{\Unf}{\mathcal{U}}

\newcommand{\SeFun}[1]{\texttt{SeFun}\ #1}
\newcommand{\ShFun}[1]{\texttt{ShFun}\ #1}
\newcommand{\Un}[1]{\texttt{Un}\ #1}
\newcommand{\Let}[3]{\texttt{let}\ #1 = #2\ \texttt{in}\ #3}
\newcommand{\Case}[2]{\texttt{case}\ #1\ \texttt{of}\ #2}
\newcommand{\Pair}[1]{\langle #1 \rangle}
\newcommand{\TODO}[1]{{\begin{framed}{\color{red} #1}\end{framed}}}
\newcommand{\HaskellF}[1]{\mintinline{haskell}{#1}}
\newcommand{\qub}{QuB\xspace}

\newcommand{\vdashs}{\vdash^s}
% used to align decimals in tables according to APA style
\usepackage{dcolumn}
\usepackage{booktabs}

% Set the title and author info
\title{\qub{}: A Resource Aware Functional Language}
\author{Apoorv Ingle}


\dept{Electrical Engineering and Computer Science}
\degreetitle{Master of Science}
\papertype{Thesis} %or Thesis (Whatever you put will appear on p.2)
%% It is vital to have 7 entries, even if some are empty for committee and role
%% I mean, it is vital to leave the empty place holders
\committee{Dr. J. Garrett Morris}{Dr. Perry Alexander}{Dr. Andy Gill}{Dr. Prasad Kulkarni}{}{}{}
\role{Chairperson}{}{}{}{}{}{}
%AT Most 7 members allowed, last here is blank on purpose to demonstrate
%flexibility

%% The following is OPTIONAL. Remove all 3 of the next 3 lines
%% to leave dates blank. If dates are included, then both dates
%% must be included.
\@printd@testrue
\datedefended{July 21, 2018}
\dateapproved{July 21, 2018}

%% These settings are now in the kuthesis.cls file, but users are free
% to customize. listings has great documentation online
%% When listings are used, break lines
%\lstset{
 %    breaklines=true,  % sets automatic line breaking
 %    breakindent=2em,
 %    breakatwhitespace=true,  % sets if automatic breaks should
 %   breakautoindent=true
%}

\@ifundefined{showcaptionsetup}{}{%
 \PassOptionsToPackage{caption=false}{subfig}}
\usepackage{subfig}
\makeatother

\usepackage{listings}
\renewcommand{\lstlistingname}{\inputencoding{latin9}Listing}

\begin{document}
\begin{romanpages}

\maketitle
\begin{abstract}
  Modern programming languages treat resources as normal values.
  They have different semantics at runtime that does not match with its static
  semantics. For example, file handles or database connections are treated as integers
  in C. Modern functional programming languages like Haskell try to improve this
  by wrapping them in IO monads. While it is an improvement over imperative languages,
  well typed programs may have runtime errors. Type systems based on linear logic[\cite{girard_linear_1987}]
  promise a way to introduce the efficient resource handling in programs.
  %There have been numerous attempts to make use of linear logic in programming languages.
  %It has made its way into some parts of programming languages
  %such as garbage collectors and also type systems (Rust and Clean).
  While the these is more expressive than intuistionistic logic, there are various practical issues
  that prohibit leveraging it to its fullest degree in programming languages.
  In this thesis, we take a different view over the resource management problem by concentrating
  on sharing or separation of resources which been studied in a theoritical
  setting as the logic of bunched implications (\BI)[\cite{ohearn_logic_1999}].

  We design and implement \qub, a prototype language that is Curry-Howard interpretation of the logic
  based on \BI\ [\cite{ohearn_bunched_2003}]---$\alpha\lambda$-calculus---by introducing 2 new type
  constructors of arrow type: sharing arrow ($\rightarrow$) and separating arrow($\sepimp$) to replace
  the usual notion of function arrows of simply typed lambda calculus. Sharing arrows imply that functions
  share resources with its arguments, while separating arrows imply that functions do not share resources with its arguments.
  We base our work on Quill [\cite{morris_best_2016}] which express linear types in a functional language.
  We realize that the interaction of unrestricted types and sharing and separating is harmless as program entities that
  do not have a sense of resourcefulness can be duplicated or dropped anytime without
  any adverse implications and all existing programs typecheck.
  % haivng unrestricted types
  % are realizations of existing purely functional programs.

  % Complexity of software systems has been increasing forever. Automation of tasks
  % provides consistency, correctness and speed at a much greater degree than humans.
  % complex software systems find their use in various fields ranging from automobiles, aircrafts,
  % financial institutions and even medicine. Higher level programming languages
  % are used to program these software systems and developing them requires
  % adequate understanding of the programming language along with problem domain.
  % Structured programming techniques ensures maintanability and extensibility
  % of the software application. Types and functional programming style helps maintain
  % structure by making it possible to run algorithmic checks to ensure sanity.
  % I have no idea what I am talking about. TODO! get back on track

  %  What is quil?
  % Quill is a purely functional language that uses
  % qualified types and a varient of linear logic to keep track of resources.
  % We introduce a new type class to express unrestrictedness
  % of a type, define the degree of unrestrictedness of classes
  % and a concept of sharing or separation of closures for functions

  % What am I trying to doing here?
  % Make Quill work. Write programs in them and check and see
  % how painful it is.
  % We realize that we have 2 kinds of pairs, a sharing pair and a separating pair depending
  % on the arrow we use. We try to formalize other structures such as Lists, non-empty lists
  % datatypes (recursive and non-recursive) and type classes such as functor, applicative and
  % monads with specific instances of each.

\begin{acknowledgementslong}
%%if you want a "quote" environment for acknowledgements,
%% use acknowledgements instead of acknowledgementslong
  TODO TODO TODO DO DO DO DO.
% I would like to thank all of the little people who made this thesis
% possible. Sleepy, Dopey, Grumpy, you know who you are.

\end{acknowledgementslong}
\end{abstract}
\tableofcontents{}

\listoffigures

\listoftables

\end{romanpages}

\chapter{Background Work}
% TODO should change title

\section{Type Inference Algorithm}
Algorithm $\mathcal{W}$ [\cite{damas_principal_1982}] and its varient algorithm $\mathcal{M}$ [\cite{lee_proofs_1998}]
are the basis of most of the modern statically typed programming languages. Type inference
is decidable in the sense, type checking algorithm always completes with a success or failure.
The algorithms also gurantee a most general typing scheme for an expression in
the simply typed lambda calculus extended with a polymorphic let construct having a term language
\begin{framed}
  \begin{flalign*}
    \text{Expressions}\ \ \ M, N ::= x: \sigma \mid \lambda x: \tau. M \mid M N \mid \texttt{let}\ x\ \texttt{=}\ M\ \texttt{in}\ N \nonumber
  \end{flalign*}
\end{framed}
and a type language specified by
\begin{framed}
  \begin{flalign*}
    \text{Types}\ \ \  \tau    &::= \alpha \mid \iota \mid \tau \rightarrow \tau \nonumber \\
    \text{Typing Scheme}\ \ \  &::= \tau \mid \forall \alpha. \tau \nonumber
  \end{flalign*}
\end{framed}
where $\alpha$ is a type variable, $\iota$ are primitive types in the language, $\rightarrow$
is a type constructor and $\sigma$ is a typing scheme.

Robinson's unification algorithm [\cite{robinson_machine-oriented_1965}] plays a key role
in ensuring that types are well formed. Its purely syntactic approach in creating
substitutions to unify types keeps the complete process elegent.
The algorithm works in an interesting way where the types of all well-typed terms can be
inferred automatically and if types are specified, the same algorithm can be used
to match the expression term.

\TODO{
Talks about Curry-Howard here.
Logic rules and corresponding Typing rules.
Typing rules are nothing but logic rules with terms and types.
Types correspond to predicates
}
\begin{figure}[h]
  \begin{framed}
    % var
    \begin{minipage}{.5\textwidth}
      \begin{prooftree}
        \AxiomC{$x: \sigma \in \Gamma$} \RightLabel{$[VAR]$}
        \UnaryInfC{$\Gamma \vdash x : \sigma $}
      \end{prooftree}
    \end{minipage}
    % let
    \begin{minipage}{.5\textwidth}
      \begin{prooftree}
        \AxiomC{$\Gamma \vdash M : \sigma$}
        \AxiomC{$\Gamma_{x}, x: \sigma \vdash N: \tau$} \RightLabel{$[LET]$}
        \BinaryInfC{$\Gamma \vdash (\Let{x}{M}{N}) : \tau$}
      \end{prooftree}
    \end{minipage}
    % forall I
    \begin{minipage}{0.5\textwidth}
      \begin{prooftree}
        \AxiomC{$\Gamma \vdash M : \sigma$}\RightLabel{$[\forall I]$}
        \AxiomC{$t \notin \text{fvs}(\Gamma)$}
        \BinaryInfC{$\Gamma \vdash \lambda t. M : \forall t. \sigma$}
      \end{prooftree}
    \end{minipage}
    % forall E
    \begin{minipage}{0.5\textwidth}
      \begin{prooftree}
        \AxiomC{$\Gamma \vdash M : \forall t. \sigma$} \RightLabel{$[\forall E]$}
        \UnaryInfC{$\Gamma \vdash M \tau : [\tau \backslash t] \sigma$}
      \end{prooftree}
    \end{minipage}
    % -> I
    \begin{minipage}{0.5\textwidth}
      \begin{prooftree}
        \AxiomC{$\Gamma_{x}, x: \tau \vdash M : \tau'$} \RightLabel{$[\rightarrow I]$}
        \UnaryInfC{$\Gamma \vdash \lambda x. M : \tau \rightarrow \tau'$}
      \end{prooftree}
    \end{minipage}
    % -> E
    \begin{minipage}{0.5\textwidth}
      \begin{prooftree}
        \AxiomC{$\Gamma \vdash M : \tau \rightarrow \tau'$}
        \AxiomC{$\Gamma \vdash N : \tau$} \RightLabel{$[\rightarrow E]$}
        \BinaryInfC{$\Gamma \vdash M N : \tau'$}
      \end{prooftree}
    \end{minipage}
  \end{framed}
  \caption{Logic Rules for Simply Typed Lambda Calculus}
  \label{fig:stlc-logic}
\end{figure}
The logical rules for type inference are shown in \ref{fig:stlc-logic}. $\Gamma$ is the
context or assumptions in which the expression is typed. The $[VAR]$ rule is tautology or a simple
lookup of the term variable $x$ in the context $\Gamma$. The $[LET]$ allows creating local
definitions within an expression term. $[\rightarrow I]$ and $[\rightarrow E]$ are rules
for typing lambda terms and application respectively. We also include the rules for
type application and abstraction $[\forall I]$ and $[\forall E]$ to introduce second order
quantification in predicate logic.

% This simple type sytem is powerful in its
% expressivity and can encode a large variety of computations. The type checking algorithm
% asserts that undefined programs can be be detected statically i.e. without actually
% running the program or as famously known as ``well typed programs do not go wrong''.
% This is extremely useful for programmers who are building
% complex real world softwares. Bad programs can be eleminated instantaneously while
% being written using a mechanize technique so that the programmer can concentrate on designing the logic
% rather than fighting undefinedness of the programs. This creates an excellent feedback loop
% to the programmer while building large software systems. % TODO too generic should it be in introduction?
% TODO Give examples? and come up with a Curry-Howard interpretation of

\section{Qualified Types}
Jones [\cite{jones_theory_1994}] proposed incorporating predicates in the type language.
Predicates are used to build constraints on the domain of the type of a term in the language expression.
It introduces additional layer between polymorphic and monomorphic typing of programs.
A modification of Milner-Damas algorithm to encorporate predicates ensures that type inference
is sound and complete. The types that satisfy all the predicates are called qualified types for the term.
Qualified types are powerful enough to expresses type classes with functional dependencies,
record types and subtyping [\cite{mark_type_2000}]. The type language is modified to contain
qualified types. $P$ and $T$ range over finite set of predicates. We slightly modify the typing rules
from \cref{fig:stlc-logic} to add 2 new rules for qualified types as shown in \cref{fig:qualified-types-rules}
\begin{figure}[h]
  \centering
  \begin{framed}
  \begin{flalign*}
    \text{Types}\ \ \ \tau              &::= \alpha \mid \iota \mid \tau \rightarrow \tau \nonumber \\
    \text{Qualified Types}\ \ \ \rho    &::= P \Rightarrow \tau \nonumber \\
    \text{Type Scheme}\ \ \ \sigma      &::= \tau \mid \forall T. \rho \nonumber
  \end{flalign*}
\end{framed}
\caption{Qualified Types}
\label{fig:qualifed-types}
\end{figure}
\begin{figure}[h]
  \begin{framed}
    % => I
    \begin{minipage}{0.5\textwidth}
      \begin{prooftree}
        \AxiomC{$P, \pi \mid \Gamma \vdash M : \rho$} \RightLabel{$[=> I]$}
        \UnaryInfC{$P \mid \Gamma \vdash M : \pi \Rightarrow \rho$}
      \end{prooftree}
    \end{minipage}
    % => E
    \begin{minipage}{0.5\textwidth}
      \begin{prooftree}
        \AxiomC{$P \mid \Gamma \vdash M : \pi \Rightarrow \rho$}
        \AxiomC{$P \Rightarrow \pi$} \RightLabel{$[=> E]$}
        \BinaryInfC{$P \mid \Gamma \vdash M: \rho$}
      \end{prooftree}
    \end{minipage}
  \end{framed}
  \caption{Modified Typing Rules}
  \label{fig:qualified-types-rules}
\end{figure}
\section{Linear Logic}
% TODO: points to cover
% what is linearity
% restricting weakening and contraction
While classical logic deals with truth of propositions, linear logic deals with availability of resources.
Linear logic [\cite{girard_linear_1987}] promises to help cope with the resource and resource control problem.
It is refinement of classical intuistionistic logic. The core idea is that propositions
cannot be freely duplicated or discarded as in classical instuistionistic logic.
In formal terms, the contraction and weakening of logical rules are restricted.
This instigates a view of propositions to behave like resources. In real world software applications,
resources may not be freely copied or dropped from a program context.
Program entities like database connections, file handles or even
in memory shared state are pet peeves for programmers writing
industry grade software. Linear logic hopes to be a remedy for
these problems. If contraction and weakening is completely abandoned,
the system gets overly restrictive. Wadler describes a refinement of
linear logic based on Girard's Logic of Unity [\cite{wadler_taste_1993}, \cite{girard_unity_1993}].
It works around the problem of linear logic being too restrictive by allowing
instuistionistic rules in fragments. It can be considered as a disjoint union
of classical linear logic and intuistionistic logic. The grammar of classical intuistionistic logic is shown in \ref{fig:intu-logic-grammar}
where $A \rightarrow B$ implies implication, $A \times B$ is conjunction and $A \plus B$ is disjunction.
\begin{figure}
  \centering
  \begin{framed}
  \begin{flalign*}
    A, B, C ::= X \mid A \vdash B \mid A \rightarrow B \mid A \times B \mid A \plus B
  \end{flalign*}
\end{framed}
\caption{Grammar for Intuistionistic Logic}
\label{fig:intu-logic-grammar}
\end{figure}

In a pure linear logic setting, none of the assumptions can be used more than once (weakening prohibited) and they cannot be discarded
(contraction prohibited) This gives rise to a different flavor of all the logical connectives.
$A \rightspoon B$ describes the new implication meaning and is read as {\em``consume A to give B''} its logical rules
is given by $\rightspoon I$ and $\rightspoon E$. Similarly there are 2 kinds of connectives, multiplicative and additive that
arise in this logic system. More symbols are added inplace of $\plus$ and $\times$ to differentiate between the
multipicative conjunction and disjuntion ($\otimes$ and $\parr$), and additive conjuntion and disjunction ($\with $ and $\oplus$).
While working in intuistionistic linear logic, $\parr$ is dropped from the system as it can be encoded by other connectives.

\begin{figure}
  \centering
  \begin{framed}
    \begin{flalign*}
      A, B, C ::= X \mid \oc A \mid A \vdash B \mid A \rightspoon B \mid A \with B \mid A \otimes B
    \end{flalign*}
  \end{framed}
  \caption{Grammar for Intuistionistic Linear Logic}
\end{figure}

\begin{figure}[h]
  \begin{framed}
    % -o I
    \begin{minipage}{0.5\textwidth}
      \begin{prooftree}
        \AxiomC{$\Gamma, A \vdash B$} \RightLabel{$[\rightspoon I]$}
        \UnaryInfC{$\Gamma \vdash A \rightspoon B$}
      \end{prooftree}
    \end{minipage}
    % -o E
    \begin{minipage}{0.5\textwidth}
      \begin{prooftree}
        \AxiomC{$\Gamma \vdash  A \rightspoon B$}
        \AxiomC{$\Delta \vdash A$} \RightLabel{$[\rightspoon E]$}
        \BinaryInfC{$\Gamma, \Delta \vdash B$}
      \end{prooftree}
    \end{minipage}
  % & I
  \begin{minipage}{.3\textwidth}
    \begin{prooftree}
      \AxiomC{$\Gamma \vdash A$}
      \AxiomC{$\Gamma \vdash B$} \RightLabel{$[\with I]$}
      \BinaryInfC{$\Gamma \vdash A \with B$}
    \end{prooftree}
  \end{minipage}
  \begin{minipage}{.3\textwidth}
    \begin{prooftree}
      \AxiomC{$\Gamma \vdash A \with B$} \RightLabel{$[\with E_1]$}
      \UnaryInfC{$\Gamma \vdash A$}
    \end{prooftree}
  \end{minipage}
  \begin{minipage}{.3\textwidth}
    \begin{prooftree}
      \AxiomC{$\Gamma \vdash A \with B$} \RightLabel{$[\with E_2]$}
      \UnaryInfC{$\Gamma \vdash B$}
    \end{prooftree}
  \end{minipage}
  % otimes I
  \begin{minipage}{.3\textwidth}
    \begin{prooftree}
      \AxiomC{$\Gamma \vdash A$}
      \AxiomC{$\Delta \vdash B$} \RightLabel{$[\otimes I]$}
      \BinaryInfC{$\Gamma, \Delta \vdash A \otimes B$}
    \end{prooftree}
  \end{minipage}
  \begin{minipage}{.7\textwidth}
    \begin{prooftree}
      \AxiomC{$\Gamma \vdash A \otimes B$} \RightLabel{$[\otimes E]$}
      \AxiomC{$\Gamma, A, B \vdash C$}
      \BinaryInfC{$\Gamma \vdash C$}
    \end{prooftree}
  \end{minipage}

    % % par I
    % \begin{minipage}{0.5\textwidth}
    %   \begin{prooftree}
    %     \AxiomC{$\Gamma, A \vdash B$} \RightLabel{$[\parr I]$}
    %     \UnaryInfC{$\Gamma \vdash A \rightspoon B$}
    %   \end{prooftree}
    % \end{minipage}
    % % par E
    % \begin{minipage}{0.5\textwidth}
    %   \begin{prooftree}
    %     \AxiomC{$\Gamma \vdash  A \rightspoon B$}
    %     \AxiomC{$\Delta \vdash A$} \RightLabel{$[\parr E]$}
    %     \BinaryInfC{$\Gamma, \Delta \vdash B$}
    %   \end{prooftree}
    % \end{minipage}

    % oplus
    \begin{minipage}{.20\textwidth}
      \begin{prooftree}
        \AxiomC{$\Gamma \vdash A$} \RightLabel{$[\oplus I_1]$}
        \UnaryInfC{$\Gamma \vdash A \oplus B$}
      \end{prooftree}
    \end{minipage}
    \begin{minipage}{.20\textwidth}
      \begin{prooftree}
        \AxiomC{$\Delta \vdash B$} \RightLabel{$[\oplus I_2]$}
        \UnaryInfC{$\Delta \vdash A \oplus B$}
      \end{prooftree}
    \end{minipage}
    \begin{minipage}{0.6\textwidth}
      \begin{prooftree}
        \AxiomC{$\Gamma \vdash A \oplus B$}
        \AxiomC{$\Delta, A \vdash C$}
        \AxiomC{$\Delta, B \vdash C$}\RightLabel{$[\oplus E]$}
        \TrinaryInfC{$\Gamma, \Delta \vdash C$}
      \end{prooftree}
    \end{minipage}
  \end{framed}
  \caption{Intuionistic Linear Logic Rules}
  \label{fig:linear-logic-rules}
\end{figure}

To escape linearity, exponential $\oc$ is used, which signifies that an assumption can
be duplicted or dropped without restriction. $\oc A$ can be thought of as {\em``as many A's as needed''}.
Thus the intitionsistic $A \rightarrow B$ can be encoded in linear logic by $\oc A \rightspoon B$.
Similarly $A \plus B$ would be represented as $\oc A \otimes \oc B$ and $A \times B$ would be represented as $A \with B$. We clearly see that
this is a much powerful system in contrast to classical intuistionistic logic because of its enhanced expressivity.

\section{Bunched Implications and $\alpha\lambda$ Calculus}

In intuitionistic linear logic, the context is considered as a list or a set. In the theory of
bunched implications (\textbf{\em BI}), the context is treated as a tree in contrast to other logics. Contexts are syntactically
combined using 2 connectives comma ($,$) or a semicolon ($;$) and are called bunches. The logic of \textbf{\em BI}
tries to glue together intuistionistic linear logic with intuistionistic logic by
permitting contexts connected with semicolon to undergo contraction and weakening while the context connected with comma
are prohibited to undergo contraction and weakening. Comma and semicolon do not distributive over each other.
Thus $A,(B;C) \neq A, B ; A,C$ and $A;(B,C) \neq A;B,A;C$ where A B C are contexts.
There are two flavours of implication---additive and multiplicative---which is closely related to the idea of conjunction.
\begin{framed}
\begin{minipage}{1.0\linewidth}
  \begin{prooftree}
    \AxiomC{$\Gamma, A \vdash B$}
    \UnaryInfC{$\Gamma \vdash A \lozenge B$}
  \end{prooftree}
\end{minipage}
\end{framed}
In the logic of {\textbf{\em BI}} the question then faced is choosing what kind of
implication should be used inplace of $\lozenge$---the additive kind or the multiplicative kind.
O'Hearn and Pym [\cite{ohearn_logic_1999}] argue by introducing 2 kinds of arrows
and using them depending on the connectives used for the context. A multiplicative implication ($\sepimp$)
is used when the context is connected with a comma and an additive implication ($\rightarrow$) is used when the
context is connected using semicolon. This gives rise to 2 rules
\begin{framed}
\begin{minipage}{0.5\linewidth}
  \begin{prooftree}
    \AxiomC{$\Gamma, A \vdash B$} \RightLabel{$[\sepimp I]$}
    \UnaryInfC{$\Gamma \vdash A \sepimp B$}
  \end{prooftree}
\end{minipage}
\begin{minipage}{0.5\linewidth}
  \begin{prooftree}
    \AxiomC{$\Gamma; A \vdash B$} \RightLabel{$[\rightarrow I]$}
    \UnaryInfC{$\Gamma \vdash A \rightarrow B$}
  \end{prooftree}
\end{minipage}
\end{framed}

As we see in $[\sepimp I]$ $\Gamma, A$ cannot under go weakening or contraction to duplicate
or get rid of either $A$ or $\Gamma$. This hints to a notion that multiplicative implication ($\sepimp$)
exhibits property of the linear implication ($\rightspoon$). The linear implication cannot however
be directly converted to a multiplicative implication as the later does not exhibit properties of
counting the number of uses of its arguments. The logic of \textbf{\em BI} tries to combine the
additive logic i.e. intuistionistic logic with the multiplicative side i.e. intuistionistic linear logic seemlessly.
The multiplicative side can be used to model the behaviour of resources in the programming language
while the additive side would help the programmers fall back to the non-resource intuistionsitic parts. The logic of \textbf{\em BI}
argues that instead of looking at the number of times an argument is used within the function, it should
be viewed in terms of {\em sharing}.
$\alpha \lambda$ calculus[\cite{ohearn_resource_1999}] is interpretation of the logic of \textbf{\em BI}.
It introduces 2 kinds of arrows by modifiying the the syntax of lambda calculus:
\begin{enumerate}
  \item $\sepimp$     : Function do not share resources with their arguments
  \item $\rightarrow$ : Function may share resources with their arguments
\end{enumerate}

\begin{figure}[h]
\begin{framed}
  \begin{flalign*}
    \text{Types}\ \ \  \tau           &::= \alpha \mid \iota \mid \tau \rightarrow \tau \mid \tau \sepimp \tau \nonumber \\
    \text{Typing Scheme}\ \ \  \sigma &::= \tau \mid \forall \alpha. \tau \nonumber
  \end{flalign*}
\end{framed}
\caption{$\alpha\lambda$-Calculus Types}
\label{fig:al-cal-types}
\end{figure}

\begin{figure}[h]
\begin{framed}
  \begin{flalign*}
    \text{Expressions}\ \ \ M, N ::= x \mid \lambda^{\alpha} x. M \mid \lambda^{*} x. M \mid M N \mid \Let{x}{M}{N}\nonumber
  \end{flalign*}
\end{framed}
\caption{$\alpha\lambda$-Calculus Terms}
\label{fig:al-calc-terms}
\end{figure}

\begin{figure}[h]
  \begin{framed}
    % var
    \begin{minipage}{.5\textwidth}
      \begin{prooftree}
        \AxiomC{$x: \sigma \in \Gamma$} \RightLabel{$[VAR]$}
        \UnaryInfC{$\Gamma \vdash x : \sigma $}
      \end{prooftree}
    \end{minipage}
    % let
    \begin{minipage}{.5\textwidth}
      \begin{prooftree}
        \AxiomC{$\Gamma \vdash M : \sigma \ \ \ \ \
          \Gamma_{x}, x: \sigma \vdash N: \tau$} \RightLabel{$[LET]$}
        \UnaryInfC{$\Gamma \vdash (\texttt{let}\ x\ \texttt{=}\ M\ \texttt{in}\ N) : \tau$}
      \end{prooftree}
    \end{minipage}
    % forall I
    \begin{minipage}{0.5\textwidth}
      \begin{prooftree}
        \AxiomC{$\Gamma \vdash M : \sigma$}\RightLabel{$[\forall I]$}
        \AxiomC{$t \notin \text{fvs}(\Gamma)$}
        \BinaryInfC{$\Gamma \vdash \lambda t. M : \forall t. \sigma$}
      \end{prooftree}
    \end{minipage}
    % forall E
    \begin{minipage}{0.5\textwidth}
      \begin{prooftree}
        \AxiomC{$\Gamma \vdash M : \forall t. \sigma$} \RightLabel{$[\forall E]$}
        \UnaryInfC{$\Gamma \vdash M \tau : [\tau \backslash t] \sigma$}
      \end{prooftree}
    \end{minipage}
    % -> I
    \begin{minipage}{0.5\textwidth}
      \begin{prooftree}
        \AxiomC{$\Gamma_{x}, x: \tau \vdash M : \tau'$} \RightLabel{$[\sepimp I]$}
        \UnaryInfC{$\Gamma \vdash \lambda^{*} x. M : \tau \sepimp \tau'$}
      \end{prooftree}
    \end{minipage}
    % -> E
    \begin{minipage}{0.5\textwidth}
      \begin{prooftree}
        \AxiomC{$\Gamma \vdash M : \tau \sepimp \tau' \ \ \ \ \
          \Gamma \vdash N : \tau$} \RightLabel{$[\sepimp E]$}
        \UnaryInfC{$\Gamma \vdash M N : \tau'$}
      \end{prooftree}
    \end{minipage}
    % -o I
    \begin{minipage}{0.5\textwidth}
      \begin{prooftree}
        \AxiomC{$\Gamma_{x}; x: \tau \vdash M : \tau'$} \RightLabel{$[\rightarrow I]$}
        \UnaryInfC{$\Gamma \vdash \lambda^{\alpha} x. M : \tau \rightarrow \tau'$}
      \end{prooftree}
    \end{minipage}
    % -o E
    \begin{minipage}{0.5\textwidth}
      \begin{prooftree}
        \AxiomC{$\Gamma \vdash M : \tau \rightarrow \tau' \ \ \ \ \
          \Gamma \vdash N : \tau$} \RightLabel{$[\rightarrow E]$}
        \UnaryInfC{$\Gamma \vdash M N : \tau'$}
      \end{prooftree}
    \end{minipage}
  \end{framed}
  \caption{Typing Rules for $\alpha\lambda$ Calculus}
  \label{fig:bi-logic}
\end{figure}


% This is kind of a big jump here.
% probably shift qualified types after linear logic section to have better continuity
\section{Linear logic with Qualified Types: Quill}
We start our work based on Quill [\cite{morris_best_2016}]. It tries
to implement linear types using qualified types. The novelty of using a qualified
types is that it gives a complete and decidable type inference system. By using
a modified version of Algorithm M we can compute principal types of the terms.
In reality due to higher ordered kind system, we may not be able to deduce the
type of the terms but we can work around it by annotating some or all parts of
the terms. This is usually done at a top level function declaration. Specifying types
also serves as some kind of documentation for the programmers.
The key idea of Morris is to introduce 2 types of predicates into the language: \texttt{Un} and \texttt{Fun}.
\texttt{Un $\tau$} implies that the type $\tau$ is unrestricted, meaning it does not
contain any resources or the resources that it captures can be easily duplicated and dropped.
The \texttt{Fun $\tau$} implies that the type $\tau$ is of a function type. The function
depending on its use, may or may not capture resources in its closure. Thus the functions
themselves can be of restricted or unrestricted type. In traditional sense of typeclasses
in haskell, we can think of the \texttt{Un} to be a typeclass with methods supporting the operation
of duplication and dropping.
\begin{figure}[h]
  \centering
  \begin{framed}
    \begin{minted}[escapeinside=||,mathescape=true]{haskell}
      class Un where
          dup  :: t |$\overset{!}{\rightarrow}$| (t |$\otimes$| t)
          drop :: t |$\overset{!}{\rightarrow}$| 1
    \end{minted}
  \end{framed}
  \caption{\texttt{Un} as a Typeclass}
  \label{fig:un-typeclass}
\end{figure}

Simple types such as integers and booleans are all of unrestricted type as
they can be duplicated or dropped freely.
While program resources such as file handles, database connections
should be treated as restricted type as we cannot freely duplicate it
or drop it. Although there would be certain portions of the program where we would
like to close the file handle or free the memory space allocated on the heap.
(This is where we expected the bunched implications would play a crucial role. I guess.)

Combining linear logic with qualified types in Quill
%%% Local Variables:
%%% mode: latex
%%% TeX-master: "../thesis-ku"
%%% End:
       % Extended abstract
\chapter{Background Work}
\section{Hindley-Milner Type System and Inference Algorithm}\label{hm-system}
Hindley-Milner (\textbf{HM}) type system \citeyearpar{milner_theory_1978} for lambda calculus with parametric polymorphism or restricted version
of System F \citep{girard_proofs_1989} forms the basis of ML like languages and also many modern functional programming languages.
The type language contains type variables, primitive types
(such as integers, floats), the type constructor ($\rightarrow$) which constructs function types, and type scheme ($\sigma$) as shown in \cref{fig:hm-lang}.
The expression language contains variables, lambda expressions
and applications extended with a polymorphic \texttt{let} construct. Type inference algorithm $\mathcal{W}$ \citep{damas_principal_1982} and its varient
type checking algorithm $\mathcal{M}$ \citep{lee_proofs_1998} is decidable in the sense, the algorithm always completes with a success or failure.
The algorithms also guarantee a most general typing scheme or principal types for an expression.

\begin{figure}[h]
\begin{framed}
  \begin{flalign*}
                              t, u, v &\in \text{Type Variables}\\
    \text{Types}\ \ \  \tau           &::= \iota \mid \tau \rightarrow \tau \\
    \text{Typing Scheme}\ \ \  \sigma &::= \tau \mid \forall t. \tau\\
         \text{Expressions}\ \ \ M, N &::= x: \sigma \mid \lambda x. M \mid M N \mid \Let{x}{M}{N}
  \end{flalign*}
\end{framed}
\caption{Hindley-Milner Type and Expression Language}
\label{fig:hm-lang}
\end{figure}

Robinson's \citeyearpar{robinson_machine-oriented_1965} unification algorithm plays a key role
in computation of well-formed types. Its purely syntactic approach in creating
substitutions to unify types keeps the complete process elegant.
Algorithm $\mathcal{M}$ works in an interesting way where the types of all well-typed terms can be
inferred automatically and if types are specified, the same algorithm can be used
to verify that the specified type of expression term matches the actual type. The same
can be obtained using algorithm $\mathcal{W}$ but it would require an additional machinery of
computing equality of types.

\begin{figure}[h]
  \begin{framed}
    % var
    \begin{minipage}{.5\textwidth}
      \begin{prooftree}
        \AxiomC{$x: \sigma \in \Gamma$} \RightLabel{[VAR]}
        \UnaryInfC{$\Gamma \vdash x : \sigma $}
      \end{prooftree}
    \end{minipage}
    % let
    \begin{minipage}{.5\textwidth}
      \begin{prooftree}
        \AxiomC{$\Gamma \vdash M : \sigma$}
        \AxiomC{$\Gamma_{x}, x: \sigma \vdash N: \tau$} \RightLabel{[LET]}
        \BinaryInfC{$\Gamma \vdash (\Let{x}{M}{N}) : \tau$}
      \end{prooftree}
    \end{minipage}
    % forall I
    \begin{minipage}{0.5\textwidth}
      \begin{prooftree}
        \AxiomC{$\Gamma \vdash M : \sigma$}\RightLabel{[$\forall$ I]}
        \AxiomC{$t \notin \text{fvs}(\Gamma)$}
        \BinaryInfC{$\Gamma \vdash M : \forall t. \sigma$}
      \end{prooftree}
    \end{minipage}
    % forall E
    \begin{minipage}{0.5\textwidth}
      \begin{prooftree}
        \AxiomC{$\Gamma \vdash M : \sigma$}
        \AxiomC{$(\sigma' \sqsubset \sigma)$}\RightLabel{[$\forall$ E]}
        \BinaryInfC{$\Gamma \vdash M : \sigma'$}
      \end{prooftree}
    \end{minipage}
    % -> I
    \begin{minipage}{0.5\textwidth}
      \begin{prooftree}
        \AxiomC{$\Gamma_{x}, x: \tau \vdash M : \tau'$} \RightLabel{[$\rightarrow$ I]}
        \UnaryInfC{$\Gamma \vdash \lambda x. M : \tau \rightarrow \tau'$}
      \end{prooftree}
    \end{minipage}
    % -> E
    \begin{minipage}{0.5\textwidth}
      \begin{prooftree}
        \AxiomC{$\Gamma \vdash M : \tau \rightarrow \tau'$}
        \AxiomC{$\Gamma \vdash N : \tau$} \RightLabel{[$\rightarrow$ E]}
        \BinaryInfC{$\Gamma \vdash M N : \tau'$}
      \end{prooftree}
    \end{minipage}
  \end{framed}
  \caption{Typing Rules for \textbf{HM} Type System}
  \label{fig:stlc-logic}
\end{figure}
The rules for type inference for \textbf{HM} type system are shown in \cref{fig:stlc-logic}. $\Gamma$ is the
collection of assumptions or context in which the expression is typed. The $[VAR]$ rule is tautology or a simple
lookup of the term variable $x$ in the context $\Gamma$. The [LET] allows creating local
definitions within an expression term. [$\rightarrow$ I] and [$\rightarrow$ E] are rules
for typing lambda terms and application respectively. We also include the rules for
type abstraction and application [$\forall$I] and [$\forall$ E] i.e. parametric polymorphism.


\begin{figure}[h]\centering
  \begin{framed}
    % ID
    \begin{minipage}{0.20\linewidth}
      \begin{prooftree}
        \AxiomC{${\color{white}\Phi\Gamma\Psi}$}\RightLabel{[Ax]}
        \UnaryInfC{$\Phi \vdash \Phi $}
      \end{prooftree}
    \end{minipage}
    \begin{minipage}{0.33\linewidth}
      \begin{prooftree}
        \AxiomC{$\Gamma \vdash \Phi$}\RightLabel{[WKN]}
        \UnaryInfC{$\Gamma, \Gamma \vdash \Phi$}
      \end{prooftree}
    \end{minipage}
    \begin{minipage}{0.33\linewidth}
      \begin{prooftree}
        \AxiomC{$\Gamma, \Gamma \vdash \Phi $}\RightLabel{[CTR]}
        \UnaryInfC{$\Gamma \vdash \Phi $}
      \end{prooftree}
    \end{minipage}
    % \forall I
    \begin{minipage}{0.33\linewidth}
      \begin{prooftree}
        \AxiomC{$\Psi$}
        \AxiomC{$x \notin \Psi$}\RightLabel{[$\forall$I]}
        \BinaryInfC{$\forall x. \Psi$}
      \end{prooftree}
    \end{minipage}
    % \forall E
    \begin{minipage}{0.33\linewidth}
      \begin{prooftree}
        \AxiomC{$\Gamma$}
        \AxiomC{$\forall x. \Psi$}\RightLabel{[$\forall$E]}
        \BinaryInfC{$\Psi[x/\Gamma]$}
      \end{prooftree}
    \end{minipage}
    % -> I
    \begin{minipage}{0.33\linewidth}
      \begin{prooftree}
        \AxiomC{$\Gamma \vdash \Phi$}
        \AxiomC{$\Phi \vdash \Psi $}\RightLabel{[$\Rightarrow$I]}
        \BinaryInfC{$\Gamma \vdash \Phi \rightarrow \Psi$}
      \end{prooftree}
    \end{minipage}
    % -> E
    \begin{minipage}{0.33\linewidth}
      \begin{prooftree}
        \AxiomC{$\Gamma \vdash \Phi \rightarrow \Psi$}
        \AxiomC{$\Gamma \vdash \Phi$}\RightLabel{[$\Rightarrow$E]}
        \BinaryInfC{$\Gamma \vdash \Psi $}
      \end{prooftree}
    \end{minipage}
  \end{framed}
  \caption{Logic rules: Implication fragment of second order Intuitionistic Propositional Logic}
  \label{fig:intuitionistic-logic}
\end{figure}

The Curry-Howard correspondence the \textbf{HM} type system correspondes to the implication fragment of intuionistic propositional logic
the rules of which are given in \cref{fig:intuitionistic-logic} in Gentzen style natural deduction
where $\Gamma$, $\Phi$ and $\Psi$ are propositions. The [Ax] rule corresponds to [ID] rule while [$\rightarrow$I] and [$\rightarrow$E] correspond
to [$\Rightarrow$I] and [$\Rightarrow$E] respectively. The structural rules of weakening [WKN] and contraction [CTR] are implicitly obeyed
by \textbf{HM} type-system. The weakening rule states that we can duplicate assumptions to derivations without
affecting the proofs while contraction states that we can remove unrelated assumptions in our derivations and the
proof will still hold.

% This simple type system is powerful in its
% expressiveness and can encode a large variety of computations. The type checking algorithm
% asserts that undefined programs can be be detected statically i.e. without actually
% running the program or as famously known as ``well typed programs do not go wrong''.
% This is extremely useful for programmers who are building
% complex real world softwares. Bad programs can be eliminated instantaneously while
% being written using a mechanize technique so that the programmer can concentrate on designing the logic
% rather than fighting undefinedness of the programs. This creates an excellent feedback loop
% to the programmer while building large software systems.

\section{Linear Logic}\label{sec:linear-logic}
% what is linearity
% restricting weakening and contraction
While propositional logic deals with truth of propositions and their connectives, linear logic deals with availability of resources.
Linear logic \citep{girard_linear_1987} promises to help cope with the resource and resource control problem.
It is refinement of classical intuitionistic logic. The core idea is that propositions
cannot be freely duplicated or discarded as allowed in intuitionistic logic.
In formal terms, the contraction and weakening of logical rules are restricted.
This instigates a view of propositions to behave like resources. In real world software applications,
resources may not be freely copied or dropped from a program context.
Program entities like database connections, file handles or even
in memory shared state are pet peeves for programmers writing
industry grade software. Linear logic hopes to be a remedy for
these problems. If contraction and weakening is completely abandoned,
the system gets overly restrictive. Wadler describes a refinement of
linear logic based on Girard's Logic of Unity \citep{wadler_taste_1993, girard_unity_1993}.
It works around the problem of linear logic being too restrictive by allowing
intuitionistic rules in fragments. It can be considered as a disjoint union
of classical linear logic and intuitionistic logic. The grammar of intuitionistic logic is shown in \ref{fig:intu-logic-grammar}
where $A \rightarrow B$ denotes implication, $A \times B$ denotes conjunction and $A \plus B$ denotes disjunction.
\begin{figure}
  \centering
  \begin{framed}
  \begin{flalign*}
    A, B, C ::= X \mid A \vdash B \mid A \rightarrow B \mid A \times B \mid A \plus B
  \end{flalign*}
\end{framed}
\caption{Grammar for Intuitionistic Logic}
\label{fig:intu-logic-grammar}
\end{figure}

In a pure linear logic setting, none of the assumptions can be used more than once (weakening is prohibited)
and they cannot be discarded as well (contraction prohibited). This gives rise to two different flavor of all the logical connectives.
$A \rightspoon B$ describes the new implication meaning and is read as {\em``consume A to give B''} its logical rules
is given by $\rightspoon I$ and $\rightspoon E$. Similarly there are 2 kinds of connectives, multiplicative and additive that
arise in this logic system. More symbols are added in place of $\plus$ and $\times$ to differentiate between the
multiplicative conjunction and disjunction ($\otimes$ and $\parr$), and additive conjunction and disjunction ($\with $ and $\oplus$).
While working in intuitionistic linear logic, $\parr$ is dropped from the system as it can be encoded by other connectives.

\begin{figure}
  \centering
  \begin{framed}
    \begin{flalign*}
      A, B, C ::= X \mid \oc A \mid A \vdash B \mid A \rightspoon B \mid A \with B \mid A \otimes B
    \end{flalign*}
  \end{framed}
  \caption{Grammar for Intuitionistic Linear Logic}
\end{figure}

\begin{figure}[h]
  \begin{framed}
    % -o I
    \begin{minipage}{0.5\textwidth}
      \begin{prooftree}
        \AxiomC{$\Gamma, A \vdash B$} \RightLabel{$[\rightspoon I]$}
        \UnaryInfC{$\Gamma \vdash A \rightspoon B$}
      \end{prooftree}
    \end{minipage}
    % -o E
    \begin{minipage}{0.5\textwidth}
      \begin{prooftree}
        \AxiomC{$\Gamma \vdash  A \rightspoon B$}
        \AxiomC{$\Delta \vdash A$} \RightLabel{$[\rightspoon E]$}
        \BinaryInfC{$\Gamma, \Delta \vdash B$}
      \end{prooftree}
    \end{minipage}
  % & I
  \begin{minipage}{.3\textwidth}
    \begin{prooftree}
      \AxiomC{$\Gamma \vdash A$}
      \AxiomC{$\Gamma \vdash B$} \RightLabel{$[\with I]$}
      \BinaryInfC{$\Gamma \vdash A \with B$}
    \end{prooftree}
  \end{minipage}
  \begin{minipage}{.3\textwidth}
    \begin{prooftree}
      \AxiomC{$\Gamma \vdash A \with B$} \RightLabel{$[\with E_1]$}
      \UnaryInfC{$\Gamma \vdash A$}
    \end{prooftree}
  \end{minipage}
  \begin{minipage}{.3\textwidth}
    \begin{prooftree}
      \AxiomC{$\Gamma \vdash A \with B$} \RightLabel{$[\with E_2]$}
      \UnaryInfC{$\Gamma \vdash B$}
    \end{prooftree}
  \end{minipage}
  % otimes I
  \begin{minipage}{.3\textwidth}
    \begin{prooftree}
      \AxiomC{$\Gamma \vdash A$}
      \AxiomC{$\Delta \vdash B$} \RightLabel{$[\otimes I]$}
      \BinaryInfC{$\Gamma, \Delta \vdash A \otimes B$}
    \end{prooftree}
  \end{minipage}
  \begin{minipage}{.7\textwidth}
    \begin{prooftree}
      \AxiomC{$\Gamma \vdash A \otimes B$} \RightLabel{$[\otimes E]$}
      \AxiomC{$\Gamma, A, B \vdash C$}
      \BinaryInfC{$\Gamma \vdash C$}
    \end{prooftree}
  \end{minipage}

    % % par I
    % \begin{minipage}{0.5\textwidth}
    %   \begin{prooftree}
    %     \AxiomC{$\Gamma, A \vdash B$} \RightLabel{$[\parr I]$}
    %     \UnaryInfC{$\Gamma \vdash A \rightspoon B$}
    %   \end{prooftree}
    % \end{minipage}
    % % par E
    % \begin{minipage}{0.5\textwidth}
    %   \begin{prooftree}
    %     \AxiomC{$\Gamma \vdash  A \rightspoon B$}
    %     \AxiomC{$\Delta \vdash A$} \RightLabel{$[\parr E]$}
    %     \BinaryInfC{$\Gamma, \Delta \vdash B$}
    %   \end{prooftree}
    % \end{minipage}

    % oplus
    \begin{minipage}{.20\textwidth}
      \begin{prooftree}
        \AxiomC{$\Gamma \vdash A$} \RightLabel{$[\oplus I_1]$}
        \UnaryInfC{$\Gamma \vdash A \oplus B$}
      \end{prooftree}
    \end{minipage}
    \begin{minipage}{.20\textwidth}
      \begin{prooftree}
        \AxiomC{$\Delta \vdash B$} \RightLabel{$[\oplus I_2]$}
        \UnaryInfC{$\Delta \vdash A \oplus B$}
      \end{prooftree}
    \end{minipage}
    \begin{minipage}{0.6\textwidth}
      \begin{prooftree}
        \AxiomC{$\Gamma \vdash A \oplus B$}
        \AxiomC{$\Delta, A \vdash C$}
        \AxiomC{$\Delta, B \vdash C$}\RightLabel{$[\oplus E]$}
        \TrinaryInfC{$\Gamma, \Delta \vdash C$}
      \end{prooftree}
    \end{minipage}
  \end{framed}
  \caption{Intuitionistic Linear Logic Rules}
  \label{fig:linear-logic-rules}
\end{figure}

To escape linearity, exponential modality $\oc$ is used, which signifies that an assumption can
be duplicated or dropped without restriction. $\oc A$ can be thought of as {\em``as many A's as needed''}.
Thus the intuitionistic $A \rightarrow B$ can be encoded in linear logic by $\oc A \rightspoon B$.
Similarly $A \plus B$ would be represented as $\oc A \otimes \oc B$ and $A \times B$ would be represented as $A \with B$.
We clearly see that this is a much powerful system in contrast to intuitionistic logic because of its enhanced expressivity.
But there is an awkward asymmetry between the multiplicative and additive constructs. While $\rightspoon$ is a right adjoint
for multiplicative $\times$, there is no such right adjoint counter part for additive $\with$.
{\color{red}There have been quite a few prototypes of languages using linear logic. L3 etc}

\section{Qualified Types}\label{sec:qualified-types}
Jones \citeyearpar{jones_theory_1994} proposed incorporating predicates in the type language.
Predicates are used to build constraints on the domain of the type of a term in the language expression.
It introduces additional layer between polymorphic and monomorphic typing of programs.
A modification of Milner-Damas algorithm to incorporate predicates ensures that type inference
is sound and complete. The types that satisfy all the predicates are called qualified types for the term.
Qualified types are powerful enough to expresses type classes with functional dependencies,
record types and sub-typing \citep{mark_type_2000}. The type language is modified to contain
qualified types. $P$ and $Q$ range over finite set of predicates. We slightly modify the typing rules
from \cref{fig:stlc-logic} to add 2 new rules for qualified types as shown in \cref{fig:qualified-types-rules}
\begin{figure}[h]
  \centering
  \begin{framed}
  \begin{flalign*}
                               t, u, v  &\in \text{Type Variables}\nonumber\\
    \text{Types}\ \ \ \tau              &::= t \mid \iota \mid \tau \rightarrow \tau \nonumber \\
    \text{Qualified Types}\ \ \ \rho    &::= P \Rightarrow \tau \nonumber \\
    \text{Type Scheme}\ \ \ \sigma      &::= \tau \mid \forall T. \rho \nonumber
  \end{flalign*}
\end{framed}
\caption{Qualified Types}
\label{fig:qualifed-types}
\end{figure}

\begin{figure}[h]
  \begin{framed}
    % => I
    \begin{minipage}{0.5\textwidth}
      \begin{prooftree}
        \AxiomC{$P, \pi \mid \Gamma \vdash M : \rho$} \RightLabel{$[=> I]$}
        \UnaryInfC{$P \mid \Gamma \vdash M : \pi \Rightarrow \rho$}
      \end{prooftree}
    \end{minipage}
    % => E
    \begin{minipage}{0.5\textwidth}
      \begin{prooftree}
        \AxiomC{$P \mid \Gamma \vdash M : \pi \Rightarrow \rho$}
        \AxiomC{$P \Rightarrow \pi$} \RightLabel{$[=> E]$}
        \BinaryInfC{$P \mid \Gamma \vdash M: \rho$}
      \end{prooftree}
    \end{minipage}
  \end{framed}
  \caption{Typing Rules for Qualified Types}
  \label{fig:qualified-types-rules}
\end{figure}

\section{Linear logic with Qualified Types: Quill}\label{sec:quill}
Quill \citep{morris_best_2016} implements a sound and complete linear type system using qualified types.
It uses a modified version of Algorithm $\M$ to compute principal types of the terms.
In practice, due to higher ordered kind system, it may not be possible to deduce the
type of the terms but a work around is to annotate some or all parts of
the terms. Annotation is usually done at a top level function declaration. Specifying types
also serves as some kind of documentation for the programmers.
The key idea of Morris is to introduce two predicates for types into the language: \texttt{Un} and \texttt{Fun}.
\texttt{Un $\tau$} implies that the type $\tau$ is unrestricted which means it does not
contain any resources or the resources that it captures can be easily duplicated and dropped.
In traditional sense of type classes in Haskell, we can think of the \texttt{Un} to be a type-class with methods supporting the operation
of duplication and dropping shown in \cref{fig:un-typeclass}. In a proof theoritic setting, it would mean
that it admits to weakening and contraction.
The \texttt{Fun $\tau$} implies that the type $\tau$ is of a function type. The function
depending on its use, may or may not capture resources in its closure and the functions
themselves can be of restricted or unrestricted type.
\begin{figure}[h]
  \centering
  \begin{framed}\centering
    \begin{minted}[escapeinside=||,mathescape=true]{haskell}
      class Un where
          dup  :: t |$\overset{!}{\rightarrow}$| (t |$\otimes$| t)
          drop :: t |$\overset{!}{\rightarrow}$| 1
    \end{minted}
  \end{framed}
  \caption{\texttt{Un} as a Typeclass}
  \label{fig:un-typeclass}
\end{figure}

Simple types such as integers and booleans are all of unrestricted type as
they can be duplicated or dropped freely. While program resources such as file handles, database connections
should be treated as restricted or linear types as we cannot freely duplicate
or drop them. Consider a lambda expression that represents function application $\lambda f. \lambda x. f x$ and it is applied to
some function $\mathcal{F}$. The linearity of this function $\lambda y. \mathcal{F} x$
would depend on the linearity of $\mathcal{F}$. To generalize, we can say that the linearity of the lambda expression depends
on its closure. The type of $\lambda f. \lambda x. f x$ can be written as $(\tau \overset{f}{\rightarrow} \upsilon) \rightarrow \tau \overset{g}{\rightarrow}\upsilon$
This function would be well typed only if $f$ is more restricting i.e. admits more structural rules than $g$ so to say $f \geq g$.
The system thus also introduces the concept of comparing types in terms of
their linearity. The predicate $\tau \geq \tau'$ holds if the type $\tau'$
is less restricting than $\tau$ or to say $\tau$ has admits more structural rules than $\tau'$.
Given this definition of $\geq$ we see that the types form a partially ordered lattice where
unrestricted types ($\Un{\tau}$) would be the meet as they are least restricting and
admit to both weakening and and contraction rules.

\section{Logic of Bunched Implications and $\alpha\lambda$-Calculus}\label{sec:bi}
In intuitionistic logic, the context is considered as a list or a set. In the theory of \BI,
the context is treated as a tree in contrast to other logics. Contexts are syntactically
combined using 2 connectives comma ($,$) or a semicolon ($;$) and are called bunches. The logic of \BI
tries to glue together intuitionistic linear logic with intuitionistic logic by
permitting contexts connected with semicolon to undergo contraction and weakening while the context connected with comma
are prohibited to undergo contraction and weakening. Comma and semicolon do not distributive over each other.
Thus $A,(B;C) \neq A, B ; A,C$ and $A;(B,C) \neq A;B,A;C$ where $A$, $B$ and $C$ are assumptions or propositions.
There are two flavors of implication---additive and multiplicative---which is closely related to the idea of conjunction.
\begin{framed}\centering
\begin{minipage}{1.0\linewidth}
  \begin{prooftree}
    \AxiomC{$\Gamma, A \vdash B$}
    \UnaryInfC{$\Gamma \vdash A \lozenge B$}
  \end{prooftree}
\end{minipage}
\end{framed}
In the logic of \BI the question then faced is choosing what kind of
implication should be used in place of $\lozenge$---the additive kind or the multiplicative kind.
O'Hearn et al. \citeyearpar{ohearn_logic_1999} introduce 2 kinds of arrows
and use them depending on the connectives used in the context. A multiplicative implication ($\sepimp$)
is used when the context is connected with a comma and an additive implication ($\rightarrow$) is used when the
context is connected using semicolon. This gives rise to two rules
\begin{framed}
\begin{minipage}{0.5\linewidth}
  \begin{prooftree}
    \AxiomC{$\Gamma, A \vdash B$} \RightLabel{$[\sepimp I]$}
    \UnaryInfC{$\Gamma \vdash A \sepimp B$}
  \end{prooftree}
\end{minipage}
\begin{minipage}{0.5\linewidth}
  \begin{prooftree}
    \AxiomC{$\Gamma; A \vdash B$} \RightLabel{$[\rightarrow I]$}
    \UnaryInfC{$\Gamma \vdash A \rightarrow B$}
  \end{prooftree}
\end{minipage}
\end{framed}

As we see in $[\sepimp I]$ $\Gamma, A$ cannot under go weakening or contraction to duplicate
or get rid of either $A$ or $\Gamma$. This hints to a notion that multiplicative implication ($\sepimp$)
exhibits property of the linear implication ($\rightspoon$). The linear implication cannot however
be directly converted to a multiplicative implication as the latter does not exhibit properties of
counting the number of uses of its arguments. Also, in contrast to linear logic, the multiplicative implication
cannot be converted into an intuitionistic implication by using modality. The logic of \BI tries to combine the
additive logic i.e. intuitionistic logic with the multiplicative side i.e. intuitionistic linear logic seamlessly.
The multiplicative side can be used to model the behavior of resources in the programming language
while the additive side would help the programmers fall back to the non-resource intuitionistic parts. This patches
up the awkward asymmetry experienced in linear logic. The multiplicative conjunction $\otimes$ has a right adjoint
counter part as $\sepimp$ while additive conjunction $\with$ has a right adjoint counterpart of $\rightarrow$.
The logic of \BI argues that instead of looking at the number of times an argument is used within the function,
it should be viewed in terms of {\em sharing}. $\alpha \lambda$-calculus \citep{ohearn_resource_1999, pym_semantics_2002}
is interpretation of the logic of \BI. It introduces 2 kinds of arrows by modifying the the syntax of lambda calculus:
\begin{enumerate}
  \item $\sepimp$     : Functions do not share resources with their arguments
  \item $\rightarrow$ : Functions share resources with their arguments
\end{enumerate}

\begin{figure}[h]
\begin{framed}
  \begin{flalign*}
                              t, u, v &\in \text{Type Variables}\nonumber\\
    \text{Types}\ \ \  \tau           &::= t \mid \iota \mid \tau \rightarrow \tau \mid \tau \sepimp \tau \nonumber\\
    \text{Expressions}\ \ \ M, N      &::= x \mid \lambda^{\alpha} x. M \mid \lambda^{*} x. M \mid M N \nonumber
  \end{flalign*}
\end{framed}
\caption{$\alpha\lambda$-Calculus Types and Terms}
\label{fig:al-cal-types}
\end{figure}

\begin{figure}[h]
  \begin{framed}
    % var
    \begin{minipage}{1.0\textwidth}
      \begin{prooftree}
        \AxiomC{$x: \tau \in \Gamma$} \RightLabel{$[VAR]$}
        \UnaryInfC{$\Gamma \vdash x : \tau $}
      \end{prooftree}
    \end{minipage}
    % -> I
    \begin{minipage}{0.5\textwidth}
      \begin{prooftree}
        \AxiomC{$\Gamma_{x}, x: \tau \vdash M : \tau'$} \RightLabel{$[\sepimp I]$}
        \UnaryInfC{$\Gamma \vdash \lambda^{*} x. M : \tau \sepimp \tau'$}
      \end{prooftree}
    \end{minipage}
    % -> E
    \begin{minipage}{0.5\textwidth}
      \begin{prooftree}
        \AxiomC{$\Gamma \vdash M : \tau \sepimp \tau' \ \ \ \ \
          \Gamma \vdash N : \tau$} \RightLabel{$[\sepimp E]$}
        \UnaryInfC{$\Gamma \vdash M N : \tau'$}
      \end{prooftree}
    \end{minipage}
    % -* I
    \begin{minipage}{0.5\textwidth}
      \begin{prooftree}
        \AxiomC{$\Gamma_{x}; x: \tau \vdash M : \tau'$} \RightLabel{$[\rightarrow I]$}
        \UnaryInfC{$\Gamma \vdash \lambda^{\alpha} x. M : \tau \rightarrow \tau'$}
      \end{prooftree}
    \end{minipage}
    % -* E
    \begin{minipage}{0.5\textwidth}
      \begin{prooftree}
        \AxiomC{$\Gamma \vdash M : \tau \rightarrow \tau' \ \ \ \ \
          \Gamma \vdash N : \tau$} \RightLabel{$[\rightarrow E]$}
        \UnaryInfC{$\Gamma \vdash M N : \tau'$}
      \end{prooftree}
    \end{minipage}
  \end{framed}
  \caption{Typing Rules for $\alpha\lambda$ Calculus}
  \label{fig:bi-logic}
\end{figure}


Due to the rules of $\alpha\lambda$-calculus $f: \tau \sepimp \tau'; x:\tau \nvdash f x:\tau'$.
As $f$ needs an argument that does not share any resources with its context.
Also, $\lambda^{*}f. \lambda^{*}x. f x x: (\tau \rightarrow \tau \rightarrow \tau') \sepimp \tau \rightarrow \tau \rightarrow \tau'$ is a typable term
This illustrates the difference between logic of \BI and linear logic, as even tough the argument is separate from the function,
it may be used twice. Linear logic would prohibit such use of arguments to function.

%%% Local Variables:
%%% mode: latex
%%% TeX-master: "../thesis-ku"
%%% End:
     % description of algorithm M linear logic, BI, qualified types
\chapter{Programming in \qub{}}
In the following chapter we illustrate using different examples to show how \qub{} is different
from other functional languages and how a powerful type system backed by logic of Bunched Implications
would be useful in keeping track of resources. The examples show how a resource can be tracked at compile
time and resource leaks can be avoided.

\section{File Handles}\label{sec:file-handle-example}
In an imperative language file handles are treated as normal variables
and it is the programmers responsiblity to check that that files are not closed twice
or there are no files that remain open when the program exits. This seemingly non-trivial
responsibility becomes tedious and erroneous as soon as there are multiple conditionals in the programming
logic, and as programs get larger in size. Modern functional languages such as Haskell treat file handles as
a special datatype and make them an instance of Monad. Consider the functions for file handling
as shown in \cref{fig:file-handling-function}. The type system however is not powerful enough
to detect whether a file handle is closed twice or is not closed at all. A simple program in Hasell that opens a file and reads
a line from it and then closes the file handle is shown in \cref{fig:file-read-close}.

\begin{figure}[h]
  \begin{framed}
    \begin{minted}{haskell}
openFile :: FilePath -*> IO FileHandle
closeFile :: FileHandle -*> IO ()
readFile :: FileHandle -*> IO (String, FileHandle)
writeFile :: String -*> FileHandle -> IO ((), FileHandle)
    \end{minted}
  \end{framed}
  \caption{File Handling functions}
  \label{fig:file-handling-function}
\end{figure}

\begin{figure}[h]
  \begin{framed}
    \begin{minted}{haskell}
do f  <- openFile "sample.txt"
   (s, f)  <- readLine f
   () <- close f
    \end{minted}
  \end{framed}
  \caption{Reading from a file in Haskell}
  \label{fig:file-read-close}
\end{figure}

Consider an incorrect version of the above program where the file handle is closed twice after reading a line from it \cref{fig:file-read-close-2times}.
It may not be a problem in a single threaded environment, but in a multithreaded environment
the second close may accidently close the file handle that may have been reused in the background by another thread.
When another thread tries to write on this closed file handle, it would throw an exception.
Haskell's type system would happily accept this program but it might generate a runtime exception.
\begin{figure}[h]
  \begin{framed}
    \begin{minted}{haskell}
do f  <- openFile "sample.txt"
   (s, f)  <- readLine f
   () <- close f
   () <- close f
    \end{minted}
  \end{framed}
  \caption{Reading from a file and closing it twice}
  \label{fig:file-read-close-2times}
\end{figure}

Another example of incorrect way of using file handle is by not closing the file handle after using it shown in \cref{fig:file-read-noclose}.
In a short lived process, when the program exits, file handles that are not closed are freed by the operating system.
But if it is a long running process it would run out of file handles and the whole process would crash with an error that
it cannot open any more file handles. Abnormal exit from the process would interfere in the write process
and the operating system would close the file handle without waiting for the buffer to be completely
written on the file system.
\begin{figure}[h]
  \begin{framed}
    \begin{minted}{haskell}
do f <- openFile "sample.txt"
   (s, f)  <- readLine f
   return s
    \end{minted}
  \end{framed}
  \caption{Reading from a file and not closing file handle}
  \label{fig:file-read-noclose}
\end{figure}
We take a deeper dive into this problem by seeing the desugared version of the ``do'' notation.
Both the programs would be translated into bind ($>>=$) operations . Recall that type signature of bind function is given as
\mintinline{haskell}{(>>=) :: (Monad m) => m t -> (t -> m u) -> m u}. Desugared version of \cref{fig:file-read-close-2times}
would look like as shown in \cref{fig:file-read-close-2times-desugared}
and the desugared version of \cref{fig:file-read-noclose} would look like \cref{fig:file-read-noclose-desugared}.

\begin{figure}[h]
  \begin{framed}
    \begin{minted}[escapeinside=||,mathescape=true]{haskell}
(>>=|$_1$|) (openFile "sample.txt") (\f ->
            (>>=|$_2$|) (readLine f) (\ (s, f) ->
                        (>>=|$_3$|) (close f) (\_ -> close f)))
    \end{minted}
  \end{framed}
  \caption{Reading from a file in Haskell and closing twice}
  \label{fig:file-read-close-2times-desugared}
\end{figure}

\begin{figure}[h]
  \begin{framed}
\begin{minted}{haskell}
(>>=) (openFile "sample.txt" ReadMode) (\ f ->
       >>= (readLine f) (\ (s, f) -> return s)
\end{minted}
  \end{framed}
  \caption{Reading from a file in Haskell and not closing it}
  \label{fig:file-read-noclose-desugared}
\end{figure}

In both the cases, the types of the programs would be computed by haskell compiler as \mintinline{haskell}{IO ()}
This well typed looking program should be flagged by the compiler as it would cause problems at runtime.
To overcome this, we introduce the concept of unrestrictedness and sharing and separation of resources.

The $\rightarrow$ constructor now has 2 meanings---shared or separated---we are
forced to describe the bind operator in terms of $\sepimp$ or $\rightarrow$. In Quill the bind operation will
be typed as given in \cref{fig:quill-bind-type}. This means all resources used by the funtion are separate.
\begin{figure}[h]
  \begin{framed}
    \begin{minted}{haskell}
          (>>=) :: m t -*> (t -!*> m u) -*> m u
    \end{minted}
  \end{framed}
  \caption{Type signature for bind in Quill}
  \label{fig:quill-bind-type}
\end{figure}
Thus all the resources used should be separating, but as we see in \cref{fig:file-read-close-2times-desugared} the file handle
\texttt{f} is shared in the third bind operation.
The bind operations will have the signatures as shown in \cref{fig:bind-signatures}
\begin{figure}[h]
\begin{framed}
\begin{minted}[escapeinside=||,mathescape=true]{haskell}
(>>=|$_1$|):: IO FileHandle -*> (FileHandle -!*> IO ()) -*> IO ()
(>>=|$_1$|) (openFile "sample.txt" ReadMode) (\f -> ...)

(>>=|$_2$|) :: IO FileHandle
            -*> (FileHandle -!*> IO (String, Filehandle))
            -*> IO (String, FileHandle)}
(>>=|$_2$|) (readLine f) (\ (s,f) -> ... )

(>>=|$_3$|) :: IO () -*> (() -!*> IO ()) -&> IO ()
(>>=|$_3$|) (closeFile f) (\_ -> closeFile f)}
\end{minted}
\end{framed}
\caption{Bind Signatures}
\label{fig:bind-signatures}
\end{figure}

The type of the third bind would be computed as \\
\mintinline{haskell}{IO () -*> (() -!*> IO ()) -&> IO ()} which would be a type error
as the bind operation should have a type signature of
\mintinline{haskell}{IO () -*> (() -!*> IO ()) -*> IO ()}. The types here would not match.
The types do match up for the correct implementation of the file read and close shown in \cref{fig:file-read-close}.
The first bind \mintinline{haskell}{(>>=) (openFile "sample.txt" ReadMode) (\ f -> ...)} has a type signature of
\mintinline{haskell}{IO FileHandle -*> (FileHandle -!*> IO ()) -*> IO ()} and the second bind
\mintinline{haskell}{(>>=) (readLine f) (\ (s, f) -> close f)} has a type signature of\\
\mintinline{haskell}{IO (String, FileHandle) -*> (FileHandle -!*> IO ()) -*> IO ()}

In \cref{fig:file-read-noclose-desugared} the file handle \texttt{f} is not closed. It is declared, but not used
in its scope, it would be tagged as an unrestricted value by the Quill type checker.
This is in voilation of our assumption that resources cannot be of the unrestricted type. Thus
the program would not typecheck due to mismatch of the file handle type to be unrestricted.

\section{Exception handling}
We expand on the file handling scenario and consider the code that can throw runtime exceptions.
The motivation to do so lies in the fact that memory leaks are caused becuase of runtime
exceptions where the part of code that is responsible to clean up resources or in this case
closing the file is skipped due to an alternate execution path.

The \HaskellF{IOMonad} is a class that encapsulates two kinds of IOs. \HaskellF{IO} that does not fail
or throw exceptions, and \HaskellF{IOF} that can fail and throw exceptions.
For a concrete example we will assume that \HaskellF{readLine} can throw an exception during runtime, where
it might fail to read a line due to the file mode being used has incorrect permissions.
For the sake of simplicity \HaskellF{openFile} and \HaskellF{closeFile} do not throw exceptions.
\HaskellF{onExcept} accepts a code that can fail and executes the second
parameter only if the actual code throws an exception. The \HaskellF{onExcept} gives a chance
to clean up resources in a systematic way. It may convert a code that can fail to a code
that does not fail by using \HaskellF{catch} function, or it may re throw the exception after
cleaning up the resources.

In the exceptionless execution path in \cref{fig:file-exceptions}, a single line would be read from the file and its uppercased
version would be returned after closing it. If suppose, an exception is thrown by \HaskellF{readLine}
the \HaskellF{onExcept} would close the filehandle. Notice that \HaskellF{onExcept}
shares the context with the file readline code, thus the appropriate file handle would be closed and
we would not have any memory leak.

\begin{figure}[h]
  \begin{framed}
    \begin{minted}{haskell}
openFile :: FilePath -*> IO FileHandle
closeFile :: FileHandle -*> IO ()
readFile :: FileHandle -*> IOF (String, FileHandle)
writeFile :: String -*> FileHandle -*> IOF ((), FileHandle)

throw :: Exception -> IO a
catch :: IOMonad IOM => IO a -*> (IO a -> IOM b) -&> IOM b

onExcept :: IOF a -*> IOF () -&> IOF a
m `onExcept` n = m `catch` (\e -> n >> throw e)

readFromFile :: String -*> IO (Either String String)
readFromFile f =
do fh  <- openFile f WriteMode
   ((s, fh)  <- readLine fh
   let l = caps s
   () <- close fh
   return $ Right l) `onExcept` (do () <- closeFile fh)
                     `catch` (\e ->
                              return $ Left "Could not read file")
    \end{minted}
  \end{framed}
  \caption{Exception Handling in Files}
  \label{fig:file-exceptions}
\end{figure}



% \section{Non Empty List}\label{sec:dll-example}
% A non empty list in Quill would look like
% \begin{minted}{haskell}
% data NEList a = Last a | Cons a (NEList a)
% \end{minted}

% The operations one could perfrom on a non-empty list are
% \begin{minted}{haskell}
% head :: NEList a -&> a
% head (Last h) = h
% head (Cons h t) = h

% tail :: NEList a -&> a
% tail (Last h) = h
% tail (Cons h t) = tail t

% concat :: NEList a -*> NEList a -&> NEList a
% concat (Last h') t = Cons h' t
% concat (Cons h t') t'' = Cons h (concat t' t'')

% append :: a -*> NEList a -&> NEList a
% append a t = Cons a t
% \end{minted}

% The underlying implementation of this list can be thought of a doubly
% linked list. There would be pointers that would connect the the individual nodes to form a doubly linked list.
% The programmer would be agnostic of the pointer handling as we expect the type system to be powerful enough to
% make sure that the lower level code generated would be free of memory leaks and runtime exceptions.
% It makes sense for this doubly linked list implementation to fit the description of a shared resource as nodes
% are interconnected. Getting access to one node gives us access to the whole list.



% \section{Circular Linked List as a Queue}\label{sec:queue-example}

% Circular list as a queue example



%%% Local Variables:
%%% mode: latex
%%% TeX-master: "../thesis-ku"
%%% End:
           % file handles and exceptions
\chapter{Core Language Syntax and Types}
In the following chapter we give the formal description of the language syntax and types. We explain what
it means for a judgement to exist as binary trees and then how we approximate the tree judgement
to a multiset inorder to simplify our type assignment algorithm.

\begin{figure}[h]
  \begin{framed}
    \begin{flalign*}
      \text{Type Variables}\ \ \      t, u, v         &\in \text{Type Variables}  \nonumber\\
      \text{Kinds}\ \ \               \kappa          &::= \star \mid \kappa \rightarrow \kappa \nonumber\\
      \text{Type constructors}\ \ \   T^{\kappa}       &::= \mathcal{T}^{\kappa}\ \text{where}
                                                      \{\overset{!}{\sepimp}, \sepimp, \xrightarrow{!}, \rightarrow \} \subseteq \mathcal{T}^{* \rightarrow * \rightarrow *}\nonumber\\
      \text{Types}\ \ \               \tau^{\kappa}    &::= t \mid T^{\kappa} \mid \tau^{\kappa' \rightarrow \kappa} \tau^{\kappa'}\nonumber\\
      \text{Predicates}\ \ \          \pi             &::= \texttt{Un}\ \tau \mid \texttt{SeFun}\ \tau \mid \texttt{ShFun}\ \tau \mid \tau \geq \tau' \nonumber\\
      \text{Qualified Types}\ \ \     \rho            &::= \tau^{*} \mid \pi => \rho \nonumber\\
      \text{Type schemes}\ \ \        \sigma          &::= \rho \mid \forall t. \sigma \nonumber
    \end{flalign*}
  \end{framed}
  \caption{Types \qub{}}
  \label{fig:qub-types}
\end{figure}
% Describe types
The type language consists of type variables and two kinds of binary type constructors the
sharing arrow ($\rightarrow$) and the separating arrow ($\sepimp$). The sharing arrow
would mean that the function shares resources with its argument and the separating
arrow would mean that the function does not share resources with its arguments.
We would write both the arrows in an infix notation. The kind system is simple where we use
$\star$ to denote all the types. We use $\tau$, $v$ and $\phi$ to denote types of any kind.
We exclude additive and multiplicative product types and sum types in our core language. The
definitions of all the four type constructors is described in \cref{sec:pairs} and \cref{sec:sums}.
The system is powerful enough to let programmers define their own {\color{red} types using the type constructors.}
% We support user defined data types following
% Jones' [\cite{jones_system_1993}] by adding kind support to our language.

% Describe Predicates
The predicate system enhances the expressibility of the type system. Following the same route taken
by Quill \citep{morris_best_2016} we use the predicate \texttt{Un} $\tau$ to denote
that the type $\tau$ is unrestricted. We write \texttt{ShFun} $\tau$ to describe that type $\tau$ may share resources with its
argument types and \texttt{SeFun} $\tau$ to describe that $\tau$ is
does not share any resources from its argument types. The function types can also be of the unrestricted type.
Thus if a type $\tau$ is unrestricted i.e. it qualifies with predicate \texttt{Un} and it is also one of the function types
i.e. \texttt{SeFun} or \texttt{ShFun}, we write them as $\overset{!}{\sepimp}$ and $\xrightarrow{!}$ respectively.
This can be considered as an improving substitution following Jones notion of improvement of qualified types \citep{jones_simplifying_1995}.
We also define an ordering on types by using the predicate $\geq$. The predicate $\tau \geq \tau'$ holds if the type $\tau'$
is less restricting than $\tau$. The predicate entailment relations $P => Q$ are given in \cref{fig:entailment-rules}.
% \TODO{Explain what does it mean for a type to be less that other type}
\begin{figure}[h]
  \begin{framed}
    \begin{minipage}{0.20\linewidth}
      \begin{prooftree}
        \AxiomC{$\pi \in P$}
        \UnaryInfC{$P => \pi$}
      \end{prooftree}
    \end{minipage}
    \begin{minipage}{0.20\linewidth}
      \begin{prooftree}
        \AxiomC{$\bigwedge_{\pi \in Q} P => \pi$}
        \UnaryInfC{$P => Q$}
      \end{prooftree}
    \end{minipage}
    \begin{minipage}{0.20\linewidth}
      \begin{prooftree}
        \AxiomC{}
        \UnaryInfC{$P => \Un{(\tau \sepimp \tau')}$}
      \end{prooftree}
    \end{minipage}
    \begin{minipage}{0.20\linewidth}
      \begin{prooftree}
        \AxiomC{}
        \UnaryInfC{$P => \Un{(\tau \rightarrow \tau')}$}
      \end{prooftree}
    \end{minipage}
    \begin{minipage}{0.20\linewidth}
      \begin{prooftree}
        \AxiomC{$\tau = \sepimp \vee \tau = \overset{!}{\sepimp}$}
        \UnaryInfC{$P => \SeFun{\tau}$}
      \end{prooftree}
    \end{minipage}
    \begin{minipage}{0.20\linewidth}
      \begin{prooftree}
        \AxiomC{$\tau = \rightarrow \vee \tau = \overset{!}{\rightarrow}$}
        \UnaryInfC{$P => \ShFun{\tau}$}
      \end{prooftree}
    \end{minipage}
    \begin{minipage}{0.20\linewidth}
      \begin{prooftree}
        \AxiomC{$P => \Un{\tau}$}
        \UnaryInfC{$P => \tau \geq (v \overset{!}{\sepimp} v')$}
      \end{prooftree}
    \end{minipage}
    \begin{minipage}{0.20\linewidth}
      \begin{prooftree}
        \AxiomC{$P => \Un{\tau}$}
        \UnaryInfC{$P => \tau \geq (v \overset{!}{\rightarrow} v')$}
      \end{prooftree}
    \end{minipage}
    \begin{minipage}{0.5\linewidth}
      \begin{prooftree}
        \AxiomC{$$}
        \UnaryInfC{$P => \tau \geq (v \sepimp v')$}
      \end{prooftree}
    \end{minipage}
    \begin{minipage}{0.5\linewidth}
      \begin{prooftree}
        \AxiomC{$$}
        \UnaryInfC{$P => \tau \geq (v \rightarrow v')$}
      \end{prooftree}
    \end{minipage}
    \begin{minipage}{0.5\linewidth}
      \begin{prooftree}
        \AxiomC{$P => \tau \geq \phi t$}
        \AxiomC{$t\ \text{fresh}$}
        \BinaryInfC{$P => \tau \geq \phi$}
      \end{prooftree}
    \end{minipage}
    \begin{minipage}{0.5\linewidth}
      \begin{prooftree}
        \AxiomC{$P => \tau t \geq \phi$}
        \AxiomC{$t\ \text{fresh}$}
        \BinaryInfC{$P => \tau \geq \phi$}
      \end{prooftree}
    \end{minipage}
  \end{framed}
  \caption{Entailment Rules}
  \label{fig:entailment-rules}
\end{figure}


\begin{figure}[h]
  \begin{framed}
    \begin{flalign*}
      \text{Environments}\ \ \      \Gamma,\Delta     &::= \epsilon \mid x^{\bar{y}}:\sigma \mid \Gamma \varoplus \Delta \mid \Gamma \circledast \Delta \nonumber\\
      \text{Multienvironment}\ \ \ \mathcal{H}     &::= \epsilon \mid \mathcal{H};\Gamma \mid \mathcal{H}, \Gamma \mid \square \nonumber
  \end{flalign*}
\end{framed}
  \caption{Typing Context}
  \label{fig:typing-context}
\end{figure}
\TODO{Need some more reading/writing about multienvironment}
% Describe Typing judgments
In normal type systems, the contexts are represented as sets or lists. In \BI\ they are represented as binary trees.
The leaf nodes contain the pair of term and its associated type. Internal nodes of the context tree are
connectives which can either be a semicolon ($;$) or a comma ($,$).
If a bunch $\Delta$ is a subtree of $\Gamma$, then we denote a subtree relation by $\Gamma(\Delta)$.
Two context are equivalent ($\Gamma \equiv \Delta$)if they can be transformed into one another by renaming the identifiers.
The bunches have a restriction that no identifier appears more than once. We restrict certain structural rules on the context
depending on the connectives being used. If contexts are combined using a comma ($,$), contraction and weakening is not admissible,
but if the contexts are combined using a semicolon ($;$) then it can undergo contraction and weakening. Exchange rule is admissible
in both the connectives. This distinction enables us to have a special treatment for resources in our language.
By associating a resource with a comma constructor, our type system will not disposed it off by using the contraction rule.
While, non-resourceful objects (or normal propositions) can be combined using the semi-colon constructor.
An example bunch is shown in \cref{fig:bunches-bi}. a and b have a shared context while c is separate from the bunch a and b.
If $\Gamma$ represents the complete bunch of \cref{fig:bunches-bi}, $\Delta \equiv (a:A; b:B)$ and $\Delta' \equiv (c:C)$
then $\Gamma \equiv \Delta,\Delta'$ and $\Gamma(\Delta)$.

\begin{figure}[h]
  \centering
  \tikzset{every tree node/.style={minimum width=2em},
         blank/.style={draw=none},
         edge from parent/.style=
         {draw,edge from parent path={(\tikzparentnode) -- (\tikzchildnode)}},
         level distance=1.5cm}
\begin{tikzpicture}
\Tree
[.,
    [.;
        [.a:A ]
        [.b:B ]
    ]
    [.c:C ]
    ]
\end{tikzpicture}
\caption{Bunches in \textbf{\em BI}}
\label{fig:bunches-bi}
\end{figure}

We try to emulate the behaviour of bunches using multisets and denote multiset union using $\Gamma \sqcup \Gamma'$.
The type assignment will be a collection of variables with an added annotation of sharing. If a resource $x$ is in
sharing with one or many resources $\bar{y}$ we would represent it as $x^{\bar{y}}:\tau$.
Further $\Gamma, x^{\bar{y}}:\tau$ would mean $\Gamma \sqcup \{x^{\bar{y}}:\tau\}$. We define a few auxilary functions on the
type assigments. \texttt{Vars}($\Gamma$) is the set of all the term variables in $\Gamma$. \texttt{Shared}($\Gamma$) computes
the set of all the term variables that are in sharing with each other. \texttt{Used}($\Gamma$) computes the
union of all the term variables in the type assignment and the term variables shared by each of those.
We define two partial operators on type assigments as shown in \cref{fig:type-assignment-operations}.
The mapping function ($\Gamma^{[\vec{a} \mapsto \vec{b}]}$) extends the sharing relation between the terms. One can imagine the operation forming a graph, where
node represents a term, and an undirected edge represents sharing.
Two type assigments are said to be in disjoint union ($\circledast$) if either of the type assignments used terms are not in common
with other type assignment's shared term. If the type assignments have an exact overlapping of terms being used,
it is said to be in a sharing union ($\varoplus$). The ($\#$) in ($\circledast$) represents disjoint check and we use
the standard notion of set equality for checking sharing union. {\color{red} TODO: The sharing relation is a symmetric, reflexive and transitive operation.
  Thus we can partition the terms into disjoint union sets. Should we define it more fomally. Something like the separating function
of \cite{atkey_lambda_sep_2004}}

\begin{figure}[h]
  \begin{framed}
    \noindent
    \begin{flalign*}
      \texttt{Vars}(\Gamma, x^{\vec{y}}) &= \texttt{Vars}(\Gamma) \cup \{ x \}\\
      \texttt{Shared}(\Gamma, x^{\vec{y}}) &= \texttt{Shared}(\Gamma) \cup \{ \vec{y} \}\\
      \texttt{Used}(\Gamma) &= \texttt{Vars}(\Gamma) \cup \texttt{Shared}(\Gamma)\\
      (\Gamma, x^{\vec{y}})^{[a \mapsto \vec{b}]} &= \begin{cases}
        x \notin \vec{y}\ \ \ \ (\Gamma^{[a \mapsto \vec{b}]}, x^{\vec{y}}:\tau)\\
        x \in \vec{y}\ \ \ \  (\Gamma^{[a \mapsto \vec{b}]}, x^{(\vec{y}\backslash a)\cup\vec{b}}:\tau)
      \end{cases}\\
      \Gamma^{[\vec{a} \mapsto \vec{b}]} &= (\dots((\Gamma^{[a_1 \mapsto \vec{b}]})^{[a_2 \mapsto \vec{b}]})^{\dots})^{[a_n \mapsto \vec{b}]}
    \end{flalign*}
  \end{framed}
  \caption{Auxilary Functions on Multisets}
  \label{fig:multiset-aux-function}
\end{figure}
\begin{figure}[h]
  \begin{framed}
    \begin{flalign*}
      \Gamma \circledast \Gamma' &= \Gamma \sqcup \Gamma' => \texttt{if}\ \texttt{Vars}(\Gamma) \# \texttt{Used}(\Gamma') \wedge \texttt{Vars}(\Gamma')\# \texttt{Used}(\Gamma) \\
      \Gamma \varoplus \Gamma'   &= \Gamma \sqcup \Gamma' => \texttt{if}\ \texttt{Used}(\Gamma) = \texttt{Used}(\Gamma')
    \end{flalign*}
  \end{framed}
  \caption{Type Assignment Operations}
  \label{fig:type-assignment-operations}
\end{figure}


\begin{figure}[h]
  \begin{framed}
    \begin{flalign*}
      \text{Term Variables}\ \ \  x, y, z  &\in \text{Var} \nonumber\\
      % \text{Patterns}\ \ \        p        &::= x \mid C \vec{x}\nonumber\\
      \text{Expressions}\ \ \     M, N     &::= x \mid \lambda^{*}x. M \mid \lambda^{\alpha}x. M \mid M N \mid \Let{x}{M}{N}\nonumber
      %&\mid \Case{M}{\{\texttt{inl}\ x \mapsto N ; \texttt{inr}\ y \mapsto N'\}}\mid \texttt{inl}\ x \mid \texttt{inr}\ y \nonumber\\
     % &\mid  \mid \Pair{M,N} \mid \Pair{M;N}
    \end{flalign*}
  \end{framed}
  \caption{Language Syntax}
  \label{fig:quill-terms}
\end{figure}
% Describe the language here

% Describe terms and patterns
Our term language is similar to that of simply typed lambda calculus involving variables and application
but we have two types of lambda expressions, the alpha lambda ($\lambda^{\alpha}$) denotes sharing
of the argument term with the expression $M$ and the separating lambda term ($\lambda^{*}$) that implies
the argument term has a separating context with the expression $M$. We also have polymorphic $\texttt{let}$
expressions to be able to define expressions with a limited scope.  % The type constructors \texttt{inl} and \texttt{inr} build
% a sum type while \texttt{case} expression match on the sum type to specify what should be done in for the two ways
% in which the sum type was created.

% The type constructors are added in order to allow programmers to define their own data types. They can be used to define sum and product types.
% \texttt{case} expression can be used to pattern match on the expression to express it in terms
% of individual sum types. Patterns are either term variables or constructor terms.

%%% Local Variables:
%%% mode: latex
%%% TeX-master: "../thesis-ku.tex"
%%% End:
            % Core language syntax and Types, contexts and related definitions
\chapter{Type System and Type Inference}
In the following chapter we go over the type system using the notion of types and terms defined in previous
chapter. We describe type system based on the logic of \BI\ in \cref{sec:type-system}. We then describe a
syntax directed type system in \cref{sec:syntax-typing-rules} and give a modified algorithm $\M$ in \cref{sec:algorithm-m}.
First we give some conventions and notations that will be used throughout the sections that follow.

\section{Conventions and Notations}
Vector $\vec{t}$ is a shorthand for a finite set of variables $\{t_1, t_2, \dots, t_n\}$ and $\forall \vec{t}. Q => \tau$
abbreviates $\forall t_1 \dots \forall t_n. P_1 => \dots => P_m => \tau$.
The typing context $\mathcal{H}$ is a finite multiset of pairs of terms and types such that if two term variables occur
more than once, all of them must have same type. i.e. if $(x:\sigma) \in \mathcal{H}$ and $x:\sigma' \in \mathcal{H}$ then $\sigma \approx \sigma'$.
The equality ($\approx$) on typing schemes is up-to renaming of bound variables.
$\Gamma_{x}$ denotes the type assignment excluding the type variable $x$.
We write $\sigma = \Gamma(x)$ for the type scheme assigned to the term x in $\Gamma$.
\texttt{dom}($\Gamma$) is the set of identifiers in the type assignment i.e.
\texttt{dom}($\Gamma$) = $\{ x \mid (x:\sigma) \in \Gamma\}$.
$\Gamma \odot \Delta$ would mean that the contexts can either
be sharing union ($\Gamma \varoplus \Delta$) or separating union ($\Gamma \circledast \Delta$).
Similar to Jones' \citeyearpar{jones_theory_1994} qualified types and type schemes are overloaded Hindley-Milner calculi.
\begin{defn}[Free Type Variables]
  $\texttt{fvs}(\tau)$ is the set of free type variables in the type $\tau$\\
  $\texttt{fvs}(\sigma)$ is the set of free type variables in a typing scheme $\sigma = \forall \vec{t} Q. => \tau$.\\
  $\texttt{fvs}(\sigma) = (\texttt{fvs}(\tau) \cup \texttt{fvs}(Q)) \backslash \vec{t})${\color{red} Needs confirmation on \texttt{fvs}(Q).}\\
  $\texttt{fvs}(\Gamma)$ is the set of free type variables in the type assignment $\Gamma$.\\
  $\texttt{fvs}(\Gamma) = \bigcup_{x \in \texttt{dom}(\Gamma)} \{ \texttt{fvs}(\Gamma(x)) \}$
\end{defn}
\begin{defn}[Typing Judgment]
The expression $P \mid \Gamma \vdash M : \sigma$ denotes the assertion that the term $M$ has a typing scheme $\sigma$
when the predicates in $P$ are satisfied and the free type variables in $M$ are specified in type assignment $\Gamma$.
\end{defn}

\section{Type System}\label{sec:type-system}
% Structural Rules
% Connective Rules
% forall, => Qualified type rules
We split our type system into multiple parts for legibility. The first part includes structural rules
shown in \cref{fig:structural-rules} and the second part includes connectives with
introduction and elimination rules shown in \cref{fig:typing-rules}.

% structural rules
\begin{figure}[h]
  \begin{framed}
    % ID
    \begin{minipage}{.5\textwidth}
      \begin{prooftree}
        \AxiomC{} \RightLabel{[ID]}
        \UnaryInfC{$P \mid x^{\vec{y}} : \sigma \vdash x : \sigma $}
      \end{prooftree}
    \end{minipage}
    % EXCH
    \begin{minipage}{.5\textwidth}
      \begin{prooftree}
        \AxiomC{$\Gamma \odot \Delta \vdash M : \sigma$} \RightLabel{[EXCH]}
        \UnaryInfC{$P \mid \Delta \odot \Gamma \vdash M : \sigma $}
      \end{prooftree}
    \end{minipage}
        % CTR UN
    \begin{minipage}{.5\textwidth}
      \begin{prooftree}
        \AxiomC{$P \mid \Gamma \circledast \Delta \vdash M : \sigma$}
        \AxiomC{$P'\vdash \Delta\ \texttt{un}$} \RightLabel{[CTR-UN]}
        \BinaryInfC{$P \mid \Gamma \vdash M : \sigma$}
      \end{prooftree}
    \end{minipage}
    % WKN UN
    \begin{minipage}{.5\textwidth}
      \begin{prooftree}
        \AxiomC{$P \mid \Gamma \vdash M : \sigma$}
        \AxiomC{$P' \vdash \Delta\ \texttt{un}$} \RightLabel{[WKN-UN]}
        \BinaryInfC{$P \mid \Gamma \circledast \Delta \vdash M : \sigma$}
      \end{prooftree}
    \end{minipage}
    % CTR Sh
    \begin{minipage}{.5\textwidth}
      \begin{prooftree}
        \AxiomC{$P \mid \Gamma \varoplus \Delta \vdash M : \sigma$} \RightLabel{[CTR-SH]}
        \AxiomC{$P \mid \Gamma \vdash M : \sigma$} \RightLabel{[CTR-SH]}
        \BinaryInfC{$P \mid \Gamma \vdash M : \sigma$}
      \end{prooftree}
    \end{minipage}
    % WKN Sh
    \begin{minipage}{.5\textwidth}
      \begin{prooftree}
        \AxiomC{$P  \mid \Gamma \vdash M : \sigma$} \RightLabel{[WKN-SH]}
        \UnaryInfC{$P \mid \Gamma \varoplus \Delta \vdash M : \sigma$}
      \end{prooftree}
    \end{minipage}
   \end{framed}
    \caption{Structural Typing Rules}
    \label{fig:structural-rules}
  \end{figure}

The tautology rule ([ID]) is a simple type assignment lookup for checking the type of the term.
The exchange rule ([EXCH]) justifies the fact that the order in which the type assignments are witnessed
in the multiset is insignificant and we can commute the individual term typing schemes within the multiset context
with no effect to the typing derivations. The weakening and contraction rules are made explicit in contrast to standard
Hindley-Milner type system. The contraction sharing rule [CTR-SH] and weakening sharing rule [WKN-SH]
says that we can duplicate or drop certain pairs of type assignments as per our convenience as we know they are in sharing with other
terms that remain in the context. The contraction separation rule [CTR-UN] and weakening separation rule [WKN-SH] can be
applied to terms only if we can prove that they are of unrestricted type which is captured by introducing
the ($\Delta$ \texttt{un}) predicate on the type that is dropped or duplicated.

% Connective rules
  \begin{figure}[h]
    \begin{framed}
    % let
    \begin{minipage}{1\textwidth}
      \begin{prooftree}
        \AxiomC{$P \mid \Gamma \vdash M : \sigma$}
        \AxiomC{$P' \mid \Gamma'_{x} \sqcup x: \sigma \vdash N: \tau$} \RightLabel{[LET]}
        \BinaryInfC{$P \cup P' \mid \Gamma \sqcup \Gamma' \vdash \Let{x}{M}{N}: \tau$}
      \end{prooftree}
    \end{minipage}
    \newline\newline\newline
    % forall I
    \begin{minipage}{0.5\textwidth}
      \begin{prooftree}
        \AxiomC{$P \mid \Gamma \vdash M: \sigma$}
        \AxiomC{$t \notin \texttt{fvs}(\Gamma) \cup \texttt{fvs}(P)$}\RightLabel{$[\forall I]$}
        \BinaryInfC{$P \mid \Gamma \vdash M: \forall t. \sigma$}
      \end{prooftree}
    \end{minipage}
    % forall E
    \begin{minipage}{0.5\textwidth}
      \begin{prooftree}
        \AxiomC{$P \mid \Gamma \vdash M: \forall t.\sigma$}\RightLabel{$[\forall E]$}
        \UnaryInfC{$P \mid \Gamma \vdash M: [\tau \backslash t] \sigma $}
      \end{prooftree}
    \end{minipage}
    \newline\newline\newline
    % => I
    \begin{minipage}{0.5\textwidth}
      \begin{prooftree}
        \AxiomC{$P, \pi \mid \Gamma \vdash M : \rho$} \RightLabel{$[=> I]$}
        \UnaryInfC{$P \mid \Gamma \vdash M : \pi \Rightarrow \rho$}
      \end{prooftree}
    \end{minipage}
    % => E
    \begin{minipage}{0.5\textwidth}
      \begin{prooftree}
        \AxiomC{$P \mid \Gamma \vdash M : \pi \Rightarrow \rho$}
        \AxiomC{$P \Rightarrow \pi$} \RightLabel{$[=> E]$}
        \BinaryInfC{$P \mid \Gamma \vdash M: \rho$}
      \end{prooftree}
    \end{minipage}
    \newline\newline\newline
    % -&> I
    \begin{minipage}{0.5\textwidth}
      \begin{prooftree}
        \AxiomC{$P \Rightarrow \texttt{ShFun}\ \phi\ \ \ \ \
          P \vdash \Gamma \geq \phi$}\noLine
        \UnaryInfC{$P \mid \Gamma^{[\emptyset\mapsto \{x\}]},x^{\text{Vars}(\Gamma)}: \tau \vdash M : \tau'$}\RightLabel{$[\rightarrow I]$}
        \UnaryInfC{$P \mid \Gamma \vdash \lambda^{\alpha}x. M : \phi \tau \tau'$}
      \end{prooftree}
    \end{minipage}
    % -&> E
    \begin{minipage}{0.5\textwidth}
      \begin{prooftree}
        \AxiomC{$P => \texttt{ShFun}\ \phi\ \ \ \ \
          P \vdash \Gamma \geq \phi$}\noLine
        \UnaryInfC{$P \mid \Gamma \vdash M : \phi \tau \tau'\ \ \ \ \
          P \mid \Delta \vdash N : \tau'$} \RightLabel{$[\rightarrow E]$}
        \UnaryInfC{$P \mid \Gamma \varoplus \Delta \vdash M N : \tau'$}
      \end{prooftree}
    \end{minipage}
    \newline\newline\newline
    % -*> I
    \begin{minipage}{0.5\textwidth}
      \begin{prooftree}
        \AxiomC{$P => \texttt{SeFun}\ \phi\ \ \ \ \ P \vdash \Gamma \geq \phi$}\noLine
        \UnaryInfC{$P \mid \Gamma,x^{\emptyset}: \tau \vdash M : \tau'$} \RightLabel{$[\sepimp I]$}
        \UnaryInfC{$P \mid \Gamma \vdash \lambda^{*}x. M : \phi \tau \tau'$}
      \end{prooftree}
    \end{minipage}
    % -*> E
    \begin{minipage}{0.5\textwidth}
      \begin{prooftree}
        \AxiomC{$P => \texttt{SeFun}\ \phi\ \ \ \ \ P \vdash \Gamma \geq \phi$}\noLine
        \UnaryInfC{$P \mid \Gamma \vdash M : \phi \tau \tau'\ \ \ \ P \mid \Delta \vdash N : \tau$} \RightLabel{$[\sepimp E]$}
        \UnaryInfC{$P \mid \Gamma \circledast \Delta \vdash M N : \tau'$}
      \end{prooftree}
    \end{minipage}
  \end{framed}
    \caption{Connective Typing Rules}
    \label{fig:typing-rules}
\end{figure}

The rules of [$\rightarrow$I] and [$\sepimp$I] describes the abstraction over shared and
separating resources respectively, while [$\rightarrow$E] and [$\sepimp$E] is the application
rule for shared and separating resources respectively. [$=>$I] and [$=>$E] are the rules for
qualified types that would add constraints on the type being computed. [$\forall$I] introduces
polymorphism and [$\forall$E] eliminates it. We assume that the free type variables that
would be substituted in the type scheme would not interfere with the sharing and separation
of resources of the fully concretized type. The $\lambda$ abstractions $\lambda^{*} x. M$ and $\lambda^{\alpha}.M$
have a function type $\ShFun{\phi}$ or $\SeFun{\phi}$ only if it is more restricting that its environment.
This is specified in the judgments $\cdot \geq \cdot$. To avoid name shadowing, we would assume that
the binders introduce fresh names. The reason to use multisets can now be understood
as the elimination rule ([$\rightarrow$E]) can be expressed by simply duplicating the typing
judgments that are in sharing after we split the type assignment into two parts,
for typing each sub expression individually.{\color{red} I am not sure about this/can be better articulated}.
\begin{figure}[h]
  \begin{framed}
  \begin{minipage}{0.5\linewidth}
    \begin{prooftree}
      \AxiomC{$P => \texttt{Un}\ \tau$}\RightLabel{[\texttt{Un}-$\tau$]}
      \UnaryInfC{$P \vdash \tau\ \texttt{un}$}
    \end{prooftree}
  \end{minipage}
  \begin{minipage}{0.5\linewidth}
    \begin{prooftree}
      \AxiomC{$P,\pi \vdash \rho\ \texttt{un}$}\RightLabel{[\texttt{Un}-$\rho$]}
      \UnaryInfC{$P \vdash \pi => \rho\ \texttt{un}$}
    \end{prooftree}
  \end{minipage}
  \begin{minipage}{0.5\linewidth}
    \begin{prooftree}
      \AxiomC{$P, \texttt{Un}\ t \vdash \sigma\ \texttt{Un}$}\RightLabel{[\texttt{Un}-$\sigma$]}
      \UnaryInfC{$P \vdash \forall t.\sigma\ \texttt{un}$}
    \end{prooftree}
  \end{minipage}
  \begin{minipage}{0.5\linewidth}
    \begin{prooftree}
      \AxiomC{$\bigwedge_{x:\sigma \in \Gamma}P \vdash \rho\ \texttt{un}$}\RightLabel{[\texttt{Un}-$\Gamma$]}
      \UnaryInfC{$P \vdash \Gamma\ \texttt{un}$}
    \end{prooftree}
  \end{minipage}
  \begin{minipage}{0.5\linewidth}
    \begin{prooftree}
      \AxiomC{$P => \tau \geq \phi$}\RightLabel{[$\geq$-$\tau$]}
      \UnaryInfC{$P \vdash \tau \geq \phi$}
    \end{prooftree}
  \end{minipage}
  \begin{minipage}{0.5\linewidth}
    \begin{prooftree}
      \AxiomC{$P,\pi \vdash \rho \geq \phi$}\RightLabel{[$\geq$-$\rho$]}
      \UnaryInfC{$P \vdash (\pi => \rho) \geq \phi$}
    \end{prooftree}
  \end{minipage}
  \begin{minipage}{0.5\linewidth}
    \begin{prooftree}
      \AxiomC{$P, \texttt{Un}\ t \vdash \sigma \geq \phi$}\RightLabel{[$\geq$-$\sigma$]}
      \UnaryInfC{$P \vdash (\forall t.\sigma) \geq \phi$}
    \end{prooftree}
  \end{minipage}
  \begin{minipage}{0.5\linewidth}
    \begin{prooftree}
      \AxiomC{$\bigwedge_{x:\sigma \in \Gamma}P \vdash \rho \geq \phi$}\RightLabel{[$\geq$-$\Gamma$]}
      \UnaryInfC{$P \vdash \Gamma \geq \phi$}
    \end{prooftree}
  \end{minipage}
\end{framed}
  \caption{Typing Rules for Base cases}
  \label{fig:bi-base-typing-rules}
\end{figure}
The rules given in \cref{fig:bi-base-typing-rules}
are convenience rules for base cases that compute predicate constraints for types within a context.
The $P \mid \cdot \geq \cdot$ lifts $\geq$ predicate into the typing environment.


\section{Kind Inference}
Our system supports introduction of new types via type constructors. We include a
richer kind system for this purpose. A modified version of Robinson's algorithm \citeyearpar{robinson_machine-oriented_1965}
is used in order to deduce the most general unifier for type constructors. Our treatment to type unification
is similar to Jones' \citeyearpar{jones_system_1993}.
Formally, we define $S$ to be the {\it most general unifier} for type constructors $T$ and $T'$ if:
\begin{enumerate}
  \item $S$ is a unifier for $C$ and $C'$.
  \item For every unifier $S'$ of $C$ and $C'$ we can write $S'$ in a form of
    $R S$ where $R$ is a kind preserving substitution.
\end{enumerate}
We write $C \overset{S}{\sim}_{\kappa} C'$ for assertion that $S$ is the unifier
of the constructor types $C, C' \in C^{\kappa}$. The rules in \cref{fig:unify-types}
describe the unification algorithm for type constructors. [Var] and [Var'] contain
and additional constraint of the type variable $t$ to not be free in the type constructor's $C$
type variables to ensure the unification does not lead to infinite types.

\begin{figure}[h]
  \begin{framed}
    \begin{minipage}[l]{0.5\linewidth}
      \begin{flalign*}
        t \overset{id}{\sim}_{\kappa} t \tag{[ID-Var]}
      \end{flalign*}
    \end{minipage}
    \begin{minipage}[r]{0.5\linewidth}
      \begin{flalign*}
        \tau \overset{id}{\sim}_{\kappa} \tau \tag{[ID-Const]}
      \end{flalign*}
    \end{minipage}

    \begin{minipage}[l]{0.5\linewidth}
      \begin{flalign*}
        t \overset{[C/t]}{\sim}_{\kappa} C, t \notin \texttt{fvs}(C) \tag{[Var]}
      \end{flalign*}
    \end{minipage}
    \begin{minipage}[r]{0.5\linewidth}
      \begin{flalign*}
        C \overset{[C/t]}{\sim}_{\kappa} t, t \notin \texttt{fvs}(C) \tag{[Var']}
      \end{flalign*}
    \end{minipage}

    \begin{minipage}[c]{1.0\linewidth}
      \begin{prooftree}
        \AxiomC{$C \overset{S}{\sim}_{\kappa'\rightarrow \kappa} D$}
        \AxiomC{$S C' \overset{S'}{\sim}_{\kappa'} S D'$}\RightLabel{[Apply]}
        \BinaryInfC{$CC' \overset{SS'}{\sim}_{\kappa} DD'$}
      \end{prooftree}
    \end{minipage}
  \end{framed}
  \caption{Kind Preserving Unification of Type Constructors}
  \label{fig:unify-types}
\end{figure}

\section{Syntax Directed Typing rules}\label{sec:syntax-typing-rules}
The type system explained in the previous section are not syntax directed and will not be fit
to develop a type inference algorithm as the typing rules and syntactic forms should have one-to-one
correspondence. In this section we will define syntax directed typing rules
that will simplify our type inference system shown in \cref{fig:syntax-typing-rules}

We define generalization and instantiation to express introduction and elimination of polymorphism in our
syntax direct typing rules as follows:
\begin{defn}[Instantiation]
  For a type scheme $\sigma := \forall \vec{t}. P => \tau'$, we say $(Q => \tau)$ is
  an instance of $\sigma$ and write it as $(Q => \tau) \sqsubseteq \sigma$, if there exists a $\vec{v}$
  such that $\tau = [\vec{v} / \vec{t}] \tau'$ and $Q = [\vec{v} / \vec{t}]P$.
\end{defn}

\begin{defn}[Generalization]
  For a type assignment $\Gamma$ and qualified type $\rho$, we define type scheme
  $\texttt{Gen}(\Gamma, \rho) = \forall (\texttt{fvs}(\rho) \backslash \texttt{fvs}(\Gamma)). \rho$.
\end{defn}

\begin{defn}[Qualified Type Scheme]
  A qualified type scheme is a pair of type scheme with a set of predicates written as $(P \mid \sigma)$,
  where $\sigma = \forall \vec{t}. Q => \tau$.
\end{defn}
\begin{defn}[Qualified Type Scheme Instantiation]
  For two qualified type schemes $(P \mid \sigma)$ where $\sigma = \forall \vec{t}. Q => \tau$ and
  $(P' \mid \sigma')$ where $\sigma' = \forall \vec{t'}. Q' => \tau'$ we say $(P' \mid \sigma')$ is an
  instance of $(P \mid \sigma)$ iff there exists $\vec{v}$ such that $\tau' = [\vec{v}/ \vec{t}]\tau$ and
  $P',Q' => P, [\vec{v}/ \vec{t}]Q$. We write it as $(P' \mid \sigma') \sqsubseteq (P \mid \sigma)$. A type scheme
  $\sigma$ is an abbreviation of $(\emptyset \mid \sigma)$.
\end{defn}

Similar to Quill typing system \citep{morris_best_2016} elimination of polymorphism [$\forall$E] and qualified
types[$=>$E] is always done in the [Var$^s$], introduction of polymorphism [$\forall$I] and qualified types[$=>$I] is
done at let bindings [Let$^s$]. This collapses the rules [$\forall$E], [$=>$E] and [ID] in one rule [VAR$^s$] where
we use instantiation, and [$\forall$I], [$=>$I] and [Let] in one rule [Let$^s$] where we use generalization.
[$\sepimp$I$^s$] is used in occurence of $\lambda^{*}$, and [$\rightarrow$I$^s$] is used in occurence of $\lambda^{\alpha}$.
We would add the introduced abstraction variable into the sharing context in case of [$\rightarrow$I$^s$].
The type variable $\tau$ in both [$\sepimp$I$^s$] and [$\rightarrow$I$^s$] are new. We collapse the application rules
[$\sepimp$E] and [$\rightarrow$E] into one rule [App$^s$] where we check for sharing of the used variables in both
the expressions and then assign a predicate of $\texttt{ShFun}$ or $\texttt{SeFun}$ depending on whether the variables
are shared or not. The \texttt{un} predicates are added to the types of the terms that are not used directly in the expression
or which are not in sharing with the terms used.
\begin{figure}[h]
  \begin{framed}
    % VAR^s
    \begin{minipage}{1.0\textwidth}
      \begin{prooftree}
        \AxiomC{$P \vdash \Gamma_{\vec{y}}\ \texttt{un}$}
        \AxiomC{$(P => \tau) \sqsubseteq \sigma$} \RightLabel{[VAR$^s$]}
        \BinaryInfC{$P \mid \Gamma, x^{\vec{y}} : \sigma \vdashs x : \tau $}
      \end{prooftree}
    \end{minipage}
    \newline\newline\newline
    % Let^s
    \begin{minipage}{1.0\textwidth}
      \begin{prooftree}
        \AxiomC{$Q \mid (\Gamma_x' \varoplus \Gamma_x'') \circledast \Delta \vdashs M: v\ \ \ \ \
          P \vdash \Delta\ \texttt{un}$}
        \noLine
        \UnaryInfC{$P \mid (\Gamma_x \sqcup x:\sigma) \varoplus \Gamma_x'' \circledast \Delta \vdashs N:\tau\ \ \ \ \
          \sigma = \texttt{Gen}(\{\Gamma' \varoplus \Gamma_x'' \circledast \Delta \}, Q => v)$}\RightLabel{[Let$^s$]}
        \UnaryInfC{$P \mid (\Gamma \circledast \Gamma') \varoplus \Gamma'' \circledast \Delta \vdashs (\Let{x}{M}{N} : \tau) $}
      \end{prooftree}
    \end{minipage}
    \newline\newline\newline
    % -*>I^s
    \begin{minipage}{0.5\textwidth}
      \begin{prooftree}
        \AxiomC{$P => \SeFun{\phi}$}
        \AxiomC{$P \vdash \Gamma \geq \phi$}\noLine
        \BinaryInfC{$\ \ \ \ \ \ \ P \mid \Gamma \circledast x^{\emptyset}:\tau \vdashs M: v\ \ \ \ \ \ \ $} \RightLabel{[$\sepimp$I$^s$]}
        \UnaryInfC{$P \mid \Gamma \vdashs \lambda^{*}x. M : \phi \tau v$}
      \end{prooftree}
    \end{minipage}
    % -&>I^s
    \begin{minipage}{0.5\textwidth}
      \begin{prooftree}
        \AxiomC{$P => \ShFun{\phi}\ \ \ \ \ P \vdash \Gamma \geq \phi$}\noLine
        \UnaryInfC{$P \mid \Gamma^{[\emptyset \mapsto \{x\}]} \varoplus x^{\texttt{Vars}(\Gamma)}:\tau \vdashs M: v$}\RightLabel{[$\rightarrow$I$^s$]}
        \UnaryInfC{$P \mid \Gamma \vdashs \lambda^{\alpha}x. M : \phi \tau v$}
      \end{prooftree}
    \end{minipage}
    \newline\newline\newline
    % App^s
    \begin{minipage}{1.0\textwidth}
      \begin{prooftree}
        \AxiomC{$P \mid \Gamma \circledast \Delta \vdashs M: \phi v \tau\ \ \ \ \
          P \mid  \Gamma' \circledast \Delta \vdashs N: v$}
        \AxiomC{$P \vdash \Delta\ \texttt{un}$}\noLine
        \BinaryInfC{$(\Gamma \varoplus \Gamma' \wedge (P => \ShFun{\phi}))
          \vee (\Gamma \circledast \Gamma' \wedge (P => \SeFun{\phi}))$}\RightLabel{[App$^s$]}
        \UnaryInfC{$P \mid \Gamma \sqcup \Gamma' \circledast \Delta \vdashs M N: \tau$}
      \end{prooftree}
    \end{minipage}
    % sum types
    % \begin{minipage}{0.5\textwidth}
    %   \begin{prooftree}
    %     \AxiomC{$P \mid  \Gamma \vdashs M: \tau$}\RightLabel{[$\oplus_l$I$^s$]}
    %     \UnaryInfC{$P \mid \Gamma \vdashs \texttt{inl}\ M: \tau \oplus \tau' $}
    %   \end{prooftree}
    % \end{minipage}
    % \begin{minipage}{0.5\textwidth}
    %   \begin{prooftree}
    %     \AxiomC{$P \mid  \Gamma \vdashs M: \tau'$}\RightLabel{[$\oplus_r$I$^s$]}
    %     \UnaryInfC{$P \mid \Gamma \vdashs \texttt{inr}\ M: \tau \oplus \tau' $}
    %   \end{prooftree}
    % \end{minipage}
    % \begin{minipage}{1.0\textwidth}
    %   \begin{prooftree}
    %     \AxiomC{$P \vdash  \Delta\ \texttt{un}\ \ \ \ \ 
    %       P \mid  \Gamma \circledast \Delta \vdashs M: \tau \oplus \tau'$}\noLine
    %     \UnaryInfC{$P \mid  \Gamma'_x \circledast \Delta,x:\tau \vdashs N: v\ \ \ \ \
    %       P \mid  \Gamma'_y \circledast \Delta, y:\tau' \vdashs N': v$}\RightLabel{[$\oplus$E$^s$]}
    %     \UnaryInfC{$P \mid (\Gamma \sqcup \Gamma') \circledast \Delta \vdashs \Case{M}{\{\texttt{inl}\ x \mapsto N; \texttt{inr}\ y \mapsto N'\}}:v$}
    %   \end{prooftree}
    % \end{minipage}
    % \begin{minipage}{1\linewidth}
    %   \TODO{get rid of the red symbols}
    % \end{minipage}
    \end{framed}
  \caption{Syntax Directed Typing Rules}
  \label{fig:syntax-typing-rules}
\end{figure}

The [Let$^s$] rule defines an expression within another expression locally i.e. x would
not be in scope other than using it in $N$. We divide the typing context into multiple parts.
$\Gamma$ contains the variables that exists exclusively in $M$ and $\Gamma'$ which
are exclusively in $N$. $\Gamma''$ is common to both $M$ and $N$ while $\Delta$ is not used in both the expressions.
Thus $\Gamma$ and $\Gamma'$ will be completely separate from each other while $\Gamma''$ would be in sharing with both $\Gamma$ and $\Gamma'$.
$\Delta$ would have to be unrestricted as it is not being used either in $M$ or in $N$. The sharing of $x$ with $\Gamma$ would depend on
whether $\Gamma$ and $\Gamma'$ are completely disjoint or empty. For the application rule [App$^s$] the type assignment $\Gamma$ would contain
variables for $M$ and $\Gamma'$ for $N$. If they are completely separate, it would be a separating function application and $M$ would be
assigned a type $\tau' \sepimp \tau$ else, they would have to be completely sharing and M would be assigned a type $\tau' \rightarrow \tau$.
The conditions do not clearly look as if they are syntax directed and in the cases where the type checker cannot directly infer if the
resources used are separate or shared, the user would be expected to provide it using type annotations  {\color{red} meh.}.

\begin{theorem}[Soundness of $\vdash^s$]
  If $P \mid \Gamma \vdash^s M:\tau$ then $P \mid \Gamma \vdash M : \tau$
\end{theorem}
The soundness property captures the essence that derivations in the syntax directed type system follow the original type system.
The proof is by induction on derivation of $P \mid \Gamma \vdash^s M:\tau$ and is detailed in \cref{prf:soundness-syntax-directed}.

The syntax directed typing system is however not complete i.e. the syntax directed typing may not be able to derive the same
typing judgement as the original system. \TODO{What is the reason? How does this relate to future work? needs elaboration}

\section{Type Inference and Algorithm $\M$}\label{sec:algorithm-m}
We now describe the type inference algorithm based on the previously defined syntax directed
type system. We use a variation of algorithm $\M$ [\cite{lee_proofs_1998}] for type inference.
We address three independent concerns in the type inference algorithm.
The first being treatment of polymorphism to be same as Hindley-Milner style. The second, we introduce
\texttt{Un} predicates for types that are unrestricted. We track this with the help of carrying a
collection of used variables throughout the algorithm which detects whether a variable is discarded
or used multiple types. The third being accounting for sharing of the variables.
The complete algorithm is outlined in \cref{fig:algorithm-m}. The input to the algorithm includes the term $M$ whose
type has to be inferred, $\tau$ is the expected type of the term, $S$ is the current substition and the
sharing information $\Psi$. The output includes the set of new predicates $P$ that are generated,
the new set of substitutions $S'$, the used variables $\Sigma$, and the new sharing information $\Psi'$.
The type variables $u$ with a subscript denote fresh variables.

We now formally define $\Psi$, the sharing information that is defined as a pair of variable and a collection of variables it is in sharing with.
$\Psi(x)$ will denote all the variables that $x$ is in sharing with.  Domain of $\Psi$ defined as
$\texttt{dom}(\Psi) = \{x \mid (x, \bar{y}) \in \Psi \}$, where $\bar{y}$ is the collection of variables shared with x.
One can think of $\Psi$ similar to $\Gamma$, but instead
of the type of the variable it contains the sharing information.
Extending the sharing for a variable will be denoted by $\Psi(x) + y$, which would mean the variable $y$ is in sharing with $x$. Because the sharing
operation is a reflexive, symmetric and transitive we observe certain properties given below to be true.
\begin{flalign*}
 &\forall_{x \in \texttt{dom}(\Psi)}\ x \in \Psi(x) \tag{reflexive}\\
 &\forall_{x,y \in \texttt{dom}(\Psi)}\ \text{if}\ y \in \Psi(x)\ \text{then}\ x \in \Psi(y) \tag{symmetric}\\
 &\forall_{x,y,z\in \texttt{dom}(\Psi)}\ \text{if}\ y \in \Psi(x) \wedge z \in \Psi(y)\ \text{then}\ z \in \Psi(x)\tag{transitive}
\end{flalign*}
Thus sharing relation defines an equivalence relation on $\texttt{dom}(\Psi)$.

{\color{red} needs more formalization? state some
properties. Similar to what we do while defining $\Gamma$}.

We define some auxilary functions in \cref{fig:aux-defs} to lift predicates into the type system.
$\texttt{Weaken}(x, \sigma, \Sigma)$ adds the unrestricted predicate to the type $\sigma$
if it does not belong to $\Sigma$. $\texttt{Un}(\Gamma)$ adds an unrestricted
predicate to all the types of the variables that are in the domain of $\Gamma$. $\mathcal{C}(\Psi, \Sigma)$ is the transitive
closure of sharing relation for all the variables that are in sharing with all the variables in $\Sigma$.

% Explain each case of algorithm M here.
% Var
The first case describes the variable case, where we are given the variable identifier and
the expected type. We try to unify the expected type $\tau$ with the derived type scheme
$\nu$ from the type assignment $\Gamma$. The return values of the algorithm are
a new set of predicates which are nothing but instantiated version of the predicates
obtained from the typing scheme, the variable $x$ being used and the new
substitution which is combination of the unification algorithms output and the original
substitution. There is no change in the sharing information and $\Psi$ is returned as is.

% \&y
In next case of sharing function introduction rule $\lambda^{\alpha}x.M$, as per the [$\rightarrow$I] rules, the entitites returned
are union of four predicate categories. The first is assigning a the predicate of the function to be
a sharing function $\ShFun{u_1}$, where $u_1$ is a new type variable for the type of function argument $x$.
The second assigning the function to be less restricting than the other variables in the typing assignment $\Gamma$.
The third, assigning an unrestricted predicate to the binding variable $x$ if it has not been used
anywhere in the lambda body $M$, which is done by the $\texttt{Weaken}$ function and the fourth
being the predicates generated by recursively type checking the body of the lambda expression $M$.
We create sharing links for $x$ with all the variables within the typing context $\Gamma$ in updated
$\Psi''$.
% \*x
The case of separating function $\lambda^{*}x. M$ is very similar to the previous case of $\lambda^{\alpha}x. M$ except
the that there is no sharing information to be updated as the argument to the function is
separate from its body and the function predicate assigned is $\SeFun{u_1}$ to denote this very
separating relation instead of $\ShFun u_1$

% App
In the application case the algorithm typechecks the subexpressions $M$ as a function type having
an input of the type of $N$. The additional check done here is to identify whether $M$ has a sharing
application or a separating application. If all the variables used in $M$ are also used in $N$ then
it is a sharing application i.e. $M$ is assigned a sharing function predicate $\ShFun{}$
else if the used variables in $M$ are disjoint to the variables that
are shared by $N$ or if the variables shared by variables in $M$ are disjoint to the variables
that are used in $N$, $M$ is assigned the predicate type $\SeFun{}$. Incase of a separating function
application, the variables that are used in both are marked as unrestricted. This is captured by
$\texttt{Un}(\Gamma|_{\Sigma \cap \Sigma'})$. $\Gamma|_{\Sigma}$ means the typing assignment $\Gamma$
restricted to the variables in $\Sigma$.
% Let
In the polymorphic \texttt{let} case we first check for the type of the expression $M$ and
ensure that the variable binding $x$ is not usd in it to avoid recursive definition which would
lead to infinite types. We then generalize the type to $\sigma$ and then check the type of expression $N$
by expanding the type assignment $\Gamma$ with the variable $x$ and type scheme $\sigma$.

% \begin{framed}\centering
%   $\tau \sim \tau' => \exists S. S \tau = \tau'$
% \end{framed}

\begin{figure}[h]
  \begin{framed}
    \begin{minipage}[ht]{1\linewidth}
      \centering
      \fbox{
        $\M(S, \Psi, \Gamma \vdash M : \tau) = P, S', \Sigma, \Psi'$
      }
    \end{minipage}
    % x var
    \begin{minipage}{1\linewidth}
      \begin{flalign*}
        \M(S, \Psi, \Gamma \vdash x : \tau) &= ([\vec{u} / \vec{t}]P), S' \circ S, \{x\}, \Psi \\
        \text{where}\ (x : \forall \vec{t}. P => \nu) &\in S \Gamma \\
        S' &= \Unf([\vec{u} / \vec{t}]\nu, S \tau)
      \end{flalign*}
    \end{minipage}

    % \&x. M: t
    \begin{minipage}{1\linewidth}
      \begin{flalign*}
        \M(S, \Psi, \Gamma \vdash \lambda ^{\alpha} x. M : \tau) &= \{P \cup Q \}, S', \Sigma \backslash x, \Psi''  \\
        \text{where}\ P; S'; \Sigma; \Psi' &= \M(\Unf(\tau, u_1 u_2 u_3) \circ S, \Psi, \Gamma, x:u_2 \vdash M: u_3) \\
        Q &= \{\ShFun{u_1}\} \cup \text{Leq}(u_1, \Gamma|_{\Sigma}) \cup \text{Weaken}(x, u_2, \Sigma)\\
        \Psi'' &= \{\forall_{y \in \texttt{dom}(\Psi')}. \Psi'(y) + x\} \cup \{x \mapsto \{ x \}\}
      \end{flalign*}
    \end{minipage}

    % \*x. M: t
    \begin{minipage}{1\linewidth}
      \begin{flalign*}
        \M(S, \Psi, \Gamma \vdash \lambda ^{*} x. M : \tau) &= \{ P \cup Q \}, S', \Sigma \backslash x, \Psi''\\
        \text{where}\ P; S'; \Sigma; \Psi' &= \M(\Unf(\tau, u_1 u_2 u_3) \circ S, X; \Gamma, x:u_2 \vdash M: u_3) \\
        Q &= \{\SeFun{u_1}\} \cup \text{Leq}(u_1, \Gamma\mid_{\Sigma}) \cup \text{Weaken}(x, u_2, \Sigma)\\
        \Psi'' &= \Psi' \cup \{x \mapsto \{ x \}\}
      \end{flalign*}
    \end{minipage}

    % M N: t
    \begin{minipage}{1\linewidth}
      \begin{flalign*}
        \M(S, \Psi, \Gamma \vdash M N : \tau) &= \{ P \cup P' \cup Q \}, R', \Sigma \cup \Sigma', \Psi'' \\
        \text{where}\ P; R; \Sigma; \Psi' &= \M(S, \Psi, \Gamma \vdash M:  u_1 u_2 \tau) \\
        P'; R'; \Sigma'; \Psi'' &= \M(S R, \Psi', S \Gamma \vdash N: u_2)\\
        \text{if}\ \mathcal{C}(\Gamma, \Psi'', \Sigma) &= \mathcal{C}(\Gamma, \Psi'', \Sigma')\\
        \text{then}\ Q &= \{\ShFun{u_1}\} \\
        \text{else}\ \text{if}\ &(\Sigma \# \mathcal{C}(R\Gamma, \Psi'', \Sigma')\ \text{and}\ \Sigma' \# \mathcal{C}(R\Gamma, \Psi'', \Sigma))\\
        &\text{then}\ Q = \{\SeFun{u_1}\}\\
        \text{else}\ Q &= \{\SeFun{u_1}\} \cup \text{Un}(\Gamma|_{\Sigma \cap \Sigma'})
      \end{flalign*}
    \end{minipage}

    % let x = M in N: t
    \begin{minipage}{1\linewidth}
      \begin{flalign*}
        \M(S, \Psi, \Gamma \vdash \Let{x}{M}{N} : \tau) &= (P \cup Q), R', \Sigma \cup \{\Sigma' \backslash x \}, \Psi'' \\
        \text{where}\ P; R; \Sigma; \Psi' &= \M(S, \Psi, \Gamma \vdash M:u_1)  \\
        \sigma &= \text{GenI}(R\Gamma; R(P => u_1)) \\
        P'; R'; \Sigma'; \Psi'' &= \M(R, \Psi', \Gamma, x:\sigma \vdash N : \tau) \\
        Q &= \text{Un}(\Gamma|_{\Sigma \cap \Sigma'}) \cup \text{Weaken}(x, \sigma, \Sigma')
      \end{flalign*}
    \end{minipage}

    % \begin{minipage}{1\linewidth}
    %   \begin{flalign*}
    %     \M(S, X;\Gamma \vdash \Case{M}{\{p_i \mapsto N_i\}_i}) &= (P_M \cup \{P_i\}), Q, \Sigma_M \cup \{\Sigma_i\} \nonumber\\
    %     \text{where}\ P_M, R_M, \Sigma_M &= \M(S, X;\Gamma \vdash M:u_1 \oplus u_2 \oplus \cdots \oplus u_n) \nonumber \\
    %     P_1, R_1, \Sigma_1 &= \M(R_M, X;\Gamma \vdash p_1:u_1 \vdash N_1: v) \nonumber \\
    %     P_i, R_i, \Sigma_i &= \M(R_{i-1}, X;\Gamma \vdash p_i:u_i \vdash N_i: v) \nonumber \\
    %     \Sigma' &= \bigcup_{i,j \leq n} (\Sigma_i \cap \Sigma_{j}) \nonumber\\
    %     Q &= \text{Un}(\Gamma|_{\Sigma'}) \cup \{\text{Weaken}(p_i, u_i, \Sigma_i)\}
    %   \end{flalign*}
    % \end{minipage}
  \end{framed}
  \caption{Type Inference Algorithm $\mathcal{M}$}
  \label{fig:algorithm-m}


\end{figure}
\begin{figure}[h]
  \begin{framed}
    \begin{minipage}{0.5\linewidth}
      \begin{flalign*}
        Leq(\phi, \Gamma)  = \bigcup_{(x:\tau) \in \Gamma} \{P \mid P \vdash \phi \leq \tau \}
      \end{flalign*}
    \end{minipage}
    \begin{minipage}{0.5\linewidth}
      \begin{flalign*}
        \texttt{Un}(\Gamma)  = \bigcup\{P \mid (y:\sigma) \in \Gamma, P \vdash \sigma\ \texttt{un} \}
      \end{flalign*}
    \end{minipage}
    \begin{minipage}{0.5\linewidth}
      \begin{flalign*}
        \texttt{Weaken}(x, \sigma, \Sigma)  = \begin{cases}
          P\ \ \ \ &\text{if}\ x \notin \Sigma, P \vdash \sigma\ \texttt{un}\\
          \emptyset\ \ \ &otherwise
        \end{cases}
      \end{flalign*}
    \end{minipage}
    \begin{minipage}{0.5\linewidth}
      \begin{flalign*}
        \texttt{GenI}(\Gamma, P &=> \tau)  = \forall (ftv(S P, \tau)).S P => \tau \nonumber\\
        \text{where}\ &S\ \text{improves}\ \texttt{ftv}(P) \backslash \texttt{ftv}(\Gamma, \tau)\ \text{in}\ P
      \end{flalign*}
    \end{minipage}
    \begin{minipage}{1\linewidth}
      \begin{flalign*}
        \mathcal{C}(\Psi, \Sigma)  = \bigcup_{x \in \texttt{dom}(\Sigma)} \Psi(x)
      \end{flalign*}
    \end{minipage}
  \end{framed}
  \caption{Auxiliary definitions}
  \label{fig:aux-defs}
\end{figure}

% \begin{figure}[h]
%   \begin{framed}
%     \begin{minipage}{1\linewidth}
%       \begin{flalign*}
%         \M(S, X;\Gamma \vdash C\ x) &= (P_M \cup \{P_i\}), Q, \Sigma_M \cup \{\Sigma_i\} \nonumber\\
%         \text{where}\ C &= \forall \vec{t_1}. (\forall \vec{t_2}. \exists \vec{t_3}. Q => v') \sepimp v \nonumber\\
%         P_M, R_M, \Sigma_M &= \M(S, X;\Gamma \vdash M:u_1 \oplus u_2 \oplus \cdots \oplus u_n) \nonumber \\
%         P_1, R_1, \Sigma_1 &= \M(R_M, X;\Gamma \vdash p_1:u_1 \vdash N_1: v) \nonumber \\
%         P_i, R_i, \Sigma_i &= \M(R_{i-1}, X;\Gamma \vdash p_i:u_i \vdash N_i: v) \nonumber \\
%         \Sigma' &= \bigcup_{i,j \leq n} (\Sigma_i \cap \Sigma_{j}) \nonumber\\
%         Q &= \text{Un}(\Gamma|_{\Sigma'}) \cup \{\text{Weaken}(p_i, u_i, \Sigma_i)\}
%       \end{flalign*}
%     \end{minipage}
%   \end{framed}
%   \caption{Type Inference Algorithm $\mathcal{M}$ (continued)}
%   \label{fig:algorithm-m-cont}
% \end{figure}

% Here is how the naive recursive algorithm works

% For Lambda
% \begin{enumerate}
% \item Get the bound variable
% \item Assign a new type for bound variable
% \item assign a new type variable for the body
% \item Evaluate type of the body and assign it to the body type variable
% \item return the type as (type of var) -> (type of body)
% \end{enumerate}

% Body can be made up of another lambda. In this case the same 5 steps will be triggered
% or it is an application of 2 or more variables
% In case of application 2 new type variable:
% \begin{enumerate}
% \item introduce type variables for left expression and right expression
%   left expression a type $A \rightarrow B$ right expression gets the type $A$
% \item recursively type check both of them
% \item return the type of the complete application as $B$ if the type checking in previous step is successfully
% \end{enumerate}


% In \qub{} there are 2 kinds of lambdas:
% \begin{enumerate}
% \item Sharing Lambda $\lambda^{\alpha}$
% \item Separating Lambda $\lambda^{*}$
% \end{enumerate}
% This helps us specify if the 2 variables are separated or have sharing
% of resources between them.
% The church encoding of a sharing pair will be represented as\\
% $shPair = \lambda^{*} x \rightarrow \lambda^{\alpha} y \rightarrow \lambda^{*} sh \rightarrow sh\ x\ y$\\
% This means that x and y may share resources
% A separating pair is represented as\\
% $sePair = \lambda^{*} x \rightarrow \lambda^{*}y \rightarrow \lambda^{*}se \rightarrow se\ x\ y$\\
% This means that x and y do not share resources.
% % By default we may assume that resources are always separating unless explicitly specified
% % that they are sharing.

\section{Implementing Algorithm $\M$}
The typing environment in standard Milner-Damas algorithm
is a pair of identifier and its type.
We need to modify the typing environment so that it describes sharing.
% There can be many different ways of doing it.
In the current implementation we have
extended the typing environment to hold 2 more entities along with the
type of the identifier, a list of list of identifiers --- that describes the sharing of variables,
a scope tag---that identifies if the variable is global in the complete module or local to the definition.
Global variables can be used anywhere in the file or other code file if it is imported
All function names will be defaulted to global scope.
Local variables can be used only after they have been bound in the typing environment.
The new typing environment can be realized as:
\begin{minted}{haskell}
  type Env = Map Id (Type, [[Id]], Scope)
\end{minted}

% % how is the list of list of ids help in identifying sharing

% % how do you define a closure

% % What do you mean by having a break in the closure

% % The used field in the type-checker state

% \section{Modification to Typechecking Algorithm}

% To incorporate the sharing, we have to modify the typechecking algorithm.
% The main reason to do that is to avoid adding $Un$ predicates to
% shared variables that are not used. Take an example of $fst$ function
% that returns the first argument of a shared pair
% \begin{verbatim}
% fst = \x -> \&y -> x
% \end{verbatim}
% In this case y should not be assigned an $Un$ predicate because
% it is shared with x and x is indeed used.

% The type checking algorithm has to be tweaked so that we keep track
% of what all variables are shared and which ones are separate.

% When we encounter an Alpha Lambda
% \begin{enumerate}
% \item Get the bound variable
% \item Assign a new type for bound variable
% \item Assign a new type variable for the body
% \item add the variable in a sharing context
% \item Evaluate type of the body and assign it to the body type variable
% \item return the type as (type of var) -> (type of body)
% \end{enumerate}


% % The main problem is that we do not know when a sharing variable should be kept in scope
% % or removed. In some cases we may need to keep it around for introducing the (>:=) predicates
% % Some cases are 2 types:
% % 1) The variable is used
% % 2) The variable is not used

% The type checking calls occur from left to right recursively.
% While going down the recursion we keep on adding variables to the environment
% We stop recursing at application. where we compute whether we have complete sharing to assign ShFun or SeFun.

% While folding out of of the recursion we do 2 things:
% 1) keep track of the used variables (including implicitly used variables due to sharing)
% 2) generate new goals
% a) depending on whether the the introduced variable was used
% or its sharing member was used. Assign Un predicates and weaken if it is not used at all
% b) introduce lesser-restricted predicates
% 3) Generate other new assumptions

% We cannot determine when to get rid of the complete bunch.
% \begin{verbatim}
% \z -> \*x -> \&y -> y
% \end{verbatim}
% hence, when we are folding out of the final recursive but 1 call ie. for Lambda bound variable x

% \section{Sharing}
% What do we exactly mean by sharing?
% There are 2 interpretations of sharing that i can think of
% 1) We have a resource $\mathcal{R}$, and 2 pointers $\alpha$, $\beta$. we say $alpha$, $\beta$ share if both of them point to the same resource $\mathcal{R}$
% $\mathcal{R}$ is never exposed to the user space and can be manipulated only by using $\alpha$, $\beta$.
% 2) We have resource

%%% Local Variables:
%%% mode: latex
%%% TeX-master: "../thesis-ku"
%%% End:
       % typing rules, syntax directed typing rules and Algorithm M
\chapter{\qub{} Extention and Datatypes}\label{chp:datatypes}

In this chapter we discuss how \qub{} can be extended to have a kind system which makes
the type language powerful enough to accept user defined datatypes\cref{sec:kind-system}. We also discuss
how we can encode sums and multiplicative and additive products in \cref{sec:pairs-types}.


\section{Kind System}\label{sec:kind-system}
In the original system, there is no concept of type constructors, it would be tedious to add in
new types into the system as we would have to introduce new syntax and associated typing rules for
each of the new syntax. The kind system alleviates this tedious process by adding in the language
of type constructors. This idea was introduced by Barendregt \citeyearpar{barendregt_1991} and
is desribed in detail by Jones \citeyearpar{jones_system_1993}. We will follow Jones' approach
to add the language of type constructors and kinds in our system. The modified type system is shown in \cref{fig:qub-kind-extension}. All types have kind $\star$ and the
kind of the type constructors depends on its arity. The language of type constructors is extendible
by using a specialized syntax on surface level. In Haskell, this is done by using \mintinline{Haskell}{class} keywork.
and then declaring datatypes as instances of the defined class.

\begin{figure}[h]
  \begin{framed}
    \begin{flalign*}
      \text{Type Variables}\ \ t, u, \upsilon         &\in \text{Type Variables}  \\
      \text{Kinds}\ \ \               \kappa          &::= \kappa \mid \kappa' \rightarrow \kappa\\
      \text{Types}\ \ \             \tau^{\kappa}      &::= t^\kappa \mid T^{\kappa} \mid \tau^{\kappa' \rightarrow \kappa}\tau^{\kappa'}\\
      \text{Type Constructors}\ \ \   T^{\kappa}       &\in \mathcal{T}^{\kappa}\ 
                         \text{where}\ \{\otimes, \with, \oplus, \overset{!}{\sepimp}, \sepimp, \xrightarrow{!}, \rightarrow \} \subseteq \mathcal{T}^{\star \rightarrow \star \rightarrow \star}\\
      \text{Predicates}\ \ \          \pi             &::= \texttt{Un}\ \tau \mid \texttt{SeFun}\ \tau \mid \texttt{ShFun}\ \tau \mid \tau \geq \tau' \\
      \text{Qualified Types}\ \ \     \rho            &::= \tau \mid \pi => \rho \\
      \text{Type schemes}\ \ \        \sigma          &::= \rho \mid \forall t. \sigma
    \end{flalign*}
  \end{framed}
  \caption{Extended \qub{} Types and Kinds}
  \label{fig:qub-kind-extension}
\end{figure}

Addition of a kind system changes the treatement to the type inference algorithm in some detail.
All types and type constructors have to be annotated with their kind. $T^\kappa$ denotes type constructors
and $\tau^{\kappa' \rightarrow \kappa}\tau^{\kappa'}$ denotes application of types.
The type constructor application rule that computes kinds is given in \cref{fig:constructor-application} where
$\tau$ is of kind $\kappa' \rightarrow \kappa$ and $\tau'$ is of kind $\kappa'$. The application
of both the constructors would result in a kind $\kappa$.
$\sepimp$, $\overset{!}{\sepimp}$, $\rightarrow$ and $\overset{!}{\rightarrow}$
are now treated as type constructors with an arity of two and would have a kind $\star \rightarrow \star \rightarrow \star$ while
type like Int and Float will have a kind $\star$.

\begin{figure}
  \begin{framed}
    \begin{prooftree}
      \AxiomC{$\tau :: \kappa' \rightarrow \kappa$}
      \AxiomC{$\tau' :: \kappa'$}
      \BinaryInfC{$\tau \tau' :: \kappa$}
    \end{prooftree}
  \end{framed}
  \caption{Constructor Application Rule}
  \label{fig:constructor-application}
\end{figure}

The unification of types is now done via a modified version of Robinson's algorithm \citeyearpar{robinson_machine-oriented_1965}
is used in order to deduce the most general unifier for type constructors.
Formally we define $S$ to be the {\it most general unifier} for type constructors $T$ and $T'$ if:
\begin{enumerate}
  \item $S$ is a unifier for $T$ and $T'$.
  \item For every unifier $S'$ of $T$ and $T'$ we can write $T'$ in a form of
    $R S$ where $R$ some kind preserving substitution.
\end{enumerate}
We write $T \overset{S}{\sim}_{\kappa} T'$ for assertion that $S$ is the unifier
of the constructor types $T, T' \in T^{\kappa}$. The rules in \cref{fig:unify-types}
describe the unification algorithm for type constructors. [KVar] and [KVar'] contain
and additional constraint of the type variable $t$ to not be free in the type constructor's $T$
type variables to ensure the unification does not lead to infinite types. The [KApply] rule
states that type constructors of the form $T T'$ can be unified with $H H'$ only if $T$ and $H$
can be unified which asserts that they have to have the same kind $\kappa' \rightarrow \kappa$.

\begin{figure}[h]
  \begin{framed}
    \begin{minipage}[l]{0.5\linewidth}
      \begin{flalign*}
        t \overset{id}{\sim}_{\kappa} t \tag{[ID-KVar]}
      \end{flalign*}
    \end{minipage}
    \begin{minipage}[r]{0.5\linewidth}
      \begin{flalign*}
        T \overset{id}{\sim}_{\kappa} T \tag{[ID-KConst]}
      \end{flalign*}
    \end{minipage}
    \begin{minipage}[l]{0.5\linewidth}
      \begin{flalign*}
        t \overset{[C/t]}{\sim}_{\kappa} T, t \notin \texttt{fvs}(T) \tag{[KVar]}
      \end{flalign*}
    \end{minipage}
    \begin{minipage}[r]{0.5\linewidth}
      \begin{flalign*}
        T \overset{[C/t]}{\sim}_{\kappa} t, t \notin \texttt{fvs}(T) \tag{[KVar']}
      \end{flalign*}
    \end{minipage}
    \begin{minipage}[c]{1.0\linewidth}
      \begin{prooftree}
        \AxiomC{$T \overset{S}{\sim}_{\kappa'\rightarrow \kappa} T'$}
        \AxiomC{$S H \overset{S'}{\sim}_{\kappa'} S H'$}\RightLabel{([KApply])}
        \BinaryInfC{$T T' \overset{SS'}{\sim}_{\kappa} H H'$}
      \end{prooftree}
    \end{minipage}
  \end{framed}
  \caption{Kind Preserving Unification of Type Constructors}
  \label{fig:unify-types}
\end{figure}


\begin{figure}[h]
  \begin{framed}
    \begin{flalign*}
      \text{Term Variables}\ \ \  x, y, z  &\in \text{Var}\\
      \text{Expressions}\ \ \     M, N     &::= x \mid \lambda^{*}x. M \mid \lambda^{\alpha}x. M \mid M N \mid \Let{x}{M}{N}\\
      &\mid \Pair{M,N} \mid \Let{\Pair{x,y}}{M}{N} \mid \Pair{M;N} \mid \texttt{fst}(M) \mid \texttt{snd}(M)\\
      &\mid \Case{M}{\{\texttt{inl}\ x \mapsto N ; \texttt{inr}\ y \mapsto N'\}}\mid \texttt{inl}\ x \mid \texttt{inr}\ y\\
    \end{flalign*}
  \end{framed}
  \caption{Extended \qub{} Language Syntax}
  \label{fig:ext-qub-terms}
\end{figure}

\TODO{add more details of how this helps and proofs remain the same}

\section{Pairs and Sums in  \qub{}}\label{sec:pairs-types}
Introduction of two kinds of arrows in our type system leads to different flavors of pairs. This distinction cannot be made
in intuitionistic logic as the structural rules allow re-use of propositions. But due to restrictions in weakening and
contraction we obtain two kinds of pairs, additive and multiplicative. In this section we illustrate how the
extented \qub{} can be used to introduce new types. We introduce syntax and type constructors for multiplicative pairs in \cref{subsec:mul-pairs}
and the same for additive pairs in \cref{subsec:mul-pairs}. We then introduce the
syntax and type constructors for sum types in \cref{subsec:sums} and verify that they indeed work as expected.

\subsection{Multiplicative Pair Type}\label{subsec:mul-pairs}
Lambda encoding of muliplicative pairs is given in \cref{fig:mul-pair}. The typing rules
are given in \cref{fig:mul-pair-rules}. The significance of a multiplicative pair can be thought of as
having separate resouce entities together in the program environment context and they would have to be
explicitly disposed off. Failure to do so, would raise a type error regarding either of the resources
not being unrestricted. This is different from an additive pair where we can easily use projection functions
\texttt{fst} or \texttt{snd} to retrieve one of the components from the pair and convinently forget about the other.

\begin{figure}[h]
  \begin{framed}
    \begin{flalign*}
      \otimes &\in \mathcal{T}^{\star \rightarrow \star \rightarrow \star}\\
      \otimes &= \tau \sepimp \tau' \sepimp (\tau \sepimp \tau' \sepimp \upsilon) \sepimp \upsilon\\
      (,) &= \lambda^{*} x. \lambda^{*} y. \lambda^{*} f. f x y
    \end{flalign*}
  \end{framed}
\caption{Multiplicative Pair}
\label{fig:mul-pair}
\end{figure}
\begin{figure}[h]
  \begin{framed}
    \begin{minipage}{1\linewidth}
      \begin{prooftree}
        \AxiomC{$P \mid \Gamma  \vdash M : \tau$}
        \AxiomC{$P \mid \Delta \vdash N: \tau'$} \RightLabel{$[\otimes I]$}
        \BinaryInfC{$P \mid \Gamma \circledast \Delta \vdash \Pair{M,N}: \tau \otimes \tau'$}
      \end{prooftree}
    \end{minipage}
    \begin{minipage}{1\linewidth}
      \begin{prooftree}
        \AxiomC{$P \mid \Gamma \vdash M: \tau \otimes \tau'$}
        \AxiomC{$P \mid \{ x^{\{\vec{z} \mid \vec{z} \subseteq \texttt{Vars}(\Gamma')\}}:\tau \} \circledast
          \{ y^{\{\bar{z'} \mid \bar{z'} \subseteq \texttt{Vars}(\Gamma')\}} : \tau'\} \sqcup \Gamma'_{x,y}$}\RightLabel{$[\otimes E]$}
        \BinaryInfC{$P \mid \Gamma \sqcup \Gamma' \vdash (\Let{\Pair{x, y}}{M}{N}): \upsilon$}
      \end{prooftree}
    \end{minipage}
  \end{framed}
  \caption{Derivable Typing Rules for Multiplicative Pair}
  \label{fig:mul-pair-rules}
\end{figure}



\subsection{Additive Pair Type}\label{subsec:add-pairs}

\begin{figure}[h]
  \centering
  \begin{framed}
    \begin{flalign*}
      \with &\in \mathcal{T}^{\star \rightarrow \rightarrow \star}\\
      \with &= \tau \sepimp \tau' \rightarrow (\tau \sepimp \tau' \rightarrow \nu) \rightarrow \nu\\
      (;) &= \lambda^{*} x. \lambda^{\alpha} by. \lambda^{\alpha} f. f x y
    \end{flalign*}
  \end{framed}
  \caption{Additive Pair}
  \label{fig:add-pair}
\end{figure}
\begin{figure}[h]
  \begin{framed}
    \begin{minipage}{1\linewidth}
      \begin{prooftree}
        \AxiomC{$P \mid \Gamma  \vdash M : \tau$}
        \AxiomC{$P \mid \Delta \vdash N: \tau'$} \RightLabel{$[\with I]$}
        \BinaryInfC{$P \mid \Gamma \varoplus \Delta \vdash \Pair{M;N}: \tau \with \tau'$}
      \end{prooftree}
    \end{minipage}
    \begin{minipage}{.5\linewidth}
      \begin{prooftree}
        \AxiomC{$P \mid \Gamma \vdash M: \tau \with \tau'$} \RightLabel{$[\with E_1]$}
        \UnaryInfC{$P \mid \Gamma \vdash \texttt{fst}\ M: \tau$}
      \end{prooftree}
    \end{minipage}
    \begin{minipage}{.5\linewidth}
      \begin{prooftree}
        \AxiomC{$P \mid \Gamma \vdash M: \tau \with \tau'$} \RightLabel{$[\with E_2]$}
        \UnaryInfC{$P \mid \Gamma \vdash \texttt{snd}\ M: \tau'$}
      \end{prooftree}
    \end{minipage}
  \end{framed}
  \caption{Derivable Typing Rules for Additive Pair}
  \label{fig:add-pair-rules}
\end{figure}


\subsection{Sum Type}\label{subsec:sums}


\section{Generic Type constructors}\label{sec:type-constructors}
We add two new typing rules for handling generic datatypes using type constructors.
We only consider a specific case of type constructors where all the type variables
involved are either all shared or all separate and give an insight of how this
can be generalized to have a mix of sharing and separation between the type parameters
for a type constructor.


It gets tricky with the surface syntax.


\TODO{How do we write about type constructors?}
\begin{figure}[h]
  % sharing constructor
  \begin{framed}
\begin{minipage}{1\textwidth}
  \begin{prooftree}
    \AxiomC{$P \mid \Gamma \vdash C: \tau_1 \sepimp \tau_2 \rightarrow \tau_3 \rightarrow \dots \rightarrow \tau_n \rightarrow T$}
    \AxiomC{$\Delta \equiv H(x_1 : \tau_1 ; x_2 : \tau_2 ; \ldots ; x_n:\tau_n)$}\RightLabel{$[\text{C-sh}]$}
    \BinaryInfC{$P \mid \Gamma; \Delta \vdash C \vec{x} : T $}
  \end{prooftree}
\end{minipage}
% Separating constructor
\begin{minipage}{1\textwidth}
  \begin{prooftree}
    \AxiomC{$P \mid \Gamma \vdash C: \tau_1 \sepimp \tau_2 \sepimp \tau_3 \sepimp \dots \sepimp \tau_n \sepimp T$}
    \AxiomC{$\Delta \equiv H(x_1: \tau_1, x_2:\tau_2, \dots ,x_n:\tau_n)$}\RightLabel{$[\text{C-se}]$}
    \BinaryInfC{$P \mid \Gamma, \Delta \vdash C \vec{x} : T $}
  \end{prooftree}
\end{minipage}
\end{framed}
  \caption{User Defined Datatypes}
  \label{fig:ud-datatypes}
\end{figure}


% Having two arrows indroduces two types of pairs. A sharing pair (\texttt{Pair$_{sh}$}) and a separating pair (\texttt{Pair$_{sep}$}) depending on
% which arrow we use. The church encoding of both the pairs and their respective types are given in \ref{fig:bi-pairs-typing}.
% Derivation proofs of both the pairs are given in \cref{sec:pairs}.


% We can now introduce 2 types in our type language $\otimes$ and $\with$
% which represent separating pair and sharing pair respectively as shown in \cref{fig:pair-typing-rules}.
% We also have a sum type whose rules are shown in \cref{fig:bi-sum-types}.
% Proof derivation and definitions for sum type is given in \cref{sec:sums}.

\begin{figure}[h]
  \begin{framed}
    \begin{minipage}{1\linewidth}
      \begin{prooftree}
        \AxiomC{$P \mid \Gamma  \vdash M : \tau$}
        \AxiomC{$P \mid \Delta \vdash N: \tau'$} \RightLabel{$[\with I]$}
        \BinaryInfC{$P \mid \Gamma \oplus \Delta \vdash \Pair{M;N} : \tau \with \tau'$}
      \end{prooftree}
    \end{minipage}
  \end{framed}
  \caption{Derivable Typing rules for Pair types}
  \label{fig:pair-typing-rules}
\end{figure}

\begin{figure}[h]
\begin{framed}
% sum type I_l
\begin{minipage}{0.5\textwidth}
  \begin{prooftree}
    \AxiomC{$P \mid \Gamma \vdash M: \tau$}
    \UnaryInfC{$P \mid \Delta \vdash \texttt{inl}\ M: \tau \oplus \tau'$}
  \end{prooftree}
\end{minipage}
% sum type I_r
\begin{minipage}{0.5\textwidth}
  \begin{prooftree}
    \AxiomC{$P \mid \Gamma \vdash M: \tau_2$}
    \UnaryInfC{$P \mid \Delta \vdash \texttt{inr}\ M: \tau \oplus \tau'$}
  \end{prooftree}
\end{minipage}
% sum type E
\begin{minipage}{1\textwidth}
  \begin{prooftree}
    \AxiomC{$P \mid \Gamma \vdash M: \tau \oplus \tau'$}
    \AxiomC{$P \mid \Gamma \circledast x:\tau \vdash N :\nu$}
    \AxiomC{$P \mid \Gamma \circledast y:\tau' \vdash N':\nu$}\RightLabel{[$\oplus$E]}
    \TrinaryInfC{$P \mid \Gamma \vdash \Case{M}{\{\texttt{inl}\ x \mapsto N; \texttt{inr}\ y \mapsto N'\}} : v $}
  \end{prooftree}
\end{minipage}
\end{framed}
\caption{Derivable Typing Rules for Sum Type}
\label{fig:bi-sum-types}
\end{figure}


The datatypes are user defined types:
There are 2 main kinds of non-recursive datatypes

1) data Choice a b = L a | R b

where the types a and b do not appear in both the constructors.
Choice here is a sum type a + b. How can we handle sum types in the current scenario?
1 way is to have a and b both not share resources. in this case we will have
a and b to be both un restricted.

a and b cannot direclty share resources. As the meaning of a sum type is that
either a or b can exist and not both. So we will always have separating arrow
for the sum types

Write the elemination rules for Either

for functions the closure has to be equal.

2) data Pair a b = P a b

The constructor uses both the resources a and b
Here we can have 2 subtypes:

a and b share resources or can be completely different.
We would want the programmer to specify whether the arguments are shared or separate.

For current purposes this syntactical specification is given via a modified syntax

data SeparatingPair a b = MkSeP a, b
data SharingPair a b = MkShP !! a b

We can of-course mix and match the sharing or separation.

data MixTuple a b c d e = MkMixTuple a,b;c,d

The question now that needs to be answered is how should the sharing be interpreted?
There are various options.

1. interpreted right to left. i.e. MkMixTuple (a, (b; (c,d)))
   so the b is shared with separating pair (c,d) and a is separate than b

2. interpreted left to right. i.e. MkMixTuple (((a, b);c),d)
   so the c is shared with separating pair (a,b) and d is separate from c

3. Sharing ; has precedence over separating , i.e. MkMixTuple a,(b;c),d

4. Separating , has lower precedence over ; i.e. MkMixTuple (a,b);(c,d)

This should be handled during the AST generating while parsing.
The type checking algorithm should be robust enough
to perform type checking either of these 4 types of sharing as they are
all perfectly valid scenarios.

Recursive datatypes have some kind of structural similarities within themselves

data Tree a = Leaf a | Node a (Tree a) (Tree a)

data Tree a b = Leaf b | Node a (Tree a b) (Tree a b)

Open questions:
How do we specify sharing of data between nodes?
eg. How can we say that all the leaves have shared data? and the data in nodes is sparating


What happens when lesser unrestrictedness with datatypes? Choice a b >:= f


\section{Examples}

\subsection{Pairs}

Our language has a notion of 2 kinds of pairs. The pairs representing resources that do not share
are denoted using separating pair $A \otimes B$. This means that when ever 



%%% Local Variables:
%%% mode: latex
%%% TeX-master: "../thesis-ku.tex"
%%% End:
      % sharing pair and weakening example Monad etc
\chapter{Conclusion and Future Work}

missing pieces here.
Categorical model?
Sematic model?
automoatic detection of  addtitive and multiplicative types using polymorphism
\global\long\def\bibname{References}

\bibliographystyle{apalike2}
\bibliography{Biblio/allcites}

\appendix
\chapter{Derivations for Products and Sums}
We start this section by adding few auxilary defitions for terms and types from first principles.
By convention, we denote an empty typing context by $I$ and empty predicate context with $\emptyset$

\section{Derivable Typing Rules For Product Types (Addiditive and Multiplicative Pairs)}\label{sec:pairs}
Pairs can now have two meanings. Either they have sharing resources or they have separating resources. We define each
of them below.

\subsection{Additive Pairs}
If the resources are in sharing we say they are additive pairs.

\begin{minipage}[h]{1.0\linewidth}
  \begin{prooftree}
    \AxiomC{$$}\RightLabel{[ID]}
    \UnaryInfC{$\emptyset \mid y^{xf}:B \vdash y: B $}

    \AxiomC{$$}
    \UnaryInfC{$\emptyset \mid x^{fy}: A \vdash x: A$}\RightLabel{[ID]}

    \AxiomC{$$}\RightLabel{[ID]}
    \UnaryInfC{$\emptyset \mid f^{xy}: A \rightarrow B \rightarrow C \vdash f: A \rightarrow B \rightarrow C $}\RightLabel{[$\sepimp E$]}
    \BinaryInfC{$\emptyset \mid x^{xy}:A \varoplus f^{xy}: A \rightarrow B \rightarrow C \vdash f x: (B \rightarrow C)$}\RightLabel{[$\rightarrow E$]}

    \BinaryInfC{$\emptyset \mid y^{xf}:B \varoplus x^{yf}:A \varoplus f^{xy}:A \rightarrow B \rightarrow C \vdash f x y: C$}\RightLabel{[$EXCH$]}
    \UnaryInfC{$\emptyset \mid x^{yf}:A \varoplus y^{xf}:B \varoplus f^{xy}: A \rightarrow B \rightarrow C \vdash f x y: C$}\RightLabel{[$\rightarrow I$]}
    \UnaryInfC{$\emptyset \mid x^{y}:A \varoplus y^{x}:B \vdash \lambda^{\alpha}f. f x y: (A \rightarrow B \rightarrow C) \rightarrow C$}\RightLabel{[$\rightarrow I$]}
    \UnaryInfC{$\emptyset \mid x^{\emptyset}:A \vdash \lambda^{\alpha}y. \lambda^{\alpha}f. f x y: B \rightarrow (A \rightarrow B \rightarrow C) \rightarrow C$}\RightLabel{[$\sepimp$ I]}
    \UnaryInfC{$\emptyset \mid I \vdash \lambda^{*}x. \lambda^{\alpha}y. \lambda^{\alpha}f. f x y: A \sepimp B \rightarrow (A \rightarrow B \rightarrow C) \rightarrow C$}
  \end{prooftree}
\end{minipage}
\noindent
We can now define sharing (additive) pair as:
\begin{framed}\centering
    $; = \lambda^{*}x. \lambda^{\alpha}y. \lambda^{\alpha}f. f x y$
\end{framed}
and assign a new type to this pair and call it $\with$
\begin{framed}\centering
  $A \with B = (A \rightarrow B \rightarrow C) \rightarrow C$
\end{framed}

\noindent
We now define left and right projections or deconstructors for sharing pairs below:
\begin{minipage}[h]{1.0\linewidth}
  \begin{prooftree}
    \AxiomC{$$}\RightLabel{[ID]}
    \UnaryInfC{$\emptyset \mid x^{y}:A \vdash x: A $}\RightLabel{[WKN-SH]}
    \UnaryInfC{$\emptyset \mid x^{y}:A \varoplus y^{x}:B \vdash x: A $}\RightLabel{[$\rightarrow I$]}
    \UnaryInfC{$\emptyset \mid x^{\emptyset}:A \vdash \lambda^{\alpha}y. x : B \rightarrow A $}\RightLabel{[$\rightarrow$I]}
    \UnaryInfC{$\emptyset \mid I \vdash \lambda^{\alpha}x. \lambda^{\alpha}y. x: A \rightarrow B \rightarrow A$}
  \end{prooftree}
\end{minipage}
\begin{framed}
  \centering
  $\texttt{fst}_{sh} = \lambda^{\alpha}x. \lambda^{\alpha}y. x$
\end{framed}

\begin{minipage}[h]{1.0\linewidth}
  \begin{prooftree}
    \AxiomC{$$}\RightLabel{[ID]}
    \UnaryInfC{$\emptyset \mid y^{x}:B \vdash y: B $}\RightLabel{[WKN-SH]}
    \UnaryInfC{$\emptyset \mid x^{y}:A \varoplus y^{x}:B \vdash y: B $}\RightLabel{[$\rightarrow I$]}
    \UnaryInfC{$\emptyset \mid x^{\emptyset}:A \vdash \lambda^{\alpha}y: B \rightarrow B $}\RightLabel{[$\rightarrow I$]}
    \UnaryInfC{$\emptyset \mid I \vdash \lambda^{\alpha}x. \lambda^{\alpha}y. y: A \rightarrow B \rightarrow B$}
  \end{prooftree}
\end{minipage}

\begin{framed}\centering
    $\texttt{snd}_{sh} = \lambda^{\alpha}x. \lambda^{\alpha}y. y$
\end{framed}

\subsection{Multiplicative Pairs}
If the resources are separate we say they are multiplicative pairs.

\begin{minipage}[h]{1.0\linewidth}
  \begin{prooftree}
    \AxiomC{$$}\RightLabel{[ID]}
    \UnaryInfC{$\emptyset \mid y^{\emptyset}:B \vdash y: B $}

    \AxiomC{$$}\RightLabel{[ID]}
    \UnaryInfC{$\emptyset \mid x^{\emptyset}: A \vdash x: A$}

    \AxiomC{$$}\RightLabel{[ID]}
    \UnaryInfC{$\emptyset \mid f^{\emptyset}: A \sepimp B \sepimp C \vdash f: A \sepimp B \sepimp C $}\RightLabel{[$\sepimp E$]}
    \BinaryInfC{$\emptyset \mid x^{\emptyset}:A \circledast f^{\emptyset}: A \sepimp B \sepimp C \vdash f x: (B \sepimp C)$}\RightLabel{[$\sepimp E$]}

    \BinaryInfC{$\emptyset \mid y^{\emptyset}:B \circledast x^{\emptyset}:A \circledast f^{\emptyset}:A \sepimp B \sepimp C \vdash f x y: C$}\RightLabel{[$EXCH$]}
    \UnaryInfC{$\emptyset \mid x^{\emptyset}:A \circledast y^{\emptyset}:B \circledast f^{\emptyset}:A \sepimp B \sepimp C \vdash f x y: C$}\RightLabel{[$\sepimp I$]}
    \UnaryInfC{$\emptyset \mid x^{\emptyset}:A \circledast y^{\emptyset}:B \vdash \lambda^{*}f. f x y: (A \sepimp B \sepimp C) \sepimp C$}\RightLabel{[$\sepimp I$]}
    \UnaryInfC{$\emptyset \mid x^{\emptyset}:A \vdash \lambda^{*}y. \lambda^{*}f. f x y: B \sepimp (A \sepimp B \sepimp C) \sepimp C$}\RightLabel{[$\equiv$]}
    \UnaryInfC{$\emptyset \mid I \vdash \lambda^{*}x. \lambda^{*}y. \lambda^{*}f. f x y: A \sepimp B \sepimp (A \sepimp B \sepimp C) \sepimp C$}
  \end{prooftree}
\end{minipage}

We can now define separating (multiplicative) pair as:
\begin{framed}\centering
    $,= \lambda^{*}x. \lambda^{*}y. \lambda^{*}f. f x y$
\end{framed}
We assign a new type to this pair construct and call it $\otimes$
\begin{framed}\centering
  $A \otimes B = (A \sepimp B \sepimp C) \sepimp C$
\end{framed}

\begin{minipage}[h]{1.0\linewidth}
  \begin{prooftree}
    \AxiomC{$$}\RightLabel{[ID]}
    \UnaryInfC{$\Un{B} \mid x^{\emptyset}:A \vdash x:A $}

    \AxiomC{$$}
    \UnaryInfC{$\Un{B} => \Un{B}$}\RightLabel{[UN-$\Gamma$, UN-$\sigma$, UN-$\tau$]}
    \UnaryInfC{$\Un{B} \vdash y^{\emptyset}:B\ \texttt{un} $}\RightLabel{[WKN-UN]}

    \BinaryInfC{$\Un{B} \mid x^{\emptyset}:A \circledast y^{\emptyset}:B \vdash x: A $}\RightLabel{[$\sepimp I$]}
    \UnaryInfC{$\Un{B} \mid x^{\emptyset}:A \vdash \lambda^{*}y. x: B \sepimp A $}\RightLabel{[$\sepimp I$]}
    \UnaryInfC{$\Un{B} \mid I \vdash \lambda^{*}x. \lambda^{*}y. x: A \sepimp B \sepimp A$}
  \end{prooftree}
\end{minipage}

\begin{framed}\centering
    $\texttt{fst}_{sep} = \lambda^{*}x. \lambda^{*}y. x$
\end{framed}

\begin{minipage}[h]{1.0\linewidth}
  \begin{prooftree}
    \AxiomC{$$}\RightLabel{[ID]}
    \UnaryInfC{$\Un{A} \mid y^{\emptyset}:B \vdash y: B $}

    \AxiomC{$$}
    \UnaryInfC{$\Un{A} => \Un{A}$}\RightLabel{[UN-$\Gamma$, UN-$\sigma$, UN-$\tau$]}
    \UnaryInfC{$\Un{A} \vdash x^{\emptyset}:A\ \texttt{un} $}\RightLabel{[WKN-UN]}

    \BinaryInfC{$\Un{A} \mid x^{\emptyset}:A \circledast y^{\emptyset}:B \vdash y: B $}\RightLabel{[$\sepimp I$]}
    \UnaryInfC{$\Un{A} \mid x^{\emptyset}:A \vdash \lambda^{*}y.y: B \sepimp B $}\RightLabel{[$\sepimp I$]}
    \UnaryInfC{$\Un{A} \mid I \vdash \lambda^{*}x. \lambda^{*}y. y: A \sepimp B \sepimp B$}
  \end{prooftree}
\end{minipage}

\begin{framed}\centering
    $\texttt{snd}_{sep} = \lambda^{*}x. \lambda^{*}y. y$
\end{framed}

We will abuse the notation of lambda calculus for $;$ and $,$ as use them as infix operators for syntactic convinence
\begin{flalign*}
  \langle x , y \rangle \equiv (,) x y\\
  \langle x ; y \rangle \equiv (;) x y
\end{flalign*}

We are now in a position to write the proof derivations for \cref{fig:pair-typing-rules} using
the auxilary definitions from above.
\begin{framed}
  \noindent
  \begin{minipage}{0.5\linewidth}
    \begin{prooftree}
      \AxiomC{$\emptyset \mid \texttt{fst}_{sh}\ : A \rightarrow B \rightarrow A \vdash M: A \with B$}\RightLabel{$\with E_1$}
      \UnaryInfC{$\emptyset \mid I \vdash \texttt{fst}_{sh}\  M : A$}
    \end{prooftree}
  \end{minipage}
  \begin{minipage}{0.5\linewidth}
    \begin{prooftree}
      \AxiomC{$\emptyset \mid \texttt{snd}_{sh}\ : A \rightarrow B \rightarrow B \vdash M: A \with B $}\RightLabel{$\with E_2$}
      \UnaryInfC{$\emptyset \mid I \vdash \texttt{snd}_{sh}\  M : B$}
    \end{prooftree}
  \end{minipage}
  \noindent
  \begin{minipage}{1\linewidth}
    \begin{prooftree}
      \AxiomC{$P \mid \emptyset \vdash (;) : \with $}
      \AxiomC{$P \mid \Gamma  \vdash M : A$}
      \AxiomC{$P \mid \Delta \vdash N: B $} \RightLabel{$[\with I]$}
      \TrinaryInfC{$P \mid \Gamma \varoplus \Delta \vdash \langle M ; N \rangle: A \with B$}
    \end{prooftree}
  \end{minipage}
  \noindent
  \begin{minipage}{0.5\linewidth}
    \begin{prooftree}
      \AxiomC{$\Un{B} \mid \texttt{fst}_{sep}\ : A \sepimp B \sepimp A \vdash  M: A \otimes B $}\RightLabel{$\otimes E_1$}
      \UnaryInfC{$\Un{B} \mid I \vdash \texttt{fst}_{sep}\  M : A$}
    \end{prooftree}
  \end{minipage}
  \begin{minipage}{0.5\linewidth}
    \begin{prooftree}
      \AxiomC{$\Un{A} \mid \texttt{snd}_{sep}\ : A \sepimp B \sepimp B \vdash M: A \otimes B $}\RightLabel{$\otimes E_2$}
      \UnaryInfC{$\Un{A} \mid I \vdash \texttt{snd}_{sep}\  M : B$}
    \end{prooftree}
  \end{minipage}
  \noindent
  \begin{minipage}{1\linewidth}
    \begin{prooftree}
      \AxiomC{$P \mid \emptyset \vdash (,) : \otimes $}
      \AxiomC{$P \mid \Gamma  \vdash M : A$}
      \AxiomC{$P \mid \Delta \vdash N: B $} \RightLabel{$[\otimes I]$}
      \TrinaryInfC{$P \mid \Gamma \circledast \Delta \vdash \langle M , N \rangle: A \otimes B$}
    \end{prooftree}
  \end{minipage}
\end{framed}

\section{Derivable Typing rules for Sum Types}\label{sec:sums}
Sum types can hold only one of the enclosing types.
In our case they there can be two kinds of typing for sums.
% \subsection{Muliplicative Sums}
% If the two resources are separate, we call them multipicative sums.
% \begin{landscape}
%   % choice
%   \noindent
%   \begin{small}
%   \begin{prooftree}
%     \AxiomC{$$}\RightLabel{[ID]}
%     \UnaryInfC{$\emptyset \mid  g^{\emptyset}:(B \sepimp E) \vdash g: (B \sepimp E)$}

%     \AxiomC{$$}\RightLabel{[ID]}
%     \UnaryInfC{$\emptyset \mid f^{\emptyset}: (A \sepimp E) \vdash f: (A \sepimp E)$}
%     \AxiomC{$$}\RightLabel{[ID]}
%     \UnaryInfC{$\emptyset \mid c^{\emptyset}: ((A \sepimp E) \sepimp (B \sepimp E) \sepimp E)
%       \vdash c:((A \sepimp E) \sepimp (B \sepimp E) \sepimp E)$}\RightLabel{[$\sepimp$E]}
%     \BinaryInfC{$\emptyset \mid c^{\emptyset}:((A \sepimp E) \sepimp (B \sepimp E) \sepimp E)
%       \circledast f^{\emptyset}:(A \sepimp E) \vdash c f: (B \sepimp E) \sepimp E$}\RightLabel{[$\sepimp$E]}

%     \BinaryInfC{$\emptyset \mid  c^{\emptyset}:((A \sepimp E) \sepimp (B \sepimp E) \sepimp E)  \circledast
%       f^{\emptyset}:(A \sepimp E)  \circledast g^{\emptyset}:(B \sepimp E) \vdash c f g : E$}\RightLabel{[$\sepimp$I]}
%     \UnaryInfC{$\emptyset \mid  c^{\emptyset}:((A \sepimp E) \sepimp (B \sepimp E) \sepimp E) \circledast
%       f^{\emptyset}:(A \sepimp E) \vdash \lambda^{*} g. c f g :(B \sepimp E) \sepimp E$}\RightLabel{[$\sepimp$I]}
%     \UnaryInfC{$\emptyset \mid  c^{\emptyset}:((A \sepimp E) \sepimp (B \sepimp E) \sepimp E) \vdash  \lambda^{*} f. \lambda^{*} g. c f g :(A \sepimp E) \sepimp (B \sepimp E) \sepimp E$}\RightLabel{[$\sepimp$I]}\RightLabel{[$\sepimp$I]}
%     \UnaryInfC{$\emptyset \mid I \vdash \lambda^{*} c. \lambda^{*} f. \lambda^{*} g. c f g : ((A \sepimp E) \sepimp (B \sepimp E) \sepimp E) \sepimp (A \sepimp E) \sepimp (B \sepimp E) \sepimp E$}
%   \end{prooftree}
% \end{small}

%   \noindent
%   We now define sum type to be
%   \begin{framed}\centering
%     $A \oplus B = (A \sepimp E) \sepimp (B \sepimp E) \sepimp E$\\
%     $\CaseSe{c}{\{ f;g \}} = \lambda^{*}c. \lambda^{*}f. \lambda^{*}g. c f g$
%   \end{framed}

%   We now define left and right {\it injections} or constructors for the sum type.
%   % left
%   \noindent
%   \begin{prooftree}
%     \AxiomC{$$}\RightLabel{[ID]}
%     \UnaryInfC{$\Un{A \sepimp E} \mid y: B  \vdash y : B$}

%     \AxiomC{$$}\RightLabel{}
%     \UnaryInfC{$\Un{A \sepimp E} \vdash \Un{A \sepimp E}$}\RightLabel{[WKN-UN]}
%     \BinaryInfC{$\Un{A \sepimp E} \mid f: (A \sepimp E), y: B  \vdash y : B$}

%     \AxiomC{$$}\RightLabel{[ID]}
%     \UnaryInfC{$\Un{A \sepimp E} \mid g: (B \sepimp E) \vdash g: (B \sepimp E)$}\RightLabel{[$\sepimp$I]}

%     \BinaryInfC{$\Un{A \sepimp E} \mid f: (A \sepimp E), y: B, g: (B \sepimp E) \vdash g y : E$}\RightLabel{[EXCH]}
%     \UnaryInfC{$\Un{A \sepimp E} \mid y: B, f: (A \sepimp E), g: (B \sepimp E) \vdash g y : E$}\RightLabel{[$\sepimp$I]}
%     \UnaryInfC{$\Un{A \sepimp E} \mid y: B, f: (A \sepimp E) \vdash \lambda^{*} g. g y : (B \sepimp E) \sepimp E$}\RightLabel{[$\sepimp$I]}
%     \UnaryInfC{$\Un{A \sepimp E} \mid y: B \vdash \lambda^{*} y. \lambda^{*} f. \lambda^{*} g. g y :(A \sepimp E) \sepimp (B \sepimp E) \sepimp E$}\RightLabel{[$\equiv$]}
%     \UnaryInfC{$\Un{A \sepimp E} \mid  I, y: B \vdash \lambda^{*} y. \lambda^{*} f. \lambda^{*} g. g y :(A \sepimp E) \sepimp (B \sepimp E) \sepimp E$}\RightLabel{[$\sepimp$I]}\RightLabel{[$\sepimp$I]}
%     \UnaryInfC{$\Un{A \sepimp E} \mid I \vdash \lambda^{*} y. \lambda^{*} f. \lambda^{*} g. g y: B \sepimp  (A \sepimp E) \sepimp (B \sepimp E) \sepimp E$}
%   \end{prooftree}
%   \noindent Left injection defined below as:
%   \begin{framed}\centering
%     $inl : A \sepimp A \oplus B$\\
%     $\texttt{inl} = \lambda x. \lambda f. \lambda g. f x$
%   \end{framed}

%   % right
%   \begin{prooftree}
%     \AxiomC{$$}\RightLabel{[ID]}
%     \UnaryInfC{$\Un{B \sepimp E} \mid x: A  \vdash x : B$}

%     \AxiomC{$$}\RightLabel{}
%     \UnaryInfC{$\Un{B \sepimp E} \vdash \Un{B \sepimp E}$}\RightLabel{[WKN-UN]}
%     \BinaryInfC{$\Un{B \sepimp E} \mid g: (B \sepimp E), x: A  \vdash x : A$}

%     \AxiomC{$$}\RightLabel{[ID]}
%     \UnaryInfC{$\Un{B \sepimp E} \mid f: (A \sepimp E) \vdash f: (A \sepimp E)$}\RightLabel{[$\sepimp$E]}

%     \BinaryInfC{$\Un{B \sepimp E} \mid x: A, f: (A \sepimp E), g: (B \sepimp E) \vdash f x : E$}\RightLabel{[$\sepimp$I]}
%     \UnaryInfC{$\Un{B \sepimp E} \mid x: A, f: (A \sepimp E) \vdash \lambda^{*} g. f x : (B \sepimp E) \sepimp E$}\RightLabel{[$\sepimp$I]}
%     \UnaryInfC{$\Un{B \sepimp E} \mid x: A \vdash  \lambda^{*} f. \lambda^{*} g. f x :(A \sepimp E) \sepimp (B \sepimp E) \sepimp E$}\RightLabel{[$\equiv$]}
%     \UnaryInfC{$\Un{B \sepimp E} \mid  I, x: A \vdash  \lambda^{*} f. \lambda^{*} g. f x :(A \sepimp E) \sepimp (B \sepimp E) \sepimp E$}\RightLabel{[$\sepimp$I]}\RightLabel{[$\sepimp$I]}
%     \UnaryInfC{$\Un{B \sepimp E} \mid I \vdash \lambda^{*} x. \lambda^{*} f. \lambda^{*} g. f x: A \sepimp  (A \sepimp E) \sepimp (B \sepimp E) \sepimp E$}
%   \end{prooftree}

%   \noindent \noindent Right injection defined below as:
%   \begin{framed}\centering
%     $inr : B \sepimp A \oplus B$\\
%     $\texttt{inr} = \lambda y. \lambda f. \lambda g. g y$
%   \end{framed}
% \end{landscape}

% We can now add sum types in our language using the auxilary definitions given above:
% \begin{framed}
%   \begin{minipage}[h]{0.5\linewidth}
%     \begin{prooftree}
%       \AxiomC{$\emptyset \mid \texttt{inr}: A \sepimp A \oplus B \vdash x : A$}\RightLabel{[$\oplus$I$_1$]}
%       \UnaryInfC{$\emptyset \mid I \vdash \texttt{inl}\ x: A \oplus B$}
%     \end{prooftree}
%   \end{minipage}
%   \begin{minipage}[h]{0.5\linewidth}
%     \begin{prooftree}
%       \AxiomC{$\emptyset \mid \texttt{inl}: B \sepimp A \oplus B \vdash y : B$}\RightLabel{[$\oplus$I$_2$]}
%       \UnaryInfC{$\emptyset \mid I \vdash \texttt{inl}\ y: A \oplus B$}
%     \end{prooftree}
%   \end{minipage}
%   \begin{minipage}[h]{1.0\linewidth}
%     \begin{prooftree}
%       \AxiomC{$\emptyset \mid \Gamma \vdash M : A \oplus B$}\RightLabel{[$\oplus$E]}
%       \AxiomC{$\emptyset \mid \Gamma \circledast x:A \vdash N_1 : E$}
%       \AxiomC{$\emptyset \mid \Gamma \circledast y:B \vdash N_2 : E$}
%       \TrinaryInfC{$\emptyset \mid \Gamma \vdash\Case{M}{\{ \texttt{inl}\ x \mapsto N_1; \texttt{inl}\ y \mapsto N_2\}}:E$}
%     \end{prooftree}
%   \end{minipage}
% \end{framed}

\begin{landscape}
  % choice
  \noindent
  \begin{small}
  \begin{prooftree}
    \AxiomC{$$}\RightLabel{[ID]}
    \UnaryInfC{$\emptyset \mid  g^{cf}:(B \rightarrow E) \vdash g: (B \rightarrow E)$}

    \AxiomC{$$}\RightLabel{[ID]}
    \UnaryInfC{$\emptyset \mid f^{cg}: (A \rightarrow E) \vdash f: (A \rightarrow E)$}
    \AxiomC{$$}\RightLabel{[ID]}
    \UnaryInfC{$\emptyset \mid c^{fg}: ((A \rightarrow E) \rightarrow (B \rightarrow E) \rightarrow E)
      \vdash c:((A \rightarrow E) \rightarrow (B \rightarrow E) \rightarrow E)$}\RightLabel{[$\rightarrow$E]}
    \BinaryInfC{$\emptyset \mid c^{fg}:((A \rightarrow E) \rightarrow (B \rightarrow E) \rightarrow E)
      \varoplus f^{cg}:(A \rightarrow E) \vdash c f: (B \rightarrow E) \rightarrow E$}\RightLabel{[$\rightarrow$E]}

    \BinaryInfC{$\emptyset \mid  c^{fg}:((A \rightarrow E) \rightarrow (B \rightarrow E) \rightarrow E)  \varoplus
      f^{cg}:(A \rightarrow E)  \varoplus g^{cf}:(B \rightarrow E) \vdash c f g : E$}\RightLabel{[$\rightarrow$I]}
    \UnaryInfC{$\emptyset \mid  c^{f}:((A \rightarrow E) \rightarrow (B \rightarrow E) \rightarrow E) \varoplus
      f^{c}:(A \rightarrow E) \vdash \lambda^{\alpha} g. c f g :(B \rightarrow E) \rightarrow  E$}\RightLabel{[$\rightarrow$I]}
    \UnaryInfC{$\emptyset \mid  c^{\emptyset}:((A \rightarrow E) \rightarrow (B \rightarrow E) \rightarrow E)
      \vdash  \lambda^{\alpha} f. \lambda^{\alpha} g. c f g :(A \rightarrow E) \rightarrow (B \rightarrow E) \rightarrow E$}\RightLabel{[$\sepimp$I]}
    \UnaryInfC{$\emptyset \mid I \vdash \lambda^{*} c. \lambda^{\alpha} f. \lambda^{\alpha} g. c f g
      : ((A \rightarrow E) \rightarrow (B \rightarrow E) \rightarrow E) \sepimp (A \rightarrow E) \rightarrow (B \rightarrow E) \rightarrow E$}
  \end{prooftree}
\end{small}

\noindent
We now define sum type to be
\begin{framed}\centering
  $A \oplus B = (A \rightarrow E) \rightarrow (B \rightarrow E) \rightarrow E$\\
\end{framed}

We now define left and right {\it injections} or constructors for the sum type.
% left
\begin{prooftree}
  \AxiomC{$$}\RightLabel{[ID]}
  \UnaryInfC{$\emptyset \mid x^{fg}: A  \vdash x : A$}

  \AxiomC{$$}\RightLabel{[ID]}
  \UnaryInfC{$\emptyset \mid f^{gx}: (A \rightarrow E) \vdash f: (A \rightarrow E)$}\RightLabel{[$\rightarrow$E]}

  \BinaryInfC{$\emptyset \mid x^{fg}: A \varoplus f^{xg}: (A \rightarrow E)  \vdash f x : E$}\RightLabel{[WKN-UN]}
  \UnaryInfC{$\emptyset \mid x^{fg}: A \varoplus f^{xg}: (A \rightarrow E) \varoplus g^{fx}: (B \rightarrow E) \vdash f x : E$}\RightLabel{[$\rightarrow$I]}
  \UnaryInfC{$\emptyset \mid x^{f}: A \varoplus f^{x}: (A \rightarrow E) \vdash \lambda^{\alpha} g. f x : (B \rightarrow E) \rightarrow E$}\RightLabel{[$\rightarrow$I]}
  \UnaryInfC{$\emptyset \mid  x^{\emptyset}: A \vdash  \lambda^{\alpha} f. \lambda^{\alpha} g. f x
    :(A \rightarrow E) \rightarrow (B \rightarrow E) \rightarrow E$}\RightLabel{[$\sepimp$I]}\RightLabel{[$\sepimp$I]}
  \UnaryInfC{$\emptyset \mid I \vdash \lambda^{*} x. \lambda^{\alpha} f. \lambda^{\alpha} g. f x
    : A \sepimp  (A \rightarrow E) \rightarrow (B \rightarrow E) \rightarrow E$}
\end{prooftree}

\noindent Left injection defined below as:
\begin{framed}\centering
  $\texttt{inl} : A \sepimp A \oplus B$\\
  $\texttt{inl} = \lambda^{*} x. \lambda^{\alpha} f. \lambda^{\alpha} g. f x$
\end{framed}

% right
\noindent
\begin{prooftree}
  \AxiomC{$$}\RightLabel{[ID]}
  \UnaryInfC{$\emptyset \mid y^{fg}: B  \vdash y : B$}

  \AxiomC{$$}\RightLabel{[ID]}
  \UnaryInfC{$\emptyset \mid g^{yf}: (B \sepimp E) \vdash g: (B \sepimp E)$}\RightLabel{[$\rightarrow$I]}

  \BinaryInfC{$\emptyset \mid y^{fg}: B \varoplus g^{yf}: (B \rightarrow E) \vdash g y : E$}\RightLabel{[WKN-UN]}
  \UnaryInfC{$\emptyset \mid y^{fg}: B \varoplus f^{yg}: (A \rightarrow E) \varoplus g^{yf}: (B \rightarrow E) \vdash g y : E$}\RightLabel{[$\rightarrow$I]}
  \UnaryInfC{$\emptyset \mid y^{f}: B \varoplus f^{y}: (A \rightarrow E) \vdash \lambda^{\alpha} g. g y : (B \rightarrow E) \rightarrow E$}\RightLabel{[$\rightarrow$I]}
  \UnaryInfC{$\emptyset \mid  y^{\emptyset}: B \vdash \lambda^{\alpha} f. \lambda^{\alpha} g. g y
    :(A \rightarrow E) \rightarrow (B \rightarrow E) \rightarrow E$}\RightLabel{[$\sepimp$I]}\RightLabel{[$\sepimp$I]}
  \UnaryInfC{$\emptyset \mid I \vdash \lambda^{*} y. \lambda^{\alpha} f. \lambda^{\alpha} g. g y: B \sepimp  (A \rightarrow E) \rightarrow (B \rightarrow E) \rightarrow E$}
\end{prooftree}

\noindent Right injection defined below as:
\begin{framed}\centering
  $\texttt{inr} : B \sepimp A \oplus B$\\
  $\texttt{inr} = \lambda^{*} y. \lambda^{\alpha} f. \lambda^{\alpha} g. g y$
\end{framed}
\end{landscape}

We can now derive sum types in our language using the auxilary definitions given above
and provide a new syntax for deconstructing the sum type by matching on its structure by a case statement.
\begin{framed}\centering
    $\texttt{case}\ {c}\ \texttt{of}\ {\{ f;g \}} = \lambda^{*}c. \lambda^{\alpha}f. \lambda^{\alpha}g. c f g$
\end{framed}

\begin{framed}
  \begin{minipage}[h]{0.5\linewidth}
    \begin{prooftree}
      \AxiomC{$\emptyset \mid \texttt{inl}: A \sepimp A \oplus B \vdash x : A$}\RightLabel{[$\oplus$I$_1$]}
      \UnaryInfC{$\emptyset \mid I \vdash \texttt{inl}\ x: A \oplus B$}
    \end{prooftree}
  \end{minipage}
  \begin{minipage}[h]{0.5\linewidth}
    \begin{prooftree}
      \AxiomC{$\emptyset \mid \texttt{inr}: B \sepimp A \oplus B \vdash y : B$}\RightLabel{[$\oplus$I$_2$]}
      \UnaryInfC{$\emptyset \mid I \vdash \texttt{inr}\ y: A \oplus B$}
    \end{prooftree}
  \end{minipage}
  \begin{minipage}[h]{1.0\linewidth}
    \begin{prooftree}
      \AxiomC{$\emptyset \mid \Gamma \vdash M : A \oplus B$}\RightLabel{[$\oplus$E]}
      \AxiomC{$\emptyset \mid \Gamma \varoplus x:A \vdash N_1 : E$}
      \AxiomC{$\emptyset \mid \Gamma \varoplus y:B \vdash N_2 : E$}
      \TrinaryInfC{$\emptyset \mid \Gamma \vdash \Case{M}{\{ \texttt{inl}\ x \mapsto N_1; \texttt{inr}\ y \mapsto N_2\}}:E$}
    \end{prooftree}
  \end{minipage}
\end{framed}


%%% Local Variables:
%%% mode: latex
%%% TeX-master: "../thesis-ku"
%%% End:

\chapter{Proofs}
\begin{theorem}[Soundness of $\vdashs$]\label{thm:soundness-syntax-directed}
   If $P \mid \Gamma \vdashs M:\tau$ then $P \mid \Gamma \vdash M:\tau$
\end{theorem}
\begin{proof}\label{prf:soundness-syntax-directed}
  Proof by structural induction on the derivation of $P \mid \Gamma \vdashs M:\tau$.

  \begin{itemize}
  \item{Case [VAR$^s$].}
    By induction hypothesis we have a derivation of $P \mid x:\sigma \vdash x:\sigma$ by [VAR] rule.
    We proceed by repeated application of [$\forall$E] as $(Q => \tau) \sqsubseteq \sigma$, and
    we construct a derivation of $P \mid x:\sigma \vdash x: Q => \tau$. Then as $P => Q$ we can
    repeatedly apply [$=>$E] to construct a derivation of $P \mid x:\sigma \vdash x:\tau$.
    Finally, depending on whether the bindings are in sharing with or separate from
    $x$ we repeatedly apply [WKN-SH] or [WKN-UN] respectively for all the bindings in $\Gamma$ to construct
    a derivation of $P \mid \Gamma \sqcup x:\sigma \vdash x:\tau$.
  \item{Case [$\rightarrow$I$^s$].}
    By induction hypothesis we have a derivation of $P \mid \Gamma \varoplus x:\tau \vdash M:\tau'$.
    We apply [$\rightarrow$I] and reuse the derivations for $\ShFun{\phi}$ and $\Gamma \geq \phi$  to
    construct a derivation of $P \mid \Gamma \vdash \lambda^{\alpha}x. M: \phi \tau \tau'$.
  \item{Case [$\sepimp$I$^s$].}
    By induction hypothesis we have a derivation of $P \mid \Gamma \circledast x:\tau \vdash M:\tau'$.
    Similar to previous case, we apply [$\sepimp$I] and reuse the derivations for $\SeFun{\phi}$ and $\Gamma \geq \phi$  to
    construct a derivation of $P \mid \Gamma \vdash \lambda^{*}x. M: \phi \tau \tau'$.
  \item{Case [App$^s$].}
    By induction hypothesis we have derivations of $P \mid \Gamma \circledast \Delta \vdash M: \phi v \tau$ and
    $P \mid \Gamma' \circledast \Delta \vdash: N:v$ we check for $\texttt{Used}(\Gamma) = \texttt{Used}(\Gamma')$ and if it is true
    and apply [$\rightarrow$E] or check for $\texttt{Used}(\Gamma)\#\texttt{Shared}(\Gamma') \wedge \texttt{Shared}(\Gamma')\#\texttt{Used}(\Gamma)$
    and if it is true we apply [$\sepimp$E] to reuse derivations of $\ShFun{\phi}$ or $\SeFun{\phi}$ respectively to construct the derivation
    of $P \mid (\Gamma \varoplus \Gamma') \circledast \Delta \vdash M N:\tau$ or $P \mid (\Gamma \circledast \Gamma') \circledast \Delta \vdash M N:\tau$.
  \item{Case [Let$^s$].}
    By induction hypothesis have a derivation of $P \mid \Gamma \circledast \Delta \vdash M: \tau$ and $P \mid \Gamma' \sqcup x:\tau \circledast \Delta \vdash N:\tau$
    Applying [$\forall$I] and [$=>$I] on the first hypothesis we derive $\emptyset \mid \Gamma \circledast \Delta \vdash M: \sigma$. Now by applying
    the [LET] rule and reusing $P \vdash \Delta\ \texttt{un}$ we construct the derivation of
    $P \mid \Gamma \sqcup \Gamma' \circledast \Delta \vdash \Let{x}{M}{N}:\tau$.
  \end{itemize}
\end{proof}

\begin{theorem}[Completeness of $\vdashs$]\label{thm:completeness-syntax-directed}
  If $P \mid \Gamma \vdash M:\tau$ then
  $\exists ! Q,\tau. Q \mid \Gamma \vdashs M:\tau$
  and $(P \mid \sigma) \sqsubseteq \texttt{Gen}(\Gamma, Q => \tau)$
\end{theorem}
\begin{proof}
  \TODO{some induction here}.
\end{proof}

\begin{theorem}[Soundness of $\mathcal{M}$.]
   If $\mathcal{M}(S, X; \Gamma \vdash M : \tau) = P, S', \Sigma$ then $S' P | S' (\Gamma\mid_{\Sigma}) \vdash M : S' \tau$
\end{theorem}
\begin{proof}
  Proof by induction on structure of $M$
  \begin{itemize}
  \item Case 1. $x$
  \item Case 2. $\lambda^{*} x. M$
  \item Case 3. $\lambda ^{\alpha}x. M$
  \item Case 4. $M\ N$
  \item Case 5. $\texttt{let}\ x = M\ \texttt{in}\ N$
  \end{itemize}
\end{proof}

\begin{theorem}[Completeness of $\M$.]
  If $S$ is a substitution ans X is a set of type variables such that
  $P \mid S\Gamma \vdashs M: S\tau$, and $S|_X = id$ then $\M(id, X; \Gamma \vdash M:\tau) = Q, S', \Sigma$, such that
  $P => S\tau \sqsubset \texttt{GenI}(S'\Gamma, S' Q => S' \tau)$
\end{theorem}
\begin{proof}
\TODO{some more induction here}
\end{proof}

\begin{theorem}[Principal types.]
  If $P_0 \mid  \Gamma \vdash M : \sigma_0$ and $P_1 \mid  \Gamma \vdash M : \sigma_1$ then there is some $\sigma$ such that
  $\emptyset \mid \Gamma \vdash M : \sigma$ and $(P_0 | \sigma_0) \subseteq \sigma$, and $(P_1 | \sigma_1) \subseteq \sigma$.
\end{theorem}
\begin{proof}
\TODO{some more induction here. Makes use of theorems \ref{thm:soundness-syntax-directed} and \ref{thm:completeness-syntax-directed}}
\end{proof}



\TODO{
  \begin{theorem}[Progress.]
    If $M:\tau$ then either  $M \leadsto M'$ or $M$ is a value.
  \end{theorem}
\begin{proof}
  we have to define $\beta \eta$ rules.
\end{proof}

\begin{theorem}[Preservation.]
  If $M:\tau$ and $M \leadsto M'$ then $M':\tau$
\end{theorem}
\begin{proof}
\end{proof}
}


%%% Local Variables:
%%% mode: latex
%%% TeX-master: "../thesis-ku"
%%% End:

\end{document}

%%% Local Variables:
%%% mode: latex
%%% TeX-master: t
%%% End:
