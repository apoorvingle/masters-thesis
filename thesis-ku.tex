%% LyX 2.2.0 created this file.  For more info, see http://www.lyx.org/.
%% Do not edit unless you really know what you are doing.
\documentclass[12pt,english]{kuthesis}
\usepackage{mathptmx}
\usepackage{MnSymbol}
\renewcommand{\sfdefault}{lmss}
\renewcommand{\ttdefault}{lmtt}
\usepackage[T1]{fontenc}
\usepackage[utf8]{inputenc}
\usepackage{geometry}
\geometry{verbose,tmargin=1in,bmargin=1in,lmargin=1in,rmargin=1in}
\setcounter{secnumdepth}{3}
\setcounter{tocdepth}{3}
\usepackage{xcolor}
\usepackage{babel}
\usepackage{url}
\usepackage{graphicx}
\usepackage{setspace}
\usepackage{esint}
\usepackage[authoryear]{natbib}
\usepackage{bussproofs}
\usepackage{semantic}
\usepackage{framed}
\usepackage{minted}
% \usepackage{amsmath, amssymb}

\doublespacing
\usepackage[unicode=true,
 bookmarks=true,bookmarksnumbered=false,bookmarksopen=false,
 breaklinks=true,pdfborder={0 0 0},pdfborderstyle={},backref=false,colorlinks=true]
 {hyperref}
\hypersetup{pdftitle={University of Kansas Thesis Template},
 pdfauthor={Apoorv Ingle},
 pdfsubject={A Thesis},
 urlcolor={blue},citecolor={blue},allcolors={blue}}

\makeatletter

%%%%%%%%%%%%%%%%%%%%%%%%%%%%%% LyX specific LaTeX commands.
\providecommand{\LyX}{\texorpdfstring%
  {L\kern-.1667em\lower.25em\hbox{Y}\kern-.125emX\@}
  {LyX}}
%% Because html converters don't know tabularnewline
\providecommand{\tabularnewline}{\\}
%%%%%%%%%%%%%%%%%%%%%%%%%%%%%% User specified LaTeX commands.
\newcommand\sepimp{\mathrel{-\mkern-6mu*}}

%used to align decimals in tables according to APA style
\usepackage{dcolumn}
\usepackage{booktabs}

% Set the title and author info
\title{Resource Aware Functional Language}
\author{Apoorv Ingle}


\dept{Electrical Engineering and Computer Science}
\degreetitle{Master of Science}
\papertype{Thesis} %or Thesis (Whatever you put will appear on p.2)
%% It is vital to have 7 entries, even if some are empty for committee and role
%% I mean, it is vital to leave the empty place holders
\committee{Dr. J. Garrett Morris}{Dr. Perry Alexandar}{Dr. Andy Gill}{Dr. Prasad Kulkarni}{}{}{}
\role{Chairperson}{}{}{}{}{}{}
%AT Most 7 members allowed, last here is blank on purpose to demonstrate
%flexibility

%% The following is OPTIONAL. Remove all 3 of the next 3 lines
%% to leave dates blank. If dates are included, then both dates
%% must be included.
\@printd@testrue
\datedefended{July 21, 2018}
\dateapproved{July 21, 2018}

%% These settings are now in the kuthesis.cls file, but users are free
% to customize. listings has great documentation online
%% When listings are used, break lines
%\lstset{
 %    breaklines=true,  % sets automatic line breaking
 %    breakindent=2em,
 %    breakatwhitespace=true,  % sets if automatic breaks should
 %   breakautoindent=true
%}

\@ifundefined{showcaptionsetup}{}{%
 \PassOptionsToPackage{caption=false}{subfig}}
\usepackage{subfig}
\makeatother

\usepackage{listings}
\renewcommand{\lstlistingname}{\inputencoding{latin9}Listing}

\begin{document}
\begin{romanpages}

\maketitle
\begin{abstract}

  Complexity of software systems has been increasing forever. Automation of tasks
  provides consistency, correctness and speed at a much greater degree than humans.
  complex software systems find their use in various fields ranging from automobiles, aircrafts,
  financial institutions and even medicine. Higher level programming languages
  are used to program these software systems and developing them requires
  adequate understanding of the programming language along with problem domain.
  Structured programming techniques ensures maintanability and extensibility
  of the software application. Types and functional programming style helps maintain
  structure by making it possible to run algorithmic checks to ensure sanity.
  I have no idea what I am talking about. TODO! get back on track

  %  What is quil?
  % Quill is a purely functional language that uses
  % qualified types and a varient of linear logic to keep track of resources.
  % We introduce a new type class to express unrestrictedness
  % of a type, define the degree of unrestrictedness of classes
  % and a concept of sharing or separation of closures for functions

  We introduce 2 new types of type constructors of arrow kind: sharing arrow and
  separating arrow over already existing unrestricted types formalized in
  Quill [\cite{morris_best_2016}]. Sharing arrows imply that the objects may share resources, while separating
  arrows imply that they do not share any resources. For unrestricted types
  sharing and separating does not make any difference as they can be easily
  duplicated or dropped without any adverse implications. So all the existing
  programs would type check correctly with only difference being everying
  regarded as unrestricted. This work is similar to bunched implications
  [\cite{ohearn_resource_1999}, \cite{ohearn_logic_1999}] with the difference being
  I do not know exactly what. TODO!

  % What am I trying to doing here?
  % Make Quill work. Write programs in them and check and see
  % how painful it is.
  We realize that we have 2 kinds of pairs, a sharing pair and a separating pair depending
  on the arrow we use. We try to formalize other structures such as Lists, non-empty lists
  datatypes (recursive and non-recursive) and type classes such as functor, applicative and
  monads with specific instances of each. (do we also do try catch finally here?)


\begin{acknowledgementslong}
%%if you want a "quote" environment for acknowledgements,
%% use acknowledgements instead of acknowledgementslong
  God speed everyone.
  TODO TODO TODO DO DO DO DO.
% I would like to thank all of the little people who made this thesis
% possible. Sleepy, Dopey, Grumpy, you know who you are.

\end{acknowledgementslong}
\end{abstract}
\tableofcontents{}

\listoffigures

\listoftables

\end{romanpages}

\chapter{Background Work}
% TODO should change title

\section{Type Inference Algorithm}
Algorithm $\mathcal{W}$ [\cite{damas_principal_1982}] and its varient algorithm $\mathcal{M}$ [\cite{lee_proofs_1998}]
are the basis of most of the modern statically typed programming languages. Type inference
is decidable in the sense, type checking algorithm always completes with a success or failure.
The algorithms also gurantee a most general typing scheme for an expression in
the simply typed lambda calculus extended with a polymorphic let construct having a term language
\begin{framed}
  \begin{flalign*}
    \text{Expressions}\ \ \ M, N ::= x: \sigma \mid \lambda x: \tau. M \mid M N \mid \texttt{let}\ x\ \texttt{=}\ M\ \texttt{in}\ N \nonumber
  \end{flalign*}
\end{framed}
and a type language specified by
\begin{framed}
  \begin{flalign*}
    \text{Types}\ \ \  \tau    &::= \alpha \mid \iota \mid \tau \rightarrow \tau \nonumber \\
    \text{Typing Scheme}\ \ \  &::= \tau \mid \forall \alpha. \tau \nonumber
  \end{flalign*}
\end{framed}
where $\alpha$ is a type variable, $\iota$ are primitive types in the language, $\rightarrow$
is a type constructor and $\sigma$ is a typing scheme.

Robinson's unification algorithm [\cite{robinson_machine-oriented_1965}] plays a key role
in ensuring that types are well formed. Its purely syntactic approach in creating
substitutions to unify types keeps the complete process elegent.
The algorithm works in an interesting way where the types of all well-typed terms can be
inferred automatically and if types are specified, the same algorithm can be used
to match the expression term.

\TODO{
Talks about Curry-Howard here.
Logic rules and corresponding Typing rules.
Typing rules are nothing but logic rules with terms and types.
Types correspond to predicates
}
\begin{figure}[h]
  \begin{framed}
    % var
    \begin{minipage}{.5\textwidth}
      \begin{prooftree}
        \AxiomC{$x: \sigma \in \Gamma$} \RightLabel{$[VAR]$}
        \UnaryInfC{$\Gamma \vdash x : \sigma $}
      \end{prooftree}
    \end{minipage}
    % let
    \begin{minipage}{.5\textwidth}
      \begin{prooftree}
        \AxiomC{$\Gamma \vdash M : \sigma$}
        \AxiomC{$\Gamma_{x}, x: \sigma \vdash N: \tau$} \RightLabel{$[LET]$}
        \BinaryInfC{$\Gamma \vdash (\Let{x}{M}{N}) : \tau$}
      \end{prooftree}
    \end{minipage}
    % forall I
    \begin{minipage}{0.5\textwidth}
      \begin{prooftree}
        \AxiomC{$\Gamma \vdash M : \sigma$}\RightLabel{$[\forall I]$}
        \AxiomC{$t \notin \text{fvs}(\Gamma)$}
        \BinaryInfC{$\Gamma \vdash \lambda t. M : \forall t. \sigma$}
      \end{prooftree}
    \end{minipage}
    % forall E
    \begin{minipage}{0.5\textwidth}
      \begin{prooftree}
        \AxiomC{$\Gamma \vdash M : \forall t. \sigma$} \RightLabel{$[\forall E]$}
        \UnaryInfC{$\Gamma \vdash M \tau : [\tau \backslash t] \sigma$}
      \end{prooftree}
    \end{minipage}
    % -> I
    \begin{minipage}{0.5\textwidth}
      \begin{prooftree}
        \AxiomC{$\Gamma_{x}, x: \tau \vdash M : \tau'$} \RightLabel{$[\rightarrow I]$}
        \UnaryInfC{$\Gamma \vdash \lambda x. M : \tau \rightarrow \tau'$}
      \end{prooftree}
    \end{minipage}
    % -> E
    \begin{minipage}{0.5\textwidth}
      \begin{prooftree}
        \AxiomC{$\Gamma \vdash M : \tau \rightarrow \tau'$}
        \AxiomC{$\Gamma \vdash N : \tau$} \RightLabel{$[\rightarrow E]$}
        \BinaryInfC{$\Gamma \vdash M N : \tau'$}
      \end{prooftree}
    \end{minipage}
  \end{framed}
  \caption{Logic Rules for Simply Typed Lambda Calculus}
  \label{fig:stlc-logic}
\end{figure}
The logical rules for type inference are shown in \ref{fig:stlc-logic}. $\Gamma$ is the
context or assumptions in which the expression is typed. The $[VAR]$ rule is tautology or a simple
lookup of the term variable $x$ in the context $\Gamma$. The $[LET]$ allows creating local
definitions within an expression term. $[\rightarrow I]$ and $[\rightarrow E]$ are rules
for typing lambda terms and application respectively. We also include the rules for
type application and abstraction $[\forall I]$ and $[\forall E]$ to introduce second order
quantification in predicate logic.

% This simple type sytem is powerful in its
% expressivity and can encode a large variety of computations. The type checking algorithm
% asserts that undefined programs can be be detected statically i.e. without actually
% running the program or as famously known as ``well typed programs do not go wrong''.
% This is extremely useful for programmers who are building
% complex real world softwares. Bad programs can be eleminated instantaneously while
% being written using a mechanize technique so that the programmer can concentrate on designing the logic
% rather than fighting undefinedness of the programs. This creates an excellent feedback loop
% to the programmer while building large software systems. % TODO too generic should it be in introduction?
% TODO Give examples? and come up with a Curry-Howard interpretation of

\section{Qualified Types}
Jones [\cite{jones_theory_1994}] proposed incorporating predicates in the type language.
Predicates are used to build constraints on the domain of the type of a term in the language expression.
It introduces additional layer between polymorphic and monomorphic typing of programs.
A modification of Milner-Damas algorithm to encorporate predicates ensures that type inference
is sound and complete. The types that satisfy all the predicates are called qualified types for the term.
Qualified types are powerful enough to expresses type classes with functional dependencies,
record types and subtyping [\cite{mark_type_2000}]. The type language is modified to contain
qualified types. $P$ and $T$ range over finite set of predicates. We slightly modify the typing rules
from \cref{fig:stlc-logic} to add 2 new rules for qualified types as shown in \cref{fig:qualified-types-rules}
\begin{figure}[h]
  \centering
  \begin{framed}
  \begin{flalign*}
    \text{Types}\ \ \ \tau              &::= \alpha \mid \iota \mid \tau \rightarrow \tau \nonumber \\
    \text{Qualified Types}\ \ \ \rho    &::= P \Rightarrow \tau \nonumber \\
    \text{Type Scheme}\ \ \ \sigma      &::= \tau \mid \forall T. \rho \nonumber
  \end{flalign*}
\end{framed}
\caption{Qualified Types}
\label{fig:qualifed-types}
\end{figure}
\begin{figure}[h]
  \begin{framed}
    % => I
    \begin{minipage}{0.5\textwidth}
      \begin{prooftree}
        \AxiomC{$P, \pi \mid \Gamma \vdash M : \rho$} \RightLabel{$[=> I]$}
        \UnaryInfC{$P \mid \Gamma \vdash M : \pi \Rightarrow \rho$}
      \end{prooftree}
    \end{minipage}
    % => E
    \begin{minipage}{0.5\textwidth}
      \begin{prooftree}
        \AxiomC{$P \mid \Gamma \vdash M : \pi \Rightarrow \rho$}
        \AxiomC{$P \Rightarrow \pi$} \RightLabel{$[=> E]$}
        \BinaryInfC{$P \mid \Gamma \vdash M: \rho$}
      \end{prooftree}
    \end{minipage}
  \end{framed}
  \caption{Modified Typing Rules}
  \label{fig:qualified-types-rules}
\end{figure}
\section{Linear Logic}
% TODO: points to cover
% what is linearity
% restricting weakening and contraction
While classical logic deals with truth of propositions, linear logic deals with availability of resources.
Linear logic [\cite{girard_linear_1987}] promises to help cope with the resource and resource control problem.
It is refinement of classical intuistionistic logic. The core idea is that propositions
cannot be freely duplicated or discarded as in classical instuistionistic logic.
In formal terms, the contraction and weakening of logical rules are restricted.
This instigates a view of propositions to behave like resources. In real world software applications,
resources may not be freely copied or dropped from a program context.
Program entities like database connections, file handles or even
in memory shared state are pet peeves for programmers writing
industry grade software. Linear logic hopes to be a remedy for
these problems. If contraction and weakening is completely abandoned,
the system gets overly restrictive. Wadler describes a refinement of
linear logic based on Girard's Logic of Unity [\cite{wadler_taste_1993}, \cite{girard_unity_1993}].
It works around the problem of linear logic being too restrictive by allowing
instuistionistic rules in fragments. It can be considered as a disjoint union
of classical linear logic and intuistionistic logic. The grammar of classical intuistionistic logic is shown in \ref{fig:intu-logic-grammar}
where $A \rightarrow B$ implies implication, $A \times B$ is conjunction and $A \plus B$ is disjunction.
\begin{figure}
  \centering
  \begin{framed}
  \begin{flalign*}
    A, B, C ::= X \mid A \vdash B \mid A \rightarrow B \mid A \times B \mid A \plus B
  \end{flalign*}
\end{framed}
\caption{Grammar for Intuistionistic Logic}
\label{fig:intu-logic-grammar}
\end{figure}

In a pure linear logic setting, none of the assumptions can be used more than once (weakening prohibited) and they cannot be discarded
(contraction prohibited) This gives rise to a different flavor of all the logical connectives.
$A \rightspoon B$ describes the new implication meaning and is read as {\em``consume A to give B''} its logical rules
is given by $\rightspoon I$ and $\rightspoon E$. Similarly there are 2 kinds of connectives, multiplicative and additive that
arise in this logic system. More symbols are added inplace of $\plus$ and $\times$ to differentiate between the
multipicative conjunction and disjuntion ($\otimes$ and $\parr$), and additive conjuntion and disjunction ($\with $ and $\oplus$).
While working in intuistionistic linear logic, $\parr$ is dropped from the system as it can be encoded by other connectives.

\begin{figure}
  \centering
  \begin{framed}
    \begin{flalign*}
      A, B, C ::= X \mid \oc A \mid A \vdash B \mid A \rightspoon B \mid A \with B \mid A \otimes B
    \end{flalign*}
  \end{framed}
  \caption{Grammar for Intuistionistic Linear Logic}
\end{figure}

\begin{figure}[h]
  \begin{framed}
    % -o I
    \begin{minipage}{0.5\textwidth}
      \begin{prooftree}
        \AxiomC{$\Gamma, A \vdash B$} \RightLabel{$[\rightspoon I]$}
        \UnaryInfC{$\Gamma \vdash A \rightspoon B$}
      \end{prooftree}
    \end{minipage}
    % -o E
    \begin{minipage}{0.5\textwidth}
      \begin{prooftree}
        \AxiomC{$\Gamma \vdash  A \rightspoon B$}
        \AxiomC{$\Delta \vdash A$} \RightLabel{$[\rightspoon E]$}
        \BinaryInfC{$\Gamma, \Delta \vdash B$}
      \end{prooftree}
    \end{minipage}
  % & I
  \begin{minipage}{.3\textwidth}
    \begin{prooftree}
      \AxiomC{$\Gamma \vdash A$}
      \AxiomC{$\Gamma \vdash B$} \RightLabel{$[\with I]$}
      \BinaryInfC{$\Gamma \vdash A \with B$}
    \end{prooftree}
  \end{minipage}
  \begin{minipage}{.3\textwidth}
    \begin{prooftree}
      \AxiomC{$\Gamma \vdash A \with B$} \RightLabel{$[\with E_1]$}
      \UnaryInfC{$\Gamma \vdash A$}
    \end{prooftree}
  \end{minipage}
  \begin{minipage}{.3\textwidth}
    \begin{prooftree}
      \AxiomC{$\Gamma \vdash A \with B$} \RightLabel{$[\with E_2]$}
      \UnaryInfC{$\Gamma \vdash B$}
    \end{prooftree}
  \end{minipage}
  % otimes I
  \begin{minipage}{.3\textwidth}
    \begin{prooftree}
      \AxiomC{$\Gamma \vdash A$}
      \AxiomC{$\Delta \vdash B$} \RightLabel{$[\otimes I]$}
      \BinaryInfC{$\Gamma, \Delta \vdash A \otimes B$}
    \end{prooftree}
  \end{minipage}
  \begin{minipage}{.7\textwidth}
    \begin{prooftree}
      \AxiomC{$\Gamma \vdash A \otimes B$} \RightLabel{$[\otimes E]$}
      \AxiomC{$\Gamma, A, B \vdash C$}
      \BinaryInfC{$\Gamma \vdash C$}
    \end{prooftree}
  \end{minipage}

    % % par I
    % \begin{minipage}{0.5\textwidth}
    %   \begin{prooftree}
    %     \AxiomC{$\Gamma, A \vdash B$} \RightLabel{$[\parr I]$}
    %     \UnaryInfC{$\Gamma \vdash A \rightspoon B$}
    %   \end{prooftree}
    % \end{minipage}
    % % par E
    % \begin{minipage}{0.5\textwidth}
    %   \begin{prooftree}
    %     \AxiomC{$\Gamma \vdash  A \rightspoon B$}
    %     \AxiomC{$\Delta \vdash A$} \RightLabel{$[\parr E]$}
    %     \BinaryInfC{$\Gamma, \Delta \vdash B$}
    %   \end{prooftree}
    % \end{minipage}

    % oplus
    \begin{minipage}{.20\textwidth}
      \begin{prooftree}
        \AxiomC{$\Gamma \vdash A$} \RightLabel{$[\oplus I_1]$}
        \UnaryInfC{$\Gamma \vdash A \oplus B$}
      \end{prooftree}
    \end{minipage}
    \begin{minipage}{.20\textwidth}
      \begin{prooftree}
        \AxiomC{$\Delta \vdash B$} \RightLabel{$[\oplus I_2]$}
        \UnaryInfC{$\Delta \vdash A \oplus B$}
      \end{prooftree}
    \end{minipage}
    \begin{minipage}{0.6\textwidth}
      \begin{prooftree}
        \AxiomC{$\Gamma \vdash A \oplus B$}
        \AxiomC{$\Delta, A \vdash C$}
        \AxiomC{$\Delta, B \vdash C$}\RightLabel{$[\oplus E]$}
        \TrinaryInfC{$\Gamma, \Delta \vdash C$}
      \end{prooftree}
    \end{minipage}
  \end{framed}
  \caption{Intuionistic Linear Logic Rules}
  \label{fig:linear-logic-rules}
\end{figure}

To escape linearity, exponential $\oc$ is used, which signifies that an assumption can
be duplicted or dropped without restriction. $\oc A$ can be thought of as {\em``as many A's as needed''}.
Thus the intitionsistic $A \rightarrow B$ can be encoded in linear logic by $\oc A \rightspoon B$.
Similarly $A \plus B$ would be represented as $\oc A \otimes \oc B$ and $A \times B$ would be represented as $A \with B$. We clearly see that
this is a much powerful system in contrast to classical intuistionistic logic because of its enhanced expressivity.

\section{Bunched Implications and $\alpha\lambda$ Calculus}

In intuitionistic linear logic, the context is considered as a list or a set. In the theory of
bunched implications (\textbf{\em BI}), the context is treated as a tree in contrast to other logics. Contexts are syntactically
combined using 2 connectives comma ($,$) or a semicolon ($;$) and are called bunches. The logic of \textbf{\em BI}
tries to glue together intuistionistic linear logic with intuistionistic logic by
permitting contexts connected with semicolon to undergo contraction and weakening while the context connected with comma
are prohibited to undergo contraction and weakening. Comma and semicolon do not distributive over each other.
Thus $A,(B;C) \neq A, B ; A,C$ and $A;(B,C) \neq A;B,A;C$ where A B C are contexts.
There are two flavours of implication---additive and multiplicative---which is closely related to the idea of conjunction.
\begin{framed}
\begin{minipage}{1.0\linewidth}
  \begin{prooftree}
    \AxiomC{$\Gamma, A \vdash B$}
    \UnaryInfC{$\Gamma \vdash A \lozenge B$}
  \end{prooftree}
\end{minipage}
\end{framed}
In the logic of {\textbf{\em BI}} the question then faced is choosing what kind of
implication should be used inplace of $\lozenge$---the additive kind or the multiplicative kind.
O'Hearn and Pym [\cite{ohearn_logic_1999}] argue by introducing 2 kinds of arrows
and using them depending on the connectives used for the context. A multiplicative implication ($\sepimp$)
is used when the context is connected with a comma and an additive implication ($\rightarrow$) is used when the
context is connected using semicolon. This gives rise to 2 rules
\begin{framed}
\begin{minipage}{0.5\linewidth}
  \begin{prooftree}
    \AxiomC{$\Gamma, A \vdash B$} \RightLabel{$[\sepimp I]$}
    \UnaryInfC{$\Gamma \vdash A \sepimp B$}
  \end{prooftree}
\end{minipage}
\begin{minipage}{0.5\linewidth}
  \begin{prooftree}
    \AxiomC{$\Gamma; A \vdash B$} \RightLabel{$[\rightarrow I]$}
    \UnaryInfC{$\Gamma \vdash A \rightarrow B$}
  \end{prooftree}
\end{minipage}
\end{framed}

As we see in $[\sepimp I]$ $\Gamma, A$ cannot under go weakening or contraction to duplicate
or get rid of either $A$ or $\Gamma$. This hints to a notion that multiplicative implication ($\sepimp$)
exhibits property of the linear implication ($\rightspoon$). The linear implication cannot however
be directly converted to a multiplicative implication as the later does not exhibit properties of
counting the number of uses of its arguments. The logic of \textbf{\em BI} tries to combine the
additive logic i.e. intuistionistic logic with the multiplicative side i.e. intuistionistic linear logic seemlessly.
The multiplicative side can be used to model the behaviour of resources in the programming language
while the additive side would help the programmers fall back to the non-resource intuistionsitic parts. The logic of \textbf{\em BI}
argues that instead of looking at the number of times an argument is used within the function, it should
be viewed in terms of {\em sharing}.
$\alpha \lambda$ calculus[\cite{ohearn_resource_1999}] is interpretation of the logic of \textbf{\em BI}.
It introduces 2 kinds of arrows by modifiying the the syntax of lambda calculus:
\begin{enumerate}
  \item $\sepimp$     : Function do not share resources with their arguments
  \item $\rightarrow$ : Function may share resources with their arguments
\end{enumerate}

\begin{figure}[h]
\begin{framed}
  \begin{flalign*}
    \text{Types}\ \ \  \tau           &::= \alpha \mid \iota \mid \tau \rightarrow \tau \mid \tau \sepimp \tau \nonumber \\
    \text{Typing Scheme}\ \ \  \sigma &::= \tau \mid \forall \alpha. \tau \nonumber
  \end{flalign*}
\end{framed}
\caption{$\alpha\lambda$-Calculus Types}
\label{fig:al-cal-types}
\end{figure}

\begin{figure}[h]
\begin{framed}
  \begin{flalign*}
    \text{Expressions}\ \ \ M, N ::= x \mid \lambda^{\alpha} x. M \mid \lambda^{*} x. M \mid M N \mid \Let{x}{M}{N}\nonumber
  \end{flalign*}
\end{framed}
\caption{$\alpha\lambda$-Calculus Terms}
\label{fig:al-calc-terms}
\end{figure}

\begin{figure}[h]
  \begin{framed}
    % var
    \begin{minipage}{.5\textwidth}
      \begin{prooftree}
        \AxiomC{$x: \sigma \in \Gamma$} \RightLabel{$[VAR]$}
        \UnaryInfC{$\Gamma \vdash x : \sigma $}
      \end{prooftree}
    \end{minipage}
    % let
    \begin{minipage}{.5\textwidth}
      \begin{prooftree}
        \AxiomC{$\Gamma \vdash M : \sigma \ \ \ \ \
          \Gamma_{x}, x: \sigma \vdash N: \tau$} \RightLabel{$[LET]$}
        \UnaryInfC{$\Gamma \vdash (\texttt{let}\ x\ \texttt{=}\ M\ \texttt{in}\ N) : \tau$}
      \end{prooftree}
    \end{minipage}
    % forall I
    \begin{minipage}{0.5\textwidth}
      \begin{prooftree}
        \AxiomC{$\Gamma \vdash M : \sigma$}\RightLabel{$[\forall I]$}
        \AxiomC{$t \notin \text{fvs}(\Gamma)$}
        \BinaryInfC{$\Gamma \vdash \lambda t. M : \forall t. \sigma$}
      \end{prooftree}
    \end{minipage}
    % forall E
    \begin{minipage}{0.5\textwidth}
      \begin{prooftree}
        \AxiomC{$\Gamma \vdash M : \forall t. \sigma$} \RightLabel{$[\forall E]$}
        \UnaryInfC{$\Gamma \vdash M \tau : [\tau \backslash t] \sigma$}
      \end{prooftree}
    \end{minipage}
    % -> I
    \begin{minipage}{0.5\textwidth}
      \begin{prooftree}
        \AxiomC{$\Gamma_{x}, x: \tau \vdash M : \tau'$} \RightLabel{$[\sepimp I]$}
        \UnaryInfC{$\Gamma \vdash \lambda^{*} x. M : \tau \sepimp \tau'$}
      \end{prooftree}
    \end{minipage}
    % -> E
    \begin{minipage}{0.5\textwidth}
      \begin{prooftree}
        \AxiomC{$\Gamma \vdash M : \tau \sepimp \tau' \ \ \ \ \
          \Gamma \vdash N : \tau$} \RightLabel{$[\sepimp E]$}
        \UnaryInfC{$\Gamma \vdash M N : \tau'$}
      \end{prooftree}
    \end{minipage}
    % -o I
    \begin{minipage}{0.5\textwidth}
      \begin{prooftree}
        \AxiomC{$\Gamma_{x}; x: \tau \vdash M : \tau'$} \RightLabel{$[\rightarrow I]$}
        \UnaryInfC{$\Gamma \vdash \lambda^{\alpha} x. M : \tau \rightarrow \tau'$}
      \end{prooftree}
    \end{minipage}
    % -o E
    \begin{minipage}{0.5\textwidth}
      \begin{prooftree}
        \AxiomC{$\Gamma \vdash M : \tau \rightarrow \tau' \ \ \ \ \
          \Gamma \vdash N : \tau$} \RightLabel{$[\rightarrow E]$}
        \UnaryInfC{$\Gamma \vdash M N : \tau'$}
      \end{prooftree}
    \end{minipage}
  \end{framed}
  \caption{Typing Rules for $\alpha\lambda$ Calculus}
  \label{fig:bi-logic}
\end{figure}


% This is kind of a big jump here.
% probably shift qualified types after linear logic section to have better continuity
\section{Linear logic with Qualified Types: Quill}
We start our work based on Quill [\cite{morris_best_2016}]. It tries
to implement linear types using qualified types. The novelty of using a qualified
types is that it gives a complete and decidable type inference system. By using
a modified version of Algorithm M we can compute principal types of the terms.
In reality due to higher ordered kind system, we may not be able to deduce the
type of the terms but we can work around it by annotating some or all parts of
the terms. This is usually done at a top level function declaration. Specifying types
also serves as some kind of documentation for the programmers.
The key idea of Morris is to introduce 2 types of predicates into the language: \texttt{Un} and \texttt{Fun}.
\texttt{Un $\tau$} implies that the type $\tau$ is unrestricted, meaning it does not
contain any resources or the resources that it captures can be easily duplicated and dropped.
The \texttt{Fun $\tau$} implies that the type $\tau$ is of a function type. The function
depending on its use, may or may not capture resources in its closure. Thus the functions
themselves can be of restricted or unrestricted type. In traditional sense of typeclasses
in haskell, we can think of the \texttt{Un} to be a typeclass with methods supporting the operation
of duplication and dropping.
\begin{figure}[h]
  \centering
  \begin{framed}
    \begin{minted}[escapeinside=||,mathescape=true]{haskell}
      class Un where
          dup  :: t |$\overset{!}{\rightarrow}$| (t |$\otimes$| t)
          drop :: t |$\overset{!}{\rightarrow}$| 1
    \end{minted}
  \end{framed}
  \caption{\texttt{Un} as a Typeclass}
  \label{fig:un-typeclass}
\end{figure}

Simple types such as integers and booleans are all of unrestricted type as
they can be duplicated or dropped freely.
While program resources such as file handles, database connections
should be treated as restricted type as we cannot freely duplicate it
or drop it. Although there would be certain portions of the program where we would
like to close the file handle or free the memory space allocated on the heap.
(This is where we expected the bunched implications would play a crucial role. I guess.)

Combining linear logic with qualified types in Quill
%%% Local Variables:
%%% mode: latex
%%% TeX-master: "../thesis-ku"
%%% End:

\chapter{Implementing Typechecking in Quill}


\section{Basic Definitions}

We go over the definitions of the base language and some conventions that we would
follow thorough out this thesis.

\section{Language Syntax and Types}

Language will be a pumped up version of simply typed lambda calculus with kind support

\begin{figure}[h]
  \begin{framed}
      \begin{flalign*}
      \text{Term Variables}\ \ \      x, y, z         &\in \text{Var} \nonumber  \\
      \text{Type Variables}\ \ \      t, u, v         &\in \text{TVar}  \nonumber\\
      \text{Kinds}\ \ \               \kappa          &::= * \mid \kappa \rightarrow \kappa \nonumber\\
      \text{Types}\ \ \               \tau^{\kappa}    &::= t \mid C^{\kappa} \mid \tau^{\kappa \rightarrow \kappa \rightarrow \kappa}\nonumber\\
      \text{Type constructors}\ \ \   C^{\kappa}       &::= C^{* \rightarrow * \rightarrow *} | \overset{!}{\sepimp} | \sepimp | \xrightarrow{!} | \rightarrow \nonumber\\
      \text{Predicates}\ \ \          \pi             &::= \texttt{Un}\ \tau \mid \texttt{SeFun}\ \tau \mid \texttt{ShFun}\ \tau \mid \tau \geq \tau' \nonumber\\
      \text{Qualified Types}\ \ \     \rho            &::= \tau^{*} \mid \pi \Rightarrow \rho \nonumber\\
      \text{Type schemes}\ \ \        \sigma          &::= \rho \mid \forall t. \sigma \nonumber\\
      \text{Environments}\ \ \      \Gamma,\Delta     &::= \epsilon \mid x:\sigma \mid \Gamma, \Delta \mid \Gamma; \Delta \nonumber\\
      \text{Environment Context}\ \ \ H               &::= H,H' \mid H;H' \mid \fbox{$\phantom{5}$} \nonumber\\ %\text{what is this how is it different from Environment?}
      \text{Expressions}\ \ \         M, N            &::= x \mid \lambda^{*}x. M \mid \lambda^{\alpha}x. M \mid M N\nonumber\\
                                                      &      \mid \texttt{let}\ x\ \texttt{=}\ M\ \texttt{in}\ N\nonumber\\
                                                      &      \mid \texttt{case}\ M\ \texttt{of}\ \{K_i\ x_{1i}\ x_{2i}\ \ldots\ x_{ji}\}_i\nonumber
    \end{flalign*}

% TODO work on this

\end{framed}
\caption{Language Syntax}
\label{fig:language-syntax}
\end{figure}


\section{Conventions and Notations}
$\Gamma_{x}$ is the typing environment excluding the type variable $x$. $TV(\Gamma)$ are the free
variables in the environment $\Gamma$.
%\begin{itemize}
% \item $\Pi$ varies over predicates
% \item $\Rightarrow$ qualified types
% \item $\Gamma$ and $\Delta$ are typing environments
% \item $,$ is a separating concatination of environment
% \item $;$ is a sharing concatination of environment
% \item $SeFun$ is a predicate of separating functions
% \item $\sepimp$ for separating function application
% \item $ShFun$ is a predicate of sharing functions
% \item $\rightarrow$ is for sharing function application
% \item $\geq$ lesser restricting
% \end{itemize}

\section{Syntax Directed Typing Judgements}

\begin{figure}[h]
  \begin{framed}
    % var
    \begin{minipage}{.5\textwidth}
      \begin{prooftree}
        \AxiomC{} \RightLabel{$[VAR]$}
        \UnaryInfC{$\Pi \mid x : \sigma \vdash x : \sigma $}
      \end{prooftree}
    \end{minipage}
    % let
    \begin{minipage}{.5\textwidth}
      \begin{prooftree}
        \AxiomC{$\Pi \mid \Gamma \vdash M : \sigma \ \ \ \ \
          \Pi' \mid \Gamma_{x}, x: \sigma \vdash N: \tau$} \RightLabel{$[LET]$}
        \UnaryInfC{$\Pi \cup \Pi' \mid \Gamma \vdash \texttt{let}\ x\ \texttt{=}\ M\ \texttt{in}\ N: \tau$}
      \end{prooftree}
    \end{minipage}
    % => I
    \begin{minipage}{0.5\textwidth}
      \begin{prooftree}
        \AxiomC{$\Pi, \pi \mid \Gamma \vdash M : \rho$} \RightLabel{$[\Rightarrow I]$}
        \UnaryInfC{$\Pi \mid \Gamma \vdash M : \pi \Rightarrow \rho$}
      \end{prooftree}
    \end{minipage}
    % => E
    \begin{minipage}{0.5\textwidth}
      \begin{prooftree}
        \AxiomC{$\Pi \mid \Gamma \vdash M : \pi \Rightarrow \rho \ \ \ \ \
          \Pi \Rightarrow \pi$} \RightLabel{$[\Rightarrow E]$}
        \UnaryInfC{$\Pi \mid \Gamma \vdash M: \rho$}
      \end{prooftree}
    \end{minipage}

    % forall I
    \begin{minipage}{0.5\textwidth}
      \begin{prooftree}
        \AxiomC{$\Pi \mid \Gamma \vdash M: \sigma \ \ \ \ \ \
          t \notin TV(\Gamma) \cup TV(\Pi)$}\RightLabel{$[\forall I]$}
        \UnaryInfC{$\Pi \mid \Gamma \vdash M: \forall t. \sigma$}
      \end{prooftree}
    \end{minipage}
    % forall E
    \begin{minipage}{0.5\textwidth}
      \begin{prooftree}
        \AxiomC{$\Pi \mid \Gamma \vdash M: \forall t.\sigma$}\RightLabel{$[\forall E]$}
        \UnaryInfC{$\Pi \mid \Gamma \vdash M: [\tau / t] \sigma $}
      \end{prooftree}
    \end{minipage}

    % -&> I
    \begin{minipage}{1\textwidth}
      \begin{prooftree}
        \AxiomC{$\Pi \mid \Gamma_{x};x: \tau \vdash M : \tau' \ \ \ \ \
          \Pi \Rightarrow \texttt{ShFun} \phi \ \ \ \
          \Pi \vdash \Gamma \geq \phi$} \RightLabel{$[\rightarrow I]$}
        \UnaryInfC{$\Pi \mid \Gamma \vdash \lambda^{\alpha}x. M \tau : \phi \tau \tau'$}
      \end{prooftree}
    \end{minipage}
    % -&> E
    \begin{minipage}{1\textwidth}
      \begin{prooftree}
        \AxiomC{$\Pi \mid \Gamma \vdash M : \phi \tau \tau' \ \ \ \ \
          \Pi \mid \Delta \vdash N : \tau' \ \ \ \ \
          \Pi \Rightarrow \texttt{ShFun} \phi \ \ \ \
          \Pi \vdash \Gamma \geq \phi$} \RightLabel{$[\sepimp E]$}
        \UnaryInfC{$\Pi \mid \Gamma,\Delta \vdash M N : \tau'$}
      \end{prooftree}
    \end{minipage}

    % -*> I
    \begin{minipage}{1\textwidth}
      \begin{prooftree}
        \AxiomC{$\Pi \mid \Gamma_{x},x: \tau \vdash M : \tau' \ \ \ \ \
          \Pi \Rightarrow \texttt{SeFun} \phi \ \ \ \
          \Pi \vdash \Gamma \geq \phi$} \RightLabel{$[\sepimp I]$}
        \UnaryInfC{$\Pi \mid \Gamma \vdash \lambda^{*}x. M : \phi \tau \tau'$}
      \end{prooftree}
    \end{minipage}
    % -*> E
    \begin{minipage}{1\textwidth}
      \begin{prooftree}
        \AxiomC{$\Pi \mid \Gamma \vdash M : \phi \tau \tau' \ \ \ \ \
          \Pi \mid \Delta \vdash N : \tau \ \ \ \ \
          \Pi \Rightarrow \texttt{ShFun} \phi \ \ \ \
          \Pi \vdash \Gamma \geq \phi$} \RightLabel{$[\rightarrow E]$}
        \UnaryInfC{$\Pi \mid \Gamma;\Delta \vdash M N : \tau'$}
      \end{prooftree}
    \end{minipage}
    % user defined type constructor I
    \begin{minipage}{1\textwidth}
      \begin{prooftree}
        \AxiomC{$\Pi \mid \Gamma \vdash M: \tau       \ \ \ \ \ \
          \{\vdash K_i:\bar{x}_i \rightsquigarrow H    \ \ \ \ \ \
          \Delta \cdot H \vdash M: v \}_i$}\RightLabel{$[K_i I]$}
        \UnaryInfC{$\Pi \mid \Gamma \cdot \Delta \vdash \texttt{case}\ M\ \texttt{of}\ \{K_i\ x_{1i}\ x_{2i} \ldots\ x_{ji} \rightarrow N_i\}_i : v $}
      \end{prooftree}
    \end{minipage}
    % user defined type constructor E
    \begin{minipage}{1\textwidth}
      \begin{prooftree}
        \AxiomC{$\Pi \mid \Gamma \cdot \Delta \vdash \texttt{case}\ M\ \texttt{of}\ \{K_i\ x_{1i}\ x_{2i} \ldots\ x_{ji} \rightarrow N_i\}_i : v $}\RightLabel{$[K_i E]$}
        \UnaryInfC{$Elemenation TODO!$}
      \end{prooftree}
    \end{minipage}

    \caption{Typing Rules}
    \label{fig:typing-rules}
  \end{framed}
\end{figure}


Here is how the naive recursive algorithm works

For Lambda
\begin{enumerate}
\item Get the bound variable
\item Assign a new type for bound variable
\item assign a new type variable for the body
\item Evaluate type of the body and assign it to the body type varialbe
\item return the type as (type of var) -> (type of body)
\end{enumerate}

Body can be made up of another lambda. In this case the same 5 steps will be triggered
or it is an application of 2 or more variables
In case of application 2 new type variable:
\begin{enumerate}
\item introduce type variables for left expression and right expression
    left expression a type $A \rightarrow B$ right expression gets the type $A$
\item recursively type check both of them
\item return the type of the complete application as $B$ if the type checking in previous step is successfull
\end{enumerate}


In Quill there are 2 kinds of lambdas:
\begin{enumerate}
\item Sharing Lambda $\lambda^{\alpha}$
\item Separating Lambda $\lambda^{*}$
\end{enumerate}
This helps us specify if the 2 variables are separated or have sharing
of resources between them.
The church encoding of a sharing pair will be represented as\\
$shPair = \lambda^{*} x \rightarrow \lambda^{\alpha} y \rightarrow \lambda^{*} sh \rightarrow sh\ x\ y$\\
This means that x and y may share resources
A sparating pair is represented as\\
$sePair = \lambda^{*} x \rightarrow \lambda^{*}y \rightarrow \lambda^{*}se \rightarrow se\ x\ y$\\
This means that x and y do not share resources.
%By default we may assume that resources are always separating unless explicitly specified
%that they are sharing.

\section{Typing environment}

The typing environment in standard Milner-Damas algorithm
is a pair of identifier and its type.
We need to modify the typing envrionment so that it describes sharing.
% There can be many different ways of doing it.
In the current implmenetation we have
extended the typing environment to hold 2 more entities along with the
type of the indentifier, a list of list of identifiers --- that describes the sharing of variables,
a scope tag---that identifies if the variable is global in the complete module or local to the definition.
Global variables can be used anywhere in the file or other code file if it is imported
All function names will be defaulted to global scope.
Local variables can be used only after they have been bound in the typing environment.
The new typing envionment can be realized as:
\begin{minted}{haskell}
  type Env = Map Id (Type, [[Id]], Scope)
\end{minted}

% how is the list of list of ids help in identifying sharing

% how do you define a closure

% What do you mean by having a break in the closure

% The used field in the typechecker state

\section{Modification to Typechecking Algorithm}

To incorporate the sharing, we have to modify the typechecking algorithm.
The main reason to do that is to avoid adding $Un$ predicates to
shared variables that are not used. Take an example of $fst$ function
that returns the first argument of a shared pair
\begin{verbatim}
fst = \x -> \&y -> x
\end{verbatim}
In this case y should not be assigend an $Un$ predicate because
it is shared with x and x is indeed used.

The type checking algorithm has to be tweeked so that we keep track
of what all variables are shared and which ones are separate.

When we encounter an Alpha Lambda
\begin{enumerate}
\item Get the bound variable
\item Assign a new type for bound variable
\item Assign a new type variable for the body
\item add the variable in a sharing context
\item Evaluate type of the body and assign it to the body type variable
\item return the type as (type of var) -> (type of body)
\end{enumerate}


% The main problem is that we do not know when a sharing variable should be kept in scope
% or removed. In some cases we may need to keep it around for introducing the (>:=) predicates
% Some cases are 2 types:
% 1) The variable is used
% 2) The variable is not used

The type checking calls occur from left to right recursively.
While going down the recursion we keep on adding variables to the envionment
We stop recursing at application. where we compute whether we have complete sharing to assign ShFun or SeFun.

While folding out of of the recursion we do 2 things:
1) keep track of the used variables (including implicitly used variables due to sharing)
2) generate new goals
   a) depending on whether the the introduced variable was used
      or its sharing member was used. Assign Un predicates and weaken if it is not used at all
   b) introduce lesser-restricted predicates
3) Generate other new assumptions

We cannot determine when to get rid of the complete bunch.
\begin{verbatim}
\z -> \*x -> \&y -> y
\end{verbatim}
hence, when we are folding out of the final recursive but 1 call ie. for Lambda bound variable x


%%% Local Variables:
%%% mode: latex
%%% TeX-master: "Typechecking"
%%% End:

\chapter{Datatypes with resources}

The datatypes are user defined types:
There are 2 main kinds of non-recursive datatypes

1) data Choice a b = L a | R b

where the types a and b do not appear in both the constructors.
Choice here is a sum type a + b. How can we handle sum types in the current scenario?
1 way is to have a and b both not share resources. in this case we will have
a and b to be both un restricted.

a and b cannot direclty share resources. As the meaning of a sum type is that
either a or b can exist and not both. So we will always have separating arrow
for the sum types

Write the elemination rules for Either

for functions the closure has to be equal.

2) data Pair a b = P a b

The constructor uses both the resources a and b
Here we can have 2 subtypes:

a and b share resources or can be completely different.
We would want the programmer to specify whether the arguments are shared or separate.

For current purposes this syntactical specification is given via a modified syntax

data SeparatingPair a b = MkSeP a, b
data SharingPair a b = MkShP a; b

We can of-course mix and match the sharing or separation.

data MixTuple a b c d e = MkMixTuple a,b;c,d

The question now that needs to be answered is how should the sharing be interpreted?
There are various options.

1. interpreted right to left. i.e. MkMixTuple (a, (b; (c,d)))
   so the b is shared with separating pair (c,d) and a is separate than b

2. interpreted left to right. i.e. MkMixTuple (((a, b);c),d)
   so the c is shared with separating pair (a,b) and d is separate from c

3. Sharing ; has precedence over separating , i.e. MkMixTuple a,(b;c),d

4. Separating , has lower precedence over ; i.e. MkMixTuple (a,b);(c,d)

This should be handled during the AST generating while parsing.
The type checking algorithm should be robust enough
to perform type checking either of these 4 types of sharing as they are
all perfectly valid scenarios.

Recursive datatypes have some kind of structural similarities within themselves

data Tree a = Leaf a | Node a (Tree a) (Tree a)

data Tree a b = Leaf b | Node a (Tree a b) (Tree a b)

Open questions:
How do we specify sharing of data between nodes?
eg. How can we say that all the leaves have shared data? and the data in nodes is sparating


What happens when lesser unrestrictedness with datatypes? Choice a b >:= f


%%% Local Variables:
%%% mode: latex
%%% TeX-master: t
%%% End:

\section{Type Extensions}

We add 2 new type classes to our language
\begin{itemize}
\item Unrestricted values will have a type Un
  that represents the fact that the resource can be
  duplicated/discarded or they cannot.
  This expressivity helps us understand how the functions
  use the resources. If the function uses a file handle as a resource
  in a restricted fashion. It would not call duplicate or drop on it.
  Its return value would contain the reference to the resource
  On the other hand if the file handle is being opened or closed
  by a function, it will have to be in an unrestricted fashion.
\item All the functions are of class $->$ can potentially have any of the 4 choices
  \begin{lstlisting}
                  ShFun SeFun

    Un            $\overset{!}{\rightarrow}$    $\overset{!}{\sepimp}$

   Substructural   $\rightarrow$     $\sepimp$
  \end{lstlisting}

  $\rightarrow$ implies that the function application is of the sharing type.
  i.e. the resources used in the function have a may have a sharing closure.
  represented as $-\&>$

  $\sepimp$ implies that the function applciation is of a separating type.
  the function may not share resources with its arguments

  $\overset{!}{\rightarrow}$ implies that the function itself can be discarded
  or used multiple number of times. i.e. its closure
  can be duplicated or discarded. And the function shares resources with the arguments

  $\overset{!}{\sepimp}$ implies that the function itself can be discarded.
  This is useful in case of higher order functions.

\begin{minted}{haskell}
instance Monad (-!*>) Maybe where
   return a = Just a
   (>>=) a f = case a of
                 Nothing -> Nothing
                 Just v  -> f v
\end{minted}

  In the first branch of case expression, f is never used and discarded. Hence
  the type of the function should be $-!*>$
  This may not always be the case. As with a non-empty list
  the operation f is performed atleast once.

\begin{minted}{haskell}
instance Monad (-!*>) NEList where
   return a   = Last a
   (>>=) m f  = case a of
                  Last a    -> f a
                  Cons a as -> concat (f a) (as >>= f)
\end{minted}

\end{itemize}


%%% Local Variables:
%%% mode: latex
%%% TeX-master: t
%%% End:


\global\long\def\bibname{References}

\bibliographystyle{apalike2}
\bibliography{Biblio/allcites}

\appendix

\chapter{My Appendix, Next to my Spleen}

There could be lots of stuff here

\end{document}

%%% Local Variables:
%%% mode: latex
%%% TeX-master: t
%%% End:
