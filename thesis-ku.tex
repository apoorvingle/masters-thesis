%% LyX 2.2.0 created this file.  For more info, see http://www.lyx.org/.
%% Do not edit unless you really know what you are doing.
\documentclass[12pt,english]{kuthesis}
\usepackage{mathptmx}
\usepackage{MnSymbol}
\renewcommand{\sfdefault}{lmss}
\renewcommand{\ttdefault}{lmtt}
\usepackage[T1]{fontenc}
\usepackage[utf8]{inputenc}
\usepackage{geometry}
\geometry{verbose,tmargin=1in,bmargin=1in,lmargin=1in,rmargin=1in}
\setcounter{secnumdepth}{3}
\setcounter{tocdepth}{3}

\usepackage{xcolor}
\usepackage{babel}
\usepackage{url}
\usepackage{graphicx}
\usepackage{setspace}
\usepackage{esint}
\usepackage[authoryear]{natbib}

\usepackage{bussproofs}
\usepackage{semantic}
\usepackage{framed}
\usepackage{minted}
\usepackage{cmll}
\usepackage{amsthm}
\usepackage{pdflscape}
\usepackage{stmaryrd}
\usepackage{xspace}

\usepackage{tikz}
\usepackage{tikz-qtree}
%\usetikzlibrary{arrows.meta}


\doublespacing
\usepackage[unicode=true,
 bookmarks=true,bookmarksnumbered=false,bookmarksopen=false,naturalnames=true
 breaklinks=true,pdfborder={0 0 0},pdfborderstyle={},backref=false,colorlinks=true]
 {hyperref}
\hypersetup{pdftitle={QuB Resource Aware Functional Programming Language},
 pdfauthor={Apoorv Ingle},
 pdfsubject={Masters Thesis},
 urlcolor={blue},citecolor={blue},allcolors={blue}}
%\PassOptionsToPackage{unicode}{hyperref}
%\PassOptionsToPackage{naturalnames}{hyperref}

\usepackage[capitalise, nameinlink]{cleveref}
\crefformat{section}{\S#2#1#3} % see manual of cleveref, section 8.2.1
\crefformat{subsection}{\S#2#1#3}
\crefformat{subsubsection}{\S#2#1#3}


\renewcommand\qedsymbol{$\blacksquare$}

\newtheorem{theorem}{Theorem}[chapter]
\newtheorem{lemma}{Lemma}[chapter]

\newtheorem{innercustomgeneric}{\customgenericname}
\providecommand{\customgenericname}{}

\newcommand{\newcustomthm}[2]{%
  \newenvironment{#1}[1]
  {%
   \renewcommand\customgenericname{#2}%
   \renewcommand\theinnercustomgeneric{##1}%
   \innercustomgeneric
  }
  {\endinnercustomgeneric}
}

\newcommand{\newcustomlem}[2]{%
  \newenvironment{#1}[1]
  {%
   \renewcommand\customgenericname{#2}%
   \renewcommand\theinnercustomgeneric{##1}%
   \innercustomgeneric
  }
  {\endinnercustomgeneric}
}

\newcustomthm{ctmtheorem}{Theorem}
\crefname{ctmtheorem}{Theorem}{Theorems}
\Crefname{ctmtheorem}{Theorem}{Theorems}

\newcustomlem{ctmlemma}{Lemma}
\crefname{ctmlemma}{Lemma}{Lemmas}
\Crefname{ctmlemma}{Lemma}{Lemmas}

\theoremstyle{definition}
\newtheorem{defn}{Definition}

\makeatletter

%% Because html converters don't know tabularnewline
\providecommand{\tabularnewline}{\\}

\newcommand{\tightoverset}[2]{%
  \mathop{#2}\limits^{\vbox to -.5ex{\kern-0.75ex\hbox{$#1$}\vss}}}

%%%%%%%%%%%%%%%%%%%%%%%%%%%%%% User specified LaTeX commands.
\newcommand{\BI}{\textbf{\em BI}\xspace}
\newcommand\sepimp{\mathrel{-\mkern-6mu*}}
\newcommand\shimp{\twoheadrightarrow}

\newcommand{\M}{\mathcal{M}}
\newcommand{\Unf}{\mathcal{U}}

\newcommand{\SeFun}[1]{\texttt{SeFun}\ #1}
\newcommand{\ShFun}[1]{\texttt{ShFun}\ #1}
\newcommand{\Un}[1]{\texttt{Un}\ #1}
\newcommand{\vdashs}{\vdash^s}
\newcommand{\Gen}[1]{\texttt{Gen}(#1)}
\newcommand{\Let}[3]{\texttt{let}\ #1 = #2\ \texttt{in}\ #3}
\newcommand{\Case}[2]{\texttt{case}\ #1\ \texttt{of}\ #2}
\newcommand{\Fst}[1]{\texttt{fst}\ #1}
\newcommand{\Snd}[1]{\texttt{snd}\ #1}
\newcommand{\Inr}[1]{\texttt{inr}\ #1}
\newcommand{\Inl}[1]{\texttt{inl}\ #1}
\newcommand{\Pair}[1]{\langle #1 \rangle}
\newcommand{\HaskellF}[1]{\mintinline[mathescape=true, escapeinside=||]{haskell}{#1}}
\newcommand{\qub}{QuB\xspace}

%\newcommand{\TODO}[1]{\fbox{\color{red} #1}}
% used to align decimals in tables according to APA style
\usepackage{dcolumn}
\usepackage{booktabs}

% Set the title and author info
\title{\qub{}: A Resource Aware Functional Programming Language}
\author{Apoorv Ingle}


\dept{Electrical Engineering and Computer Science}
\degreetitle{Master of Science}
\papertype{Thesis} %or Thesis (Whatever you put will appear on p.2)
%% It is vital to have 7 entries, even if some are empty for committee and role
%% I mean, it is vital to leave the empty place holders
\committee{Dr. J. Garrett Morris}{Dr. Perry Alexander}{Dr. Andy Gill}{Dr. Prasad Kulkarni}{}{}{}
\role{Chairperson}{}{}{}{}{}{}
%AT Most 7 members allowed, last here is blank on purpose to demonstrate
%flexibility

%% The following is OPTIONAL. Remove all 3 of the next 3 lines
%% to leave dates blank. If dates are included, then both dates
%% must be included.
\@printd@testrue
\datedefended{}
\dateapproved{}

%% These settings are now in the kuthesis.cls file, but users are free
% to customize. listings has great documentation online
%% When listings are used, break lines
%\lstset{
 %    breaklines=true,  % sets automatic line breaking
 %    breakindent=2em,
 %    breakatwhitespace=true,  % sets if automatic breaks should
 %   breakautoindent=true
%}

\@ifundefined{showcaptionsetup}{}{%
 \PassOptionsToPackage{caption=false}{subfig}}
\usepackage{subfig}
\makeatother

\usepackage{listings}
\renewcommand{\lstlistingname}{\inputencoding{latin9}Listing}

\begin{document}
\begin{romanpages}

\maketitle
\begin{abstract}
  Modern programming languages treat resources as normal values.
  The static semantics of resources in such languages does not match their runtime semantics.
  In this thesis, we tackle the resource management problem by making resources
  first class citizens in the language, and concentrating on sharing or separation of resources.

  We design and implement \qub{} (pronounced: cube), a Curry-Howard interpretation of logic
  of bunched implications (\BI) \citep{ohearn_logic_1999}.
  We distinguish two kinds of values---restricted
  and unrestricted---and two kinds of function implications---sharing and separating.
  The restricted values model resources while the unrestricted values model program objects that do not contain any resources.
  Sharing functions denote that functions share resources with its arguments,
  while separating functions denote that functions do not share resources with its arguments.
  We show how the use of monads with sharing and separating functions helps in modeling patterns, such as exception handling,
  that are difficult to express in linear languages.

  \begin{acknowledgementslong}
    I am grateful to my parents and my brother. Their persistent support and encouragement has been a crucial part of my education and my life otherwise.

    I am thankful to my advisor, Dr. Garrett Morris. Without his immense patience, wholesome guidance, and
    critical insights, it would have been impossible to make this journey.

    I am appreciative of my friends, who helped me stay sane by showing me the world outside work.

    And of course, those countless cups of coffee that subsisted my morale in the writing task.
  \end{acknowledgementslong}

\end{abstract}
\tableofcontents{}

\listoffigures

\end{romanpages}

\chapter{Background Work}
% TODO should change title

\section{Type Inference Algorithm}
Algorithm $\mathcal{W}$ [\cite{damas_principal_1982}] and its varient algorithm $\mathcal{M}$ [\cite{lee_proofs_1998}]
are the basis of most of the modern statically typed programming languages. Type inference
is decidable in the sense, type checking algorithm always completes with a success or failure.
The algorithms also gurantee a most general typing scheme for an expression in
the simply typed lambda calculus extended with a polymorphic let construct having a term language
\begin{framed}
  \begin{flalign*}
    \text{Expressions}\ \ \ M, N ::= x: \sigma \mid \lambda x: \tau. M \mid M N \mid \texttt{let}\ x\ \texttt{=}\ M\ \texttt{in}\ N \nonumber
  \end{flalign*}
\end{framed}
and a type language specified by
\begin{framed}
  \begin{flalign*}
    \text{Types}\ \ \  \tau    &::= \alpha \mid \iota \mid \tau \rightarrow \tau \nonumber \\
    \text{Typing Scheme}\ \ \  &::= \tau \mid \forall \alpha. \tau \nonumber
  \end{flalign*}
\end{framed}
where $\alpha$ is a type variable, $\iota$ are primitive types in the language, $\rightarrow$
is a type constructor and $\sigma$ is a typing scheme.

Robinson's unification algorithm [\cite{robinson_machine-oriented_1965}] plays a key role
in ensuring that types are well formed. Its purely syntactic approach in creating
substitutions to unify types keeps the complete process elegent.
The algorithm works in an interesting way where the types of all well-typed terms can be
inferred automatically and if types are specified, the same algorithm can be used
to match the expression term.

\TODO{
Talks about Curry-Howard here.
Logic rules and corresponding Typing rules.
Typing rules are nothing but logic rules with terms and types.
Types correspond to predicates
}
\begin{figure}[h]
  \begin{framed}
    % var
    \begin{minipage}{.5\textwidth}
      \begin{prooftree}
        \AxiomC{$x: \sigma \in \Gamma$} \RightLabel{$[VAR]$}
        \UnaryInfC{$\Gamma \vdash x : \sigma $}
      \end{prooftree}
    \end{minipage}
    % let
    \begin{minipage}{.5\textwidth}
      \begin{prooftree}
        \AxiomC{$\Gamma \vdash M : \sigma$}
        \AxiomC{$\Gamma_{x}, x: \sigma \vdash N: \tau$} \RightLabel{$[LET]$}
        \BinaryInfC{$\Gamma \vdash (\Let{x}{M}{N}) : \tau$}
      \end{prooftree}
    \end{minipage}
    % forall I
    \begin{minipage}{0.5\textwidth}
      \begin{prooftree}
        \AxiomC{$\Gamma \vdash M : \sigma$}\RightLabel{$[\forall I]$}
        \AxiomC{$t \notin \text{fvs}(\Gamma)$}
        \BinaryInfC{$\Gamma \vdash \lambda t. M : \forall t. \sigma$}
      \end{prooftree}
    \end{minipage}
    % forall E
    \begin{minipage}{0.5\textwidth}
      \begin{prooftree}
        \AxiomC{$\Gamma \vdash M : \forall t. \sigma$} \RightLabel{$[\forall E]$}
        \UnaryInfC{$\Gamma \vdash M \tau : [\tau \backslash t] \sigma$}
      \end{prooftree}
    \end{minipage}
    % -> I
    \begin{minipage}{0.5\textwidth}
      \begin{prooftree}
        \AxiomC{$\Gamma_{x}, x: \tau \vdash M : \tau'$} \RightLabel{$[\rightarrow I]$}
        \UnaryInfC{$\Gamma \vdash \lambda x. M : \tau \rightarrow \tau'$}
      \end{prooftree}
    \end{minipage}
    % -> E
    \begin{minipage}{0.5\textwidth}
      \begin{prooftree}
        \AxiomC{$\Gamma \vdash M : \tau \rightarrow \tau'$}
        \AxiomC{$\Gamma \vdash N : \tau$} \RightLabel{$[\rightarrow E]$}
        \BinaryInfC{$\Gamma \vdash M N : \tau'$}
      \end{prooftree}
    \end{minipage}
  \end{framed}
  \caption{Logic Rules for Simply Typed Lambda Calculus}
  \label{fig:stlc-logic}
\end{figure}
The logical rules for type inference are shown in \ref{fig:stlc-logic}. $\Gamma$ is the
context or assumptions in which the expression is typed. The $[VAR]$ rule is tautology or a simple
lookup of the term variable $x$ in the context $\Gamma$. The $[LET]$ allows creating local
definitions within an expression term. $[\rightarrow I]$ and $[\rightarrow E]$ are rules
for typing lambda terms and application respectively. We also include the rules for
type application and abstraction $[\forall I]$ and $[\forall E]$ to introduce second order
quantification in predicate logic.

% This simple type sytem is powerful in its
% expressivity and can encode a large variety of computations. The type checking algorithm
% asserts that undefined programs can be be detected statically i.e. without actually
% running the program or as famously known as ``well typed programs do not go wrong''.
% This is extremely useful for programmers who are building
% complex real world softwares. Bad programs can be eleminated instantaneously while
% being written using a mechanize technique so that the programmer can concentrate on designing the logic
% rather than fighting undefinedness of the programs. This creates an excellent feedback loop
% to the programmer while building large software systems. % TODO too generic should it be in introduction?
% TODO Give examples? and come up with a Curry-Howard interpretation of

\section{Qualified Types}
Jones [\cite{jones_theory_1994}] proposed incorporating predicates in the type language.
Predicates are used to build constraints on the domain of the type of a term in the language expression.
It introduces additional layer between polymorphic and monomorphic typing of programs.
A modification of Milner-Damas algorithm to encorporate predicates ensures that type inference
is sound and complete. The types that satisfy all the predicates are called qualified types for the term.
Qualified types are powerful enough to expresses type classes with functional dependencies,
record types and subtyping [\cite{mark_type_2000}]. The type language is modified to contain
qualified types. $P$ and $T$ range over finite set of predicates. We slightly modify the typing rules
from \cref{fig:stlc-logic} to add 2 new rules for qualified types as shown in \cref{fig:qualified-types-rules}
\begin{figure}[h]
  \centering
  \begin{framed}
  \begin{flalign*}
    \text{Types}\ \ \ \tau              &::= \alpha \mid \iota \mid \tau \rightarrow \tau \nonumber \\
    \text{Qualified Types}\ \ \ \rho    &::= P \Rightarrow \tau \nonumber \\
    \text{Type Scheme}\ \ \ \sigma      &::= \tau \mid \forall T. \rho \nonumber
  \end{flalign*}
\end{framed}
\caption{Qualified Types}
\label{fig:qualifed-types}
\end{figure}
\begin{figure}[h]
  \begin{framed}
    % => I
    \begin{minipage}{0.5\textwidth}
      \begin{prooftree}
        \AxiomC{$P, \pi \mid \Gamma \vdash M : \rho$} \RightLabel{$[=> I]$}
        \UnaryInfC{$P \mid \Gamma \vdash M : \pi \Rightarrow \rho$}
      \end{prooftree}
    \end{minipage}
    % => E
    \begin{minipage}{0.5\textwidth}
      \begin{prooftree}
        \AxiomC{$P \mid \Gamma \vdash M : \pi \Rightarrow \rho$}
        \AxiomC{$P \Rightarrow \pi$} \RightLabel{$[=> E]$}
        \BinaryInfC{$P \mid \Gamma \vdash M: \rho$}
      \end{prooftree}
    \end{minipage}
  \end{framed}
  \caption{Modified Typing Rules}
  \label{fig:qualified-types-rules}
\end{figure}
\section{Linear Logic}
% TODO: points to cover
% what is linearity
% restricting weakening and contraction
While classical logic deals with truth of propositions, linear logic deals with availability of resources.
Linear logic [\cite{girard_linear_1987}] promises to help cope with the resource and resource control problem.
It is refinement of classical intuistionistic logic. The core idea is that propositions
cannot be freely duplicated or discarded as in classical instuistionistic logic.
In formal terms, the contraction and weakening of logical rules are restricted.
This instigates a view of propositions to behave like resources. In real world software applications,
resources may not be freely copied or dropped from a program context.
Program entities like database connections, file handles or even
in memory shared state are pet peeves for programmers writing
industry grade software. Linear logic hopes to be a remedy for
these problems. If contraction and weakening is completely abandoned,
the system gets overly restrictive. Wadler describes a refinement of
linear logic based on Girard's Logic of Unity [\cite{wadler_taste_1993}, \cite{girard_unity_1993}].
It works around the problem of linear logic being too restrictive by allowing
instuistionistic rules in fragments. It can be considered as a disjoint union
of classical linear logic and intuistionistic logic. The grammar of classical intuistionistic logic is shown in \ref{fig:intu-logic-grammar}
where $A \rightarrow B$ implies implication, $A \times B$ is conjunction and $A \plus B$ is disjunction.
\begin{figure}
  \centering
  \begin{framed}
  \begin{flalign*}
    A, B, C ::= X \mid A \vdash B \mid A \rightarrow B \mid A \times B \mid A \plus B
  \end{flalign*}
\end{framed}
\caption{Grammar for Intuistionistic Logic}
\label{fig:intu-logic-grammar}
\end{figure}

In a pure linear logic setting, none of the assumptions can be used more than once (weakening prohibited) and they cannot be discarded
(contraction prohibited) This gives rise to a different flavor of all the logical connectives.
$A \rightspoon B$ describes the new implication meaning and is read as {\em``consume A to give B''} its logical rules
is given by $\rightspoon I$ and $\rightspoon E$. Similarly there are 2 kinds of connectives, multiplicative and additive that
arise in this logic system. More symbols are added inplace of $\plus$ and $\times$ to differentiate between the
multipicative conjunction and disjuntion ($\otimes$ and $\parr$), and additive conjuntion and disjunction ($\with $ and $\oplus$).
While working in intuistionistic linear logic, $\parr$ is dropped from the system as it can be encoded by other connectives.

\begin{figure}
  \centering
  \begin{framed}
    \begin{flalign*}
      A, B, C ::= X \mid \oc A \mid A \vdash B \mid A \rightspoon B \mid A \with B \mid A \otimes B
    \end{flalign*}
  \end{framed}
  \caption{Grammar for Intuistionistic Linear Logic}
\end{figure}

\begin{figure}[h]
  \begin{framed}
    % -o I
    \begin{minipage}{0.5\textwidth}
      \begin{prooftree}
        \AxiomC{$\Gamma, A \vdash B$} \RightLabel{$[\rightspoon I]$}
        \UnaryInfC{$\Gamma \vdash A \rightspoon B$}
      \end{prooftree}
    \end{minipage}
    % -o E
    \begin{minipage}{0.5\textwidth}
      \begin{prooftree}
        \AxiomC{$\Gamma \vdash  A \rightspoon B$}
        \AxiomC{$\Delta \vdash A$} \RightLabel{$[\rightspoon E]$}
        \BinaryInfC{$\Gamma, \Delta \vdash B$}
      \end{prooftree}
    \end{minipage}
  % & I
  \begin{minipage}{.3\textwidth}
    \begin{prooftree}
      \AxiomC{$\Gamma \vdash A$}
      \AxiomC{$\Gamma \vdash B$} \RightLabel{$[\with I]$}
      \BinaryInfC{$\Gamma \vdash A \with B$}
    \end{prooftree}
  \end{minipage}
  \begin{minipage}{.3\textwidth}
    \begin{prooftree}
      \AxiomC{$\Gamma \vdash A \with B$} \RightLabel{$[\with E_1]$}
      \UnaryInfC{$\Gamma \vdash A$}
    \end{prooftree}
  \end{minipage}
  \begin{minipage}{.3\textwidth}
    \begin{prooftree}
      \AxiomC{$\Gamma \vdash A \with B$} \RightLabel{$[\with E_2]$}
      \UnaryInfC{$\Gamma \vdash B$}
    \end{prooftree}
  \end{minipage}
  % otimes I
  \begin{minipage}{.3\textwidth}
    \begin{prooftree}
      \AxiomC{$\Gamma \vdash A$}
      \AxiomC{$\Delta \vdash B$} \RightLabel{$[\otimes I]$}
      \BinaryInfC{$\Gamma, \Delta \vdash A \otimes B$}
    \end{prooftree}
  \end{minipage}
  \begin{minipage}{.7\textwidth}
    \begin{prooftree}
      \AxiomC{$\Gamma \vdash A \otimes B$} \RightLabel{$[\otimes E]$}
      \AxiomC{$\Gamma, A, B \vdash C$}
      \BinaryInfC{$\Gamma \vdash C$}
    \end{prooftree}
  \end{minipage}

    % % par I
    % \begin{minipage}{0.5\textwidth}
    %   \begin{prooftree}
    %     \AxiomC{$\Gamma, A \vdash B$} \RightLabel{$[\parr I]$}
    %     \UnaryInfC{$\Gamma \vdash A \rightspoon B$}
    %   \end{prooftree}
    % \end{minipage}
    % % par E
    % \begin{minipage}{0.5\textwidth}
    %   \begin{prooftree}
    %     \AxiomC{$\Gamma \vdash  A \rightspoon B$}
    %     \AxiomC{$\Delta \vdash A$} \RightLabel{$[\parr E]$}
    %     \BinaryInfC{$\Gamma, \Delta \vdash B$}
    %   \end{prooftree}
    % \end{minipage}

    % oplus
    \begin{minipage}{.20\textwidth}
      \begin{prooftree}
        \AxiomC{$\Gamma \vdash A$} \RightLabel{$[\oplus I_1]$}
        \UnaryInfC{$\Gamma \vdash A \oplus B$}
      \end{prooftree}
    \end{minipage}
    \begin{minipage}{.20\textwidth}
      \begin{prooftree}
        \AxiomC{$\Delta \vdash B$} \RightLabel{$[\oplus I_2]$}
        \UnaryInfC{$\Delta \vdash A \oplus B$}
      \end{prooftree}
    \end{minipage}
    \begin{minipage}{0.6\textwidth}
      \begin{prooftree}
        \AxiomC{$\Gamma \vdash A \oplus B$}
        \AxiomC{$\Delta, A \vdash C$}
        \AxiomC{$\Delta, B \vdash C$}\RightLabel{$[\oplus E]$}
        \TrinaryInfC{$\Gamma, \Delta \vdash C$}
      \end{prooftree}
    \end{minipage}
  \end{framed}
  \caption{Intuionistic Linear Logic Rules}
  \label{fig:linear-logic-rules}
\end{figure}

To escape linearity, exponential $\oc$ is used, which signifies that an assumption can
be duplicted or dropped without restriction. $\oc A$ can be thought of as {\em``as many A's as needed''}.
Thus the intitionsistic $A \rightarrow B$ can be encoded in linear logic by $\oc A \rightspoon B$.
Similarly $A \plus B$ would be represented as $\oc A \otimes \oc B$ and $A \times B$ would be represented as $A \with B$. We clearly see that
this is a much powerful system in contrast to classical intuistionistic logic because of its enhanced expressivity.

\section{Bunched Implications and $\alpha\lambda$ Calculus}

In intuitionistic linear logic, the context is considered as a list or a set. In the theory of
bunched implications (\textbf{\em BI}), the context is treated as a tree in contrast to other logics. Contexts are syntactically
combined using 2 connectives comma ($,$) or a semicolon ($;$) and are called bunches. The logic of \textbf{\em BI}
tries to glue together intuistionistic linear logic with intuistionistic logic by
permitting contexts connected with semicolon to undergo contraction and weakening while the context connected with comma
are prohibited to undergo contraction and weakening. Comma and semicolon do not distributive over each other.
Thus $A,(B;C) \neq A, B ; A,C$ and $A;(B,C) \neq A;B,A;C$ where A B C are contexts.
There are two flavours of implication---additive and multiplicative---which is closely related to the idea of conjunction.
\begin{framed}
\begin{minipage}{1.0\linewidth}
  \begin{prooftree}
    \AxiomC{$\Gamma, A \vdash B$}
    \UnaryInfC{$\Gamma \vdash A \lozenge B$}
  \end{prooftree}
\end{minipage}
\end{framed}
In the logic of {\textbf{\em BI}} the question then faced is choosing what kind of
implication should be used inplace of $\lozenge$---the additive kind or the multiplicative kind.
O'Hearn and Pym [\cite{ohearn_logic_1999}] argue by introducing 2 kinds of arrows
and using them depending on the connectives used for the context. A multiplicative implication ($\sepimp$)
is used when the context is connected with a comma and an additive implication ($\rightarrow$) is used when the
context is connected using semicolon. This gives rise to 2 rules
\begin{framed}
\begin{minipage}{0.5\linewidth}
  \begin{prooftree}
    \AxiomC{$\Gamma, A \vdash B$} \RightLabel{$[\sepimp I]$}
    \UnaryInfC{$\Gamma \vdash A \sepimp B$}
  \end{prooftree}
\end{minipage}
\begin{minipage}{0.5\linewidth}
  \begin{prooftree}
    \AxiomC{$\Gamma; A \vdash B$} \RightLabel{$[\rightarrow I]$}
    \UnaryInfC{$\Gamma \vdash A \rightarrow B$}
  \end{prooftree}
\end{minipage}
\end{framed}

As we see in $[\sepimp I]$ $\Gamma, A$ cannot under go weakening or contraction to duplicate
or get rid of either $A$ or $\Gamma$. This hints to a notion that multiplicative implication ($\sepimp$)
exhibits property of the linear implication ($\rightspoon$). The linear implication cannot however
be directly converted to a multiplicative implication as the later does not exhibit properties of
counting the number of uses of its arguments. The logic of \textbf{\em BI} tries to combine the
additive logic i.e. intuistionistic logic with the multiplicative side i.e. intuistionistic linear logic seemlessly.
The multiplicative side can be used to model the behaviour of resources in the programming language
while the additive side would help the programmers fall back to the non-resource intuistionsitic parts. The logic of \textbf{\em BI}
argues that instead of looking at the number of times an argument is used within the function, it should
be viewed in terms of {\em sharing}.
$\alpha \lambda$ calculus[\cite{ohearn_resource_1999}] is interpretation of the logic of \textbf{\em BI}.
It introduces 2 kinds of arrows by modifiying the the syntax of lambda calculus:
\begin{enumerate}
  \item $\sepimp$     : Function do not share resources with their arguments
  \item $\rightarrow$ : Function may share resources with their arguments
\end{enumerate}

\begin{figure}[h]
\begin{framed}
  \begin{flalign*}
    \text{Types}\ \ \  \tau           &::= \alpha \mid \iota \mid \tau \rightarrow \tau \mid \tau \sepimp \tau \nonumber \\
    \text{Typing Scheme}\ \ \  \sigma &::= \tau \mid \forall \alpha. \tau \nonumber
  \end{flalign*}
\end{framed}
\caption{$\alpha\lambda$-Calculus Types}
\label{fig:al-cal-types}
\end{figure}

\begin{figure}[h]
\begin{framed}
  \begin{flalign*}
    \text{Expressions}\ \ \ M, N ::= x \mid \lambda^{\alpha} x. M \mid \lambda^{*} x. M \mid M N \mid \Let{x}{M}{N}\nonumber
  \end{flalign*}
\end{framed}
\caption{$\alpha\lambda$-Calculus Terms}
\label{fig:al-calc-terms}
\end{figure}

\begin{figure}[h]
  \begin{framed}
    % var
    \begin{minipage}{.5\textwidth}
      \begin{prooftree}
        \AxiomC{$x: \sigma \in \Gamma$} \RightLabel{$[VAR]$}
        \UnaryInfC{$\Gamma \vdash x : \sigma $}
      \end{prooftree}
    \end{minipage}
    % let
    \begin{minipage}{.5\textwidth}
      \begin{prooftree}
        \AxiomC{$\Gamma \vdash M : \sigma \ \ \ \ \
          \Gamma_{x}, x: \sigma \vdash N: \tau$} \RightLabel{$[LET]$}
        \UnaryInfC{$\Gamma \vdash (\texttt{let}\ x\ \texttt{=}\ M\ \texttt{in}\ N) : \tau$}
      \end{prooftree}
    \end{minipage}
    % forall I
    \begin{minipage}{0.5\textwidth}
      \begin{prooftree}
        \AxiomC{$\Gamma \vdash M : \sigma$}\RightLabel{$[\forall I]$}
        \AxiomC{$t \notin \text{fvs}(\Gamma)$}
        \BinaryInfC{$\Gamma \vdash \lambda t. M : \forall t. \sigma$}
      \end{prooftree}
    \end{minipage}
    % forall E
    \begin{minipage}{0.5\textwidth}
      \begin{prooftree}
        \AxiomC{$\Gamma \vdash M : \forall t. \sigma$} \RightLabel{$[\forall E]$}
        \UnaryInfC{$\Gamma \vdash M \tau : [\tau \backslash t] \sigma$}
      \end{prooftree}
    \end{minipage}
    % -> I
    \begin{minipage}{0.5\textwidth}
      \begin{prooftree}
        \AxiomC{$\Gamma_{x}, x: \tau \vdash M : \tau'$} \RightLabel{$[\sepimp I]$}
        \UnaryInfC{$\Gamma \vdash \lambda^{*} x. M : \tau \sepimp \tau'$}
      \end{prooftree}
    \end{minipage}
    % -> E
    \begin{minipage}{0.5\textwidth}
      \begin{prooftree}
        \AxiomC{$\Gamma \vdash M : \tau \sepimp \tau' \ \ \ \ \
          \Gamma \vdash N : \tau$} \RightLabel{$[\sepimp E]$}
        \UnaryInfC{$\Gamma \vdash M N : \tau'$}
      \end{prooftree}
    \end{minipage}
    % -o I
    \begin{minipage}{0.5\textwidth}
      \begin{prooftree}
        \AxiomC{$\Gamma_{x}; x: \tau \vdash M : \tau'$} \RightLabel{$[\rightarrow I]$}
        \UnaryInfC{$\Gamma \vdash \lambda^{\alpha} x. M : \tau \rightarrow \tau'$}
      \end{prooftree}
    \end{minipage}
    % -o E
    \begin{minipage}{0.5\textwidth}
      \begin{prooftree}
        \AxiomC{$\Gamma \vdash M : \tau \rightarrow \tau' \ \ \ \ \
          \Gamma \vdash N : \tau$} \RightLabel{$[\rightarrow E]$}
        \UnaryInfC{$\Gamma \vdash M N : \tau'$}
      \end{prooftree}
    \end{minipage}
  \end{framed}
  \caption{Typing Rules for $\alpha\lambda$ Calculus}
  \label{fig:bi-logic}
\end{figure}


% This is kind of a big jump here.
% probably shift qualified types after linear logic section to have better continuity
\section{Linear logic with Qualified Types: Quill}
We start our work based on Quill [\cite{morris_best_2016}]. It tries
to implement linear types using qualified types. The novelty of using a qualified
types is that it gives a complete and decidable type inference system. By using
a modified version of Algorithm M we can compute principal types of the terms.
In reality due to higher ordered kind system, we may not be able to deduce the
type of the terms but we can work around it by annotating some or all parts of
the terms. This is usually done at a top level function declaration. Specifying types
also serves as some kind of documentation for the programmers.
The key idea of Morris is to introduce 2 types of predicates into the language: \texttt{Un} and \texttt{Fun}.
\texttt{Un $\tau$} implies that the type $\tau$ is unrestricted, meaning it does not
contain any resources or the resources that it captures can be easily duplicated and dropped.
The \texttt{Fun $\tau$} implies that the type $\tau$ is of a function type. The function
depending on its use, may or may not capture resources in its closure. Thus the functions
themselves can be of restricted or unrestricted type. In traditional sense of typeclasses
in haskell, we can think of the \texttt{Un} to be a typeclass with methods supporting the operation
of duplication and dropping.
\begin{figure}[h]
  \centering
  \begin{framed}
    \begin{minted}[escapeinside=||,mathescape=true]{haskell}
      class Un where
          dup  :: t |$\overset{!}{\rightarrow}$| (t |$\otimes$| t)
          drop :: t |$\overset{!}{\rightarrow}$| 1
    \end{minted}
  \end{framed}
  \caption{\texttt{Un} as a Typeclass}
  \label{fig:un-typeclass}
\end{figure}

Simple types such as integers and booleans are all of unrestricted type as
they can be duplicated or dropped freely.
While program resources such as file handles, database connections
should be treated as restricted type as we cannot freely duplicate it
or drop it. Although there would be certain portions of the program where we would
like to close the file handle or free the memory space allocated on the heap.
(This is where we expected the bunched implications would play a crucial role. I guess.)

Combining linear logic with qualified types in Quill
%%% Local Variables:
%%% mode: latex
%%% TeX-master: "../thesis-ku"
%%% End:
       % Extended abstract, contributions
\chapter{Background Work}\label{chp:background-work}
\section{Hindley-Milner Type System and Type Inference Algorithm}\label{hm-system}

Hindley-Milner (\textbf{HM}) type system\citeyearpar{milner_theory_1978} for lambda calculus extended with parametric polymorphism (i.e restricted version
of System F\citep{girard_proofs_1989}) forms the basis many modern functional programming languages such as Haskell and ML.
\cref{fig:hm-lang} shows the type language contains type variables, primitive types (such as integers, floats),
the type constructor ($\rightarrow$)---which constructs function types---and type scheme ($\sigma$).
The expression language contains variables, function absstraction and applications and
polymorphic \texttt{let} construct. Type inference algorithm $\mathcal{W}$\citep{damas_principal_1982} and its variant
type checking algorithm $\mathcal{M}$\citep{lee_proofs_1998} is decidable in the sense, the algorithm always completes with a success or failure.
The algorithms compute a most general, or principal typing scheme for an expression.

\begin{figure}[h]
  \begin{framed}
    \begin{minipage}{0.35\linewidth}
      \begin{flalign*}
        t, u, \upsilon &\in \text{Type Variables}
      \end{flalign*}
    \end{minipage}%
  \begin{minipage}{0.65\linewidth}
  \begin{flalign*}
    \text{Types}\ \ \  \tau           &::= t \mid \iota \mid \tau \rightarrow \tau\\
    \text{Typing Scheme}\ \ \  \sigma &::= \tau \mid \forall t. \tau\\
    \text{Typing Context}\ \ \ \Gamma &::= \epsilon \mid \Gamma, x:\sigma\\
         \text{Expressions}\ \ \ M, N &::= x \mid \lambda x. M \mid M N \mid \Let{x}{M}{N}
       \end{flalign*}
     \end{minipage}
     \end{framed}
\caption{Hindley-Milner Type and Expression Language}
\label{fig:hm-lang}
\end{figure}

Robinson's\citeyearpar{robinson_machine-oriented_1965} unification algorithm plays a key role
in computation of well-formed principal types. Its purely syntactic approach in creating
substitutions to unify types keeps the complete process elegant.
Algorithm $\mathcal{M}$ can either be used to either infer the types of all well-typed expressions it
can be used to verify that the specified type of expression term matches the actual type. The same
can be obtained using algorithm $\mathcal{W}$ with an additional machinery of
computing equality over types.

\begin{figure}[h]
  \begin{framed}
    % var
    \begin{minipage}{.5\textwidth}
      \begin{prooftree}
        \AxiomC{$x: \sigma \in \Gamma$} \RightLabel{[VAR]}
        \UnaryInfC{$\Gamma \vdash x : \sigma $}
      \end{prooftree}
    \end{minipage}
    % let
    \begin{minipage}{.5\textwidth}
      \begin{prooftree}
        \AxiomC{$\Gamma \vdash M : \sigma$}
        \AxiomC{$\Gamma_{x}, x: \sigma \vdash N: \tau$} \RightLabel{[LET]}
        \BinaryInfC{$\Gamma \vdash (\Let{x}{M}{N}) : \tau$}
      \end{prooftree}
    \end{minipage}
    % forall I
    \begin{minipage}{0.5\textwidth}
      \begin{prooftree}
        \AxiomC{$\Gamma \vdash M : \sigma$}\RightLabel{[$\forall$ I]}
        \AxiomC{$t \notin \texttt{fvs}(\Gamma)$}
        \BinaryInfC{$\Gamma \vdash M : \forall t. \sigma$}
      \end{prooftree}
    \end{minipage}
    % forall E
    \begin{minipage}{0.5\textwidth}
      \begin{prooftree}
        \AxiomC{$\Gamma \vdash M : \sigma$}
        \AxiomC{$(\sigma' \sqsubseteq \sigma)$}\RightLabel{[$\forall$ E]}
        \BinaryInfC{$\Gamma \vdash M : \sigma'$}
      \end{prooftree}
    \end{minipage}
    % -> I
    \begin{minipage}{0.5\textwidth}
      \begin{prooftree}
        \AxiomC{$\Gamma_{x}, x: \tau \vdash M : \tau'$} \RightLabel{[$\rightarrow$ I]}
        \UnaryInfC{$\Gamma \vdash \lambda x. M : \tau \rightarrow \tau'$}
      \end{prooftree}
    \end{minipage}
    % -> E
    \begin{minipage}{0.5\textwidth}
      \begin{prooftree}
        \AxiomC{$\Gamma \vdash M : \tau \rightarrow \tau'$}
        \AxiomC{$\Gamma \vdash N : \tau$} \RightLabel{[$\rightarrow$ E]}
        \BinaryInfC{$\Gamma \vdash M N : \tau'$}
      \end{prooftree}
    \end{minipage}
  \end{framed}
  \caption{Typing Rules for \textbf{HM} Type System}
  \label{fig:hm-system}
\end{figure}

The rules for \textbf{HM} type system are shown in \cref{fig:hm-system}. $\Gamma$ is the
collection of assumptions, or context, in which the expression M is typed. It can be thought of as a collection--list or set--of
an ordered pair of identifier and its type scheme. The [VAR] rule is the tautology; a simple
lookup of variable $x$ in context $\Gamma$ for the type scheme.  [$\rightarrow$I] and [$\rightarrow$E] type
lambda terms and application respectively. Rules for parametric polymorphism are implicit in expression language.
[$\forall$I] rule generalizes the type scheme by adding universal quantifiers and [$\forall$E] generates an instance
of the type scheme by substituting free type variables. While generalizing types, the rule
[$\forall$I], has a side condition that ensures the new type variable introduced should not be free in the typing context.
The $\texttt{fvs}(\Gamma)$ denotes free type variables in $\Gamma$. The [LET] rules allows implicit parametric polymorphism.
For example, the expression, \texttt{g = $\lambda$f.(f True, f 1)} is ill-typed. The type unification will fail for \texttt{f} as
the expression \texttt{f True} asserts \texttt{f:Bool$\rightarrow$u$_1$} while the expression \texttt{f 1} asserts \texttt{f:Int$\rightarrow$u$_2$}.
However, there indeed exists a polymorphic type \texttt{$\forall$u$_1$.u$_1$ $\rightarrow$ u$_1$}
that types \texttt{f}. The \texttt{let} construct makes this possible. The expression \texttt{g} can be defined
as \texttt{g = let f = $\lambda$ x. x in (f True, f 1)}. The type of the expression \texttt{f} is now computed
as \texttt{f : $\forall$u$_1$. u$_1\rightarrow$u$_1$}, using [$\rightarrow$I] and [$\forall$I] rules and \texttt{g} is assigned a type \texttt{(Bool, Int)}.

\begin{figure}[h]
  \centering
  \begin{framed}
  \begin{flalign*}
    A, B, C &::= x \mid A \supset B \mid \forall x. B \mid A \wedge B \mid A \vee B\\
    \Gamma,\Delta &::= \epsilon \mid \Gamma, A
  \end{flalign*}
\end{framed}
\caption{Grammar for Intuitionistic Logic}
\label{fig:intu-logic-grammar}
\end{figure}


Using the Curry-Howard correspondence theorem, the \textbf{HM} type system
corresponds to the second order implication fragment of intuitionistic propositional logic.
The grammar of intuitionistic logic is shown in \cref{fig:intu-logic-grammar} where $\Gamma \supset \Delta$ denotes implication,
$A \wedge B$ denotes conjunction and $A \vee B$ denotes disjunction.

\begin{figure}[h]\centering
  \begin{framed}
    % ID
    \begin{minipage}{0.20\linewidth}
      \begin{prooftree}
        \AxiomC{${\color{white}A \Gamma A}$}\RightLabel{[Ax]}
        \UnaryInfC{$A \vdash A $}
      \end{prooftree}
    \end{minipage}%
    % WKN
    \begin{minipage}{0.33\linewidth}
      \begin{prooftree}
        \AxiomC{$\Gamma \vdash B$}\RightLabel{[WKN]}
        \UnaryInfC{$\Gamma, A \vdash B$}
      \end{prooftree}
    \end{minipage}%
    % CTR
    \begin{minipage}{0.33\linewidth}
      \begin{prooftree}
        \AxiomC{$\Gamma, A, A \vdash B$}\RightLabel{[CTR]}
        \UnaryInfC{$\Gamma, A \vdash B $}
      \end{prooftree}
    \end{minipage}

    % \forall I
    \begin{minipage}{0.33\linewidth}
      \begin{prooftree}
        \AxiomC{$\Gamma \vdash B$}
        \AxiomC{$x \notin \Gamma$}\RightLabel{[$\forall$I]}
        \BinaryInfC{$\forall x. B$}
      \end{prooftree}
    \end{minipage}%
    % \forall E
    \begin{minipage}{0.33\linewidth}
      \begin{prooftree}
        \AxiomC{$\Gamma \vdash \forall x. B$}
        \AxiomC{$\Gamma \vdash A$}\RightLabel{[$\forall$E]}
        \BinaryInfC{$B[x/A]$}
      \end{prooftree}
    \end{minipage}

    % -> I
    \begin{minipage}{0.33\linewidth}
      \begin{prooftree}
        \AxiomC{$\Gamma,A \vdash B$}\RightLabel{[$\supset$I]}
        \UnaryInfC{$\Gamma \vdash A \supset B$}
      \end{prooftree}
    \end{minipage}%
    % -> E
    \begin{minipage}{0.33\linewidth}
      \begin{prooftree}
        \AxiomC{$\Gamma \vdash A \supset B$}
        \AxiomC{$\Gamma \vdash A$}\RightLabel{[$\supset$E]}
        \BinaryInfC{$\Gamma \vdash B $}
      \end{prooftree}
    \end{minipage}

    % \Gamma and A
    \begin{minipage}{0.33\linewidth}
      \begin{prooftree}
        \AxiomC{$\Gamma \vdash A$}
        \AxiomC{$\Gamma \vdash B$}\RightLabel{[$\wedge$I]}
        \BinaryInfC{$\Gamma \vdash A \wedge B$}
      \end{prooftree}
    \end{minipage}%
    % \Gamma and delta E_1
    \begin{minipage}{0.33\linewidth}
      \begin{prooftree}
        \AxiomC{$\Gamma \vdash A \wedge B$}\RightLabel{[$\wedge$E$_1$]}
        \UnaryInfC{$\Gamma \vdash A $}
      \end{prooftree}
    \end{minipage}%
    % \Gamma and delta E_2
    \begin{minipage}{0.33\linewidth}
      \begin{prooftree}
        \AxiomC{$\Gamma \vdash A \wedge B$}\RightLabel{[$\wedge$E$_2$]}
        \UnaryInfC{$\Gamma \vdash B$}
      \end{prooftree}
    \end{minipage}

    % \Gamma or A I_1
    \begin{minipage}{0.25\linewidth}
      \begin{prooftree}
        \AxiomC{$\Gamma \vdash A$}\RightLabel{[$\vee$I$_1$]}
        \UnaryInfC{$\Gamma \vdash A \vee B$}
      \end{prooftree}
    \end{minipage}%
    % \Gamma or delta I_2
    \begin{minipage}{0.25\linewidth}
      \begin{prooftree}
        \AxiomC{$\Gamma \vdash B$}\RightLabel{[$\vee$I$_2$]}
        \UnaryInfC{$\Gamma \vdash A \vee B$}
      \end{prooftree}
    \end{minipage}%
    % \Gamma or delta E
    \begin{minipage}{0.50\linewidth}
      \begin{prooftree}
        \AxiomC{$\Gamma \vdash A \supset C$}
        \AxiomC{$\Gamma \vdash B \supset C$}
        \AxiomC{$\Gamma \vdash A \vee B$}\RightLabel{[$\vee$E$$]}
        \TrinaryInfC{$\Gamma \vdash C$}
      \end{prooftree}
    \end{minipage}
  \end{framed}
  \caption{Logic rules: Implication fragment of second order Intuitionistic Propositional Logic}
  \label{fig:intuitionistic-logic}
\end{figure}

The rules of the logic system are shown in \cref{fig:intuitionistic-logic} in Gentzen style natural deduction
where $\Gamma$, $\Delta$ and $\Phi$ are propositions. The [Ax] rule corresponds to [ID] rule while [$\rightarrow$I] and [$\rightarrow$E] correspond
to [$\supset$I] and [$\supset$E] respectively. The weakening [WKN] and contraction [CTR] rules are implicit in the term structue of \textbf{HM} type-system.
The weakening rule states that we can add unrelated assumptions to derivations without
affecting the proofs while contraction states that we can discard duplicate assumptions in our derivations and the
proof will still hold. The [$\forall$I] introduces the universal quantifier for a proposition and [$\forall$E] instantiates
the quantifier by replacing the variable with given proposition. The rules [$\forall$I] and [$\forall$E] correspond to parametric polymorphism
introduction [$\forall$I] and elimination [$\forall$E] rules in \cref{fig:hm-system}. [$\wedge$I], [$\wedge$E$_2$] and [$\wedge$E$_2$] introduces and eleminates
conjunction. Conjunction is equivalent to pairs or product types. [$\vee$I$_2$], [$\vee$I$_2$] and [$\vee$E] introduces and eleminates disjunction respectively.
Disjunction is equivalent to sum types. While, \textbf{HM} type system does not treat disjunction and conjunction types as first class citizens,
they can be church encoded using lambda expressions\citep{pierce_tapl_2005}.


\begin{figure}[h]
  \begin{framed}
    \singlespacing
    \centering
    {\small
      \fbox{$\M(\Gamma \vdash M:\tau) = S$}
      % VAR
      \begin{minipage}{0.45\linewidth}
        \begin{flalign*}
            \M(\Gamma \vdash &x:\tau)  = \mathcal{U}(\tau, [\vec{u}/\vec{t}]\upsilon)\\
            &\text{where\qquad}\  \forall \vec{t}. \upsilon = \Gamma(x)
        \end{flalign*}
      \end{minipage}%
      % \x. M
      \begin{minipage}{0.50\linewidth}
        \begin{flalign*}
          \M(\Gamma \vdash &\lambda x. M:\tau) = S  \circ S' \\
            \text{where\qquad}\ S  &= \mathcal{U}(\tau, u_1 \rightarrow u_2)\\
            S'  &= \M(S \Gamma, x: S  u_1 \vdash M : S u_2)
          \end{flalign*}
      \end{minipage}

      % M N
      \begin{minipage}{0.45\linewidth}
        \begin{flalign*}
          \M(\Gamma \vdash &M N:\tau)  = S  \circ S' \\
          \text{where\qquad}\ S  &= \M(\Gamma \vdash M: u \rightarrow \tau)\\
          S'  &= \M(S  \Gamma \vdash N: S u)
        \end{flalign*}
      \end{minipage}%
      % let x = M in N
      \begin{minipage}{0.50\linewidth}
        \begin{flalign*}
          \M(\Gamma \vdash (&\Let{x}{M}{N}):\tau) = S  \circ S' \\
          \text{where\qquad}\ S  &= \M(\Gamma \vdash M: u)\\
                              \sigma &= \texttt{Gen}(S \Gamma, S u)\\
                              S' &= \M(S \Gamma, x:\sigma \vdash N:\tau)\\
        \end{flalign*}
      \end{minipage}

      \fbox{Auxiliary Definitions}

      \begin{minipage}{0.45\linewidth}
        \begin{flalign*}
          \texttt{Gen}(\Gamma, \tau) &= \forall \vec{t}. \tau\\
          \text{where\qquad}\ \vec{t} &= \texttt{fvs}(\tau)\backslash\texttt{fvs}(\Gamma)
        \end{flalign*}
      \end{minipage}%
      \begin{minipage}{0.45\linewidth}
        \begin{flalign*}
          \texttt{fvs}(t) &= \{t\}\\
          \texttt{fvs}(\forall \vec{t}. \tau) &= \texttt{fvs}(\tau) \backslash \vec{t}\\
          \texttt{fvs}(\Gamma) &= \bigcup_{\forall (x:\sigma) \in \Gamma} \texttt{fvs}(\sigma)
        \end{flalign*}
      \end{minipage}

    }
  \end{framed}
  \caption{Algorithm $\M$ for \textbf{HM} type system}
  \label{fig:hm-algo-m}
\end{figure}

The algorithm $\M$ is given in figure \cref{fig:hm-algo-m}. $\mathcal{U}$ is the Robinson's unification
algorithm that computes the most general unifier required to unify two types
and $\texttt{Gen}(\Gamma,\tau)$ generalizes a type to a principle type scheme. All the type variables denoted by $u$ are
fresh and do not shadow the existing type variables in the context. $S$ denotes substitutions that are combined using $\circ$ operator
can be applied on types or type schemes. When substitutions are applied to a context $\Gamma$, they are applied to each type scheme contained by $\Gamma$.
$\vec{t}$ is shorthand for a set of type variables $\{t_1, t_2, \dots, t_n\}$ and $\vec{t} \backslash \vec{u}$ denotes the set difference.

\section{Linear Logic}\label{sec:linear-logic}
While propositional logic deals with truth of propositions and their connectives, linear logic\citep{girard_linear_1987}
deals with availability of resources.
Linear logic addresses resource and resource control problem.
The core idea is that propositions cannot be freely duplicated or discarded in contrast to intuitionistic logic.
In formal terms, the contraction and weakening rules are restricted.
Propositions are modeled as resources. In real world software applications,
if resources---such as database connections, file handles or even
in-memory shared state---are freely copied or dropped from a program context, it can introduce bugs or crashes in
industry grade software. If contraction and weakening are completely abandoned,
the system gets overly restrictive. As a work around, the modality operator $\oc$ is introduced for controlled use
of contraction and weakening. Wadler\citeyearpar{wadler_taste_1993} describes a refinement of
linear logic based on Girard's Logic of Unity\citeyearpar{girard_unity_1993}.
It is a disjoint union of linear logic and intuitionistic logic:
$[A]$ means that $A$ is an intuitionistic assumption and the rules of weakening and contraction are admissible,
while $\Pair{A}$ would mean that it is a linear assumption and weakening and contraction are prohibited.

\begin{figure}[h]
  \centering
  \begin{framed}
    \begin{flalign*}
      A, B, C &::= X \mid \oc A \mid A \rightspoon B \mid A \with B \mid A \otimes B \mid A \oplus B\\
      \Gamma,\Delta &::= \epsilon \mid \Gamma, \Pair{A} \mid \Gamma, [A]
    \end{flalign*}
  \end{framed}
  \caption{Grammar for Intuitionistic Linear Logic}
  \label{fig:linear-logic-syntax}
\end{figure}

In a linear logic, the assumptions can not be used more than once. In the proposition $A \rightspoon B$, the assumption $A$ cannot be used again after
it has been used to obtain $B$. This gives a new view of implication as consumption.
Due to the absense of weakening and contraction rules in this system we obtain two fragments---multiplicative and additive---of connectives.
Multiplicative conjunction is represented as $A \otimes B$, and additive conjunction and disjunction as $A \with B$ and $A \oplus B$ respectively.
The syntax of linear logic is shown in \cref{fig:linear-logic-syntax}.

\begin{figure}[h]
  \begin{framed}
    \centering
    \centering\fbox{
      Structural Rules
    }

    % []ID
    \begin{minipage}{0.30\textwidth}
      \begin{prooftree}
        \AxiomC{{\color{white}$\Gamma, \Delta \vdash A$}} \RightLabel{[ID$_{[]}$]}
        \UnaryInfC{$[A] \vdash A$}
      \end{prooftree}
    \end{minipage}
  % !ID
    \begin{minipage}{0.30\textwidth}
      \begin{prooftree}
        \AxiomC{{\color{white}$\Gamma, \Delta \vdash A$}} \RightLabel{[ID$_{\Pair{}}$]}
        \UnaryInfC{$\Pair{A} \vdash A$}
      \end{prooftree}
    \end{minipage}

    % EXCH
    \begin{minipage}{0.30\textwidth}
      \begin{prooftree}
        \AxiomC{$\Gamma, \Delta \vdash A$} \RightLabel{[EXCH]}
        \UnaryInfC{$\Delta, \Gamma \vdash A$}
      \end{prooftree}
    \end{minipage}
    % CTRN
    \begin{minipage}{0.30\textwidth}
      \begin{prooftree}
        \AxiomC{$\Gamma, [A], [A] \vdash B$} \RightLabel{[CTRN]}
        \UnaryInfC{$\Gamma, [A] \vdash B$}
      \end{prooftree}
    \end{minipage}
    % WKN
    \begin{minipage}{0.30\textwidth}
      \begin{prooftree}
        \AxiomC{$\Gamma \vdash B$} \RightLabel{[WKN]}
        \UnaryInfC{$\Gamma, [A] \vdash B$}
      \end{prooftree}
    \end{minipage}

    % ! I
    \begin{minipage}{0.30\textwidth}
      \begin{prooftree}
        \AxiomC{$[\Gamma] \vdash A$} \RightLabel{[$\oc$I]}
        \UnaryInfC{$[\Gamma] \vdash \oc A$}
      \end{prooftree}
    \end{minipage}
    % ! E
    \begin{minipage}{0.30\textwidth}
      \begin{prooftree}
        \AxiomC{$\Gamma \vdash \oc A$}
        \AxiomC{$\Delta, [A] \vdash B$} \RightLabel{[$\oc$E]}
        \BinaryInfC{$\Gamma, \Delta \vdash B$}
      \end{prooftree}
    \end{minipage}

    \fbox{    \centering
      Connective Rules
    }

    % -o I
    \begin{minipage}{0.45\textwidth}
      \begin{prooftree}
        \AxiomC{$\Gamma, A \vdash B$} \RightLabel{$[\rightspoon I]$}
        \UnaryInfC{$\Gamma \vdash A \rightspoon B$}
      \end{prooftree}
    \end{minipage}
    % -o E
    \begin{minipage}{0.5\textwidth}
      \begin{prooftree}
        \AxiomC{$\Gamma \vdash  A \rightspoon B$}
        \AxiomC{$\Delta \vdash A$} \RightLabel{$[\rightspoon E]$}
        \BinaryInfC{$\Gamma, \Delta \vdash B$}
      \end{prooftree}
    \end{minipage}
    % & I
    \begin{minipage}{.3\textwidth}
      \begin{prooftree}
        \AxiomC{$\Gamma \vdash A$}
        \AxiomC{$\Gamma \vdash B$} \RightLabel{$[\with I]$}
        \BinaryInfC{$\Gamma \vdash A \with B$}
      \end{prooftree}
    \end{minipage}
    % & E_1
    \begin{minipage}{.3\textwidth}
      \begin{prooftree}
        \AxiomC{$\Gamma \vdash A \with B$} \RightLabel{$[\with E_1]$}
        \UnaryInfC{$\Gamma \vdash A$}
      \end{prooftree}
    \end{minipage}
    % & E_2
    \begin{minipage}{.3\textwidth}
      \begin{prooftree}
        \AxiomC{$\Gamma \vdash A \with B$} \RightLabel{$[\with E_2]$}
        \UnaryInfC{$\Gamma \vdash B$}
      \end{prooftree}
    \end{minipage}

    % otimes I
    \begin{minipage}{.3\textwidth}
      \begin{prooftree}
        \AxiomC{$\Gamma \vdash A$}
        \AxiomC{$\Delta \vdash B$} \RightLabel{$[\otimes I]$}
        \BinaryInfC{$\Gamma, \Delta \vdash A \otimes B$}
      \end{prooftree}
    \end{minipage}
    % otimes E
    \begin{minipage}{.6\textwidth}
      \begin{prooftree}
        \AxiomC{$\Gamma \vdash A \otimes B$} \RightLabel{$[\otimes E]$}
        \AxiomC{$\Delta, A, B \vdash C$}
        \BinaryInfC{$\Gamma,\Delta \vdash C$}
      \end{prooftree}
    \end{minipage}

    % oplus I_1
    \begin{minipage}{.20\textwidth}
      \begin{prooftree}
        \AxiomC{$\Gamma \vdash A$} \RightLabel{$[\oplus I_1]$}
        \UnaryInfC{$\Gamma \vdash A \oplus B$}
      \end{prooftree}
    \end{minipage}
    % oplus I_1
    \begin{minipage}{.20\textwidth}
      \begin{prooftree}
        \AxiomC{$\Delta \vdash B$} \RightLabel{$[\oplus I_2]$}
        \UnaryInfC{$\Delta \vdash A \oplus B$}
      \end{prooftree}
    \end{minipage}
    % oplus E
    \begin{minipage}{0.5\textwidth}
      \begin{prooftree}
        \AxiomC{$\Gamma \vdash A \oplus B$}
        \AxiomC{$\Delta, A \vdash C$}
        \AxiomC{$\Delta, B \vdash C$}\RightLabel{$[\oplus E]$}
        \TrinaryInfC{$\Gamma, \Delta \vdash C$}
      \end{prooftree}
    \end{minipage}
  \end{framed}
  \caption{Intuitionistic Linear Logic Rules}
  \label{fig:linear-logic-rules}
\end{figure}

The structural and connective logical rules for the complete system is given in \cref{fig:linear-logic-rules}.
To relax linearity constraints, exponential modality $\oc$ is used, which signifies that an assumption can
be duplicated or dropped without restriction. $\oc A$ can be thought of as {\em``as many A's as needed''}.
The intuitionistic implication $A \supset B$ can be encoded in linear logic by using the modality operator as $\oc A \rightspoon B$.
% Similarly, $A \wedge B$ can be encoded as $\oc A \with \oc B$ and $A \vee B$ as $\oc A \otimes \oc B$\citep{wadler_taste_1993}.
We clearly see that this is a much powerful system in comparison to intuitionistic logic because of its enhanced expressivity for handling
resources. However, there is an awkward asymmetry between the connectives. The absense of structural rules enables conjuntion to have
multiplicative and additive fragments but to express additive implication the use of modality is necessary.

There have been several prototype languages based on linear logic. L$^3$\citep{ahmed_l3_2007} is an intermediate
language that is built on a linear type system and supports strong updates. Lolliproc\citep{mazurak_lolliproc_2010} explores the use of
linear logic in concurrent functional programming while F$^{\circ}$\citep{mazurak_lightweight_2010} uses kinds to distinguish between
linear and unrestricted types for general purpose programming. Linear Haskell\citep{bernardy_linear_2017} is a surface level language
that overloads function arrows to incorporate linearity.

\section{Qualified Types}\label{sec:qualified-types}
Jones\citeyearpar{jones_theory_1994} proposed a general framework of incorporating predicates in the type language.
Predicates are used to build constraints on the domain of the type of a term in the language expression.
It introduces additional layer between polymorphic and monomorphic typing of programs.
A modification of Damas-Milner algorithm $\mathcal{W}$ to incorporate predicates ensures that type inference
is sound and complete. Qualified types are the types that satisfy all the predicates for the term.
Qualified types are powerful enough to expresses type classes with functional dependencies\citep{mark_type_2000},
record types\citep{gaster_polymorphic_1996}, sub-typing\citep{jones_theory_1994} and first class polymorphism\citep{jones_first-class_1997}.

\begin{figure}[h]
  \centering
  \begin{framed}
    \begin{minipage}{0.35\linewidth}
      \begin{flalign*}
        t, u, \upsilon  &\in \text{Type Variables}\\
        \pi,\omega &\in \text{Predicates}\\
        P, Q &\in \text{Finite Predicate Set}\\
      \end{flalign*}
    \end{minipage}%
    \begin{minipage}{0.65\linewidth}
  \begin{flalign*}
    \text{Types}\ \ \ \tau              &::= t \mid \iota \mid \tau \rightarrow \tau\\
    \text{Qualified Types}\ \ \ \rho    &::= \tau \mid \pi \Rightarrow \rho\\
    \text{Type Scheme}\ \ \ \sigma      &::= \rho \mid \forall t. \sigma\\
    \text{Typing Context}\ \ \Gamma     &::= \epsilon \mid \Gamma, x : \sigma\\
    \text{Expressions}\ \ \ M, N &::= x: \sigma \mid \lambda x. M \mid M N \\
                                 &\mid \Let{x}{M}{N}
  \end{flalign*}
\end{minipage}
\end{framed}
\caption{Qualified Types and Expression Language}
\label{fig:qualifed-types}
\end{figure}

The type language from \cref{fig:hm-lang} is modified to incorporate
qualified types shown in \cref{fig:qualifed-types}. $\pi$ and $\omega$ range over predicates and  $P$ and $Q$ range over finite set of predicates.
Types of the form $\pi \Rightarrow \sigma$ denote those instances of $\sigma$ that satisfy the predicate $\pi$, in general
$P \Rightarrow \sigma$ would mean the instances of $\sigma$ that satisfy all the predicates $\pi \in P$. The predicate entailment
relation $P \Vdash \pi$ asserts that the predicate $\pi$ can be inferred from the predicates in $P$.
The typing rules in \textbf{HM} type system are slightly modified and 2 new rules are added
as shown in \cref{fig:qualified-types-rules}. [$\Rightarrow$I] and [$\Rightarrow$E] serve the purpose of for introduction and elimination of qualified types respectively.
The new rules are \colorbox{blue!30}{highlighted}. The predicate set $P$ is threaded throughout the other rules
but is not used anywhere except [$\Rightarrow$I] and [$\Rightarrow$E].

\begin{figure}[h]
  \begin{framed}\centering
    \begin{minipage}{.45\textwidth}
      \begin{prooftree}
        \AxiomC{$x: \sigma \in \Gamma$} \RightLabel{[VAR]}
        \UnaryInfC{$P \mid \Gamma \vdash x : \sigma $}
      \end{prooftree}
    \end{minipage}
    % let
    \begin{minipage}{.45\textwidth}
      \begin{prooftree}
        \AxiomC{$P \mid \Gamma \vdash M : \sigma$}
        \AxiomC{$Q \mid \Gamma_{x}, x: \sigma \vdash N: \tau$} \RightLabel{[LET]}
        \BinaryInfC{$P,Q \mid \Gamma \vdash (\Let{x}{M}{N}) : \tau$}
      \end{prooftree}
    \end{minipage}

    % forall I
    \begin{minipage}{0.45\textwidth}
      \begin{prooftree}
        \AxiomC{$P \mid \Gamma \vdash M : \sigma$}\RightLabel{[$\forall$ I]}
        \AxiomC{$t \notin \texttt{fvs}(\Gamma) \cup \texttt{fvs}(P)$}
        \BinaryInfC{$P \mid \Gamma \vdash M : \forall t. \sigma$}
      \end{prooftree}
    \end{minipage}
    % forall E
    \begin{minipage}{0.45\textwidth}
      \begin{prooftree}
        \AxiomC{$P \mid \Gamma \vdash M : \forall t. \sigma$}\RightLabel{[$\forall$ E]}
        \UnaryInfC{$P \mid \Gamma \vdash M: [\tau / t] \sigma$}
      \end{prooftree}
    \end{minipage}

    % -> I
    \begin{minipage}{0.45\textwidth}
      \begin{prooftree}
        \AxiomC{$P \mid \Gamma_{x}, x: \tau \vdash M : \tau'$} \RightLabel{[$\rightarrow$ I]}
        \UnaryInfC{$P \mid \Gamma \vdash \lambda x. M : \tau \rightarrow \tau'$}
      \end{prooftree}
    \end{minipage}
    % -> E
    \begin{minipage}{0.45\textwidth}
      \begin{prooftree}
        \AxiomC{$P \mid \Gamma \vdash M : \tau \rightarrow \tau'$}
        \AxiomC{$P \mid \Gamma \vdash N : \tau$} \RightLabel{[$\rightarrow$ E]}
        \BinaryInfC{$P \mid \Gamma \vdash M N : \tau'$}
      \end{prooftree}
    \end{minipage}

    % => I
    \colorbox{blue!30}{
      \begin{minipage}{0.45\textwidth}
      \begin{prooftree}
        \AxiomC{$P, \pi \mid \Gamma \vdash M : \rho$} \RightLabel{$[=> I]$}
        \UnaryInfC{$P \mid \Gamma \vdash M : \pi => \rho$}
      \end{prooftree}
    \end{minipage}
    % => E
    \begin{minipage}{0.45\textwidth}
      \begin{prooftree}
        \AxiomC{$P \mid \Gamma \vdash M : \pi => \rho$}
        \AxiomC{$P \Vdash \pi$} \RightLabel{$[=> E]$}
        \BinaryInfC{$P \mid \Gamma \vdash M: \rho$}
      \end{prooftree}
    \end{minipage}
    }
  \end{framed}
  \caption{Typing Rules for Qualified Types}
  \label{fig:qualified-types-rules}
\end{figure}

\section{Quill: Linear Logic with Qualified Types}\label{sec:quill}
Quill\citep{morris_best_2016} uses the framework of qualified types to implement a linear type system with a sound and complete type inference.
It uses a modified version of Algorithm $\M$ to compute principal types of the terms.
% In practice, due to higher ordered kind system, it may not be possible to deduce the
% type of the terms but a work around is to annotate some or all parts of
% the terms. Annotation is usually done at a top level function declaration. Specifying types
% also serves as some kind of documentation for the programmers.
The key idea of Morris is to introduce two predicates for types into the language: \Un and \texttt{Fun} with a predicate for ordering
types depending on their admittance to structural rules. The predicate $\tau \geq \tau'$ will hold only if $\tau$ admits more
structural rules than $\tau'$ or, in other words, if $\tau'$ is more restricting than $\tau$.
The predicate \Un{$\tau$} implies that the type $\tau$ is unrestricted, which means it does not
contain any resources or the resources that it captures can be easily duplicated and dropped.
In traditional sense of type classes in Haskell, \Un can be thought to be a type-class with methods supporting the operation
of duplication and dropping shown in \cref{fig:un-typeclass}. In a proof theoretic setting, it would mean
that it admits weakening and contraction. The predicate \texttt{Fun $\tau$} implies that the type $\tau$ is of a function type. The function
may or may not capture resources in its closure and the functions themselves can be of restricted or unrestricted type.
\begin{figure}[h]
  \begin{framed}\centering
    \begin{minted}[escapeinside=||,mathescape=true,xleftmargin=\parindent,linenos]{haskell}
class Un where
    dup  :: t |$\overset{!}{\rightarrow}$| (t |$\otimes$| t)
    drop :: t |$\overset{!}{\rightarrow}$| 1
    \end{minted}
  \end{framed}
  \caption{\texttt{Un} as a Typeclass}
  \label{fig:un-typeclass}
\end{figure}

Simple types such as integers and Booleans are all of unrestricted type, as
they can be duplicated or dropped freely. Program resources such as file handles, database connections
are treated as restricted or linear types as we cannot freely duplicate
or drop them. Consider a lambda expression that represents function application $\lambda f. \lambda x. f x$ and it is applied to
some function $\mathcal{F}$. The linearity of this function $\lambda x. \mathcal{F} x$
would depend on the linearity of $\mathcal{F}$. To generalize, we can say that the linearity of the lambda expression depends
on its closure. The type of $\lambda f. \lambda x. f x$ can be written as $(\tau \overset{f}{\rightarrow} \upsilon) \rightarrow \tau \overset{g}{\rightarrow}\upsilon$.
This function would be well typed only if $f$ is less restricting i.e. admits more structural rules than $g$, so to say $f \geq g$.

\section{Logic of Bunched Implications and $\alpha\lambda$-Calculus}\label{sec:bi}
In intuitionistic logic, the context is considered as a list or a set. In the theory of \BI{},
the context is treated as a tree. Contexts are called bunches and are syntactically
combined using 2 connectives comma ($,$) or a semicolon ($;$). The logic of \BI{} combines
substructural logic with intuitionistic logic by permitting contexts connected with
semicolon to undergo contraction and weakening while the context connected with comma
are prohibited to undergo contraction and weakening. Comma and semicolon do not distribute over each other.
Thus for propositions $A$, $B$ and $C$: $A,(B;C) \neq A, B ; A,C$ and $A;(B,C) \neq A;B,A;C$.
% There are two flavors of implication---additive and multiplicative---which is closely related to the idea of conjunction.
% \begin{framed}\centering
% \begin{minipage}{1.0\linewidth}
%   \begin{prooftree}
%     \AxiomC{$\Gamma, A \vdash B$}
%     \UnaryInfC{$\Gamma \vdash A \lozenge B$}
%   \end{prooftree}
% \end{minipage}
% \end{framed}
% In the logic of \BI the question then faced is choosing what kind of
% implication should be used in place of $\lozenge$---the additive kind or the multiplicative kind.
\cite{ohearn_logic_1999} introduce 2 kinds of arrows
and use them depending on the connectives used in the context. A multiplicative implication ($\sepimp$)
is used when the context is connected with a comma and an additive implication ($\shimp$) is used when the
context is connected using semicolon. This gives rise to two introduction rules for implication:
\begin{framed}
\begin{minipage}{0.5\linewidth}
  \begin{prooftree}
    \AxiomC{$\Gamma, A \vdash B$} \RightLabel{$[\sepimp I]$}
    \UnaryInfC{$\Gamma \vdash A \sepimp B$}
  \end{prooftree}
\end{minipage}
\begin{minipage}{0.5\linewidth}
  \begin{prooftree}
    \AxiomC{$\Gamma; A \vdash B$} \RightLabel{$[\shimp I]$}
    \UnaryInfC{$\Gamma \vdash A \shimp B$}
  \end{prooftree}
\end{minipage}
\end{framed}

$\Gamma, A$ cannot under go weakening or contraction to duplicate
or get rid of either $A$ or $\Gamma$. This hints to a notion that multiplicative implication ($\sepimp$)
exhibits property of the linear implication ($\rightspoon$). The linear implication cannot however
be directly converted to a multiplicative implication (or vice versa) as the latter does not exhibit properties of
counting the number of uses of its arguments. Also, in contrast to linear logic, the multiplicative implication
cannot be converted into an additive implication as there is there is no modality introduced in the system.
The logic of \BI{} combines the additive logic i.e. classical intuitionistic logic with the multiplicative side
i.e. intuitionistic substructural logic. The promise of this logic system is that the multiplicative side can
be used to model the behavior of resources in the programming language while the additive side would help the
programmers fall back to the non-resource intuitionistic parts.
\begin{figure}[h]
  \begin{framed}
  \begin{flalign*}
    A, B, C &::= A \sepimp B \mid A \shimp B \mid A \with B \mid A \otimes B \mid A \oplus B\\
    \Gamma,\Delta &::= \{\}_m \mid \{\}_A \mid \Gamma, A \mid \Gamma ; A
  \end{flalign*}
\end{framed}
  \caption{Grammar for Logic of \BI{}}
  \label{fig:grammar-bi}
\end{figure}

\begin{figure}[h]
  \begin{framed}\centering
    \fbox{Strucutral}
    %ID
    \begin{minipage}{0.33\linewidth}
      \begin{prooftree}
        \AxiomC{} \RightLabel{[ID]}
        \UnaryInfC{$A \vdash A$}
      \end{prooftree}
    \end{minipage}%
    %WKN
    \begin{minipage}{0.33\linewidth}
      \begin{prooftree}
        \AxiomC{$\Gamma(\Delta) \vdash A$} \RightLabel{[WKN]}
        \UnaryInfC{$\Gamma(\Delta;\Delta') \vdash A$}
      \end{prooftree}
    \end{minipage}%
    %CTR
    \begin{minipage}{0.33\linewidth}
      \begin{prooftree}
        \AxiomC{$\Gamma(\Delta;\Delta) \vdash A$} \RightLabel{[CTR]}
        \UnaryInfC{$\Gamma(\Delta) \vdash A$}
      \end{prooftree}
    \end{minipage}

    \fbox{Multiplicative Connectives}

    % -*I
    \begin{minipage}{0.5\linewidth}
      \begin{prooftree}
        \AxiomC{$\Gamma, A \vdash B$} \RightLabel{[$\sepimp$I]}
        \UnaryInfC{$\Gamma \vdash A \sepimp B$}
      \end{prooftree}
    \end{minipage}%
    % -* E
    \begin{minipage}{0.5\linewidth}
      \begin{prooftree}
        \AxiomC{$\Gamma \vdash A \sepimp B$}
        \AxiomC{$\Delta \vdash A$} \RightLabel{[$\sepimp$E]}
        \BinaryInfC{$\Gamma,\Delta \vdash B$}
      \end{prooftree}
    \end{minipage}

    % \otimes I
    \begin{minipage}{.5\textwidth}
      \begin{prooftree}
        \AxiomC{$\Gamma \vdash A$}
        \AxiomC{$\Delta \vdash B$} \RightLabel{$[\otimes I]$}
        \BinaryInfC{$\Gamma, \Delta \vdash A \otimes B$}
      \end{prooftree}
    \end{minipage}%
    % \otimes E
    \begin{minipage}{.5\textwidth}
      \begin{prooftree}
        \AxiomC{$\Gamma(A, B) \vdash C $} \RightLabel{$[\otimes E]$}
        \AxiomC{$\Delta \vdash A \otimes B$}
        \BinaryInfC{$\Gamma(\Delta) \vdash C$}
      \end{prooftree}
    \end{minipage}

    \fbox{Additive Connectives}

    % ->> I
    \begin{minipage}{0.5\linewidth}
      \begin{prooftree}
        \AxiomC{$\Gamma; A \vdash B$} \RightLabel{[$\shimp$I]}
        \UnaryInfC{$\Gamma \vdash A \shimp B$}
      \end{prooftree}
    \end{minipage}%
    % ->> E
    \begin{minipage}{0.5\linewidth}
      \begin{prooftree}
        \AxiomC{$\Gamma \vdash A \shimp B$}
        \AxiomC{$\Delta \vdash A$} \RightLabel{[$\shimp$E]}
        \BinaryInfC{$\Gamma;\Delta \vdash B$}
      \end{prooftree}
    \end{minipage}

    % \with I
    \begin{minipage}{.5\textwidth}
      \begin{prooftree}
        \AxiomC{$\Gamma \vdash A$}
        \AxiomC{$\Delta \vdash B$} \RightLabel{[$\with$I]}
        \BinaryInfC{$\Gamma; \Delta \vdash A \with B$}
      \end{prooftree}
    \end{minipage}%
    % \with E
    \begin{minipage}{.5\textwidth}
      \begin{prooftree}
        \AxiomC{$\Gamma(A; B) \vdash C $} \RightLabel{[$\with$E]}
        \AxiomC{$\Delta \vdash A \with B$}
        \BinaryInfC{$\Gamma(\Delta) \vdash C$}
      \end{prooftree}
    \end{minipage}

    % \oplus I_1
    \begin{minipage}{.20\textwidth}
      \begin{prooftree}
        \AxiomC{$\Gamma \vdash A$} \RightLabel{[$\oplus$I$_1$]}
        \UnaryInfC{$\Gamma \vdash A \oplus B$}
      \end{prooftree}
    \end{minipage}
    % \oplus I_1
    \begin{minipage}{.20\textwidth}
      \begin{prooftree}
        \AxiomC{$\Delta \vdash B$} \RightLabel{[$\oplus$I$_2$]}
        \UnaryInfC{$\Delta \vdash A \oplus B$}
      \end{prooftree}
    \end{minipage}
    % \oplus E
    \begin{minipage}{0.5\textwidth}
      \begin{prooftree}
        \AxiomC{$\Gamma \vdash A \oplus B$}
        \AxiomC{$\Delta(A) \vdash C$}
        \AxiomC{$\Delta(B) \vdash C$}\RightLabel{[$\oplus$E]}
        \TrinaryInfC{$\Gamma(\Delta) \vdash C$}
      \end{prooftree}
    \end{minipage}
  \end{framed}
  \caption{Rules for Logic of \BI{}}
  \label{fig:rules-bi}
\end{figure}

The complete grammar for logic of \BI{} is shown in \cref{fig:grammar-bi}. The contexts $\{\}_m$ represents a multiplicative empty context
while the $\{\}_a$ represents an additive empty context. The rules for the logic system are shown in \cref{fig:rules-bi}.
$\Gamma(\Delta)$ means that $\Delta$ is a sub-tree within $\Gamma$.

$\alpha\lambda$-calculus\citep{ohearn_resource_1999, pym_semantics_2002}
is the Curry-Howard interpretation of the logic of \BI{}. It views implication in terms of sharing rather than
the number of times it is used. It introduces 2 kinds of arrows by modifying the the syntax of lambda calculus:
\begin{enumerate}
  \item $\sepimp$: Functions do not share resources with their arguments
  \item $\shimp$ : Functions share resources with their arguments
\end{enumerate}

The types and terms of $\alpha\lambda$-calculus are shown in \cref{fig:al-cal-types}.
The structural and connective rules for $\alpha\lambda$-calculus are summarized in \cref{fig:bi-typing-rules}.
We have left out the terms for additive and multiplicative conjunction and additive disjunction for the
sake of simplicity.

\begin{figure}[h]
\begin{framed}
  \begin{flalign*}
                       t, u, \upsilon &\in \text{Type Variables}\\
   \text{Context}\ \ \ \Gamma, \Delta &::= \{\}_m \mid \{\}_a \mid x:\tau \mid \Gamma, \Delta \mid \Gamma;\Delta\\
    \text{Types}\ \ \  \tau           &::= t \mid \iota \mid \tau \shimp \tau \mid \tau \sepimp \tau \\
    \text{Expressions}\ \ \ M, N      &::= x \mid \lambda x. M \mid \alpha x. M \mid M N
  \end{flalign*}
\end{framed}
\caption{$\alpha\lambda$-Calculus Types and Terms}
\label{fig:al-cal-types}
\end{figure}

\begin{figure}[h]
  \begin{framed}\centering
    % var
    \begin{minipage}{0.30\textwidth}
      \begin{prooftree}
        \AxiomC{$x: \tau \in \Gamma$} \RightLabel{[VAR]}
        \UnaryInfC{$\Gamma \vdash x : \tau $}
      \end{prooftree}
    \end{minipage}%
    % CTRN
    \begin{minipage}{0.30\textwidth}
      \begin{prooftree}
        \AxiomC{$\Gamma; \Gamma \vdash M:\tau$} \RightLabel{[CTRN]}
        \UnaryInfC{$\Gamma \vdash M:\tau$}
      \end{prooftree}
    \end{minipage}%
    % WKN
    \begin{minipage}{0.30\textwidth}
      \begin{prooftree}
        \AxiomC{$\Gamma \vdash M:\tau$} \RightLabel{[WKN]}
        \UnaryInfC{$\Gamma;\Delta \vdash M:\tau $}
      \end{prooftree}
    \end{minipage}

    % -* I
    \begin{minipage}{0.45\textwidth}
      \begin{prooftree}
        \AxiomC{$\Gamma_{x}, x: \tau \vdash M : \tau'$} \RightLabel{[$\sepimp$I]}
        \UnaryInfC{$\Gamma \vdash \lambda  x. M : \tau \sepimp \tau'$}
      \end{prooftree}
    \end{minipage}%
    % -* E
    \begin{minipage}{0.45\textwidth}
      \begin{prooftree}
        \AxiomC{$\Gamma \vdash M : \tau \sepimp \tau' \ \ \ \ \
          \Delta \vdash N : \tau$} \RightLabel{[$\sepimp$E]}
        \UnaryInfC{$\Gamma,\Delta \vdash M N : \tau'$}
      \end{prooftree}
    \end{minipage}

    % -> I
    \begin{minipage}{0.45\textwidth}
      \begin{prooftree}
        \AxiomC{$\Gamma_{x}; x: \tau \vdash M : \tau'$} \RightLabel{[$\shimp$I]}
        \UnaryInfC{$\Gamma \vdash \alpha  x. M : \tau \shimp \tau'$}
      \end{prooftree}
    \end{minipage}%
    % -> E
    \begin{minipage}{0.45\textwidth}
      \begin{prooftree}
        \AxiomC{$\Gamma \vdash M : \tau \shimp \tau' \ \ \ \ \
          \Delta \vdash N : \tau$} \RightLabel{[$\shimp$E]}
        \UnaryInfC{$\Gamma;\Delta \vdash M N : \tau'$}
      \end{prooftree}
    \end{minipage}
  \end{framed}

  \caption{Typing Rules for $\alpha\lambda$-Calculus}
  \label{fig:bi-typing-rules}
\end{figure}

The logic of \BI{} patches up the awkward asymmetry experienced in linear logic. The multiplicative conjunction $\otimes$ has a right adjoint
counter part as $\sepimp$ while additive conjunction $\with$ has a right adjoint counterpart of $\shimp$.
In other words, $A \otimes B \vdash C$ iff $A \vdash B \sepimp C$ and $A \with B \vdash C$ iff $A \vdash B \shimp C$.
This relieves us from introducing modality into the system.

Due to the rules of $\alpha\lambda$-calculus $f: \tau \sepimp \tau'; x:\tau \nvdash f x:\tau'$,
as $f$ needs an argument that does not share any resources with its context.
The term $\lambda x. \alpha  f. f x x: \tau \sepimp (\tau \shimp \tau \shimp \tau') \shimp \tau'$ is a typable term in
$\alpha\lambda$-calculus as shown in \cref{fig:multi-bi-example}. This illustrates the difference between logic of \BI{} and linear logic
as even tough the argument is separate from the function, it may be used twice. In linear logic this expression will be ill-typed.

\begin{figure}[h]
  \begin{framed}
    \begin{minipage}{1.0\linewidth}
      \begin{prooftree}
        \AxiomC{}\RightLabel{[VAR]}
        \UnaryInfC{$f:\tau \shimp \tau \shimp \tau' \vdash f: \tau \shimp \tau \shimp \tau'$}

        \AxiomC{}\RightLabel{[VAR]}
        \UnaryInfC{$x:\tau \vdash x:\tau$} \RightLabel{[$\shimp$E]}
        \BinaryInfC{$f:\tau \shimp \tau \shimp \tau'; x:\tau \vdash f x: \tau \shimp \tau'$}

        \AxiomC{}\RightLabel{[VAR]}
        \UnaryInfC{$x:\tau \vdash x: \tau$}\RightLabel{[$\shimp$E]}

        \BinaryInfC{$x:\tau; f:\tau \shimp \tau \shimp \tau';x:\tau \vdash f x x: \tau'$}\RightLabel{[CTRN]}
        \UnaryInfC{$x:\tau; f:\tau \shimp \tau \shimp \tau' \vdash f x x: \tau'$} \RightLabel{[$\shimp$I]}
        \UnaryInfC{$x:\tau \vdash \lambda x. f x x: (\tau \shimp \tau \shimp \tau') \shimp  \tau'$}\RightLabel{[$\sepimp$I]}
        \UnaryInfC{$\vdash \lambda x. \alpha f. f x x: \tau \sepimp (\tau \shimp \tau \shimp \tau') \shimp \tau'$}
      \end{prooftree}
    \end{minipage}
  \end{framed}
  \caption{Multiplicative Argument used Twice in $\alpha\lambda$-calculus}
  \label{fig:multi-bi-example}
\end{figure}

The use of logic of \BI{} as a type inference system is an active area of research in functional programming language implementation.
There has been research on building proof theoretic and semantic models of the logic system\citep{pym_semantics_2002}. The use of bunches instead of
lists as typing environment makes it difficult to have a direct implementation of the type inference algorithm. \cite{atkey_lambda_sep_2004}
designs $\lambda_{sep}$ calculus which is based on the affine variant of $\alpha\lambda$-calculus focusing on separation of resources used by objects.
\cite{collinson_bunched_2005} designs a polymorphic variant of $\alpha\lambda$-calculus.

%%% Local Variables:
%%% mode: latex
%%% TeX-master: "../thesis-ku"
%%% End:
     % HM type system, algorithm M, linear logic, BI, qualified types, Quill
\chapter{Programming in \qub{}}
In this chapter, we illustrate using examples how \qub{} is different
from other functional languages and how a powerful type system based on logic of \BI
would be used to track resources. The examples show how the resources use
can be tracked at compile time and resource leaks can be avoided.

\section{File Handles}\label{sec:file-handle-example}
In modern languages file are treated as normal variables
and it is the programmer's responsiblity to check that that files are not closed twice
and that there are no files that remain open when the program exits. This seemingly trivial
responsibility becomes tedious and error prone as soon as the programming logic gets complex.
Modern functional languages such as Haskell enforces the file input/output to be wrapped in a IO Monad.
This is more declarative than imperative languages, but the type system is not powerful enough
to detect whether a file handle is closed twice or is not closed at all.
Consider the functions for file handling as shown in \cref{fig:file-handling-function}. A simple program in Hasell that opens a file and reads
a line from it and then closes the file handle is shown in \cref{fig:file-read-close}.
We write \mintinline{haskell}{-*>} in haskell syntax to mean $\sepimp$ or the function argument is separate from
the function and \mintinline{haskell}{-&>} to mean $\rightarrow$ or that the function arugment is in sharing
with the function.

\begin{figure}[h]
  \begin{framed}
    \begin{minted}{haskell}
openFile :: FilePath -*> IO FileHandle
closeFile :: FileHandle -*> IO ()
readLine :: FileHandle -*> IO (String, FileHandle)
writeFile :: String -*> FileHandle -> IO ((), FileHandle)
    \end{minted}
  \end{framed}
  \caption{File Handling functions}
  \label{fig:file-handling-function}
\end{figure}

\begin{figure}[h]
  \begin{framed}
    \begin{minted}{haskell}
do f  <- openFile "sample.txt"
   (s, f)  <- readLine f
   () <- closeFile f
    \end{minted}
  \end{framed}
  \caption{Reading from a file in Haskell}
  \label{fig:file-read-close}
\end{figure}

Consider an incorrect version of the program in \cref{fig:file-read-close}
where the file handle is closed twice after reading a line from it \cref{fig:file-read-close-2times}.
It may not be a problem in a single threaded environment, but in a multithreaded environment
the second close may accidently close the file handle that may have been reused in the background by another thread.
When another thread tries to write on this closed file handle, it would throw an exception.
Haskell's type system would happily accept this program but it might generate a runtime exception.
\begin{figure}[h]
  \begin{framed}
    \begin{minted}{haskell}
do f  <- openFile "sample.txt"
   (s, f)  <- readLine f
   () <- closeFile f
   () <- closeFile f
    \end{minted}
  \end{framed}
  \caption{Reading from a file and closing it twice}
  \label{fig:file-read-close-2times}
\end{figure}

Apple's goto fail bug that appeared in iOS 7.0 and caused a security vulnerability in 2012 is a similar example of closing the file twice.
The code snippet that caused the SSL/TLS handshake to be completely skipped looked like in \cref{fig:goto-fail}.
The cases in which the condition is false, the second \mintinline{C}{goto fail;} on line 6 would force the protocol to skip the
steps to be taken after the if-block. This made the system vulnerable to a man-in-middle attack.
\begin{figure}[h]
  \begin{framed}
\begin{minted}[linenos]{C}
...
if ((err = SSLHashSHA1.update(&hashCtx, &serverRandom)) != 0)
    goto fail;
if ((err = SSLHashSHA1.update(&hashCtx, &signedParams)) != 0)
    goto fail;
    goto fail;
if ((err = SSLHashSHA1.final(&hashCtx, &hashOut)) != 0)
    goto fail;

err = sslRawVerify(...);
...
\end{minted}
  \end{framed}
  \caption{Goto Fail bug}
  \label{fig:goto-fail}
\end{figure}

Another example of incorrect way of using file handle is by not closing the file handle after using it shown in \cref{fig:file-read-noclose}.
In a short lived process, when the program exits, file handles that are not closed are freed by the operating system.
But if it is a long running process it would run out of file handles and the whole process would crash with an error that
it cannot open any more file handles. Abnormal exit from the process would interfere in the write process
and the operating system would close the file handle without waiting for the buffer to be completely
written on the file system.
\begin{figure}[h]
  \begin{framed}
    \begin{minted}{haskell}
do f <- openFile "sample.txt"
   (s, f)  <- readLine f
   return s
    \end{minted}
  \end{framed}
  \caption{Reading from a file and not closing file handle}
  \label{fig:file-read-noclose}
\end{figure}
We take a deeper dive into this problem by seeing the desugared version of the ``do'' notation.
Both the programs would be translated into bind ($>>=$) operations . Recall that type signature of bind function is given as
\mintinline{haskell}{(>>=) :: (Monad m) => m t -> (t -> m u) -> m u}. Desugared version of \cref{fig:file-read-close-2times}
would look like as shown in \cref{fig:file-read-close-2times-desugared}
and the desugared version of \cref{fig:file-read-noclose} would look like \cref{fig:file-read-noclose-desugared}.

\begin{figure}[h]
  \begin{framed}
    \begin{minted}[escapeinside=||,mathescape=true]{haskell}
(>>=|$_1$|) (openFile "sample.txt") (\f ->
            (>>=|$_2$|) (readFile f) (\ (s, f) ->
                        (>>=|$_3$|) (close f) (\_ -> closeFile f)))
    \end{minted}
  \end{framed}
  \caption{Reading from a file in Haskell and closing twice}
  \label{fig:file-read-close-2times-desugared}
\end{figure}

\begin{figure}[h]
  \begin{framed}
\begin{minted}{haskell}
(>>=) (openFile "sample.txt" ReadMode) (\ f ->
       >>= (closeFile f) (\ (s, f) -> return s)
\end{minted}
  \end{framed}
  \caption{Reading from a file in Haskell and not closing it}
  \label{fig:file-read-noclose-desugared}
\end{figure}

In both the cases, the types of the programs would be computed by haskell compiler as \mintinline{haskell}{IO ()}
This well typed looking program should be flagged by the compiler as it would cause problems at runtime.
To overcome this, we introduce the concept of unrestrictedness and sharing and separation of resources.

The $\rightarrow$ constructor now has 2 meanings---shared or separated---we are
forced to describe the bind operator in terms of $\sepimp$ or $\rightarrow$. In Quill the bind operation will
be typed as given in \cref{fig:quill-bind-type}. This means all resources used by the funtion are separate.
\begin{figure}[h]
  \begin{framed}
    \begin{minted}{haskell}
          (>>=) :: m t -*> (t -!*> m u) -*> m u
    \end{minted}
  \end{framed}
  \caption{Type signature for bind in Quill}
  \label{fig:quill-bind-type}
\end{figure}
Thus all the resources used should be separating, but as we see in \cref{fig:file-read-close-2times-desugared} the file handle
\texttt{f} is shared in the third bind operation.
The bind operations will have the signatures as shown in \cref{fig:bind-signatures}
\begin{figure}[h]
\begin{framed}
\begin{minted}[escapeinside=||,mathescape=true]{haskell}
(>>=|$_1$|):: IO FileHandle -*> (FileHandle -!*> IO ()) -*> IO ()
(>>=|$_1$|) (openFile "sample.txt" ReadMode) (\f -> ...)

(>>=|$_2$|) :: IO FileHandle
            -*> (FileHandle -!*> IO (String, Filehandle))
            -*> IO (String, FileHandle)}
(>>=|$_2$|) (readLine f) (\ (s,f) -> ... )

(>>=|$_3$|) :: IO () -*> (() -!*> IO ()) -&> IO ()
(>>=|$_3$|) (closeFile f) (\_ -> closeFile f)}
\end{minted}
\end{framed}
\caption{Bind Signatures}
\label{fig:bind-signatures}
\end{figure}

The third bind would be typed as\\
\mintinline{haskell}{IO () -*> (() -!*> IO ()) -&> IO ()} as both the arguments share the file variable \texttt{f}
which would be a type error
as the bind operation should have a type signature of\\
\mintinline{haskell}{IO () -*> (() -!*> IO ()) -*> IO ()}. The types here would not match.
The types do match up for the correct implementation of the file read and close shown in \cref{fig:file-read-close}.
The first bind \mintinline{haskell}{(>>=) (openFile "sample.txt" ReadMode) (\ f -> ...)} has a type signature of\\
\mintinline{haskell}{IO FileHandle -*> (FileHandle -!*> IO ()) -*> IO ()}\\ and the second bind
\mintinline{haskell}{(>>=) (readLine f) (\ (s, f) -> close f)} has a type signature of\\
\mintinline{haskell}{IO (String, FileHandle) -*> (FileHandle -!*> IO ()) -*> IO ()}

In \cref{fig:file-read-noclose-desugared} the file handle \texttt{f} is not closed. It is declared, but not used
in its scope, it would be tagged as an unrestricted value by the Quill type checker.
This is in voilation of our assumption that resources cannot be of the unrestricted type. Thus
the program would not typecheck due to mismatch of the file handle type to be unrestricted.

\section{Exception handling}
We expand on the file handling scenario and consider the code that can throw runtime exceptions.
The motivation to do so lies in the fact that memory leaks are caused becuase of runtime
exceptions where the part of code that is responsible to clean up resources or in this case
closing the file is skipped due to an alternate execution path.

The \HaskellF{IOMonad} is a class that encapsulates two kinds of IOs. \HaskellF{IO} that does not fail
or throw exceptions, and \HaskellF{IOF} that can fail and throw exceptions.
For a concrete example we will assume that \HaskellF{readLine} can throw an exception during runtime, where
it might fail to read a line due to the file mode being used has incorrect permissions.
For the sake of simplicity \HaskellF{openFile} and \HaskellF{closeFile} do not throw exceptions.
\HaskellF{onExcept} accepts a code that can fail and executes the second
parameter only if the actual code throws an exception. The \HaskellF{onExcept} gives a chance
to clean up resources in a systematic way. It may convert a code that can fail to a code
that does not fail by using \HaskellF{catch} function, or it may re throw the exception after
cleaning up the resources.

In the exceptionless execution path in \cref{fig:file-exceptions}, a single line would be read from the file and its uppercased
version would be returned after closing it. If suppose, an exception is thrown by \HaskellF{readLine}
the \HaskellF{onExcept} would close the filehandle. Notice that \HaskellF{onExcept}
shares the context with the file readline code, thus the appropriate file handle would be closed and
we would not have any memory leak.

\begin{figure}[h]
  \begin{framed}
    \begin{minted}{haskell}
openFile :: FilePath -*> IO FileHandle
closeFile :: FileHandle -*> IO ()
readFile :: FileHandle -*> IOF (String, FileHandle)
writeFile :: String -*> FileHandle -*> IOF ((), FileHandle)

throw :: Exception -> IO a
catch :: IOMonad IOM => IO a -*> (IO a -> IOM b) -&> IOM b

onExcept :: IOF a -*> IOF () -&> IOF a
m `onExcept` n = m `catch` (\e -> n >> throw e)

readFromFile :: String -*> IO (Either String String)
readFromFile f =
do fh  <- openFile f WriteMode
   ((s, fh)  <- readLine fh
   let l = caps s
   () <- close fh
   return $ Right l) `onExcept` (do () <- closeFile fh)
                     `catch` (\e ->
                              return $ Left "Could not read file")
    \end{minted}
  \end{framed}
  \caption{Exception Handling in Files}
  \label{fig:file-exceptions}
\end{figure}



% \section{Non Empty List}\label{sec:dll-example}
% A non empty list in Quill would look like
% \begin{minted}{haskell}
% data NEList a = Last a | Cons a (NEList a)
% \end{minted}

% The operations one could perfrom on a non-empty list are
% \begin{minted}{haskell}
% head :: NEList a -&> a
% head (Last h) = h
% head (Cons h t) = h

% tail :: NEList a -&> a
% tail (Last h) = h
% tail (Cons h t) = tail t

% concat :: NEList a -*> NEList a -&> NEList a
% concat (Last h') t = Cons h' t
% concat (Cons h t') t'' = Cons h (concat t' t'')

% append :: a -*> NEList a -&> NEList a
% append a t = Cons a t
% \end{minted}

% The underlying implementation of this list can be thought of a doubly
% linked list. There would be pointers that would connect the the individual nodes to form a doubly linked list.
% The programmer would be agnostic of the pointer handling as we expect the type system to be powerful enough to
% make sure that the lower level code generated would be free of memory leaks and runtime exceptions.
% It makes sense for this doubly linked list implementation to fit the description of a shared resource as nodes
% are interconnected. Getting access to one node gives us access to the whole list.



% \section{Circular Linked List as a Queue}\label{sec:queue-example}

% Circular list as a queue example



%%% Local Variables:
%%% mode: latex
%%% TeX-master: "../thesis-ku"
%%% End:
           % file handles and exceptions
\chapter{Core Language Syntax and Types}\label{chp:qub-language}
In this chapter we give the formal description of \qub{}'s syntax and types. We explain what
it means for a type assignment to exist as binary trees. We then show how we generalize the tree
into a sharing graph and represent it as a collection of three tuple sets in order to simplify our type inference algorithm.

\begin{figure}[h]
  \begin{framed}
    \begin{flalign*}
                         t, u, \upsilon, \phi         &\in \text{Type Variables}\\
                                      P, Q            &\in \text{Finite Predicate Set}\\
      \text{Types}\ \ \               \tau            &::= t \mid \iota \mid \tau \rightarrow \tau\
                                \qquad\text{where}\ \{\tightoverset{\scalebox{0.5}{!}}{\sepimp}, \sepimp, \tightoverset{\scalebox{0.5}{!}}{\shimp}, \shimp \} \subseteq \rightarrow \\
      \text{Predicates}\ \ \        \pi,\omega        &::= \texttt{Un}\ \tau \mid \texttt{SeFun}\ \tau \mid \texttt{ShFun}\ \tau \mid \tau \geq \tau' \\
      \text{Qualified Types}\ \ \     \rho            &::= \tau \mid \pi => \rho \\
      \text{Type schemes}\ \ \        \sigma          &::= \rho \mid \forall t. \sigma 
    \end{flalign*}
  \end{framed}
  \caption{Types \qub{}}
  \label{fig:qub-types}
\end{figure}
% Describe types
The type language consists of type variables ($t$, $u$, $\upsilon$), built-in types such as integers, booleans ($\iota$), and four binary type constructors the
sharing arrow ($\shimp$) and the separating arrow ($\sepimp$) and unrestricted
version of both the type constructors ($\tightoverset{\scalebox{0.5}{!}}{\sepimp}, \tightoverset{\scalebox{0.5}{!}}{\shimp}$). The sharing arrow
would mean that the function shares resources with its argument and the separating
arrow would mean that the function does not share resources with its arguments.
We would write both the arrows in an infix notation.

% Describe Predicates
The predicate system enhances the expressivity of the type system. Following the same route taken
in Quill \citep{morris_best_2016} we use the predicate $\Un{\tau}$ to denote
that the type $\tau$ is unrestricted. Unrestricted types are the ones that do not have any resources or whose resources can
be duplicated or deleted easily. Built-in lightweight types such as \HaskellF{Int}, \HaskellF{Char} are considered unrestricted types.
We write $\ShFun{\phi}$ to describe that type $\phi$ may share resources with its
argument types and  $\SeFun{\phi}$ to describe that type $\phi$ does not share any resources from its argument types.
Notice that function types can also be unrestricted i.e. it does not have any resources. If a type $\tau$ is unrestricted i.e. it qualifies with predicate
\texttt{Un} and it also qualifies one of the function predicates---\texttt{SeFun} or \texttt{ShFun}---we write
them as $\tightoverset{\scalebox{0.5}{!}}{\sepimp}$ and $\tightoverset{\scalebox{0.5}{!}}{\shimp}$ respectively.
We also define an ordering on types by using the predicate $\geq$. The predicate $\tau \geq \tau'$ holds if the type $\tau'$
is less restricting than $\tau$ or to say $\tau$ has admits more structural rules than $\tau'$ as previously explained in \cref{sec:quill}.
The predicate entailment relations $P => Q$ are given in \cref{fig:entailment-rules}.

To keep the current system simple we have not included kinds. They are added into this system as a language extension
to enable users to define custom types using type constructors. We describe this extension in \cref{chp:datatypes}.

\begin{figure}[h]\centering
  \begin{framed}
    \begin{minipage}{0.20\linewidth}
      \begin{prooftree}
        \AxiomC{$\pi \in P$}
        \UnaryInfC{$P => \pi$}
      \end{prooftree}
    \end{minipage}%
    \begin{minipage}{0.20\linewidth}
      \begin{prooftree}
        \AxiomC{$\bigwedge_{\pi \in Q} P => \pi$}
        \UnaryInfC{$P => Q$}
      \end{prooftree}
    \end{minipage}%
    \begin{minipage}{0.24\linewidth}
      \begin{prooftree}
        \AxiomC{${\color{white}\bigwedge_{\pi \in Q}}$}
        \UnaryInfC{$P => \Un{(\tau \tightoverset{\scalebox{0.5}{!}}{\sepimp} \tau')}$}
      \end{prooftree}
    \end{minipage}
    \begin{minipage}{0.24\linewidth}
      \begin{prooftree}
        \AxiomC{${\color{white}\bigwedge_{\pi \in Q}}$}
        \UnaryInfC{$P => \Un{(\tau \tightoverset{\scalebox{0.5}{!}}{\sepimp} \tau')}$}
      \end{prooftree}
    \end{minipage}

    \begin{minipage}{0.24\linewidth}
      \begin{prooftree}
        \AxiomC{${\color{white}P => \tau \geq \phi t}$}
        \UnaryInfC{$P => \tau \geq (\upsilon \sepimp \upsilon')$}
      \end{prooftree}
    \end{minipage}%
    \begin{minipage}{0.24\linewidth}
      \begin{prooftree}
        \AxiomC{${\color{white}P => \tau \geq \phi t}$}
        \UnaryInfC{$P => \tau \geq (\upsilon \shimp \upsilon')$}
      \end{prooftree}
    \end{minipage}%
    \begin{minipage}{0.24\linewidth}
      \begin{prooftree}
        \AxiomC{$P => \Un{\tau}$}
        \UnaryInfC{$P => \tau \geq (\upsilon \tightoverset{\scalebox{0.5}{!}}{\sepimp} \upsilon')$}
      \end{prooftree}
    \end{minipage}%
    \begin{minipage}{0.24\linewidth}
      \begin{prooftree}
        \AxiomC{$P => \Un{\tau}$}
        \UnaryInfC{$P => \tau \geq (\upsilon \tightoverset{\scalebox{0.5}{!}}{\shimp} \upsilon')$}
      \end{prooftree}
    \end{minipage}

    \begin{minipage}{0.20\linewidth}
      \begin{prooftree}
        \AxiomC{$\tau = \sepimp \vee \tau = \tightoverset{\scalebox{0.5}{!}}{\sepimp}$}
        \UnaryInfC{$P => \SeFun{\tau}$}
      \end{prooftree}
    \end{minipage}%
    \begin{minipage}{0.20\linewidth}
      \begin{prooftree}
        \AxiomC{$\tau = \shimp \vee \tau = \tightoverset{\scalebox{0.5}{!}}{\shimp}$}
        \UnaryInfC{$P => \ShFun{\tau}$}
      \end{prooftree}
    \end{minipage}%
    \begin{minipage}{0.30\linewidth}
      \begin{prooftree}
        \AxiomC{$P => \tau \geq \phi t$}
        \AxiomC{$t\ \text{fresh}$}
        \BinaryInfC{$P => \tau \geq \phi$}
      \end{prooftree}
    \end{minipage}
  \end{framed}
  \caption{Entailment Rules}
  \label{fig:entailment-rules}
\end{figure}

% Describe Typing environment
In normal type systems, the contexts are represented as sets or lists. In logic of \BI\ they are represented as binary trees and are called bunches.
The leaf nodes contain the pair of term and its associated type. Internal nodes of the context tree are
connectives which can either be a semicolon ($;$) or a comma ($,$).
If a bunch $\Delta$ is a subtree of $\Gamma$, the relation is denoted by $\Gamma(\Delta)$.
Two bunches are equivalent ($\Gamma \equiv \Delta$) if they can be transformed into another by renaming identifiers.
The bunches have a restriction that no identifier appears more than once. Certain structural rules are restricted on the context
depending on the connectives used. If contexts are combined using a comma ($,$), contraction and weakening is not admissible,
but if the contexts are combined using a semicolon ($;$), it can undergo contraction and weakening. Exchange rule is admissible
in both the connectives. This distinction enables a special treatment for resources in within the language.
By adding resources with a comma constructor in the typing envrionment, the type system will not dispose it off by using the contraction rule,
while non-resourceful objects (or normal propositions) can be added using the semi-colon constructor.
An example bunch is shown in \cref{fig:bunches-bi}. a and b are in sharing while, c is separate from the a and b.
If $\Gamma$ represents the complete context of \cref{fig:bunches-bi}, $\Delta \equiv (a:A; b:B)$ and $\Delta' \equiv (c:C)$
then $\Gamma \equiv \Delta,\Delta'$ and $\Gamma(\Delta)$.

\begin{figure}[h]
  \begin{framed}
  \begin{minipage}[c]{0.45\linewidth}
  \centering
      \tikzset{every tree node/.style={minimum width=2em},
        blank/.style={draw=none},
        edge from parent/.style=
        {draw,edge from parent path={(\tikzparentnode) -- (\tikzchildnode)}},
        level distance=1.5cm}
      \begin{tikzpicture}
        \Tree
        [.,
        [.;
        [.a:A ]
        [.b:B ]
        ]
        [.c:C ]
        ]
      \end{tikzpicture}
    \caption{Bunch in \textbf{\em BI}}
    \label{fig:bunches-bi}
  \end{minipage}\hfill%
  \begin{minipage}[c]{0.45\linewidth}
  \centering
      \begin{tikzpicture}
        \node[shape=circle,draw=black] (A) at (0,0) {a:A};
        \node[shape=circle,draw=black] (B) at (0,3) {b:B};
        \node[shape=circle,draw=black] (C) at (1.5,1.5) {c:C};

        \path [-] (A) edge node {} (B);
      \end{tikzpicture}
    \caption{Sharing Graph in \qub{}}
    \label{fig:sharing-graph}
  \end{minipage}
\end{framed}
\end{figure}

In our type system, we generalize the tree approach into a graph where each node represents variables or resources
and the edges between the nodes represent sharing between them. The example in \cref{fig:bunches-bi} can
be represented as what we would call a sharing graph shown in \cref{fig:sharing-graph}. A graph structure, in general,
can represent a binary tree structure and its associated operations. They represent more complex structures than
trees, thus will provide more flexibility in accepting well typed terms in our language making it more expressive.

We define sharing relation ($\Psi$) between variables to the collection of variables it is in sharing with.
The relation $\Psi(x, \{y_1, y_2, y_3\})$ holds if $x$ is in sharing with $\{y_1, y_2, y_3\}$.
Domain of $\Psi$ will be defined as $\texttt{dom}(\Psi) = \{x \mid (x, \vec{y}) \in \Psi \}$, where $\vec{y}$
is a shorthand for the denoting collection of variables shared with $x$. We can think of $\Psi$ to be similar to $\Gamma$,
but it contains the sharing information instead of the type of the variable. Extending the sharing for a variable will be denoted by $\Psi(x) + y$,
which would mean the variable $y$ is in sharing with $x$. We axiomatize the sharing operation to be reflexive,
symmetric and non-transitive. So to say,
\begin{flalign*}
 &\forall_{x \in \texttt{dom}(\Psi)}\ x \in \Psi(x) \tag{reflexive}\\
 &\forall_{x,y \in \texttt{dom}(\Psi)}\ \text{if}\ y \in \Psi(x)\ \text{then}\ x \in \Psi(y) \tag{symmetric}
\end{flalign*}

Our final goal is to design a simple type inferencing algorithm for a term language.
Using sharing graphs in implementing typing judgments would make the process considerably complex.
We simplify the sharing graph by flattening it into an adjacency list or a collection of 3 tuple containing the
variable identifier, its type and a collection of variables it shares with. Manipulating lists is much
easier than manipulating graphs. For example, if a resource $x$ has type $\tau$ and it shares with variables $\{y_1, y_2, y_3\}$
we would represent it as $x^{\{y_1, y_2, y_3\}}:\tau$ or just $x^{\bar{y}}:\tau$ for short.
We would write $\Gamma, x^{\vec{y}}:\tau$ to mean $\Gamma \sqcup \{x^{\vec{y}}:\tau\}$.
We can now formally define the typing context or environment as shown in \cref{fig:typing-context}.
\begin{figure}[h]
  \begin{framed}
    \begin{flalign*}
      \text{Typing Context}\ \ \      \Gamma,\Delta     &::= \epsilon \mid x^{\bar{y}}:\sigma \mid \Gamma \varoplus \Delta \mid \Gamma \circledast \Delta
  \end{flalign*}
\end{framed}
  \caption{Typing Context}
  \label{fig:typing-context}
\end{figure}

We define a few auxiliary functions on the
type assignments. \texttt{Vars}($\Gamma$) is the set of all the term variables in $\Gamma$. \texttt{Shared}($\Gamma$) computes
the set of all the term variables that are in sharing with each other. \texttt{Used}($\Gamma$) computes the
union of all the term variables in the type assignment and the term variables shared by each of those.
We define two partial operators on type assignments as shown in \cref{fig:type-assignment-operations}.
The mapping function ($\Gamma^{[\vec{a} \mapsto \vec{b}]}$) extends the sharing relation between the terms. In the sharing graph perspective
it would mean adding edges between the nodes.
\begin{figure}[h]
  \begin{framed}
    \noindent
    \begin{minipage}[h]{0.45\linewidth}
    \begin{flalign*}
      \texttt{Vars}(\Gamma, x^{\vec{y}}:\tau) &= \texttt{Vars}(\Gamma) \cup \{ x \}\\
      \texttt{Shared}(\Gamma, x^{\vec{y}}:\tau) &= \texttt{Shared}(\Gamma) \cup \{ \vec{y} \}\\
      \texttt{Used}(\Gamma) &= \texttt{Vars}(\Gamma) \cup \texttt{Shared}(\Gamma)\\
    \end{flalign*}
  \end{minipage}%
  \begin{minipage}[h]{0.45\linewidth}
    \begin{flalign*}
      (\Gamma, x^{\vec{y}}:\tau)^{[a \mapsto \vec{b}]} &= \begin{cases}
        a \notin \vec{y}\ \ \ \ (\Gamma^{[a \mapsto \vec{b}]}, x^{\vec{y}}:\tau)\\
        a \in \vec{y}\ \ \ \  (\Gamma^{[a \mapsto \vec{b}]}, x^{(\vec{y}\backslash a)\cup\vec{b}}:\tau)
      \end{cases}\\
      \Gamma^{[\vec{a} \mapsto \vec{b}]} &= (\dots((\Gamma^{[a_1 \mapsto \vec{b}]})^{[a_2 \mapsto \vec{b}]})^{\dots})^{[a_n \mapsto \vec{b}]}
    \end{flalign*}
    \end{minipage}
  \end{framed}
  \caption{Auxiliary Functions on Type Assignments}
  \label{fig:multiset-aux-function}
\end{figure}

Two type assignments are said to be in disjoint union ($\circledast$)
if either of the type assignments used terms are not in common
with other type assignment's shared term. If the type assignments have an exact overlapping of terms being used,
it is said to be in a sharing union ($\varoplus$). The ($\#$) in ($\circledast$) represents disjoint check and we use
the standard notion of set equality for checking sharing union.

\begin{figure}[h]
  \begin{framed}
    \begin{flalign*}
      \Gamma \circledast \Gamma' &= \Gamma \sqcup \Gamma' =>
           \texttt{if}\ \texttt{Vars}(\Gamma) \# \texttt{Used}(\Gamma') \wedge \texttt{Vars}(\Gamma')\# \texttt{Used}(\Gamma) \\
      \Gamma \varoplus \Gamma'   &= \Gamma \sqcup \Gamma' => \texttt{if}\ \texttt{Used}(\Gamma) = \texttt{Used}(\Gamma')
    \end{flalign*}
  \end{framed}
  \caption{Type Assignment Operations}
  \label{fig:type-assignment-operations}
\end{figure}


\begin{figure}[h]
  \begin{framed}
    \begin{flalign*}
      \text{Term Variables}\ \ \  x, y, z  &\in \text{Var} \nonumber\\
      \text{Expressions}\ \ \     M, N     &::= x \mid \lambda^{\sepimp}x. M \mid \lambda^{\shimp}x. M \mid M N \mid \Let{x}{M}{N}\nonumber
    \end{flalign*}
  \end{framed}
  \caption{Term Language}
  \label{fig:qub-terms}
\end{figure}
% Describe the language here

% Describe terms and patterns
Our term language is similar to that of simply typed lambda calculus involving variables and application
but we have two types of lambda expressions, the alpha lambda ($\lambda^{\shimp}$) denotes sharing
of the argument term with the expression $M$ and the separating lambda term ($\lambda^{\sepimp} $) that implies
the argument term has a separating context with the expression $M$. We also have $\texttt{let}$
construct to enable implicit parametric polymorphism.
%%% Local Variables:
%%% mode: latex
%%% TeX-master: "../thesis-ku.tex"
%%% End:
           % Core language syntax and Types, contexts and related definitions
\chapter{Type System and Type Inference}\label{chp:qub-type-system}
In this chapter we describe the type system using the types and terms defined in previous
chapter. We describe in \qub{}'s type system \cref{sec:type-system} and then describe a
syntax directed type system in \cref{sec:syntax-typing-rules} and give a type inference algorithm $\M$ in \cref{sec:algorithm-m}.
To begin with, we give some conventions and notations with priliminary definitions that will be used throughout the sections that follow.

\section{Conventions and Notations}
The vector $\vec{t}$ is a shorthand for a finite set of variables $\{t_1, t_2, \dots, t_n\}$ and  $\forall \vec{t}. P => \tau$ abbreviates
$\forall t_1 \dots \forall t_n. P_1 => \dots => P_m => \tau$.
$\Gamma$ denotes the type assignment. It is a list of three-tuples containg the variable, its type scheme and its sharing information.
$\Gamma_{x}$ denotes the type assignment excluding the type variable $x$.
We write $\sigma = \Gamma(x)$ for the type scheme assigned to the term x in $\Gamma$.
\texttt{dom}($\Gamma$) is the set of identifiers in the type assignment i.e.
\texttt{dom}($\Gamma$) = $\{ x \mid (x^{\vec{y}}:\sigma) \in \Gamma\}$.
$\Gamma \sqcup \Delta$ is a multiset union of two type assignments $\Gamma$ and $\Delta$.

\begin{defn}[Free Type Variables]

  $\texttt{fvs}(\tau)$ is the set of free type variables in the type $\tau$

  \noindent$\texttt{fvs}(\sigma)$ is the set of free type variables in a typing scheme $\sigma = \forall \vec{t} Q. => \tau$.\\
  $\texttt{fvs}(\sigma) = (\texttt{fvs}(\tau) \cup \texttt{fvs}(Q)) \backslash \vec{t})$

  \noindent$\texttt{fvs}(\Gamma)$ is the set of free type variables in the type assignment $\Gamma$.\\
  $\texttt{fvs}(\Gamma) = \bigcup_{x \in \texttt{dom}(\Gamma)} \{ \texttt{fvs}(\Gamma(x)) \}$
\end{defn}

\begin{defn}[Typing Judgment]
The expression $P \mid \Gamma \vdash M : \sigma$ denotes the assertion that the term $M$ has a typing scheme $\sigma$
when the predicates in $P$ are satisfied and the free type variables in $M$ are specified in type assignment $\Gamma$.
\end{defn}

\section{Type System}\label{sec:type-system}
% Structural Rules
% Connective Rules
% forall, => Qualified type rules
We split our type system into two parts. The first part includes structural rules
shown in \cref{fig:structural-rules} and the second part includes connectives with
introduction and elimination rules shown in \cref{fig:typing-rules}.

% structural rules
\begin{figure}[h]\centering
  \begin{framed}
    % ID
    \begin{minipage}{1\textwidth}
      \begin{prooftree}
        \AxiomC{{\color{white}$\Gamma \circledast \Delta \circledast$}} \RightLabel{[ID]}
        \UnaryInfC{$P \mid x^{\vec{y}} : \sigma \vdash x : \sigma $}
      \end{prooftree}
    \end{minipage}

    % CTR UN
    \begin{minipage}{.50\textwidth}
      \begin{prooftree}
        \AxiomC{$P \mid \Gamma \circledast \Delta \circledast \Delta \vdash M : \sigma$}
        \AxiomC{$P \vdash \Delta\ \texttt{un}$} \RightLabel{[CTR-UN]}
        \BinaryInfC{$P \mid \Gamma \circledast \Delta \vdash M : \sigma$}
      \end{prooftree}
    \end{minipage}%
    % CTR Sh
    \begin{minipage}{.50\textwidth}
      \begin{prooftree}
        \AxiomC{$P \mid \Gamma \varoplus \Delta \varoplus \Delta\vdash M : \sigma$}\RightLabel{[CTR-SH]}
        \UnaryInfC{$P \mid \Gamma \varoplus \Delta \vdash M : \sigma$}
      \end{prooftree}
    \end{minipage}

    % WKN UN
    \begin{minipage}{.50\textwidth}
      \begin{prooftree}
        \AxiomC{$P \mid \Gamma \vdash M : \sigma$}
        \AxiomC{$P \vdash \Delta\ \texttt{un}$} \RightLabel{[WKN-UN]}
        \BinaryInfC{$P \mid \Gamma \circledast \Delta \vdash M : \sigma$}
      \end{prooftree}
    \end{minipage}%
    % WKN Sh
    \begin{minipage}{.50\textwidth}
      \begin{prooftree}
        \AxiomC{$P  \mid \Gamma \vdash M : \sigma$} \RightLabel{[WKN-SH]}
        \UnaryInfC{$P \mid \Gamma \varoplus \Delta \vdash M : \sigma$}
      \end{prooftree}
    \end{minipage}
  \end{framed}
  \caption{Structural Typing Rules}
  \label{fig:structural-rules}
  \end{figure}

The tautology rule [ID] is a simple type assignment lookup for checking the type of the term.
The weakening and contraction rules are made explicit in contrast to standard
Hindley-Milner type system. The contraction sharing rule [CTR-SH] and weakening sharing rule [WKN-SH]
convey that we can duplicate or drop certain pairs of type assignments as we know they are in sharing with other
terms that remain in the context. The contraction separation rule [CTR-UN] and weakening separation rule [WKN-SH] can be
applied to terms only if we can prove that they are of unrestricted type which is captured by the antecedent predicate $\Delta$ \texttt{un}
on the type that is dropped or duplicated.

% Connective rules
\begin{figure}[h]\centering
  \begin{framed}
    % let
    \begin{minipage}{1\textwidth}
      \begin{prooftree}
        \AxiomC{$P \mid \Gamma \vdash M : \sigma$}
        \AxiomC{$P' \mid \Gamma'_{x} \sqcup x: \sigma \vdash N: \tau$} \RightLabel{[LET]}
        \BinaryInfC{$P \cup P' \mid \Gamma \sqcup \Gamma' \vdash (\Let{x}{M}{N}): \tau$}
      \end{prooftree}
    \end{minipage}
\newline\newline\newline
    % forall I
    \begin{minipage}{0.50\textwidth}
      \begin{prooftree}
        \AxiomC{$P \mid \Gamma \vdash M: \sigma$}
        \AxiomC{$t \notin \texttt{fvs}(\Gamma) \cup \texttt{fvs}(P)$}\RightLabel{[$\forall$ I]}
        \BinaryInfC{$P \mid \Gamma \vdash M: \forall t. \sigma$}
      \end{prooftree}
    \end{minipage}%
    % forall E
    \begin{minipage}{0.45\textwidth}
      \begin{prooftree}
        \AxiomC{$P \mid \Gamma \vdash M: \forall t.\sigma$}\RightLabel{[$\forall$ E]}
        \UnaryInfC{$P \mid \Gamma \vdash M: [\tau \backslash t] \sigma $}
      \end{prooftree}
    \end{minipage}
\newline\newline\newline
    % => I
    \begin{minipage}{0.50\textwidth}
      \begin{prooftree}
        \AxiomC{$P, \pi \mid \Gamma \vdash M : \rho$} \RightLabel{[$\Rightarrow$ I]}
        \UnaryInfC{$P \mid \Gamma \vdash M : \pi \Rightarrow \rho$}
      \end{prooftree}
    \end{minipage}%
    % => E
    \begin{minipage}{0.45\textwidth}
      \begin{prooftree}
        \AxiomC{$P \mid \Gamma \vdash M : \pi \Rightarrow \rho$}
        \AxiomC{$P \vdash \pi$} \RightLabel{[$\Rightarrow$ E]}
        \BinaryInfC{$P \mid \Gamma \vdash M: \rho$}
      \end{prooftree}
    \end{minipage}
\newline\newline\newline
    % -&> I
    \begin{minipage}{0.45\textwidth}
      \begin{prooftree}
        \AxiomC{$P \Rightarrow \texttt{ShFun}\ \phi\ \ \ \ \
          P \vdash \Gamma \geq \phi$}\noLine\def\extraVskip{-0.2pt}
        \UnaryInfC{$P \mid \Gamma^{[\emptyset\mapsto \{x\}]},x^{\text{Vars}(\Gamma)}: \tau \vdash M : \tau'$}\RightLabel{[$\shimp$ I]}\def\extraVskip{2pt}
        \UnaryInfC{$P \mid \Gamma \vdash \lambda^{\shimp}x. M : \phi \tau \tau'$}
      \end{prooftree}
    \end{minipage}%
    % -&> E
    \begin{minipage}{0.50\textwidth}
      \begin{prooftree}
        \AxiomC{$P \Rightarrow \texttt{ShFun}\ \phi$}\noLine\def\extraVskip{0pt}
        \UnaryInfC{$P \mid \Gamma \vdash M : \phi \tau \tau'\ \ \ \ \
          P \mid \Gamma' \vdash N : \tau$} \RightLabel{[$\shimp$ E]}\def\extraVskip{2pt}
        \UnaryInfC{$P \mid \Gamma \varoplus \Gamma' \vdash M N : \tau'$}
      \end{prooftree}
    \end{minipage}
\newline\newline\newline
    % -*> I
    \begin{minipage}{0.50\textwidth}
      \begin{prooftree}
        \AxiomC{$P \Rightarrow \texttt{SeFun}\ \phi\ \ \ \ \ P \vdash \Gamma \geq \phi$}\noLine\def\extraVskip{0pt}
        \UnaryInfC{${\color{white}\ \ \ \ }P \mid \Gamma,x^{\emptyset}: \tau \vdash M : \tau'{\color{white}\ \ \ \ }$} \RightLabel{[$\sepimp$ I]}\def\extraVskip{2pt}
        \UnaryInfC{$P \mid \Gamma \vdash \lambda^{\sepimp}x. M : \phi \tau \tau'$}
      \end{prooftree}
    \end{minipage}%
    % -*> E
    \begin{minipage}{0.45\textwidth}
      \begin{prooftree}
        \AxiomC{$P \Rightarrow \texttt{SeFun}\ \phi$}\noLine\def\extraVskip{0pt}
        \UnaryInfC{$P \mid \Gamma \vdash M : \phi \tau \tau'\ \ \ \
          P \mid \Gamma' \vdash N : \tau$}\RightLabel{[$\sepimp$ E]}\def\extraVskip{2pt}
        \UnaryInfC{$P \mid \Gamma \circledast \Gamma' \vdash M N : \tau'$}
      \end{prooftree}
    \end{minipage}
  \end{framed}
  \caption{Connective Typing Rules}
  \label{fig:typing-rules}
\end{figure}

The [LET] rule is used to enable implicit parametric polymorphism as usual.
The rules of [$\shimp$I] and [$\sepimp$I] describes the abstraction over shared and
separating resources respectively, while [$\shimp$E] and [$\sepimp$E] is the application
rule for shared and separating resources respectively. [$=>$I] and [$=>$E] are the rules for
qualified types that would add constraints on the type being computed. [$\forall$I] introduces
polymorphism and [$\forall$E] eliminates it. The $\lambda$ abstractions $\lambda^{\sepimp}  x. M$ and $\lambda^{\shimp}.M$
have a function type $\ShFun{\phi}$ or $\SeFun{\phi}$ only if it is more restricting that its environment.
This is specified in the judgments $\cdot \geq \cdot$. To avoid name shadowing, we would assume that
the binders introduce fresh names. In case of a $\lambda^{\shimp}$, the binder variable is added to the sharing information of all
terms present in the type assignment and in case of $\lambda^{\sepimp} $ the binder variable is skipped from
adding it to the sharing information to imply separation of resources. For example,
consider the term $\lambda^{\sepimp} c. \lambda^{\sepimp}  x. \lambda^{\shimp}y. c x$. The sharing graph and type assignment is shown in
\cref{fig:example-sharing-graph}.

\begin{figure}[h]
  \begin{framed}\centering
    \begin{minipage}{0.3\linewidth}
    \begin{tikzpicture}
      \node[shape=circle,draw=black] (A) at (0,0) {x};
      \node[shape=circle,draw=black] (B) at (0,3) {y};
      \node[shape=circle,draw=black] (C) at (1.5,1.5) {c};

      \path [-] (A) edge node {} (B);
      \path [-] (C) edge node {} (B);
    \end{tikzpicture}
  \end{minipage}
  \begin{minipage}{0.4\linewidth}
    $\Gamma = \{ c^{y}:\tau \sepimp \upsilon, x^{y}:\tau, y^{\{c,x\}}:\tau' \}$
  \end{minipage}
  \end{framed}
  \caption{Sharing graph and typing context for $\lambda^{\sepimp} c. \lambda^{\sepimp}  x. \lambda^{\shimp}y. c x$}
  \label{fig:example-sharing-graph}
\end{figure}

The rules given in \cref{fig:bi-base-typing-rules}
are convenience rules for base cases that compute predicate constraints for types within a context.

\begin{figure}[h]\centering
  \begin{framed}
    \fbox{$P \vdash \cdot\ \texttt{un}$}
  \begin{minipage}{0.50\linewidth}
    \begin{prooftree}
      \AxiomC{$P => \texttt{Un}\ \tau$}\RightLabel{[\texttt{Un}-$\tau$]}
      \UnaryInfC{$P \vdash \tau\ \texttt{un}$}
    \end{prooftree}
  \end{minipage}%
  \begin{minipage}{0.50\linewidth}
    \begin{prooftree}
      \AxiomC{$P,\pi \vdash \rho\ \texttt{un}$}\RightLabel{[\texttt{Un}-$\rho$]}
      \UnaryInfC{$P \vdash \pi => \rho\ \texttt{un}$}
    \end{prooftree}
  \end{minipage}

  \begin{minipage}{0.50\linewidth}
    \begin{prooftree}
      \AxiomC{$P, \texttt{Un}\ t \vdash \sigma\ \texttt{un}$}\RightLabel{[\texttt{Un}-$\sigma$]}
      \UnaryInfC{$P \vdash \forall t.\sigma\ \texttt{un}$}
    \end{prooftree}
  \end{minipage}%
  \begin{minipage}{0.50\linewidth}
    \begin{prooftree}
      \AxiomC{$\bigwedge_{x:\sigma \in \Gamma}P \vdash \sigma\ \texttt{un}$}\RightLabel{[\texttt{Un}-$\Gamma$]}
      \UnaryInfC{$P \vdash \Gamma\ \texttt{un}$}
    \end{prooftree}
  \end{minipage}

  \fbox{$P \vdash \cdot \geq \cdot$}
  \begin{minipage}{0.50\linewidth}
    \begin{prooftree}
      \AxiomC{$P => \tau \geq \phi$}\RightLabel{[$\geq$-$\tau$]}
      \UnaryInfC{$P \vdash \tau \geq \phi$}
    \end{prooftree}
  \end{minipage}%
  \begin{minipage}{0.50\linewidth}
    \begin{prooftree}
      \AxiomC{$P,\pi \vdash \rho \geq \phi$}\RightLabel{[$\geq$-$\rho$]}
      \UnaryInfC{$P \vdash (\pi => \rho) \geq \phi$}
    \end{prooftree}
  \end{minipage}

  \begin{minipage}{0.50\linewidth}
    \begin{prooftree}
      \AxiomC{$P, \texttt{Un}\ t \vdash \sigma \geq \phi$}\RightLabel{[$\geq$-$\sigma$]}
      \UnaryInfC{$P \vdash (\forall t.\sigma) \geq \phi$}
    \end{prooftree}
  \end{minipage}%
  \begin{minipage}{0.50\linewidth}
    \begin{prooftree}
      \AxiomC{$\bigwedge_{x:\sigma \in \Gamma}P \vdash \sigma \geq \phi$}\RightLabel{[$\geq$-$\Gamma$]}
      \UnaryInfC{$P \vdash \Gamma \geq \phi$}
    \end{prooftree}
  \end{minipage}
\end{framed}
  \caption{Entailment Rules for Base cases}
  \label{fig:bi-base-typing-rules}
\end{figure}


\section{Syntax Directed Typing rules}\label{sec:syntax-typing-rules}
Ideally the typing rules and syntactic forms should have one-to-one
correspondence. The type system explained in the \cref{sec:type-system} is not syntax directed and will not be fit
to develop a type inference algorithm. In this section we will define syntax directed typing rules
that will simplify our type inference system shown in \cref{fig:syntax-typing-rules}

We define generalization and instantiation to express introduction and elimination of polymorphism in our
syntax direct typing rules as follows:
\begin{defn}[Instantiation]
  For a type scheme $\sigma := \forall \vec{t}. P => \tau'$, we say $(Q => \tau)$ is
  an instance of $\sigma$ and write it as $(Q => \tau) \sqsubseteq \sigma$, if there exists a $\vec{v}$
  such that $\tau = [\vec{v} / \vec{t}] \tau'$ and $Q = [\vec{v} / \vec{t}]P$.
\end{defn}

\begin{defn}[Generalization]
  For a type assignment $\Gamma$ and qualified type $\rho$, we define type scheme
  $\texttt{Gen}(\Gamma, \rho) = \forall (\texttt{fvs}(\rho) \backslash \texttt{fvs}(\Gamma)). \rho$.
\end{defn}

\begin{defn}[Qualified Type Scheme]
  A qualified type scheme is a pair of type scheme with a set of predicates written as $(P \mid \sigma)$,
  where $\sigma = \forall \vec{t}. Q => \tau$.
\end{defn}
% \begin{defn}[Qualified Type Scheme Instantiation]
%   For two qualified type schemes $(P \mid \sigma)$ where $\sigma = \forall \vec{t}. Q => \tau$ and
%   $(P' \mid \sigma')$ where $\sigma' = \forall \vec{t'}. Q' => \tau'$ we say $(P' \mid \sigma')$ is an
%   instance of $(P \mid \sigma)$ iff there exists $\vec{v}$ such that $\tau' = [\vec{v}/ \vec{t}]\tau$ and
%   $P',Q' => P, [\vec{v}/ \vec{t}]Q$. We write it as $(P' \mid \sigma') \sqsubseteq (P \mid \sigma)$. A type scheme
%   $\sigma$ is an abbreviation of $(\emptyset \mid \sigma)$.
% \end{defn}

The elimination of polymorphism [$\forall$E] and qualified types[$=>$E] is always done in the [Var$^s$],
introduction of polymorphism [$\forall$I] and qualified types[$=>$I] is
done at \texttt{let} bindings [Let$^s$]. This collapses the rules [$\forall$E], [$=>$E] and [ID] in one rule [VAR$^s$] where
we use instantiation or specialization of type variables, and [$\forall$I], [$=>$I] and [Let] in one rule [Let$^s$] where we use generalization of type variables.
[$\sepimp$I$^s$] is used in occurence of $\lambda^{\sepimp} $, and [$\shimp$I$^s$] is used in occurence of $\lambda^{\shimp}$.
We would add the introduced abstraction variable, $x$, into the sharing context in case of [$\shimp$I$^s$].
We collapse the application rules [$\sepimp$E] and [$\shimp$E] into one rule [App$^s$] where we check for sharing of the used variables in both
the expressions and then assign a predicate of $\texttt{ShFun}$ or $\texttt{SeFun}$ depending on whether the variables
are shared or not. The \texttt{un} predicates are added to the types of the terms that are not used directly in the expression
or which are not in sharing with the terms used.

\begin{figure}[h]
  \begin{framed}
    % VAR^s
    \begin{minipage}{1.0\textwidth}
      \begin{prooftree}
        \AxiomC{$P \vdash \Gamma_{\vec{y}}\ \texttt{un}$}
        \AxiomC{$(P => \tau) \sqsubseteq \sigma$} \RightLabel{[VAR$^s$]}
        \BinaryInfC{$P \mid \Gamma, x^{\vec{y}} : \sigma \vdashs x : \tau $}
      \end{prooftree}
    \end{minipage}
    \newline\newline\newline
    % Let^s
    \begin{minipage}{1.0\textwidth}
      \begin{prooftree}
        \AxiomC{$Q \mid (\Gamma_x' \varoplus \Gamma_x'') \circledast \Delta \vdashs M: \upsilon\ \ \ \ \
          P \vdash \Delta\ \texttt{un}$}\def\extraVskip{0pt}\noLine
        \UnaryInfC{$P \mid (\Gamma_x \sqcup x^{\emptyset}:\sigma) \varoplus \Gamma_x'' \circledast \Delta \vdashs N:\tau\ \ \ \ \
          \sigma = \texttt{Gen}(\{\Gamma' \varoplus \Gamma_x'' \circledast \Delta \}, Q => \upsilon)$}\def\extraVskip{2pt}\RightLabel{[Let$^s$]}
        \UnaryInfC{$P \mid (\Gamma \circledast \Gamma') \varoplus \Gamma'' \circledast \Delta \vdashs (\Let{x}{M}{N}) : \tau$}
      \end{prooftree}
    \end{minipage}
    \newline\newline\newline
    % -*>I^s
    \begin{minipage}{0.5\textwidth}
      \begin{prooftree}
        \AxiomC{$P => \SeFun{\phi}$}
        \AxiomC{$P \vdash \Gamma \geq \phi$}\def\extraVskip{0pt}\noLine
        \BinaryInfC{$\ \ \ \ \ \ \ P \mid \Gamma \circledast x^{\emptyset}:\tau \vdashs M: \upsilon\ \ \ \ \ \ \ $}\def\extraVskip{2pt}\RightLabel{[$\sepimp$I$^s$]}
        \UnaryInfC{$P \mid \Gamma \vdashs \lambda^{\sepimp} x. M : \phi \tau \upsilon$}
      \end{prooftree}
    \end{minipage}%
    % -&>I^s
    \begin{minipage}{0.5\textwidth}
      \begin{prooftree}
        \AxiomC{$P => \ShFun{\phi}\ \ \ \ \ P \vdash \Gamma \geq \phi$}\def\extraVskip{0pt}\noLine
        \UnaryInfC{$P \mid \Gamma^{[\emptyset \mapsto \{x\}]} \varoplus x^{\texttt{Vars}(\Gamma)}:\tau \vdashs M: \upsilon$}\def\extraVskip{2pt}\RightLabel{[$\shimp$I$^s$]}
        \UnaryInfC{$P \mid \Gamma \vdashs \lambda^{\shimp}x. M : \phi \tau \upsilon$}
      \end{prooftree}
    \end{minipage}
    \newline\newline\newline
    % App^s
    \begin{minipage}{1.0\textwidth}
      \begin{prooftree}
        \AxiomC{$P \mid \Gamma \circledast \Delta \vdashs M: \phi \upsilon \tau\ \ \ \ \
          P \mid  \Gamma' \circledast \Delta \vdashs N: \upsilon$}
        \AxiomC{$P \vdash \Delta\ \texttt{un}$}\def\extraVskip{0pt}\noLine
        \BinaryInfC{$(\Gamma \tightoverset{\thicksim}{\varoplus} \Gamma' \wedge (P => \ShFun{\phi}))
          \vee (\Gamma \tightoverset{\thicksim}{\circledast} \Gamma' \wedge (P => \SeFun{\phi}))$}\def\extraVskip{2pt}\RightLabel{[App$^s$]}
        \UnaryInfC{$P \mid \Gamma \sqcup \Gamma' \circledast \Delta \vdashs M N: \tau$}
      \end{prooftree}
    \end{minipage}
    \end{framed}
  \caption{Syntax Directed Typing Rules}
  \label{fig:syntax-typing-rules}
\end{figure}

The [Let$^s$] rule defines an expression within another expression locally i.e. x would
not be in scope other than its use in $N$. We partition the typing context into multiple parts.
$\Gamma$ contains the variables that exists exclusively in $M$ and $\Gamma'$ which
are exclusively in $N$. $\Gamma''$ is common to both $M$ and $N$ while $\Delta$ is not used in either expressions $M$ or $N$.
Thus $\Gamma$ and $\Gamma'$ will be completely separate from each other while $\Gamma''$ would be in sharing with both $\Gamma$ and $\Gamma'$.
$\Delta$ would have to be unrestricted as it is not being used either in $M$ or in $N$. The sharing of $x$ with $\Gamma$ would depend on
whether $\Gamma$ and $\Gamma'$ are completely disjoint or empty. For the application rule [App$^s$] the type assignment $\Gamma$ would contain
variables for $M$ and $\Gamma'$ for $N$. If they are completely separate, it would be a separating function application and $M$ would be
assigned a type $\tau' \sepimp \tau$ else, they would have to be completely sharing and M would be assigned a type $\tau' \rightarrow \tau$.
The conditions outlined for [App$^s$] do not clearly look as if they are syntax directed and in the cases where the type checker cannot directly infer if the
resources used are separate or shared, the user would be expected to provide it using type annotations. $\Gamma \tightoverset{\thicksim}{\circledast} \Delta$
is an assertion that there exists a proof either possible to find by inspecting the type assignment, or provided by the user
that $\Gamma \circledast \Delta$ is defined. Similarly, $\Gamma \tightoverset{\thicksim}{\varoplus} \Delta$ means that $\Gamma \varoplus \Delta$ is well
defined.

We now state two important theorems regarding the type system and the syntax directed type system. The proofs of both the
theorems are given in \cref{chp:proofs}.

\begin{theorem}[Soundness of $\vdash^s$]
  If $P \mid \Gamma \vdash^s M:\tau$ then $P \mid \Gamma \vdash M : \tau$
\end{theorem}
The soundness property captures the essence that derivations in the syntax directed type system follow the original type system.

\begin{theorem}[Completeness of $\vdashs$]\label{thm:completeness-syntax-directed}
  If $P \mid \Gamma \vdash M:\sigma$ then
  $\exists Q, \tau$ such that $Q \mid \Gamma \vdashs M:\tau$
  and $(P \mid \sigma) \sqsubseteq \texttt{Gen}(\Gamma, Q => \tau)$
\end{theorem}

The completeness theorem states that we can always find predicates $Q$ and a suitable type $\tau$ for the term $M$ under the assumptions
$\Gamma$ using syntax directed type system if the original type system asserts that the typing judgement indeed exists. 

\section{Type Inference and Algorithm $\M$}\label{sec:algorithm-m}
We now describe the type inference algorithm based on the previously defined syntax directed
type system. We use a variation of algorithm $\M$ \citep{lee_proofs_1998} for type inference.
We address three independent concerns in the type inference algorithm.
The first being treatment of polymorphism to be same as Hindley-Milner style. The second, we introduce
\texttt{Un} predicates for types that are unrestricted. We track this with the help of carrying a
collection of used variables throughout the algorithm which detects whether a variable is discarded
or used multiple types. The third being accounting for sharing of the variables.
The complete algorithm is outlined in \cref{fig:algorithm-m}. The input to the algorithm includes the term $M$ whose
type has to be inferred, $\tau$ is the expected type of the term, $S$ is the current substition and the
sharing information $\Psi$. The output includes the set of new predicates $P$ that are generated,
the new set of substitutions $S'$, the used variables $\Sigma$, and the new sharing information $\Psi'$.
The type variables $u$ with a subscript denote fresh variables.

We define some auxilary functions in \cref{fig:aux-defs} to lift predicates into the type system.
$Leq(\phi, \Gamma)$ adds the predicate $\phi \geq \sigma$ for all $\sigma$ that are in $\Gamma$.
$\texttt{Weaken}(x, \sigma, \Sigma)$ adds the unrestricted predicate to the type $\sigma$
if $x$ does not belong to $\Sigma$. $\texttt{Un}(\Gamma)$ adds an unrestricted
predicate to all the types of the variables that are in the domain of $\Gamma$.
$\texttt{GenI}(\Gamma, P => \tau)$ generalizes the qualified type to a type scheme.
$\mathcal{C}(\Gamma, \Psi, \Sigma)$ computes the sharing information of variables in $\Gamma$
restricted to the variables in $\Sigma$.

\begin{figure}[h]
  \begin{framed}
    \begin{minipage}{0.5\linewidth}
      \begin{flalign*}
        Leq(\phi, \Gamma)  = \bigcup_{(x:\sigma) \in \Gamma} \{P \mid P \vdash \sigma \geq \phi \}
      \end{flalign*}
    \end{minipage}
    \begin{minipage}{0.5\linewidth}
      \begin{flalign*}
        \texttt{Un}(\Gamma)  = \bigcup_{(y:\sigma) \in \Gamma}\{P \mid P \vdash \sigma\ \texttt{un} \}
      \end{flalign*}
    \end{minipage}
    \newline\newline
    \begin{minipage}{0.5\linewidth}
      \begin{flalign*}
        \texttt{Weaken}(x, \sigma, \Sigma)  = \begin{cases}
          P\ \ \ \ &\text{if}\ x \notin \Sigma, P \vdash \sigma\ \texttt{un}\\
          \emptyset\ \ \ &otherwise
        \end{cases}
      \end{flalign*}
    \end{minipage}
    \begin{minipage}{0.5\linewidth}
      \begin{flalign*}
        \texttt{GenI}(\Gamma, P &=> \tau)  = \\
        &\forall (\texttt{fvs}(P) \cup \texttt{fvs}(\tau) \backslash\texttt{fvs}(\Gamma))). P => \tau
      \end{flalign*}
    \end{minipage}
    \newline\newline
    \begin{minipage}{1\linewidth}
      \begin{flalign*}
        \mathcal{C}(\Gamma, \Psi, \Sigma)  = \bigcup_{x \in \texttt{dom}(\Gamma) \wedge x \in \Sigma} \Psi(x)
      \end{flalign*}
    \end{minipage}
  \end{framed}
  \caption{Auxiliary Functions}
  \label{fig:aux-defs}
\end{figure}

\begin{figure}[h]
  \begin{framed}
    {\small
      \begin{minipage}[ht]{1\linewidth}
        \centering
        \fbox{
          $\M(S, \Psi, \Gamma \vdash M : \tau) = P, S', \Sigma, \Psi'$
        }
      \end{minipage}
      % x var
      \begin{minipage}{1\linewidth}
        \begin{flalign*}
          \M(S, \Psi, \Gamma \vdash x : \tau) &= ([\vec{u} / \vec{t}]P), S' \circ S, \{x\}, \Psi \\
          \text{where\qquad}\ (x : \forall \vec{t}. P => \upsilon) &\in S \Gamma \\
          S' &= \Unf([\vec{u} / \vec{t}]\upsilon, S \tau)
        \end{flalign*}
      \end{minipage}

      % \&x. M: t
      \begin{minipage}{1\linewidth}
        \begin{flalign*}
          \M(S, \Psi, \Gamma \vdash \lambda^{\shimp} x. M : \tau) &= \{P \cup Q \}, S', \Sigma \backslash x, \Psi''  \\
          \text{where\qquad}\ P; S'; \Sigma; \Psi' &= \M(\Unf(\tau, u_1 u_2 u_3) \circ S, \Psi, \Gamma, x:u_2 \vdash M: u_3) \\
          Q &= \{\ShFun{u_1}\} \cup \texttt{Leq}(u_1, \Gamma|_{\Sigma}) \cup \texttt{Weaken}(x, u_2, \Sigma)\\
          \Psi'' &= \{\forall_{y \in \texttt{dom}(\Psi')}. \Psi'(y) + x\} \cup \{(x, \{ y \mid y \in \texttt{dom}(\Gamma) \})\}
        \end{flalign*}
      \end{minipage}

      % \*x. M: t
      \begin{minipage}{1\linewidth}
        \begin{flalign*}
          \M(S, \Psi, \Gamma \vdash \lambda^{\sepimp}  x. M : \tau) &= \{ P \cup Q \}, S', \Sigma \backslash x, \Psi''\\
          \text{where\qquad}\ P; S'; \Sigma; \Psi' &= \M(\Unf(\tau, u_1 u_2 u_3) \circ S, X; \Gamma, x:u_2 \vdash M: u_3) \\
          Q &= \{\SeFun{u_1}\} \cup \texttt{Leq}(u_1, \Gamma\mid_{\Sigma}) \cup \texttt{Weaken}(x, u_2, \Sigma)\\
          \Psi'' &= \Psi' \cup \{(x, \{ x \})\}
        \end{flalign*}
      \end{minipage}

      % M N: t
      \begin{minipage}{1\linewidth}
        \begin{flalign*}
          \M(S, \Psi, \Gamma \vdash M N : \tau) &= \{ P \cup P' \cup Q \}, R', \Sigma \cup \Sigma', \Psi'' \\
          \text{where\qquad}\ P; R; \Sigma; \Psi' &= \M(S, \Psi, \Gamma \vdash M:  u_1 u_2 \tau) \\
          P'; R'; \Sigma'; \Psi'' &= \M(S R, \Psi', S \Gamma \vdash N: u_2)\\
          \text{if}\ \mathcal{C}(\Gamma, \Psi'', \Sigma) &= \mathcal{C}(\Gamma, \Psi'', \Sigma')\\
          \text{then}\ Q &= \{\ShFun{u_1}\} \\
          \text{else}\ \text{if}\ &(\Sigma \# \mathcal{C}(R\Gamma, \Psi'', \Sigma')\ \text{and}\ \Sigma' \# \mathcal{C}(R\Gamma, \Psi'', \Sigma))\\
          &\text{then}\ Q = \{\SeFun{u_1}\}
          %\text{else}\ Q &= \{\SeFun{u_1}\} \cup \text{Un}(\Gamma|_{\Sigma \cap \Sigma'})
        \end{flalign*}
      \end{minipage}

      % let x = M in N: t
      \begin{minipage}{1\linewidth}
        \begin{flalign*}
          \M(S, \Psi, \Gamma \vdash \Let{x}{M}{N} : \tau) &= (P \cup Q), R', \Sigma \cup \{\Sigma' \backslash x \}, \Psi'' \\
          \text{where\qquad}\ P; R; \Sigma; \Psi' &= \M(S, \Psi, \Gamma \vdash M:u_1)  \\
          \sigma &= \texttt{GenI}(R\Gamma; R(P => u_1)) \\
          P'; R'; \Sigma'; \Psi'' &= \M(R, \Psi', \Gamma, x:\sigma \vdash N : \tau) \\
          Q &= \texttt{Un}(\Gamma|_{\Sigma \cap \Sigma'}) \cup \texttt{Weaken}(x, \sigma, \Sigma')
        \end{flalign*}
      \end{minipage}
    }
  \end{framed}
  \caption{Type Inference Algorithm $\mathcal{M}$}
  \label{fig:algorithm-m}
\end{figure}

% Explain each case of algorithm M here.

% Var
The first case in algorithm $\M$ in \cref{fig:algorithm-m} describes the variable case, where we are given the variable identifier and
the expected type. We try to unify the expected type $\tau$ with the derived type scheme
$\nu$ from the type assignment $\Gamma$. The return values of the algorithm are
a new set of predicates which are nothing but instantiated version of the predicates
obtained from the typing scheme, the variable $x$ being used and the new
substitution which is combination of the unification algorithms output and the original
substitution. There is no change in the sharing information and $\Psi$ is returned as is.

% \&x
In next case of sharing function introduction rule $\lambda^{\shimp}x.M$, as per the [$\shimp$I] rules, the entitites returned
are union of four predicate categories. The first is assigning a the predicate of the function to be
a sharing function $\ShFun{u_1}$, where $u_1$ is a new type variable for the type of function argument $x$.
The second assigning the function to be less restricting than the other variables in the typing assignment $\Gamma$.
The third, assigning an unrestricted predicate to the binding variable $x$ if it has not been used
anywhere in the lambda body $M$, which is done by the $\texttt{Weaken}$ function and the fourth
being the predicates generated by recursively type checking the body of the lambda expression $M$.
We create sharing links for $x$ with all the variables within the typing context $\Gamma$ in updated
$\Psi''$.
% \*x
The case of separating function $\lambda^{\sepimp} x. M$ is very similar to the previous case of $\lambda^{\shimp}x. M$ except
the that there is no sharing information to be updated as the argument to the function is
separate from its body and the function predicate assigned is $\SeFun{u_1}$ to denote this very
separating relation instead of $\ShFun u_1$

% App
In the application case the algorithm typechecks the subexpressions $M$ as a function type having
an input of the type of $N$. The additional check done here is to identify whether $M$ has a sharing
application or a separating application. If all the variables used in $M$ are also used in $N$ then
it is a sharing application i.e. $M$ is assigned a sharing function predicate $\ShFun{}$
else if the used variables in $M$ are disjoint to the variables that
are shared by $N$ or if the variables shared by variables in $M$ are disjoint to the variables
that are used in $N$, $M$ is assigned the predicate type $\SeFun{}$. Incase of a separating function
application, the variables that are used in both are marked as unrestricted. This is captured by
$\texttt{Un}(\Gamma|_{\Sigma \cap \Sigma'})$ where $\Gamma|_{\Sigma}$ means the typing assignment $\Gamma$
restricted to the variables in $\Sigma$.
% Let
In the polymorphic \texttt{let} case we first check for the type of the expression $M$ and
ensure that the variable binding $x$ is not usd in it to avoid recursive definition which would
lead to infinite types. We then generalize the computed type to $\sigma$ and then check the type of expression $N$
by expanding the type assignment $\Gamma$ with the variable $x$ and type scheme $\sigma$.

We now state the soundness theorem for the algorithm $\M$.
\begin{theorem}[Soundness of $\M$]
  if $\M(S, \Psi, \Gamma \vdash M : \tau) = P, S', \Sigma, \Psi'$ then $S' P \mid S' (\Gamma |_\Sigma ) \vdash M : S' \tau$
\end{theorem}
The soundness theorem for algorithm $\M$ ensures that the type inference algorithm inferes a type that the original type system would have
produced. The proof is given in \cref{chp:proofs}.

This type inference algorithm is however not complete. There would be cases where a term can be typed in two ways.
Algorithm $\M$ would compute only one of them and leave out the other one. Consider the type for $\lambda^{\sepimp}f. \lambda^{\sepimp}x. f x x$.
This term can be inferred to have two types by using algorithm $\M$.
\begin{itemize}
  \item $\texttt{Un}\ A \mid \emptyset \vdash \lambda^{\sepimp}f. \lambda^{\sepimp}x. f x x: (A \sepimp A \sepimp B) \sepimp A \sepimp B$
  \item $\emptyset \mid \emptyset \vdash \vdash \lambda^{\sepimp}f. \lambda^{\sepimp}x. f x x: (A \sepimp A \shimp B) \sepimp A \shimp B$.
\end{itemize}

% We summarize the proof results in \cref{fig:proof-summary}

% \begin{figure}[h]
%   \centering
%   \fbox{$\Gamma \vdash M:\tau$} \fbox{$\Gamma \vdashs M:\tau$} \fbox{$\M(\Gamma \vdash M:\tau)$}
%   \caption{Proof Summary}
%   \label{fig:proof-summary}
% \end{figure}


%%% Local Variables:
%%% mode: latex
%%% TeX-master: "../thesis-ku"
%%% End:
         % typing rules, syntax directed typing rules and modified Algorithm M
\chapter{\qub{} Extention and Datatypes}\label{chp:datatypes}
In this chapter we discuss how \qub{} can be extended to have a kind system which makes
the type language powerful enough to accept user defined datatypes in \cref{sec:kind-system}. We also discuss
how we can encode sums and multiplicative and additive products in \cref{sec:pairs-types}.

\section{Kind System}\label{sec:kind-system}
In the original system, it would be tedious to extend the system with new types
as we would have to introduce new syntax and associated typing rules for
each of the new syntax in our core language. The kind system generalizes the
concept of adding new types by abstracting them as type
constructors. This generalization alleviates the burden of modifying the core
language by treating all types to be instances of type constructors.
The idea was introduced by Barendregt \citeyearpar{barendregt_1991} and
is used by Jones \citeyearpar{jones_system_1993} for qualified types. We will
follow Jones' approach to add the language of type constructors and kinds in our system.
The complete type system is shown in \cref{fig:qub-kind-extension}. All types have kind
$\star$ and the kind of the type constructors depends on its arity.

\begin{figure}[h]
  \begin{framed}
    \begin{flalign*}
      \text{Type Variables}\ \ t, u, \upsilon         &\in \text{Type Variables}  \\
      \text{Kinds}\ \ \               \kappa          &::= \kappa \mid \kappa' \rightarrow \kappa\\
      \text{Types}\ \ \             \tau^{\kappa}      &::= t^\kappa \mid T^{\kappa} \mid \tau^{\kappa' \rightarrow \kappa}\tau^{\kappa'}\\
      \text{Type Constructors}\ \ \   T^{\kappa}       &\in \mathcal{T}^{\kappa}\ 
                         \text{where}\qquad\{\otimes, \with, \oplus, \tightoverset{\scalebox{0.5}{!}}{\sepimp}, \sepimp, \tightoverset{\scalebox{0.5}{!}}{\shimp}, \shimp \} \subseteq \mathcal{T}^{\star \rightarrow \star \rightarrow \star}\\
      \text{Predicates}\ \ \          \pi, \omega     &::= \tau \mid \Un{\tau} \mid \SeFun{\tau} \mid \ShFun{\tau} \mid \tau \geq \tau' \\
      \text{Qualified Types}\ \ \     \rho            &::= \tau \mid \pi => \rho \\
      \text{Type schemes}\ \ \        \sigma          &::= \rho \mid \forall t. \sigma
    \end{flalign*}
  \end{framed}
  \caption{Extended \qub{} Types and Kinds}
  \label{fig:qub-kind-extension}
\end{figure}

Addition of a kinds changes the type system in a great detail.
All types and type constructors have to be annotated with their kind. $T^\kappa$ denotes type constructors
and $\tau^{\kappa' \rightarrow \kappa}\tau^{\kappa'}$ denotes application of types.
The type constructor application rule that computes kinds is given in \cref{fig:constructor-application} where
$\tau$ is of kind $\kappa' \rightarrow \kappa$ and $\tau'$ is of kind $\kappa'$. The application
of both the constructors would result in a kind $\kappa$.
$\sepimp$, $\tightoverset{\scalebox{0.5}{!}}{\sepimp}$, $\shimp$ and $\tightoverset{\scalebox{0.5}{!}}{\shimp}$
are now treated as type constructors with an arity of two and would have a kind $\star \rightarrow \star \rightarrow \star$, The
type constructor for List will have a kind $\star \rightarrow \star$ while
types like Int and Float will have a kind $\star$.

\begin{figure}
  \begin{framed}
    \begin{prooftree}
      \AxiomC{$\tau :: \kappa' \rightarrow \kappa$}
      \AxiomC{$\tau' :: \kappa'$}
      \BinaryInfC{$\tau \tau' :: \kappa$}
    \end{prooftree}
  \end{framed}
  \caption{Constructor Application Rule}
  \label{fig:constructor-application}
\end{figure}

The unification of types is now done via a modified version of Robinson's algorithm \citeyearpar{robinson_machine-oriented_1965}
is used in order to deduce the most general unifier for type constructors.
Formally we define $S$ to be the {\it most general unifier} for type constructors $T$ and $T'$ if:
\begin{enumerate}
  \item $S$ is a unifier for $T$ and $T'$.
  \item For every unifier $S'$ of $T$ and $T'$ we can write $T'$ in a form of
    $R S$ where $R$ some kind preserving substitution.
\end{enumerate}
We write $T \overset{S}{\sim}_{\kappa} T'$ for assertion that $S$ is the unifier
of the constructor types $T, T' \in T^{\kappa}$. The rules in \cref{fig:unify-types}
describe the unification algorithm for type constructors. [KVar] and [KVar'] contain
and additional constraint of the type variable $t$ to not be free in the type constructor's $T$
type variables to ensure the unification does not lead to infinite types. The [KApply] rule
states that type constructors of the form $T T'$ can be unified with $H H'$ only if $T$ and $H$
can be unified which asserts that they have to have the same kind $\kappa' \rightarrow \kappa$.

\begin{figure}[h]
  \begin{framed}\centering
    \begin{minipage}[l]{0.5\linewidth}
      \begin{flalign*}
        t \overset{id}{\sim}_{\kappa} t \tag{[ID-KVar]}
      \end{flalign*}
    \end{minipage}
    \begin{minipage}[r]{0.5\linewidth}
      \begin{flalign*}
        T \overset{id}{\sim}_{\kappa} T \tag{[ID-KConst]}
      \end{flalign*}
    \end{minipage}
    \begin{minipage}[l]{0.5\linewidth}
      \begin{flalign*}
        t \overset{[T/t]}{\sim}_{\kappa} T, t \notin \texttt{fvs}(T) \tag{[KVar]}
      \end{flalign*}
    \end{minipage}
    \begin{minipage}[r]{0.5\linewidth}
      \begin{flalign*}
        T \overset{[T/t]}{\sim}_{\kappa} t, t \notin \texttt{fvs}(T) \tag{[KVar']}
      \end{flalign*}
    \end{minipage}
    \begin{minipage}[c]{1.0\linewidth}
      \begin{prooftree}
        \AxiomC{$T \overset{S}{\sim}_{\kappa'\rightarrow \kappa} T'$}
        \AxiomC{$S H \overset{S'}{\sim}_{\kappa'} S H'$}\RightLabel{([KApply])}
        \BinaryInfC{$T T' \overset{SS'}{\sim}_{\kappa} H H'$}
      \end{prooftree}
    \end{minipage}
  \end{framed}
  \caption{Kind Preserving Unification of Type Constructors}
  \label{fig:unify-types}
\end{figure}

\begin{figure}[h]
  \begin{framed}
    \begin{flalign*}
      \text{Term Variables}\ \ \  x, y, z  &\in \text{Var}\\
      \text{Expressions}\ \ \     M, N     &::= x \mid \lambda^{\sepimp} x. M \mid \lambda^{\shimp}x. M \mid M N \mid \Let{x}{M}{N}\\
      &\mid \Pair{M,N} \mid \Let{\Pair{x,y}}{M}{N} \mid \Pair{M;N} \mid \Fst{M} \mid \Snd{M}\\
      &\mid \Case{M}{\{\texttt{inl}\ x \mapsto N ; \texttt{inr}\ y \mapsto N'\}}\mid \Inl{x} \mid \Inr{y}\\
    \end{flalign*}
  \end{framed}
  \caption{Extended \qub{} Language Syntax}
  \label{fig:ext-qub-terms}
\end{figure}

\section{Pairs and Sums in  \qub{}}\label{sec:pairs-types}
Introduction of two kinds of arrows in our type system leads to different flavors of pairs. This distinction cannot be made
in intuitionistic logic as the structural rules allow re-use of propositions. But due to restrictions in weakening and
contraction we obtain two kinds of pairs, additive and multiplicative. In this section we illustrate how the
extented \qub{} can be used to introduce new types. We introduce syntax and type constructors for multiplicative pairs in \cref{subsec:mul-pairs}
and the same for additive pairs in \cref{subsec:add-pairs}. We then introduce the
syntax and type constructors for sum types in \cref{subsec:sums} and illustrate that they indeed work as expected.

\subsection{Multiplicative Pair Type}\label{subsec:mul-pairs}
Lambda encoding of muliplicative pairs is given in \cref{fig:mul-pair}. The typing rules
are given in \cref{fig:mul-pair-rules}. We give the proofs of the typing rules in \cref{subsec:mul-pairs-deriv}.
The meaning of a multiplicative pair can be thought of as
having separate resouce entities together in the program environment context and they would have to be
explicitly disposed off. Failure to do so, would raise a type error regarding the resources
not being unrestricted.

\begin{figure}[h]
  \begin{framed}
    \begin{flalign*}
      \otimes &\in \mathcal{T}^{\star \rightarrow \star \rightarrow \star}\\
      \tau \otimes \tau' &= \tau \sepimp \tau' \sepimp (\tau \sepimp \tau' \sepimp \upsilon) \sepimp \upsilon\\
      (,) &= \lambda^{\sepimp}  x. \lambda^{\sepimp}  y. \lambda^{\sepimp}  f. f x y
    \end{flalign*}
  \end{framed}
\caption{Multiplicative Pair}
\label{fig:mul-pair}
\end{figure}
\begin{figure}[h]
  \begin{framed}
    \begin{minipage}{1\linewidth}
      \begin{prooftree}
        \AxiomC{$P \mid \Gamma  \vdash M : \tau$}
        \AxiomC{$P \mid \Delta \vdash N: \tau'$} \RightLabel{$[\otimes I]$}
        \BinaryInfC{$P \mid \Gamma \circledast \Delta \vdash \Pair{M,N}: \tau \otimes \tau'$}
      \end{prooftree}
    \end{minipage}
    \begin{minipage}{1\linewidth}
      \begin{prooftree}
        \AxiomC{$P \mid \Gamma \vdash M: \tau \otimes \tau'$}
        \AxiomC{$P \mid \{ x^{\{\vec{z} \mid \vec{z} \subseteq \texttt{Vars}(\Gamma')\}}:\tau \} \circledast
          \{ y^{\{\bar{z'} \mid \bar{z'} \subseteq \texttt{Vars}(\Gamma')\}} : \tau'\} \sqcup \Gamma'_{x,y} \vdash N:\upsilon$}\RightLabel{$[\otimes E]$}
        \BinaryInfC{$P \mid \Gamma \sqcup \Gamma' \vdash (\Let{\Pair{x, y}}{M}{N}): \upsilon$}
      \end{prooftree}
    \end{minipage}
  \end{framed}
  \caption{Derivable Typing Rules for Multiplicative Pair}
  \label{fig:mul-pair-rules}
\end{figure}

\subsection{Additive Pair Type}\label{subsec:add-pairs}
The lambda encoding for additive pairs is given in \cref{fig:add-pair}. The typing rules are given in \cref{fig:add-pair-rules}. We give
proof of derivations for the rules in \cref{subsec:add-pairs-deriv}. Additive pairs can be thought of as program
entities that have sharing resources. We use \Fst{} and \Snd{} as projection functions on the additive pair to obtain the
constituent objects. It is worthwhile to note that we do not provide these projection functions for multiplicative pairs
so as to restrict the programmer from implicitly discarding one of the components of the multipicative pair.

\begin{figure}[h]
  \centering
  \begin{framed}
    \begin{flalign*}
      \with &\in \mathcal{T}^{\star \rightarrow \star \rightarrow \star}\\
      \tau \with \tau' &= \tau \sepimp \tau' \rightarrow (\tau \sepimp \tau' \rightarrow \upsilon) \rightarrow \upsilon\\
      (;) &= \lambda^{\sepimp}  x. \lambda^{\shimp} by. \lambda^{\shimp} f. f x y
    \end{flalign*}
  \end{framed}
  \caption{Additive Pair}
  \label{fig:add-pair}
\end{figure}
\begin{figure}[h]
  \begin{framed}
    \begin{minipage}{1\linewidth}
      \begin{prooftree}
        \AxiomC{$P \mid \Gamma  \vdash M : \tau$}
        \AxiomC{$P \mid \Delta \vdash N: \tau'$} \RightLabel{[$\with$ I]}
        \BinaryInfC{$P \mid \Gamma \varoplus \Delta \vdash \Pair{M;N}: \tau \with \tau'$}
      \end{prooftree}
    \end{minipage}
    \begin{minipage}{.5\linewidth}
      \begin{prooftree}
        \AxiomC{$P \mid \Gamma \vdash M: \tau \with \tau'$} \RightLabel{[$\with$ E$_1$]}
        \UnaryInfC{$P \mid \Gamma \vdash \texttt{fst}\ M: \tau$}
      \end{prooftree}
    \end{minipage}
    \begin{minipage}{.5\linewidth}
      \begin{prooftree}
        \AxiomC{$P \mid \Gamma \vdash M: \tau \with \tau'$} \RightLabel{[$\with$ E$_2$]}
        \UnaryInfC{$P \mid \Gamma \vdash \texttt{snd}\ M: \tau'$}
      \end{prooftree}
    \end{minipage}
  \end{framed}
  \caption{Derivable Typing Rules for Additive Pair}
  \label{fig:add-pair-rules}
\end{figure}


\subsection{Sum Type}\label{subsec:sums}
The lambda encoding of sum types is given in \cref{fig:sum-types} and the derivable typing rules are given in \cref{fig:bi-sum-types}.
We give proof of derivations in \cref{sec:sums-deriv}. Sum types can be thought of as a choice between two types. At a given point of
time only one of the two types would exist. We provide two functions to be used has handlers for both the types. The $\Case{M}{\{f;g\}}$ construct
is the deconstructor that would decide which function to use $f$ or $g$ depending on the value evaluated for $M$. The return type
of both the functions $f$ and $g$ would have to be the same for them to have correct typing. Encoding sum types within the language would be
helpful in defining data structures such as lists, trees and option types.

\begin{figure}[h]
  \begin{framed}\centering
    \begin{flalign*}
      \oplus &\in \mathcal{T}^{\star \rightarrow \star \rightarrow \star}\\
      \tau \oplus \tau' &= (\tau \rightarrow \upsilon) \rightarrow (\tau' \rightarrow \upsilon) \rightarrow \upsilon\\
      \Inl{} &: \tau \sepimp (\tau \oplus \tau')\\
      \Inl{} &= \lambda^{\sepimp} x. \lambda^{\shimp} f. \lambda^{\shimp} g. f x\\
      \Inr{} &: \tau' \sepimp (\tau \oplus \tau')\\
      \Inr{} &= \lambda^{\sepimp} y. \lambda^{\shimp} f. \lambda^{\shimp} g. g y
    \end{flalign*}
  \end{framed}
  \caption{Sum Type}
  \label{fig:sum-types}
\end{figure}

\begin{figure}[h]
\begin{framed}
% sum type I_l
\begin{minipage}{0.5\textwidth}
  \begin{prooftree}
    \AxiomC{$P \mid \Gamma \vdash M: \tau$}\RightLabel{[$\oplus$I$_l$]}
    \UnaryInfC{$P \mid \Delta \vdash \texttt{inl}\ M: \tau \oplus \tau'$}
  \end{prooftree}
\end{minipage}
% sum type I_r
\begin{minipage}{0.5\textwidth}
  \begin{prooftree}
    \AxiomC{$P \mid \Gamma \vdash M: \tau'$}\RightLabel{[$\oplus$I$_r$]}
    \UnaryInfC{$P \mid \Delta \vdash \texttt{inr}\ M: \tau \oplus \tau'$}
  \end{prooftree}
\end{minipage}
% sum type E
\begin{minipage}{1\textwidth}
  \begin{prooftree}
    \AxiomC{$P \mid \Gamma \vdash M: \tau \oplus \tau'$}
    \AxiomC{$P \mid \Gamma \circledast x:\tau \vdash N :\upsilon$}
    \AxiomC{$P \mid \Gamma \circledast y:\tau' \vdash N':\upsilon$}\RightLabel{[$\oplus$E]}
    \TrinaryInfC{$P \mid \Gamma \vdash \Case{M}{\{\texttt{inl}\ x \mapsto N; \texttt{inr}\ y \mapsto N'\}} : \upsilon $}
  \end{prooftree}
\end{minipage}
\end{framed}
\caption{Derivable Typing Rules for Sum Type}
\label{fig:bi-sum-types}
\end{figure}


\section{Generic Type Constructors}\label{sec:type-constructors}

To leverage the full power of introducing the sharing concept in a programming language,
we would want to have users be able to define datatypes with sharing and separation of its
constituent resources.  We add two new typing rules as shown in \cref{fig:ud-datatypes}
for handling generic datatypes using type constructors.
In the current implimentation, we only consider a specific case of type constructors where all the type variables
involved are either all shared or all separate. % and give an insight of how this
% can be generalized to have a mix of sharing and separation between the type parameters
% for a type constructor.

\TODO{How do we write about type constructors?}
\begin{figure}[h]
  \begin{framed}\centering
    % sharing constructor
    \begin{minipage}{0.45\textwidth}
      \begin{prooftree}
        \AxiomC{$P \mid \Gamma \varoplus x:\tau^{\kappa} \vdash C:\tau^{\kappa} \rightarrow T$}\RightLabel{$[\text{C-sh}]$}
        \UnaryInfC{$P \mid \Gamma \vdash C x : T $}
      \end{prooftree}
    \end{minipage}
    % Separating constructor
    \begin{minipage}{0.45\textwidth}
      \begin{prooftree}
        \AxiomC{$P \mid \Gamma \circledast x:\tau^{\kappa} \vdash C:\tau^{\kappa} \sepimp T$}\RightLabel{$[\text{C-se}]$}
        \UnaryInfC{$P \mid \Gamma \vdash C x : T $}
      \end{prooftree}
    \end{minipage}
  \end{framed}
  \caption{User Defined Datatypes}
  \label{fig:ud-datatypes}
\end{figure}

We illustrate using examples how the type system works.
The sharing pair has shared resources and using \Fst{} and \Snd{} works
to get both the individual constituents as shown in \cref{fig:sharing-pair}
The (!!) denots that a and b are in sharing. The fstp and sndp functions
typecheck as expected.

\begin{figure}[h]
  \begin{framed}
\begin{minted}{haskell}
data Pair' a b = ShP !! a b

fstp::Pair' a b -> a
fstp (ShP x y) = x

sndp :: Pair' a b -> b
sndp (ShP x y) = y
\end{minted}
  \end{framed}

  \caption{Shared Pairs}
  \label{fig:sharing-pair}
\end{figure}

Separating pair is defined in \cref{fig:sep-pair}. The functions fst and snd do not type check
as on of the resources is being dropped and we would need the resource to be unrestricted.

\begin{figure}[h]
  \begin{framed}
\begin{minted}{haskell}
data Pair a b = SeP a b

fst :: Pair' a b -> a -- Does not type check
fst (SeP x y) = x

snd :: Pair' a b -> b -- Does not type check
snd (SeP x y) = y

swap :: Pair a b -> Pair b a -- Type checks
swap SeP x y = SeP y x
\end{minted}
  \end{framed}
  \caption{Separating Pair}
  \label{fig:sep-pair}
\end{figure}

%%% Local Variables:
%%% mode: latex
%%% TeX-master: "../thesis-ku.tex"
%%% End:
      % kinds with user defined datatypes. sharing pair, separating pair.
\chapter{Conclusion and Future Work}\label{chp:conclusion}

We have explored type system of \qub{}, based on sub-structural logic of \BI{}.
It offers a nice conglomeration of \textbf{HM} type system and intuitionistic linear type system and is more
expressive than existing type system in terms of statically tracking resources.


For future work we have quite a few areas to explore and which need deeper understanding.
The \qub{} type inference is incomplete with respect to the fact that terms can have two types depending
on the predicates and the context of the function. We would want to have a clear way of solving
the dilemma and be able to detect it automatically in the type inference algorithm.
We have not given any formal semantic models for our language to rigorously prove the soundness with respect to original
system of logic of \BI{} and neither have we tried to analyze the system with respect to categorical models.
It will be worth while to pursue them from a theoretical perspective.         % What next?

\global\long\def\bibname{References}

\bibliographystyle{apalike}
\bibliography{Biblio/allcites}

\appendix
\chapter{Derivations for Products and Sums}
We start this section by adding few auxilary defitions for terms and types from first principles.
By convention, we denote an empty typing context by $I$ and empty predicate context with $\emptyset$

\section{Derivable Typing Rules For Product Types (Addiditive and Multiplicative Pairs)}\label{sec:pairs}
Pairs can now have two meanings. Either they have sharing resources or they have separating resources. We define each
of them below.

\subsection{Additive Pairs}
If the resources are in sharing we say they are additive pairs.

\begin{minipage}[h]{1.0\linewidth}
  \begin{prooftree}
    \AxiomC{$$}\RightLabel{[ID]}
    \UnaryInfC{$\emptyset \mid y^{xf}:B \vdash y: B $}

    \AxiomC{$$}
    \UnaryInfC{$\emptyset \mid x^{fy}: A \vdash x: A$}\RightLabel{[ID]}

    \AxiomC{$$}\RightLabel{[ID]}
    \UnaryInfC{$\emptyset \mid f^{xy}: A \rightarrow B \rightarrow C \vdash f: A \rightarrow B \rightarrow C $}\RightLabel{[$\sepimp E$]}
    \BinaryInfC{$\emptyset \mid x^{xy}:A \varoplus f^{xy}: A \rightarrow B \rightarrow C \vdash f x: (B \rightarrow C)$}\RightLabel{[$\rightarrow E$]}

    \BinaryInfC{$\emptyset \mid y^{xf}:B \varoplus x^{yf}:A \varoplus f^{xy}:A \rightarrow B \rightarrow C \vdash f x y: C$}\RightLabel{[$EXCH$]}
    \UnaryInfC{$\emptyset \mid x^{yf}:A \varoplus y^{xf}:B \varoplus f^{xy}: A \rightarrow B \rightarrow C \vdash f x y: C$}\RightLabel{[$\rightarrow I$]}
    \UnaryInfC{$\emptyset \mid x^{y}:A \varoplus y^{x}:B \vdash \lambda^{\alpha}f. f x y: (A \rightarrow B \rightarrow C) \rightarrow C$}\RightLabel{[$\rightarrow I$]}
    \UnaryInfC{$\emptyset \mid x^{\emptyset}:A \vdash \lambda^{\alpha}y. \lambda^{\alpha}f. f x y: B \rightarrow (A \rightarrow B \rightarrow C) \rightarrow C$}\RightLabel{[$\sepimp$ I]}
    \UnaryInfC{$\emptyset \mid I \vdash \lambda^{*}x. \lambda^{\alpha}y. \lambda^{\alpha}f. f x y: A \sepimp B \rightarrow (A \rightarrow B \rightarrow C) \rightarrow C$}
  \end{prooftree}
\end{minipage}
\noindent
We can now define sharing (additive) pair as:
\begin{framed}\centering
    $; = \lambda^{*}x. \lambda^{\alpha}y. \lambda^{\alpha}f. f x y$
\end{framed}
and assign a new type to this pair and call it $\with$
\begin{framed}\centering
  $A \with B = (A \rightarrow B \rightarrow C) \rightarrow C$
\end{framed}

\noindent
We now define left and right projections or deconstructors for sharing pairs below:
\begin{minipage}[h]{1.0\linewidth}
  \begin{prooftree}
    \AxiomC{$$}\RightLabel{[ID]}
    \UnaryInfC{$\emptyset \mid x^{y}:A \vdash x: A $}\RightLabel{[WKN-SH]}
    \UnaryInfC{$\emptyset \mid x^{y}:A \varoplus y^{x}:B \vdash x: A $}\RightLabel{[$\rightarrow I$]}
    \UnaryInfC{$\emptyset \mid x^{\emptyset}:A \vdash \lambda^{\alpha}y. x : B \rightarrow A $}\RightLabel{[$\rightarrow$I]}
    \UnaryInfC{$\emptyset \mid I \vdash \lambda^{\alpha}x. \lambda^{\alpha}y. x: A \rightarrow B \rightarrow A$}
  \end{prooftree}
\end{minipage}
\begin{framed}
  \centering
  $\texttt{fst}_{sh} = \lambda^{\alpha}x. \lambda^{\alpha}y. x$
\end{framed}

\begin{minipage}[h]{1.0\linewidth}
  \begin{prooftree}
    \AxiomC{$$}\RightLabel{[ID]}
    \UnaryInfC{$\emptyset \mid y^{x}:B \vdash y: B $}\RightLabel{[WKN-SH]}
    \UnaryInfC{$\emptyset \mid x^{y}:A \varoplus y^{x}:B \vdash y: B $}\RightLabel{[$\rightarrow I$]}
    \UnaryInfC{$\emptyset \mid x^{\emptyset}:A \vdash \lambda^{\alpha}y: B \rightarrow B $}\RightLabel{[$\rightarrow I$]}
    \UnaryInfC{$\emptyset \mid I \vdash \lambda^{\alpha}x. \lambda^{\alpha}y. y: A \rightarrow B \rightarrow B$}
  \end{prooftree}
\end{minipage}

\begin{framed}\centering
    $\texttt{snd}_{sh} = \lambda^{\alpha}x. \lambda^{\alpha}y. y$
\end{framed}

\subsection{Multiplicative Pairs}
If the resources are separate we say they are multiplicative pairs.

\begin{minipage}[h]{1.0\linewidth}
  \begin{prooftree}
    \AxiomC{$$}\RightLabel{[ID]}
    \UnaryInfC{$\emptyset \mid y^{\emptyset}:B \vdash y: B $}

    \AxiomC{$$}\RightLabel{[ID]}
    \UnaryInfC{$\emptyset \mid x^{\emptyset}: A \vdash x: A$}

    \AxiomC{$$}\RightLabel{[ID]}
    \UnaryInfC{$\emptyset \mid f^{\emptyset}: A \sepimp B \sepimp C \vdash f: A \sepimp B \sepimp C $}\RightLabel{[$\sepimp E$]}
    \BinaryInfC{$\emptyset \mid x^{\emptyset}:A \circledast f^{\emptyset}: A \sepimp B \sepimp C \vdash f x: (B \sepimp C)$}\RightLabel{[$\sepimp E$]}

    \BinaryInfC{$\emptyset \mid y^{\emptyset}:B \circledast x^{\emptyset}:A \circledast f^{\emptyset}:A \sepimp B \sepimp C \vdash f x y: C$}\RightLabel{[$EXCH$]}
    \UnaryInfC{$\emptyset \mid x^{\emptyset}:A \circledast y^{\emptyset}:B \circledast f^{\emptyset}:A \sepimp B \sepimp C \vdash f x y: C$}\RightLabel{[$\sepimp I$]}
    \UnaryInfC{$\emptyset \mid x^{\emptyset}:A \circledast y^{\emptyset}:B \vdash \lambda^{*}f. f x y: (A \sepimp B \sepimp C) \sepimp C$}\RightLabel{[$\sepimp I$]}
    \UnaryInfC{$\emptyset \mid x^{\emptyset}:A \vdash \lambda^{*}y. \lambda^{*}f. f x y: B \sepimp (A \sepimp B \sepimp C) \sepimp C$}\RightLabel{[$\equiv$]}
    \UnaryInfC{$\emptyset \mid I \vdash \lambda^{*}x. \lambda^{*}y. \lambda^{*}f. f x y: A \sepimp B \sepimp (A \sepimp B \sepimp C) \sepimp C$}
  \end{prooftree}
\end{minipage}

We can now define separating (multiplicative) pair as:
\begin{framed}\centering
    $,= \lambda^{*}x. \lambda^{*}y. \lambda^{*}f. f x y$
\end{framed}
We assign a new type to this pair construct and call it $\otimes$
\begin{framed}\centering
  $A \otimes B = (A \sepimp B \sepimp C) \sepimp C$
\end{framed}

\begin{minipage}[h]{1.0\linewidth}
  \begin{prooftree}
    \AxiomC{$$}\RightLabel{[ID]}
    \UnaryInfC{$\Un{B} \mid x^{\emptyset}:A \vdash x:A $}

    \AxiomC{$$}
    \UnaryInfC{$\Un{B} => \Un{B}$}\RightLabel{[UN-$\Gamma$, UN-$\sigma$, UN-$\tau$]}
    \UnaryInfC{$\Un{B} \vdash y^{\emptyset}:B\ \texttt{un} $}\RightLabel{[WKN-UN]}

    \BinaryInfC{$\Un{B} \mid x^{\emptyset}:A \circledast y^{\emptyset}:B \vdash x: A $}\RightLabel{[$\sepimp I$]}
    \UnaryInfC{$\Un{B} \mid x^{\emptyset}:A \vdash \lambda^{*}y. x: B \sepimp A $}\RightLabel{[$\sepimp I$]}
    \UnaryInfC{$\Un{B} \mid I \vdash \lambda^{*}x. \lambda^{*}y. x: A \sepimp B \sepimp A$}
  \end{prooftree}
\end{minipage}

\begin{framed}\centering
    $\texttt{fst}_{sep} = \lambda^{*}x. \lambda^{*}y. x$
\end{framed}

\begin{minipage}[h]{1.0\linewidth}
  \begin{prooftree}
    \AxiomC{$$}\RightLabel{[ID]}
    \UnaryInfC{$\Un{A} \mid y^{\emptyset}:B \vdash y: B $}

    \AxiomC{$$}
    \UnaryInfC{$\Un{A} => \Un{A}$}\RightLabel{[UN-$\Gamma$, UN-$\sigma$, UN-$\tau$]}
    \UnaryInfC{$\Un{A} \vdash x^{\emptyset}:A\ \texttt{un} $}\RightLabel{[WKN-UN]}

    \BinaryInfC{$\Un{A} \mid x^{\emptyset}:A \circledast y^{\emptyset}:B \vdash y: B $}\RightLabel{[$\sepimp I$]}
    \UnaryInfC{$\Un{A} \mid x^{\emptyset}:A \vdash \lambda^{*}y.y: B \sepimp B $}\RightLabel{[$\sepimp I$]}
    \UnaryInfC{$\Un{A} \mid I \vdash \lambda^{*}x. \lambda^{*}y. y: A \sepimp B \sepimp B$}
  \end{prooftree}
\end{minipage}

\begin{framed}\centering
    $\texttt{snd}_{sep} = \lambda^{*}x. \lambda^{*}y. y$
\end{framed}

We will abuse the notation of lambda calculus for $;$ and $,$ as use them as infix operators for syntactic convinence
\begin{flalign*}
  \langle x , y \rangle \equiv (,) x y\\
  \langle x ; y \rangle \equiv (;) x y
\end{flalign*}

We are now in a position to write the proof derivations for \cref{fig:pair-typing-rules} using
the auxilary definitions from above.
\begin{framed}
  \noindent
  \begin{minipage}{0.5\linewidth}
    \begin{prooftree}
      \AxiomC{$\emptyset \mid \texttt{fst}_{sh}\ : A \rightarrow B \rightarrow A \vdash M: A \with B$}\RightLabel{$\with E_1$}
      \UnaryInfC{$\emptyset \mid I \vdash \texttt{fst}_{sh}\  M : A$}
    \end{prooftree}
  \end{minipage}
  \begin{minipage}{0.5\linewidth}
    \begin{prooftree}
      \AxiomC{$\emptyset \mid \texttt{snd}_{sh}\ : A \rightarrow B \rightarrow B \vdash M: A \with B $}\RightLabel{$\with E_2$}
      \UnaryInfC{$\emptyset \mid I \vdash \texttt{snd}_{sh}\  M : B$}
    \end{prooftree}
  \end{minipage}
  \noindent
  \begin{minipage}{1\linewidth}
    \begin{prooftree}
      \AxiomC{$P \mid \emptyset \vdash (;) : \with $}
      \AxiomC{$P \mid \Gamma  \vdash M : A$}
      \AxiomC{$P \mid \Delta \vdash N: B $} \RightLabel{$[\with I]$}
      \TrinaryInfC{$P \mid \Gamma \varoplus \Delta \vdash \langle M ; N \rangle: A \with B$}
    \end{prooftree}
  \end{minipage}
  \noindent
  \begin{minipage}{0.5\linewidth}
    \begin{prooftree}
      \AxiomC{$\Un{B} \mid \texttt{fst}_{sep}\ : A \sepimp B \sepimp A \vdash  M: A \otimes B $}\RightLabel{$\otimes E_1$}
      \UnaryInfC{$\Un{B} \mid I \vdash \texttt{fst}_{sep}\  M : A$}
    \end{prooftree}
  \end{minipage}
  \begin{minipage}{0.5\linewidth}
    \begin{prooftree}
      \AxiomC{$\Un{A} \mid \texttt{snd}_{sep}\ : A \sepimp B \sepimp B \vdash M: A \otimes B $}\RightLabel{$\otimes E_2$}
      \UnaryInfC{$\Un{A} \mid I \vdash \texttt{snd}_{sep}\  M : B$}
    \end{prooftree}
  \end{minipage}
  \noindent
  \begin{minipage}{1\linewidth}
    \begin{prooftree}
      \AxiomC{$P \mid \emptyset \vdash (,) : \otimes $}
      \AxiomC{$P \mid \Gamma  \vdash M : A$}
      \AxiomC{$P \mid \Delta \vdash N: B $} \RightLabel{$[\otimes I]$}
      \TrinaryInfC{$P \mid \Gamma \circledast \Delta \vdash \langle M , N \rangle: A \otimes B$}
    \end{prooftree}
  \end{minipage}
\end{framed}

\section{Derivable Typing rules for Sum Types}\label{sec:sums}
Sum types can hold only one of the enclosing types.
In our case they there can be two kinds of typing for sums.
% \subsection{Muliplicative Sums}
% If the two resources are separate, we call them multipicative sums.
% \begin{landscape}
%   % choice
%   \noindent
%   \begin{small}
%   \begin{prooftree}
%     \AxiomC{$$}\RightLabel{[ID]}
%     \UnaryInfC{$\emptyset \mid  g^{\emptyset}:(B \sepimp E) \vdash g: (B \sepimp E)$}

%     \AxiomC{$$}\RightLabel{[ID]}
%     \UnaryInfC{$\emptyset \mid f^{\emptyset}: (A \sepimp E) \vdash f: (A \sepimp E)$}
%     \AxiomC{$$}\RightLabel{[ID]}
%     \UnaryInfC{$\emptyset \mid c^{\emptyset}: ((A \sepimp E) \sepimp (B \sepimp E) \sepimp E)
%       \vdash c:((A \sepimp E) \sepimp (B \sepimp E) \sepimp E)$}\RightLabel{[$\sepimp$E]}
%     \BinaryInfC{$\emptyset \mid c^{\emptyset}:((A \sepimp E) \sepimp (B \sepimp E) \sepimp E)
%       \circledast f^{\emptyset}:(A \sepimp E) \vdash c f: (B \sepimp E) \sepimp E$}\RightLabel{[$\sepimp$E]}

%     \BinaryInfC{$\emptyset \mid  c^{\emptyset}:((A \sepimp E) \sepimp (B \sepimp E) \sepimp E)  \circledast
%       f^{\emptyset}:(A \sepimp E)  \circledast g^{\emptyset}:(B \sepimp E) \vdash c f g : E$}\RightLabel{[$\sepimp$I]}
%     \UnaryInfC{$\emptyset \mid  c^{\emptyset}:((A \sepimp E) \sepimp (B \sepimp E) \sepimp E) \circledast
%       f^{\emptyset}:(A \sepimp E) \vdash \lambda^{*} g. c f g :(B \sepimp E) \sepimp E$}\RightLabel{[$\sepimp$I]}
%     \UnaryInfC{$\emptyset \mid  c^{\emptyset}:((A \sepimp E) \sepimp (B \sepimp E) \sepimp E) \vdash  \lambda^{*} f. \lambda^{*} g. c f g :(A \sepimp E) \sepimp (B \sepimp E) \sepimp E$}\RightLabel{[$\sepimp$I]}\RightLabel{[$\sepimp$I]}
%     \UnaryInfC{$\emptyset \mid I \vdash \lambda^{*} c. \lambda^{*} f. \lambda^{*} g. c f g : ((A \sepimp E) \sepimp (B \sepimp E) \sepimp E) \sepimp (A \sepimp E) \sepimp (B \sepimp E) \sepimp E$}
%   \end{prooftree}
% \end{small}

%   \noindent
%   We now define sum type to be
%   \begin{framed}\centering
%     $A \oplus B = (A \sepimp E) \sepimp (B \sepimp E) \sepimp E$\\
%     $\CaseSe{c}{\{ f;g \}} = \lambda^{*}c. \lambda^{*}f. \lambda^{*}g. c f g$
%   \end{framed}

%   We now define left and right {\it injections} or constructors for the sum type.
%   % left
%   \noindent
%   \begin{prooftree}
%     \AxiomC{$$}\RightLabel{[ID]}
%     \UnaryInfC{$\Un{A \sepimp E} \mid y: B  \vdash y : B$}

%     \AxiomC{$$}\RightLabel{}
%     \UnaryInfC{$\Un{A \sepimp E} \vdash \Un{A \sepimp E}$}\RightLabel{[WKN-UN]}
%     \BinaryInfC{$\Un{A \sepimp E} \mid f: (A \sepimp E), y: B  \vdash y : B$}

%     \AxiomC{$$}\RightLabel{[ID]}
%     \UnaryInfC{$\Un{A \sepimp E} \mid g: (B \sepimp E) \vdash g: (B \sepimp E)$}\RightLabel{[$\sepimp$I]}

%     \BinaryInfC{$\Un{A \sepimp E} \mid f: (A \sepimp E), y: B, g: (B \sepimp E) \vdash g y : E$}\RightLabel{[EXCH]}
%     \UnaryInfC{$\Un{A \sepimp E} \mid y: B, f: (A \sepimp E), g: (B \sepimp E) \vdash g y : E$}\RightLabel{[$\sepimp$I]}
%     \UnaryInfC{$\Un{A \sepimp E} \mid y: B, f: (A \sepimp E) \vdash \lambda^{*} g. g y : (B \sepimp E) \sepimp E$}\RightLabel{[$\sepimp$I]}
%     \UnaryInfC{$\Un{A \sepimp E} \mid y: B \vdash \lambda^{*} y. \lambda^{*} f. \lambda^{*} g. g y :(A \sepimp E) \sepimp (B \sepimp E) \sepimp E$}\RightLabel{[$\equiv$]}
%     \UnaryInfC{$\Un{A \sepimp E} \mid  I, y: B \vdash \lambda^{*} y. \lambda^{*} f. \lambda^{*} g. g y :(A \sepimp E) \sepimp (B \sepimp E) \sepimp E$}\RightLabel{[$\sepimp$I]}\RightLabel{[$\sepimp$I]}
%     \UnaryInfC{$\Un{A \sepimp E} \mid I \vdash \lambda^{*} y. \lambda^{*} f. \lambda^{*} g. g y: B \sepimp  (A \sepimp E) \sepimp (B \sepimp E) \sepimp E$}
%   \end{prooftree}
%   \noindent Left injection defined below as:
%   \begin{framed}\centering
%     $inl : A \sepimp A \oplus B$\\
%     $\texttt{inl} = \lambda x. \lambda f. \lambda g. f x$
%   \end{framed}

%   % right
%   \begin{prooftree}
%     \AxiomC{$$}\RightLabel{[ID]}
%     \UnaryInfC{$\Un{B \sepimp E} \mid x: A  \vdash x : B$}

%     \AxiomC{$$}\RightLabel{}
%     \UnaryInfC{$\Un{B \sepimp E} \vdash \Un{B \sepimp E}$}\RightLabel{[WKN-UN]}
%     \BinaryInfC{$\Un{B \sepimp E} \mid g: (B \sepimp E), x: A  \vdash x : A$}

%     \AxiomC{$$}\RightLabel{[ID]}
%     \UnaryInfC{$\Un{B \sepimp E} \mid f: (A \sepimp E) \vdash f: (A \sepimp E)$}\RightLabel{[$\sepimp$E]}

%     \BinaryInfC{$\Un{B \sepimp E} \mid x: A, f: (A \sepimp E), g: (B \sepimp E) \vdash f x : E$}\RightLabel{[$\sepimp$I]}
%     \UnaryInfC{$\Un{B \sepimp E} \mid x: A, f: (A \sepimp E) \vdash \lambda^{*} g. f x : (B \sepimp E) \sepimp E$}\RightLabel{[$\sepimp$I]}
%     \UnaryInfC{$\Un{B \sepimp E} \mid x: A \vdash  \lambda^{*} f. \lambda^{*} g. f x :(A \sepimp E) \sepimp (B \sepimp E) \sepimp E$}\RightLabel{[$\equiv$]}
%     \UnaryInfC{$\Un{B \sepimp E} \mid  I, x: A \vdash  \lambda^{*} f. \lambda^{*} g. f x :(A \sepimp E) \sepimp (B \sepimp E) \sepimp E$}\RightLabel{[$\sepimp$I]}\RightLabel{[$\sepimp$I]}
%     \UnaryInfC{$\Un{B \sepimp E} \mid I \vdash \lambda^{*} x. \lambda^{*} f. \lambda^{*} g. f x: A \sepimp  (A \sepimp E) \sepimp (B \sepimp E) \sepimp E$}
%   \end{prooftree}

%   \noindent \noindent Right injection defined below as:
%   \begin{framed}\centering
%     $inr : B \sepimp A \oplus B$\\
%     $\texttt{inr} = \lambda y. \lambda f. \lambda g. g y$
%   \end{framed}
% \end{landscape}

% We can now add sum types in our language using the auxilary definitions given above:
% \begin{framed}
%   \begin{minipage}[h]{0.5\linewidth}
%     \begin{prooftree}
%       \AxiomC{$\emptyset \mid \texttt{inr}: A \sepimp A \oplus B \vdash x : A$}\RightLabel{[$\oplus$I$_1$]}
%       \UnaryInfC{$\emptyset \mid I \vdash \texttt{inl}\ x: A \oplus B$}
%     \end{prooftree}
%   \end{minipage}
%   \begin{minipage}[h]{0.5\linewidth}
%     \begin{prooftree}
%       \AxiomC{$\emptyset \mid \texttt{inl}: B \sepimp A \oplus B \vdash y : B$}\RightLabel{[$\oplus$I$_2$]}
%       \UnaryInfC{$\emptyset \mid I \vdash \texttt{inl}\ y: A \oplus B$}
%     \end{prooftree}
%   \end{minipage}
%   \begin{minipage}[h]{1.0\linewidth}
%     \begin{prooftree}
%       \AxiomC{$\emptyset \mid \Gamma \vdash M : A \oplus B$}\RightLabel{[$\oplus$E]}
%       \AxiomC{$\emptyset \mid \Gamma \circledast x:A \vdash N_1 : E$}
%       \AxiomC{$\emptyset \mid \Gamma \circledast y:B \vdash N_2 : E$}
%       \TrinaryInfC{$\emptyset \mid \Gamma \vdash\Case{M}{\{ \texttt{inl}\ x \mapsto N_1; \texttt{inl}\ y \mapsto N_2\}}:E$}
%     \end{prooftree}
%   \end{minipage}
% \end{framed}

\begin{landscape}
  % choice
  \noindent
  \begin{small}
  \begin{prooftree}
    \AxiomC{$$}\RightLabel{[ID]}
    \UnaryInfC{$\emptyset \mid  g^{cf}:(B \rightarrow E) \vdash g: (B \rightarrow E)$}

    \AxiomC{$$}\RightLabel{[ID]}
    \UnaryInfC{$\emptyset \mid f^{cg}: (A \rightarrow E) \vdash f: (A \rightarrow E)$}
    \AxiomC{$$}\RightLabel{[ID]}
    \UnaryInfC{$\emptyset \mid c^{fg}: ((A \rightarrow E) \rightarrow (B \rightarrow E) \rightarrow E)
      \vdash c:((A \rightarrow E) \rightarrow (B \rightarrow E) \rightarrow E)$}\RightLabel{[$\rightarrow$E]}
    \BinaryInfC{$\emptyset \mid c^{fg}:((A \rightarrow E) \rightarrow (B \rightarrow E) \rightarrow E)
      \varoplus f^{cg}:(A \rightarrow E) \vdash c f: (B \rightarrow E) \rightarrow E$}\RightLabel{[$\rightarrow$E]}

    \BinaryInfC{$\emptyset \mid  c^{fg}:((A \rightarrow E) \rightarrow (B \rightarrow E) \rightarrow E)  \varoplus
      f^{cg}:(A \rightarrow E)  \varoplus g^{cf}:(B \rightarrow E) \vdash c f g : E$}\RightLabel{[$\rightarrow$I]}
    \UnaryInfC{$\emptyset \mid  c^{f}:((A \rightarrow E) \rightarrow (B \rightarrow E) \rightarrow E) \varoplus
      f^{c}:(A \rightarrow E) \vdash \lambda^{\alpha} g. c f g :(B \rightarrow E) \rightarrow  E$}\RightLabel{[$\rightarrow$I]}
    \UnaryInfC{$\emptyset \mid  c^{\emptyset}:((A \rightarrow E) \rightarrow (B \rightarrow E) \rightarrow E)
      \vdash  \lambda^{\alpha} f. \lambda^{\alpha} g. c f g :(A \rightarrow E) \rightarrow (B \rightarrow E) \rightarrow E$}\RightLabel{[$\sepimp$I]}
    \UnaryInfC{$\emptyset \mid I \vdash \lambda^{*} c. \lambda^{\alpha} f. \lambda^{\alpha} g. c f g
      : ((A \rightarrow E) \rightarrow (B \rightarrow E) \rightarrow E) \sepimp (A \rightarrow E) \rightarrow (B \rightarrow E) \rightarrow E$}
  \end{prooftree}
\end{small}

\noindent
We now define sum type to be
\begin{framed}\centering
  $A \oplus B = (A \rightarrow E) \rightarrow (B \rightarrow E) \rightarrow E$\\
\end{framed}

We now define left and right {\it injections} or constructors for the sum type.
% left
\begin{prooftree}
  \AxiomC{$$}\RightLabel{[ID]}
  \UnaryInfC{$\emptyset \mid x^{fg}: A  \vdash x : A$}

  \AxiomC{$$}\RightLabel{[ID]}
  \UnaryInfC{$\emptyset \mid f^{gx}: (A \rightarrow E) \vdash f: (A \rightarrow E)$}\RightLabel{[$\rightarrow$E]}

  \BinaryInfC{$\emptyset \mid x^{fg}: A \varoplus f^{xg}: (A \rightarrow E)  \vdash f x : E$}\RightLabel{[WKN-UN]}
  \UnaryInfC{$\emptyset \mid x^{fg}: A \varoplus f^{xg}: (A \rightarrow E) \varoplus g^{fx}: (B \rightarrow E) \vdash f x : E$}\RightLabel{[$\rightarrow$I]}
  \UnaryInfC{$\emptyset \mid x^{f}: A \varoplus f^{x}: (A \rightarrow E) \vdash \lambda^{\alpha} g. f x : (B \rightarrow E) \rightarrow E$}\RightLabel{[$\rightarrow$I]}
  \UnaryInfC{$\emptyset \mid  x^{\emptyset}: A \vdash  \lambda^{\alpha} f. \lambda^{\alpha} g. f x
    :(A \rightarrow E) \rightarrow (B \rightarrow E) \rightarrow E$}\RightLabel{[$\sepimp$I]}\RightLabel{[$\sepimp$I]}
  \UnaryInfC{$\emptyset \mid I \vdash \lambda^{*} x. \lambda^{\alpha} f. \lambda^{\alpha} g. f x
    : A \sepimp  (A \rightarrow E) \rightarrow (B \rightarrow E) \rightarrow E$}
\end{prooftree}

\noindent Left injection defined below as:
\begin{framed}\centering
  $\texttt{inl} : A \sepimp A \oplus B$\\
  $\texttt{inl} = \lambda^{*} x. \lambda^{\alpha} f. \lambda^{\alpha} g. f x$
\end{framed}

% right
\noindent
\begin{prooftree}
  \AxiomC{$$}\RightLabel{[ID]}
  \UnaryInfC{$\emptyset \mid y^{fg}: B  \vdash y : B$}

  \AxiomC{$$}\RightLabel{[ID]}
  \UnaryInfC{$\emptyset \mid g^{yf}: (B \sepimp E) \vdash g: (B \sepimp E)$}\RightLabel{[$\rightarrow$I]}

  \BinaryInfC{$\emptyset \mid y^{fg}: B \varoplus g^{yf}: (B \rightarrow E) \vdash g y : E$}\RightLabel{[WKN-UN]}
  \UnaryInfC{$\emptyset \mid y^{fg}: B \varoplus f^{yg}: (A \rightarrow E) \varoplus g^{yf}: (B \rightarrow E) \vdash g y : E$}\RightLabel{[$\rightarrow$I]}
  \UnaryInfC{$\emptyset \mid y^{f}: B \varoplus f^{y}: (A \rightarrow E) \vdash \lambda^{\alpha} g. g y : (B \rightarrow E) \rightarrow E$}\RightLabel{[$\rightarrow$I]}
  \UnaryInfC{$\emptyset \mid  y^{\emptyset}: B \vdash \lambda^{\alpha} f. \lambda^{\alpha} g. g y
    :(A \rightarrow E) \rightarrow (B \rightarrow E) \rightarrow E$}\RightLabel{[$\sepimp$I]}\RightLabel{[$\sepimp$I]}
  \UnaryInfC{$\emptyset \mid I \vdash \lambda^{*} y. \lambda^{\alpha} f. \lambda^{\alpha} g. g y: B \sepimp  (A \rightarrow E) \rightarrow (B \rightarrow E) \rightarrow E$}
\end{prooftree}

\noindent Right injection defined below as:
\begin{framed}\centering
  $\texttt{inr} : B \sepimp A \oplus B$\\
  $\texttt{inr} = \lambda^{*} y. \lambda^{\alpha} f. \lambda^{\alpha} g. g y$
\end{framed}
\end{landscape}

We can now derive sum types in our language using the auxilary definitions given above
and provide a new syntax for deconstructing the sum type by matching on its structure by a case statement.
\begin{framed}\centering
    $\texttt{case}\ {c}\ \texttt{of}\ {\{ f;g \}} = \lambda^{*}c. \lambda^{\alpha}f. \lambda^{\alpha}g. c f g$
\end{framed}

\begin{framed}
  \begin{minipage}[h]{0.5\linewidth}
    \begin{prooftree}
      \AxiomC{$\emptyset \mid \texttt{inl}: A \sepimp A \oplus B \vdash x : A$}\RightLabel{[$\oplus$I$_1$]}
      \UnaryInfC{$\emptyset \mid I \vdash \texttt{inl}\ x: A \oplus B$}
    \end{prooftree}
  \end{minipage}
  \begin{minipage}[h]{0.5\linewidth}
    \begin{prooftree}
      \AxiomC{$\emptyset \mid \texttt{inr}: B \sepimp A \oplus B \vdash y : B$}\RightLabel{[$\oplus$I$_2$]}
      \UnaryInfC{$\emptyset \mid I \vdash \texttt{inr}\ y: A \oplus B$}
    \end{prooftree}
  \end{minipage}
  \begin{minipage}[h]{1.0\linewidth}
    \begin{prooftree}
      \AxiomC{$\emptyset \mid \Gamma \vdash M : A \oplus B$}\RightLabel{[$\oplus$E]}
      \AxiomC{$\emptyset \mid \Gamma \varoplus x:A \vdash N_1 : E$}
      \AxiomC{$\emptyset \mid \Gamma \varoplus y:B \vdash N_2 : E$}
      \TrinaryInfC{$\emptyset \mid \Gamma \vdash \Case{M}{\{ \texttt{inl}\ x \mapsto N_1; \texttt{inr}\ y \mapsto N_2\}}:E$}
    \end{prooftree}
  \end{minipage}
\end{framed}


%%% Local Variables:
%%% mode: latex
%%% TeX-master: "../thesis-ku"
%%% End:

\chapter{Proofs}
\begin{theorem}[Soundness of $\vdashs$]\label{thm:soundness-syntax-directed}
   If $P \mid \Gamma \vdashs M:\tau$ then $P \mid \Gamma \vdash M:\tau$
\end{theorem}
\begin{proof}\label{prf:soundness-syntax-directed}
  Proof by structural induction on the derivation of $P \mid \Gamma \vdashs M:\tau$.

  \begin{itemize}
  \item{Case [VAR$^s$].}
    By induction hypothesis we have a derivation of $P \mid x:\sigma \vdash x:\sigma$ by [VAR] rule.
    We proceed by repeated application of [$\forall$E] as $(Q => \tau) \sqsubseteq \sigma$, and
    we construct a derivation of $P \mid x:\sigma \vdash x: Q => \tau$. Then as $P => Q$ we can
    repeatedly apply [$=>$E] to construct a derivation of $P \mid x:\sigma \vdash x:\tau$.
    Finally, depending on whether the bindings are in sharing with or separate from
    $x$ we repeatedly apply [WKN-SH] or [WKN-UN] respectively for all the bindings in $\Gamma$ to construct
    a derivation of $P \mid \Gamma \sqcup x:\sigma \vdash x:\tau$.
  \item{Case [$\rightarrow$I$^s$].}
    By induction hypothesis we have a derivation of $P \mid \Gamma \varoplus x:\tau \vdash M:\tau'$.
    We apply [$\rightarrow$I] and reuse the derivations for $\ShFun{\phi}$ and $\Gamma \geq \phi$  to
    construct a derivation of $P \mid \Gamma \vdash \lambda^{\alpha}x. M: \phi \tau \tau'$.
  \item{Case [$\sepimp$I$^s$].}
    By induction hypothesis we have a derivation of $P \mid \Gamma \circledast x:\tau \vdash M:\tau'$.
    Similar to previous case, we apply [$\sepimp$I] and reuse the derivations for $\SeFun{\phi}$ and $\Gamma \geq \phi$  to
    construct a derivation of $P \mid \Gamma \vdash \lambda^{*}x. M: \phi \tau \tau'$.
  \item{Case [App$^s$].}
    By induction hypothesis we have derivations of $P \mid \Gamma \circledast \Delta \vdash M: \phi v \tau$ and
    $P \mid \Gamma' \circledast \Delta \vdash: N:v$ we check for $\texttt{Used}(\Gamma) = \texttt{Used}(\Gamma')$ and if it is true
    and apply [$\rightarrow$E] or check for $\texttt{Used}(\Gamma)\#\texttt{Shared}(\Gamma') \wedge \texttt{Shared}(\Gamma')\#\texttt{Used}(\Gamma)$
    and if it is true we apply [$\sepimp$E] to reuse derivations of $\ShFun{\phi}$ or $\SeFun{\phi}$ respectively to construct the derivation
    of $P \mid (\Gamma \varoplus \Gamma') \circledast \Delta \vdash M N:\tau$ or $P \mid (\Gamma \circledast \Gamma') \circledast \Delta \vdash M N:\tau$.
  \item{Case [Let$^s$].}
    By induction hypothesis have a derivation of $P \mid \Gamma \circledast \Delta \vdash M: \tau$ and $P \mid \Gamma' \sqcup x:\tau \circledast \Delta \vdash N:\tau$
    Applying [$\forall$I] and [$=>$I] on the first hypothesis we derive $\emptyset \mid \Gamma \circledast \Delta \vdash M: \sigma$. Now by applying
    the [LET] rule and reusing $P \vdash \Delta\ \texttt{un}$ we construct the derivation of
    $P \mid \Gamma \sqcup \Gamma' \circledast \Delta \vdash \Let{x}{M}{N}:\tau$.
  \end{itemize}
\end{proof}

\begin{theorem}[Completeness of $\vdashs$]\label{thm:completeness-syntax-directed}
  If $P \mid \Gamma \vdash M:\tau$ then
  $\exists ! Q,\tau. Q \mid \Gamma \vdashs M:\tau$
  and $(P \mid \sigma) \sqsubseteq \texttt{Gen}(\Gamma, Q => \tau)$
\end{theorem}
\begin{proof}
  \TODO{some induction here}.
\end{proof}

\begin{theorem}[Soundness of $\mathcal{M}$.]
   If $\mathcal{M}(S, X; \Gamma \vdash M : \tau) = P, S', \Sigma$ then $S' P | S' (\Gamma\mid_{\Sigma}) \vdash M : S' \tau$
\end{theorem}
\begin{proof}
  Proof by induction on structure of $M$
  \begin{itemize}
  \item Case 1. $x$
  \item Case 2. $\lambda^{*} x. M$
  \item Case 3. $\lambda ^{\alpha}x. M$
  \item Case 4. $M\ N$
  \item Case 5. $\texttt{let}\ x = M\ \texttt{in}\ N$
  \end{itemize}
\end{proof}

\begin{theorem}[Completeness of $\M$.]
  If $S$ is a substitution ans X is a set of type variables such that
  $P \mid S\Gamma \vdashs M: S\tau$, and $S|_X = id$ then $\M(id, X; \Gamma \vdash M:\tau) = Q, S', \Sigma$, such that
  $P => S\tau \sqsubset \texttt{GenI}(S'\Gamma, S' Q => S' \tau)$
\end{theorem}
\begin{proof}
\TODO{some more induction here}
\end{proof}

\begin{theorem}[Principal types.]
  If $P_0 \mid  \Gamma \vdash M : \sigma_0$ and $P_1 \mid  \Gamma \vdash M : \sigma_1$ then there is some $\sigma$ such that
  $\emptyset \mid \Gamma \vdash M : \sigma$ and $(P_0 | \sigma_0) \subseteq \sigma$, and $(P_1 | \sigma_1) \subseteq \sigma$.
\end{theorem}
\begin{proof}
\TODO{some more induction here. Makes use of theorems \ref{thm:soundness-syntax-directed} and \ref{thm:completeness-syntax-directed}}
\end{proof}



\TODO{
  \begin{theorem}[Progress.]
    If $M:\tau$ then either  $M \leadsto M'$ or $M$ is a value.
  \end{theorem}
\begin{proof}
  we have to define $\beta \eta$ rules.
\end{proof}

\begin{theorem}[Preservation.]
  If $M:\tau$ and $M \leadsto M'$ then $M':\tau$
\end{theorem}
\begin{proof}
\end{proof}
}


%%% Local Variables:
%%% mode: latex
%%% TeX-master: "../thesis-ku"
%%% End:

\end{document}

%%% Local Variables:
%%% mode: latex
%%% TeX-master: t
%%% End:
